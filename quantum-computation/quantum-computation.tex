%  \documentclass[DIV=12, a4]{scrartcl}
%\documentclass[12pt, a5]{scrartcl}

\documentclass[a4paper]{article}
\usepackage[
% fancytheorems, 
noindent, 
%spacingfix, 
%noheader
]{vanilla}


% \documentclass[a4paper]{scrreprt}
% \usepackage[
% fancytheorems, 
% noindent, 
% % %spacingfix, 
% % %noheader,
% fancyproofs
% ]{adam} 

\usepackage{tikz}
\usepackage{physics}

% \usepackage{subfig}

% \setcounter{chapter}{-1}

\title{Quantum Information \& Computation}
% \subtitle{Adam Kelly}
\author{Adam Kelly}
% \date{Michaelmas 2020}
\date{\today}

\begin{document}

\maketitle

% \begin{abstract}
	This set of notes is a work-in-progress account of the course `Quantum Information \& Computation', originally lectured by Prof Richard Jozsa in Lent 2020 at Cambridge. These notes are not a direct transcription of the lectures, but they do roughly follow what was lectured (in content and in structure).

	These notes are likely to be more succinct than other lecture notes of mine, and I have left out various aspects of what was taught. If you spot any errors in this set of notes, I can be contacted at \href{mailto:ak2316@cam.ac.uk}{ak2316@cam.ac.uk}.
% \end{abstract}

% \tableofcontents

% \clearpage

% \section{Introduction}

% Why bother with \emph{quantum} computation and information? To answer this, it helps to consider the natural of computation and information in a classical sense.

% Classical information is usually expressed in bits and bit strings. Formally these are strings of Boolean variables with values 0 or 1.

% Computation, at a fundamental level, can be thought of as updating bit strings by prescribed sequences of steps (`the program'). These are usually elementary Boolean operations/gates, for example AND, OR, NOT, SWAP and so on. A crucial property here is that each operation takes a `fixed effort' to perform, independent of the length of the string.

\tableofcontents

\section{Principles of Quantum Mechanics}


\subsection{Dirac Notation}

Let $V$ be a finite dimensional complex vector space with a (hermitian) inner product. In Dirac notation, we write vectors as 
$\ket{v}$ called \vocab{ket vectors}. 

We will often work with two dimensional space $V_2$ with a chosen orthonormal basis $\{\ket{0}, \ket{1}\}$, labelled by bit values.

By convention, kets are always written as column vectors in components. For example,
$$
\ket{v} = a \ket{0} + b\ket{1} = \begin{pmatrix}
	a \\ b
\end{pmatrix}, \quad \quad a, b\in \C.
$$

The \emph{conjugate transpose} of $\ket{v}$ is a \vocab{bra vector}, written in mirror image notation, 
$$
\bra{v} = \ket{v}^{\dagger} = a^* \bra{0} + b^* \bra{1} = \begin{pmatrix}
	a^* & b^*
\end{pmatrix},
$$
and bras are always written as row vectors in components.

More formally, bra vectors $\bra{v}$ is an element of the duel vector space $V^*$ of $V$ under the canonical isomorphism $V \cong V^*$, given by the inner product. That is, $\bra{v}$ is a linear map $\ket{w} \mapsto$ the inner product of $\ket{v}$ with $\ket{w}$.
Note that this inner product is linear in $\ket{w}$ and antilinear in $\ket{v}$ (linear in $\bra{v}$).

If $\ket{w} = c \ket{0} + d\ket{1}$, then the inner product of $\ket{v}$ with $\ket{w}$ is written by juxtaposing the bra and ket, 
$$
\ip{v}{w} = \ket{v}^{\dagger} \ket{w} = \begin{pmatrix}
	a^* & b^*
\end{pmatrix} \begin{pmatrix}
	c \\ d
\end{pmatrix} = a^* c + b^* d.
$$

For example, the orthonormality of basis vectors can be written as $\ip{i}{j} = \delta_{ij}$.

\subsection{Tensor Products of Vectors}

For some vector space $V$ of dimension $m$ on a basis $\ket{e_1}, \dots, \ket{e_m}$, and another vector space $V$ of dimension $n$ on a basis $\ket{f_1}, \dots, \ket{f_n}$, the \vocab{tensor product space} $V \otimes W$ has dimension $mn$ with orthonormal basis $\{ \ket{e_i} \otimes \ket{f_j} \}$, $i \in \{1, \dots, m\}$, $j \in \{1, \dots, n\}$, where $\otimes$ is bilinear.
Then a general ket vector in $V \otimes W$ is
$$
\ket{v} = \sum c_{ij} \ket{e_i} \otimes \ket{f_j}.
$$

There is a natural \emph{bilinear} map $f: V \times W \rightarrow V \otimes W$. If $\ket{\alpha} = \sum a_i \ket{e_i}$ and $\ket{\beta} = \sum b_j \ket{f_j}$, then
\begin{align*}
	(\ket{\alpha}, \ket{\beta}) &\mapsto f \mapsto \ket{a} \otimes \ket{b}\\
	&= \left(\sum a_i \ket{e_i}\right) \otimes \left(\sum b_j \ket{f_j}\right) \\
	&= \sum_{ij} a_i b_j \ket{e_i} \otimes \ket{f_j}.
\end{align*}

Note that $\otimes$ is not commutative. For example, if $V = W$ then $\ket{\alpha} otimes \ket{\beta} \neq \ket{\beta} \otimes \ket{\alpha}$ in general. We will often omit the symbol $\otimes$ and will write $\ket{\alpha} \otimes \ket{\beta}$ as $\ket{\alpha} \ket{\beta}$.

The mapping $f$ is \emph{not surjective}. 
With this in mind, we introduce the notions of \emph{product} and \emph{entangled} vectors.

\begin{definition}[Product and Entangled]
	Any $\ket{\xi} \in V \otimes W$ of the form $\ket{\xi} = \ket{\alpha} \otimes \ket{\beta}$ is called a \vocab{product vector}.
	Any $\ket{\xi}$ that is \emph{not} a product vector is called \vocab{entangled}.
\end{definition}

We will mostly be concerned with tensor products of the $2$ dimensional $V_2$ with itself (possible many times over).
For the $k$-fold tensor power, we write $\bigotimes^k V_2 = V_2 \otimes \cdots \otimes V_2$. This has dimension $2^k$ and orthonormal basis
$$
\ket{i_1} \otimes \cdots \otimes \ket{i_k}, \quad \quad i_1, \dots, i_k \in \{0, 1\}.
$$
These basis vectors are labelled by $2^k$ $k$-bitstrings. We will often write $\ket{i_1} \otimes \cdots \ket{i_k}$ as $\ket{i_1} \cdots \ket{i_k}$ or $\ket{i_1 \dots i_k}$.

\begin{example}
	The vector $\ket{v} = \ket{00} + \ket{11}$ in $V_2 \otimes V_2$ is \emph{entangled}. To see this, suppose we could write $\ket{v}$ as a product:
	\begin{align*}
	\ket{v} &= (a\ket{0} + b\ket{1})\otimes(c\ket{0} + d\ket{1}) \\
		&= ac \ket{00} + ad \ket{01} + bc \ket{10} + bd \ket{11}.
\end{align*}
Comparing with the coefficients of $\ket{v}$, we get $ad = 1$, $ad = 0$, $bc = 0$ and $ad = 1$. But then $abcd = (ac)(bd) = 1$ and $abcd = (ad)(bc) = 0$, which is a contradiction. Thus $\ket{v}$ must be entangled.
\end{example}
We can show that
$$
\ket{v} = a_{00} \ket{00} +a_{01} \ket{01} +a_{10} \ket{10} +a_{11} \ket{11} 
$$
is entangled if and only if $\det(a_{ij}) \neq 0$. For general dimensions,
$$
\sum_{i = 1, j = 1}^{m, n} A_{ij} \ket{i} \ket{j}
$$
is a product vector if and only if the matrix $[A_{ij}]$ has rank 1.

The inner product on $V \otimes W$ is induced the inner products on $V$ and $W$, `applied slotwise'. For product states $\ket{\alpha_1} \ket{\beta_2}$ and $\ket{\alpha_2} \ket{\beta_2}$, the inner product is
$$
(\bra{b_1} \bra{\alpha_1})(\ket{\alpha_2} \ket{\beta_2}) = \ip{\alpha_1}{\alpha_2} \ip{\beta_1}{\beta_2},
$$
and we extend by linearity to all $\ket{\xi} \in V \otimes W$.

\begin{remark}[Notation]
	For the bra vector of $\ket{\alpha} \ket{\beta}$, we often reverse the order and write is as $\bra{\beta} \bra{\alpha}$. It is always important to keep track of component slots. We sometimes include explicit labels (for example, $\ket{\alpha}_A \ket{\beta}_B$ has bra vector ${}_A\bra{\alpha} {}_B \ket{\beta} = {}_B\ket{\beta} {}_A \ket{\alpha}$). 
\end{remark}

\subsection{Quantum Principles}

We will now state some axioms that describe quantum mechanics.

\subsubsection{QM1 -- Physical States}

\begin{axiom*}[QM1]
	The state of any (isolated) physical system $S$ are represented by unit vectors in a complex vector space $V$ with a given inner product.
\end{axiom*}

The simplest nontrivial case is $V = V_2$, the two dimensional complex vector space. We choose a pair of orthonormal vectors $\ket{0}$ and $\ket{1}$. Then a general state is
$$
\ket{\psi} = \alpha \ket{0} + \beta \ket{1}, \quad \quad \alpha, \beta \in \C, \quad |\alpha|^2 + |\beta|^2 = 1.
$$
We say $\ket{\psi}$ is a \vocab{superposition} of $\ket{0}$ and $\ket{1}$ with \vocab{amplitudes} $\alpha$ and $\beta$ respectively.

\begin{definition}[Qubit]
	A \vocab{qubit} is any quantum system with a two dimensional state space and a chosen orthonormal basis labelled $\ket{0}$, $\ket{1}$ called the \vocab{computational} basis, \vocab{standard} basis or $Z$-basis.
\end{definition}

\begin{definition}[Conjugate Basis for a Qubit]
	Given an orthonormal pair $\ket{0}$ and $\ket{1}$, we get the orthonormal pair
	\begin{align*}
		\ket{+} &= \frac{1}{\sqrt{2}} \left(\ket{0} + \ket{1}\right),
		\ket{-} &= \frac{1}{\sqrt{2}} \left(\ket{0} - \ket{1}\right).
	\end{align*}
	We call this the \vocab{conjugate} basis or $X$-basis.
\end{definition}

\subsubsection{QM2 -- Composite Systems}

\begin{axiom*}[QM2]
	If system $S_1$ has state space $V_1$ and system $S_2$ has state space $V_2$ then the joint system $S_1 S_2$, obtained by taking $S_1$ and $S_2$ together, has state space $V_1 \otimes V_2$.
\end{axiom*}

Comparing this axiom with the classical analog,
the corresponding statement has a Cartesian product rather than a tensor product. So giving a state for a classical $S_1S_2$ is just giving a state for $S_1$ and giving a state for $S_2$. Thus the dimension of the system grows linearly with the number of systems, whereas in the quantum sense (with this axiom), the composite system grows \emph{exponentially} with the number of systems.

\begin{example}[$n$-qubit system]
	An $n$-qubit system has as state space $V_2^{\otimes n}$, with dimension $2^n$. It has the computational basis $\ket{i_1 \dots i_n}$, labelled by all $2^n$ $n$-bit string.

	An $n$ qubit state $\ket{\psi}$ is a \vocab{product state} if is the tensor product of $n$ single qubit states $\ket{\psi} = \ket{v_1} \ket{v_2} \cdots \ket{v_n}$, otherwise it is \vocab{entangled}.
\end{example}

Before we state the next axiom, we need to introduce how to work with linear maps in Dirac notation.

Consider linear maps on $V_2$ (higher-dimensions are similar) with $\ket{v} = a \ket{0} + b \ket{1}$ and $\ket{w} = c\ket{0} + d\ket{1}$. Then the `ket-bra' product is
$$
M = \op{v}{w} = \begin{pmatrix}
	a \\ b
\end{pmatrix}\begin{pmatrix}
	c^* &  d^*
\end{pmatrix} = \begin{pmatrix}
ac^* & ad^* \\
bc^* &  bd^*
\end{pmatrix},
$$
which is a linear map on $V_2$. For any $\ket{x} = (x_0, x_1)^T$, we have
$$
M \ket{x} = (\op{v}{w}) \ket{x} = \ket{v} \ip{w}{x}.
$$
These are all rank 1 mappings, and thus is not the most general linear map.

For a general linear map say
$$
A = \begin{pmatrix}
	a_{00} & a_{01} \\
	a_{10} & a_{11}
\end{pmatrix},
$$
we note that for basis states,
\begin{align*}
	\op{0}{0} = \begin{pmatrix}
		1 & 0 \\
		0 & 0 
	\end{pmatrix}, \quad \op{0}{1} = \begin{pmatrix}
		0 & 1 \\
		0 & 0 
	\end{pmatrix}, \quad \op{1}{0} = \begin{pmatrix}
		0 & 0 \\
		1 & 0 
	\end{pmatrix}, \quad \op{1}{1} = \begin{pmatrix}
		0 & 0 \\
		0 & 1 
	\end{pmatrix},
\end{align*}
and we can see that these are a basis for $2x2$ matrices, and we can express
$$
A = a_{00} \op{0}{0} +  a_{01} \op{0}{1} +  a_{10} \op{1}{0} +  a_{11} \op{1}{1} = \sum a_{ij} \op{i}{j}.
$$

Another useful property comes from the cycic property of the trace:
$$
\ip{v}{w} = \tr(\op{v}{w}).
$$

An important operation comes from the above when $\ket{v} = \ket{w}$ (in any dimension), and where $\ket{v}$ is normalized so $\ip{v}{v} = 1$. Then $\Pi_v = \op{v}{u}$ is the operation of projection onto the 1-dimensional subspace spanned by $\ket{v}$. For example, we have
\begin{align*}
\Pi_v \Pi_v = \ket{v}\ip{v}{v} = \op{v}{v} = \Pi_v.
\end{align*}
More generally if $E$ is any $d$-dimensional linear subspace of $n$-dimensional space $V$ and $\{\ket{e_1}, \dots, \ket{e_d}\}$ is any orthonormal basis of $E$, then $\Pi_E = \op{e_1}{e_1} + \cdots + \op{e_d}{e_d}$ is the operation of projection on to $E$. $\Pi_E$ is independent of the orthonormal basis chosen.

We also want to have tensor products of maps. If 
$$
A = \begin{pmatrix}
	a & b \\ c & d
\end{pmatrix}, \quad \text{and} \quad B= \begin{pmatrix}
	p & q \\ v& s
\end{pmatrix}
$$
are linear maps on $V_2$ then $A \otimes B : V_2 \otimes V_2 \rightarrow V_2 \otimes V_2$ defined by the basis action
$$
\ket{i} \ket{j} \longmapsto (A \ket{i})(B \ket{j}),
$$
with linear extension.

For example, for any product sate $\ket{v}\ket{w}$, we have $(A \otimes B) \ket{v} \ket{w} = (A \ket{v})(B \ket{w})$.

In components for the $2\times 2$ case we have
$$
\sbox0{$\begin{matrix}ap&aq\\av&as\end{matrix}$}
A \otimes B = \begin{pmatrix}
	aB & bB \\
	cB & dB
\end{pmatrix} = \left(
	\begin{array}{c|c}
	\usebox{0}&\makebox[\wd0]{\large $bB$}\\
	\hline
	  \vphantom{\usebox{0}}\makebox[\wd0]{\large $cB$}&\makebox[\wd0]{\large $dB$}
	\end{array}
	\right),
$$
with other cases done similarily.

Some important special cases are $I \otimes A$ and $A \otimes I$, the action of $A$ on the second (or first) component space of $V_2 \otimes V_2$. We often refer to these as \vocab{local operations} on subsystems of composite systems.

\subsubsection{QM3 -- Physical Evolution of Quantum Systems}

\begin{axiom*}[QM3]
	Any physical (finite time) evolution of a quantum system is represented by a \vocab{unitary} linear operation on the vector space of states.
\end{axiom*}

The analog to this from the Quantum Mechanics course would be Schr\"odinger's equation and specifically is the finite dimensional version of Schr\"odinger's equation, where instead of a hamiltonian we have a hermitian operation on the state space, where evolution is
$$
\frac{\mathrm{d}\ket{\psi}}{\mathrm{d}t} = H \ket{\psi}
$$
If $H$ is time independent, then $\ket{\psi} = e^{-\frac{i}{\hbar}Ht}\ket{\psi_0}$, which is the matrix exponential. This also implies that $A$ hermitian gives $e^{iA}$ unitary.

Recall that a matrix $U$ is unitary if $U^{-1} = U^{\dagger}$, that is, if and only if $U$ maps any orthonormal basis to an orthonormal set of vectors, which occurs if and only if the columns or rows of the matrix of $U$ form an orthonormal set of vectors.

The last piece of notation we will introduce is the partial inner products. For vectors in $V \otimes W$, then any ket $\ket{v} \in V$ defines a linear map $V \otimes W \rightarrow W$ called the `partial inner product with $\ket{v}$', defined on basis vectors $\ket{e_i} \ket{f_j} \in V \otimes W$ where $\ket{e_i}\ket{f_j} \mapsto \ip{v}{e_i} \ket{f_j}$, and similarity for $\ket{w} \in W$ giving a map $V \otimes W \rightarrow V$). If $V = W$ (as often occurs), it is important to specify which space of $V \otimes V$ is being used.


\begin{example}
	For $V = V_2$, and $\ket{\xi} = a \ket{00} + b\ket{01} + c \ket{10} + d\ket{11} \in V \otimes V$, we can form the partial inner product with $\ket{0}$ on \emph{either} space. So first label the spaces as $V_A \otimes V_B$ where $V_A = V_B = V$. Then the orthonormality relations $\ip{i}{j} = \delta_{ij}$ gives
	\begin{align*}
		{}_A\ip{0}{\xi}_{AB} &= a {}_A\ip{0}{0}_A \ket{0}_B + b {}_A\ip{0}{0}_A \ket{1}_B + c{}_A\ip{0}{1}_A \ket{0}_B + d{}_A\ip{0}{1}_A \ket{1}_B \\
		&= a\ket{0}_B + b\ket{1}_B. 
	\end{align*}
	That is, we pick out the terms of $\ket{\xi}_{AB}$ with a zero in the $A$ slot.
\end{example}

\subsubsection{QM4 -- Quantum Measurement \& Born Rule}

This axiom describes how classical information is extracted from quantum states. Fundamentally there is a physical limitation on this process and it is rather different than the extraction of classical information.

Typically in a classical setting, we assume that measurements can occur without changing the physical system being studied. For example, if you measured the height of a tree, you wouldn't expect the act of measuring to make the tree any smaller. This is different in a quantum setting, as we shall see.

We will work with a given \emph{single} instance of a quantum state $\ket{\psi}$ of a physical system, with a state space $V$ of dimension $n$.

We first have the \vocab{basic Born rule}, a complete projective measurement (or von Neumann measurement). Let $B = \{\ket{e_1}, \dots, \ket{e_n}\}$ be any orthonormal basis of $V$. We can make a measurement on $\ket{\psi}$ relative to the basis $B$.

Letting $\ket{\psi} = \sum a_i \ket{e_i}$, the possible measurement outcomes are $j = 1, 2, \dots, n$, and the probability of getting an outcome $j$ is $\operatorname{prob}(j) = |\ip{e_j}{\psi}|^2 = |a_j|^2$. If outcome $j$ is seen, then the state is no longer $\ket{\psi}$, but it has been \vocab{collapsed} to $\ket{\psi_j} = \ket{e_j}$.

Clearly this transformation is not unitary, and thus we have a completely different physical process to evolution as described before.
Notably, repeated measurement gives only the same result, with certainty. We would not get further samples of the $|a_j|^2$ distribution.

To rephrase, the probability is the squared length of $\ket{\psi}$ onto the basis direction $\ket{e_j}$, and the post measurement state is that projection, re-normalized to unit length.

The `complete' above refers to the \emph{one}-dimensionality of the orthonormal subspaces, defined by basis states. We can similarity describe incomplete projective measurements.

Let $\{E_1, E_2, \dots, E_d \}$ by a decomposition of $V$ into $d$ mutually orthogonal subspaces, so $V = E_1 \oplus \cdots \oplus E_d$, and let $\Pi_i$ be the projection onto $E_i$. Then the \vocab{incomplete} projective measurement of $\ket{\psi}$ with respect to the orthogonal decomposition is the following quantum operation. The possible outcomes are $j = 1, \dots, d$, and $\operatorname{prob}(j) = \text{squared length of projection of $\ket{\psi}$ into $E_j$} = \mel{\psi}{\Pi_j^{\dagger} \Pi_j}{\psi} = \mel{\psi}{\Pi_j}{\psi}$.
(Recall that $\Pi \Pi = \Pi$ and $\Pi^{\dagger} = \Pi$). This is also $\operatorname{tr}(\Pi_j \op{\psi}{\psi})$.

The post measurement state is the projected `collapsed' vector, re-normalised. So $\ket{\psi_j} = \frac{\Pi_j \ket{\psi}}{\operatorname{prob}(j)}$.

\begin{example}[Parity Measurement on 2-Qubits]
	The parity of a 2-bit string $b_1b_2$ is the modulo 2 sum $b_1 \oplus b_2$. The parity measurement on two qubits is teh incomplete measurement with orthogonal decomposition
	$$
	E_0 = \operatorname{span}\{\ket{00}, \ket{11}\}, \quad \quad E_1 = \operatorname{span}\{\ket{01}, \ket{10}\}.
	$$
	For $\ket{\psi} = a \ket{00} + b\ket{01} + c\ket{10} + d\ket{11}$, we see $0$ with probability $|a|^2 + |d|^2$, and the post measurement state would be $\frac{a \ket{00} + d \ket{11}}{\sqrt{|a|^2 + |d|^2}}$.
\end{example}

In some texts (especially pre-2000s) they will typically introduce measurement in the context of `quantum observables'.

\begin{definition}[Observable]
	A quantum observable $\theta$ is a hermitian operator on $V$ so that $\theta$ has real eigenvalues $\lambda_j$ for $j = 1, \dots, d$, and orthogonal eigenspaces $\Lambda_j$, with $\operatorname{dim}(\Lambda_j)$ being the multiplicity of $\lambda_j$, and $V = \Lambda_1 \oplus \cdots \oplus \Lambda_d$. 
\end{definition}

The measurement of a quantum observable $\theta$ is then just the incomplete measurement relative to this orthogonal decomposition.
However, these observables' eigenvalues relate to physical properties, and we thus normally label the outcomes by their eigenvalues $\lambda_j$, not just $j$ as before. We also have things like $\expval{\theta} = \sum \lambda_j \operatorname{prob}(j)$, which obviously depend on the eigenvalues. 

But for the purposes of providing information about $\ket{\psi}$ and its post measurement state, the choice of labelling is immaterial and thus we won't be that interested in observables.

A special case of incomplete measurement relates to measuring a component part of a quantum system, the \vocab{extended Born rule}.

Suppose $\ket{\psi}$ is a state of a composite system $S_1 S_2$ with state space $V \otimes W$. Let $B_V = \{\ket{e_1}, \dots, \ket{e_n}\}$ be an orthonormal basis of $V$. We can uniquely express $\ket{\psi}$ uniquely as $\ket{\psi}_{VW} = \sum_i \ket{e_i}_V \ket{\xi_i}_W$, where $\ket{\psi_i}$ are generally not normalized, and not orthogonal. In fact, $\ket{\xi_i}_W = {}_V\ip{e_i}{\psi}_{VW}$.
Thus $\ket{\psi}$ normalised implies $\sum_{i = 1}^n \ip{\xi_i}{\xi_i} = 1 = \ip{\psi}{\psi}$.

Now, if we say perform a measurement of $\ket{\psi}$ relative to a basis $B_V$ of $V$ (a `complete measurement on $V$ but not $V \otimes W$') with outcomes $i = 1, \dots, n$, and corresponding orthogonal subspaces (of $V \otimes W$), $E_i = \operatorname{span} \{\ket{e_i} \otimes \ket{\xi} \mid \ket{\xi} \in W\}$, which we write as $\ket{e_i} \otimes W$. The corresponding projectors are $\Pi_i = \op{e_i}{e_i}\otimes I_W $, and $\operatorname{prob}(i) = \ip{\xi_i}{\xi_i}$. The post measurement state for $i$ is $\ket{\psi_i} = \frac{\ket{e_i}_V \ket{\xi_i}_w}{\sqrt{\ip{\xi_i}{\xi_i}}}$.

\begin{remark}
	According to QM4 (all of the measurement rules), two different states with guaranteed (probability 1) different outcomes for some measurement, must lie in orthogonal subspaces (they must be orthogonal themselves). So non-orthogonal state, although physically different, can never be reliably distinguished (`as information') by any quantum process.

	Also, if $\ket{\psi}$ has dimension $n$, any measurement on it has at most $n$ outcomes. However, we can get more outcomes by adjoining an ancilla $\ket{A}$ of dimension $m$ (independent of $\ket{\psi}$) and measure $\ket{\psi}\ket{A}$ to get up to $mn$ outcomes. 
\end{remark}

\end{document}
