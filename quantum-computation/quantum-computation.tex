%  \documentclass[DIV=12, a4]{scrartcl}
%\documentclass[12pt, a5]{scrartcl}

\documentclass[a4paper]{article}
\usepackage[
% fancytheorems, 
noindent, 
%spacingfix, 
%noheader
]{vanilla}


% \documentclass[a4paper]{scrreprt}
% \usepackage[
% fancytheorems, 
% noindent, 
% % %spacingfix, 
% % %noheader,
% fancyproofs
% ]{adam} 

\usepackage{tikz}

% \usepackage{subfig}

% \setcounter{chapter}{-1}

\title{Quantum Information \& Computation}
% \subtitle{Adam Kelly}
\author{Adam Kelly}
% \date{Michaelmas 2020}
\date{\today}

\begin{document}

\maketitle

% \begin{abstract}
	This set of notes is a work-in-progress account of the course `Quantum Information \& Computation', originally lectured by Prof Richard Jozsa in Lent 2020 at Cambridge. These notes are not a direct transcription of the lectures, but they do roughly follow what was lectured (in content and in structure).

	These notes are likely to be more succinct than other lecture notes of mine, and I have left out various aspects of what was taught. If you spot any errors in this set of notes, I can be contacted at \href{mailto:ak2316@cam.ac.uk}{ak2316@cam.ac.uk}.
% \end{abstract}

% \tableofcontents

% \clearpage

\section{Introduction}

Why bother with \emph{quantum} computation and information? To answer this, it helps to consider the natural of computation and information in a classical sense.

Classical information is usually expressed in bits and bit strings. Formally these are strings of Boolean variables with values 0 or 1.

Computation, at a fundamental level, can be thought of as updating bit strings by prescribed sequences of steps (`the program'). These are usually elementary Boolean operations/gates, for example AND, OR, NOT, SWAP and so on. A crucial property here is that each operation takes a `fixed effort' to perform, independent of the length of the string.



% \clearpage
% \chapter{Introduction}

% For many people, `Graph Theory' is a first course in combinatorics. It's an area with a big focus on problem solving, and it can give a perspective on many other areas of mathematics.

% \section{Structure of the Course}

% Every part of this course is really nice, and there is an emphasis on breath rather than depth in the treatment of various areas of graph theory.

% \begin{enumerate}
% 	\item \emph{Introduction to Graph Theory}

% 	We begin with the boring lectures introducting the basic definitions such as what a graph is. We need to develop some vocabulary and prove some basic propositions about graphs, and then we will look at planar graphs and Kuratowski's theorem.

% 	\item \emph{Connectivity and Matchings}
	
% 	The next section will look at `matchings'. You may be familiar with Hall's marriage theorem, but this is just one part of the area of matchings, some of which we will discuss.

% 	\item \emph{Extremal Graph Theory}
	
% 	In this section we will look at some incredibly hard problems with simple statements. 

% 	\item \emph{Eigenvalue Methods}
	
% 	Overlapping with linear algebra, we will see in this section how graphs represented by matrices can be reasoned about through their eigenvalues.

% 	\item \emph{Graph Colouring}
	
% 	This section looks as the basics of graph colouring, including Brooks' and Vizing's theorems. We will also briefly discuss the four colour theorem.
	
% 	\item \emph{Ramsey Theory}
	
% 	This section we will discuss the basics of Ramsey theory, and we will prove some of the classical results.
	
% 	\item \emph{Probabilistic Methods}
	
% 	Lastly we will look at how probability can be used to answer questions in graph theory (and generally in combinatorics), both with randomness and with determinism.
% \end{enumerate}

% \section{Books}

% There are three books mentioned in the schedules for this course. All of them are quite good, but the book by Bollobás goes quite fast, and the West book take somewhat of a softer appraoch.

% If you are looking for additional problems, you can find many more in these books.

% \begin{itemize}
% 	\item B. Bollobás, \emph{Modern Graph Theory}.
% 	\item R. Diestel, \emph{Graph Theory}.
% 	\item D. West, \emph{Introduction to Graph Theory}.
% \end{itemize}


\end{document}
