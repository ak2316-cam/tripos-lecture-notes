%  \documentclass[DIV=12, a4]{scrartcl}
%\documentclass[12pt, a5]{scrartcl}

\documentclass[a4paper]{article}
\usepackage[
% fancytheorems, 
noindent, 
%spacingfix, 
%noheader
]{vanilla}


% \documentclass[a4paper]{scrreprt}
% \usepackage[
% fancytheorems, 
% noindent, 
% % %spacingfix, 
% % %noheader,
% fancyproofs
% ]{adam} 

\usepackage{tikz}
\usepackage{physics}

% \usepackage{subfig}

% \setcounter{chapter}{-1}

\title{Quantum Information \& Computation}
% \subtitle{Adam Kelly}
\author{Adam Kelly}
% \date{Michaelmas 2020}
\date{\today}

\begin{document}

\maketitle

% \begin{abstract}
	This set of notes is a work-in-progress account of the course `Quantum Information \& Computation', originally lectured by Prof Richard Jozsa in Lent 2020 at Cambridge. These notes are not a direct transcription of the lectures, but they do roughly follow what was lectured (in content and in structure).

	These notes are likely to be more succinct than other lecture notes of mine, and I have left out various aspects of what was taught. If you spot any errors in this set of notes, I can be contacted at \href{mailto:ak2316@cam.ac.uk}{ak2316@cam.ac.uk}.
% \end{abstract}

% \tableofcontents

% \clearpage

% \section{Introduction}

% Why bother with \emph{quantum} computation and information? To answer this, it helps to consider the natural of computation and information in a classical sense.

% Classical information is usually expressed in bits and bit strings. Formally these are strings of Boolean variables with values 0 or 1.

% Computation, at a fundamental level, can be thought of as updating bit strings by prescribed sequences of steps (`the program'). These are usually elementary Boolean operations/gates, for example AND, OR, NOT, SWAP and so on. A crucial property here is that each operation takes a `fixed effort' to perform, independent of the length of the string.


\section{Principles of Quantum Mechanics}


\subsection{Dirac Notation}

Let $V$ be a finite dimensional complex vector space with a (hermitian) inner product. In Dirac notation, we write vectors as 
$\ket{v}$ called \vocab{ket vectors}. 

We will often work with two dimensional space $V_2$ with a chosen orthonormal basis $\{\ket{0}, \ket{1}\}$, labelled by bit values.

By convention, kets are always written as column vectors in components. For example,
$$
\ket{v} = a \ket{0} + b\ket{1} = \begin{pmatrix}
	a \\ b
\end{pmatrix}, \quad \quad a, b\in \C.
$$

The \emph{conjugate transpose} of $\ket{v}$ is a \vocab{bra vector}, written in mirror image notation, 
$$
\bra{v} = \ket{v}^{\dagger} = a^* \bra{0} + b^* \bra{1} = \begin{pmatrix}
	a^* & b^*
\end{pmatrix},
$$
and bras are always written as row vectors in components.

More formally, bra vectors $\bra{v}$ is an element of the duel vector space $V^*$ of $V$ under the canonical isomorphism $V \cong V^*$, given by the inner product. That is, $\bra{v}$ is a linear map $\ket{w} \mapsto$ the inner product of $\ket{v}$ with $\ket{w}$.
Note that this inner product is linear in $\ket{w}$ and antilinear in $\ket{v}$ (linear in $\bra{v}$).

If $\ket{w} = c \ket{0} + d\ket{1}$, then the inner product of $\ket{v}$ with $\ket{w}$ is written by juxtaposing the bra and ket, 
$$
\ip{v}{w} = \ket{v}^{\dagger} \ket{w} = \begin{pmatrix}
	a^* & b^*
\end{pmatrix} \begin{pmatrix}
	c \\ d
\end{pmatrix} = a^* c + b^* d.
$$

For example, the orthonormality of basis vectors can be written as $\ip{i}{j} = \delta_{ij}$.

\subsection{Tensor Products of Vectors}

For some vector space $V$ of dimension $m$ on a basis $\ket{e_1}, \dots, \ket{e_m}$, and another vector space $V$ of dimension $n$ on a basis $\ket{f_1}, \dots, \ket{f_n}$, the \vocab{tensor product space} $V \otimes W$ has dimension $mn$ with orthonormal basis $\{ \ket{e_i} \otimes \ket{f_j} \}$, $i \in \{1, \dots, m\}$, $j \in \{1, \dots, n\}$, where $\otimes$ is bilinear.
Then a general ket vector in $V \otimes W$ is
$$
\ket{v} = \sum c_{ij} \ket{e_i} \times \ket{f_j}.
$$

There is a natural \emph{bilinear} map $f: V \times W \rightarrow V \otimes W$. If $\ket{\alpha} = \sum a_i \ket{e_i}$ and $\ket{\beta} = \sum b_j \ket{f_j}$, then
\begin{align*}
	(\ket{\alpha}, \ket{\beta}) &\mapsto f \mapsto \ket{a} \otimes \ket{b}\\
	&= \left(\sum a_i \ket{e_i}\right) \otimes \left(\sum b_j \ket{f_j}\right) \\
	&= \sum_{ij} a_i b_j \ket{e_i} \otimes \ket{f_j}.
\end{align*}

Note that $\times$ is not commutative. For example, if $V = W$ then $\ket{\alpha} otimes \ket{\beta} \neq \ket{\beta} \otimes \ket{\alpha}$ in general. We will often omit the symbol $\otimes$ and will write $\ket{\alpha} \otimes \ket{\beta}$ as $\ket{\alpha} \ket{\beta}$.

The mapping $f$ is \emph{not surjective}. 
With this in mind, we introduce the notions of \emph{product} and \emph{entangled} vectors.

\begin{definition}[Product and Entangled]
	Any $\ket{\xi} \in V \otimes W$ of the form $\ket{\xi} = \ket{\alpha} \otimes \ket{\beta}$ is called a \vocab{product vector}.
	Any $\ket{\xi}$ that is \emph{not} a product vector is called \vocab{entangled}.
\end{definition}

We will mostly be concerned with tensor products of the $2$ dimensional $V_2$ with itself (possible many times over).
For the $k$-fold tensor power, we write $\bigotimes^k V_2 = V_2 \otimes \cdots \otimes V_2$. This has dimension $2^k$ and orthonormal basis
$$
\ket{i_1} \otimes \cdots \otimes \ket{i_k}, \quad \quad i_1, \dots, i_k \in \{0, 1\}.
$$
These basis vectors are labelled by $2^k$ $k$-bitstrings. We will often write $\ket{i_1} \otimes \cdots \ket{i_k}$ as $\ket{i_1} \cdots \ket{i_k}$ or $\ket{i_1 \dots i_k}$.

\begin{example}
	The vector $\ket{v} = \ket{00} + \ket{11}$ in $V_2 \otimes V_2$ is \emph{entangled}. To see this, suppose we could write $\ket{v}$ as a product:
	\begin{align*}
	\ket{v} &= (a\ket{0} + b\ket{1})\otimes(c\ket{0} + d\ket{1}) \\
		&= ac \ket{00} + ad \ket{01} + bc \ket{10} + bd \ket{11}.
\end{align*}
Comparing with the coefficients of $\ket{v}$, we get $ad = 1$, $ad = 0$, $bc = 0$ and $ad = 1$. But then $abcd = (ac)(bd) = 1$ and $abcd = (ad)(bc) = 0$, which is a contradiction. Thus $\ket{v}$ must be entangled.
\end{example}
We can show that
$$
\ket{v} = a_{00} \ket{00} +a_{01} \ket{01} +a_{10} \ket{10} +a_{11} \ket{11} 
$$
is entangled if and only if $\det(a_{ij}) \neq 0$. For general dimensions,
$$
\sum_{i = 1, j = 1}^{m, n} A_{ij} \ket{i} \ket{j}
$$
is a product vector if and only if the matrix $[A_{ij}]$ has rank 1.

The inner product on $V \otimes W$ is induced the inner products on $V$ and $W$, `applied slotwise'. For product states $\ket{\alpha_1} \ket{\beta_2}$ and $\ket{\alpha_2} \ket{\beta_2}$, the inner product is
$$
(\bra{b_1} \bra{\alpha_1})(\ket{\alpha_2} \ket{\beta_2}) = \ip{\alpha_1}{\alpha_2} \ip{\beta_1}{\beta_2},
$$
and we extend by linearity to all $\ket{\xi} \in V \otimes W$.

\begin{remark}[Notation]
	For the bra vector of $\ket{\alpha} \ket{\beta}$, we often reverse the order and write is as $\bra{\beta} \bra{\alpha}$. It is always important to keep track of component slots. We sometimes include explicit labels (for example, $\ket{\alpha}_A \ket{\beta}_B$ has bra vector $._A\bra{\alpha} ._B \ket{\beta} = ._B\ket{\beta} ._A \ket{\alpha}$). 
\end{remark}

\subsection{Quantum Principles}

We will now state some axioms that describe quantum mechanics.

\begin{axiom*}[QM1 -- Physical States]
	The state of any (isolated) physical system $S$ are represented by unit vectors in a complex vector space $V$ with a given inner product.
\end{axiom*}

The simplest nontrivial case is $V = V_2$, the two dimensional complex vector space. We choose a pair of orthonormal vectors $\ket{0}$ and $\ket{1}$. Then a general state is
$$
\ket{\psi} = \alpha \ket{0} + \beta \ket{1}, \quad \quad \alpha, \beta \in \C, \quad |\alpha|^2 + |\beta|^2 = 1.
$$
We say $\ket{\psi}$ is a \vocab{superposition} of $\ket{0}$ and $\ket{1}$ with \vocab{amplitudes} $\alpha$ and $\beta$ respectively.

\begin{definition}[Qubit]
	A \vocab{qubit} is any quantum system with a two dimensional state space and a chosen orthonormal basis labelled $\ket{0}$, $\ket{1}$ called the \vocab{computational} basis, \vocab{standard} basis or $Z$-basis.
\end{definition}

\begin{definition}[Conjugate Basis for a Qubit]
	Given an orthonormal pair $\ket{0}$ and $\ket{1}$, we get the orthonormal pair
	\begin{align*}
		\ket{+} &= \frac{1}{\sqrt{2}} \left(\ket{0} + \ket{1}\right),
		\ket{-} &= \frac{1}{\sqrt{2}} \left(\ket{0} - \ket{1}\right).
	\end{align*}
	We call this the \vocab{conjugate} basis or $X$-basis.
\end{definition}

\begin{axiom*}[QM2 -- Composite Systems]
	If system $S_1$ has state space $V_1$ and system $S_2$ has state space $V_2$ then the joint system $S_1 S_2$, obtained by taking $S_1$ and $S_2$ together, has state space $V_1 \otimes V_2$.
\end{axiom*}

Comparing this axiom with the classical analog,
the corresponding statement has a Cartesian product rather than a tensor product. So giving a state for $S_1S_2$ is just giving a state for $S_1$ and giving a state for $S_2$. Thus the dimension of the system grows linearly with the number of systems, whereas in this axiom, the system grows \emph{exponentially} with the number of systems.

\end{document}
