\documentclass[a4paper, 10pt, twocolumn]{amsart}

\usepackage[nokoma,noindent,fancytheorems,fancyproofs]{adam}
\usepackage[margin=0.75in]{geometry}
\usepackage{tikz-cd}

\setlist[enumerate]{leftmargin=8mm}
\setlist[itemize]{leftmargin=8mm}

\newcommand{\enumpre}{\vspace{-1.5\baselineskip}}


\title{Number Theory}
\author{Adam Kelly -- Mathematical Tripos Part II}
\date{\today. Email \texttt{ak2316@cam.ac.uk}}

\begin{document}

\maketitle

\begin{center}
\emph{Familiarity with IA Numbers and Sets is assumed.}
\end{center}

\section{Divisibility}
\ 

% \begin{definition}[$\N$ and $\Z$]
%     We denote by $\N$ the set of \emph{natural numbers} $1, 2, 3, \dots$ and by $\Z$ the set of \emph{integers}.
% \end{definition}

% \begin{definition}[Division]
%     For $a, b \in \N$, we say that $b$ \emph{divides} $a$ if there exists $c \in \N$ such that $a = bc$. We then write $b \mid a$. If $b$ does not divide $a$, we write $b \nmid a$. 
% \end{definition}

\begin{theorem}[Division Algorithm]
    Given $a, b \in \Z$ with $b > 0$, there exists $q, r \in \Z$ with $a = bq + r$ and $0 \leq r < b$.
\end{theorem}
\begin{proof}
  Let $S=\{a-n b: n \mathbb{Z}\}$, then $S$ contains some nonnegative integer. Let the smallest be $r$. Then $r<b$, as otherwise $r-b \in S$ would be nonnegative and smaller than $r$. So $a-q b=r$ for some $q \in \mathbb{Z}$, or $a=q b+r$.
\end{proof}

\begin{definition}
  If $r = 0$, we write $b \mid a$ (``$b$ \emph{divides} $a$''), otherwise we write $b \nmid a$.
\end{definition}

Given $a_1, \dots, a_n \in \Z$ not all zero, let $I = \{\lambda_1 a_1 + \cdots + \lambda_n a_n \mid \lambda_i \in \Z\}$. Then for $a, b \in I$ and $l, m \in \Z$, we have $la + mb \in I$.

\begin{lemma}
  $I = d \Z = \{md \mid m \in \Z\}$ for some $d > 0$.
\end{lemma}
\begin{proof}
  Let $d$ be the least positive element in $I$. Then $d\Z \subset I$. Conversely, if $a \in I$, write $a = qd + r$ for some $0 \leq r < a$. If $r = 0$, then $a \in d \Z$. Otherwise, $r = a - qd \in I$ is positive and smaller than $d$, contradiction.
\end{proof}

In particular, $d \mid a_i$ for all $i$; Conversely, if $c \mid a_i$ for all $i$, then $d \Z = I \subset c \Z$, hence $c \mid d$.

\begin{definition}
  We write $d = \gcd(a_1, \dots, a_n)$ or $(a_1$, $\dots$, $a_n)$ and say $d$ is the \emph{greatest common divisor} of $a_1, \dots, a_n$.
\end{definition}

\begin{corollary}[Bézout]
  Suppose $a, b, c \in \Z$ and $a, b$ not both 0. There exists $x, y \in \Z$ such that $ax + by = c$ if and only if $(a, b) \mid c$.
\end{corollary}

% \begin{definition}
%     The \emph{greatest common divisor} $(a, b)$ of natural numbers $a, b$ is the largest integer $d$ such that $d\mid a$ and $d\mid b$.
% \end{definition}

% \begin{theorem}
%     \begin{enumerate}[label=(\roman*)]
%         \item Every common divisor of $a$ and $b$ divides their greatest common divisor $(a, b)$. 
%         \item Every common multiple of $a$ and $b$ is a multiple of the least common multiple $[a, b]$.
%     \end{enumerate}
% \end{theorem}
% \begin{proof}
%     We prove (ii) first. Set $m = [a, b]$ and suppose $\ell > m$ is another common multiple of $a$ and $b$. Then we can write $\ell = qm + r$. If $r \neq 0$, then $r$ is a common multiple of $a$ and $b$ with $r < m$, which is a contradiction. For (i), set $d = (a, b)$ and suppose that $e \mid a$ and $e \mid b$. Then $a$ is a common multiple of $d$ and $e$, so by (ii) it is a multiple of $[d, e]$, as is $b$. Now $[d, e]\mid a$, $[d, e] \mid b$ together with $[d, e] \geq d$ implies $[d, e]=d$. Consequently, $e \mid d$.
% \end{proof}

% \begin{theorem}
%     For $a, b \in \N$, $(a, b)[a ,b] = ab$.
% \end{theorem}
% \begin{proof}
%     $ab/(a, b)$ is a common multiple of $a$ and $b$, and so $[a, b] = ab/(a, b)k$ for some $k \in \N$. Then $(a, b)k \mid a$ and $(a, b)k \mid b$ hence $k = 1$. 
% \end{proof}

% \begin{theorem}\label{thm:4}
%     If $(a, b) = 1$ and $a \mid bc$, then $a \mid c$.
% \end{theorem}
% \begin{proof}
%     Observe that $bc$ is a common multiple of $a$ and $b$ so $[a, b] \mid bc$ implies $ab \mid bc$, and thus $a \mid c$.
% \end{proof}

% \begin{theorem}
%     $S = \{ax + by \mid x, y \in \Z\}$ is precisely the set of multiple of $(a, b)$.
% \end{theorem}
% \begin{proof}
%     Denote by $d$ the smallest element of $S$. Note that if $s, s' \in S$, then $s \pm s' \in S$ and $\lambda s \in S$ for $\lambda \in \Z$. It then follows from the division algorithm that $S$ is precisely the set of multiples of $d$. Then $d \mid a$, $d\mid b$ and $(a, b) \mid ax + by = d$, so $d = (a, b)$.
% \end{proof}

\begin{theorem}[Euclid's Algorithm]
Given $a, b\in \N$ with $a > b$, setting $b = r_0>0$ we can repeatedly apply the division algorithm to get
\begin{align*}
  a &= q_1 r_0 + r_1 \\
  b &= q_2 r_1 + r_2 \\
  r_1 &= q_3 r_2 + r_3 \\
  \vdots \\
  r_{k-1} &= q_{k+1} r_{k} + 0,
\end{align*}
where $0 < r_i < r_{i - 1}$ for $i \leq k$.
Then $r_k = (a, b)$.
\end{theorem}
\begin{proof}
  Note that $r_k \mid r_0$ and $r_k \mid a$, so $r_k \leq (a, b)$. Also if $m \mid a$ and $m \mid b$ then $m\mid r_k$. Hence $(a, b) \leq r_k$, and $(a, b) = r_k$.
\end{proof}

\begin{lemma}[Euclid's Lemma]
  Let $p$ be a prime and let $a, b\in \Z$. Then $p \mid ab$ if and only if $p \mid a$ or $p \mid b$.
\end{lemma}
\begin{proof}
  The ``if'' direction is clear. Conversely, suppose $p \mid a b$ yet $p \nmid a$, then $(a, p) \neq p$ but $(a, p) \mid p$ and $p$ is prime, so $(a, p)=1$, therefore there are some integers $n, y$ such that $a x+p y=1$. Now $b=b(a x+p y)=x(a b)+(b y) p \implies p \mid b$
\end{proof}

\begin{theorem}[Fundamental Theorem of Arithmetic] 
  Every $n>1$ can be written a a product of primes. Furthermore, this is unique up to reordering.
\end{theorem}
\begin{proof}
  Existence follows easily by strong induction. For uniqueness we also use induction. Suppose that $n$ has two prime factorisations $n=p_1 \cdots p_k=q_1 \cdots q_l$. We have $p_1 \mid q_1 \cdots q_k$, so $p_1 \mid q_i$ for some $i$. We can label that prime $q_1$. Hence $p_1=q_1$. So $n / p_1=p_2 \cdots p_k=q_2 \cdots q_l$, and by induction $k=l$ and $p_2=q_2, p_3=q_3$, etc.
\end{proof}
% \begin{definition}[Prime]
% A natural number $n > 1$ is \emph{prime} if it is divisible only by $1$ and itself.
% \end{definition}

% \begin{theorem}[Fundamental Theorem of Arithmetic]
%     Every natural number $> 1$ has a unique factorisation into prime numbers. 
% \end{theorem}
% \begin{proof}[Proof Sketch]
% Existence is constructive. For uniqueness, we use \autoref{thm:4} and induction.
% \end{proof}

\begin{theorem}[Euclid]
    The number of primes is infinite.
\end{theorem}
\begin{proof}
  If there was finitely many primes, say $p_1, \dots, p_k$, then the number $N = p_1 p_2 \cdots p_k + 1$ would have no prime factors which is a contradiction.
\end{proof}

\section{Congruences}

\ 

\begin{definition}
Let $n \geq 1$ be an integer. We say $a$ is \emph{congruent} to $b$ modulo $n$, written $a \equiv b \pmod{n}$, if $n \mid a - b$.
\end{definition}

\begin{theorem}
  There exists $x$ such that $ax \equiv 1 \pmod{n}$ if and only if $(a, n) = 1$.
\end{theorem}
\begin{proof}
    The ``only if'' direction is clear. For the ``if'' direction, by Bézout we can write $ax + ny = 1$ for some $d$, and so $ax \equiv 1 \pmod{n}$ as required.
\end{proof}

\begin{theorem}[Chinese Remainder Theorem]
    Given $m_1$, $\dots$, $m_k \in \N$ pairwise coprime, the set of congruences $x \equiv a_i \pmod{m_i}$, $1 \leq i \leq k$, has a unique solution $x$ modulo $M = m_1 \cdots m_k$.
\end{theorem}
\begin{proof}
  Write $M_i=M / m_i$, then $\left(m_i, M_i\right)=1$ for all $i$. Therefore for each $i$ there is a $b_i$ such that $M_i b_i \equiv 1\pmod{m_i}$. We also have $M_i b_i \equiv 0 \pmod{m_j}$ for all $j \neq i$. Take $x=\sum_i a_i b_i M_i$. For uniqueness, if $x, y$ satisfies the system, then $m_i \mid x-y$ for all $i$, therefore $M \mid x-y$ since there is no prime that divides two distinct $m_i$'s.
\end{proof}

\begin{theorem}[Fermat's Little Theorem]
  If $a, p \in \Z$ with $p$ prime, then $a^p \equiv a \pmod{p}$.
\end{theorem}
\begin{proof}
  For $p = 2$ this is true. Then for $p \neq 2$, $a^p - (a-1)^p \equiv 1 \pmod{p}$ by the binomial theorem. 
\end{proof}

\begin{definition}[Euler Totient Function]
Let $\phi(n)$ denote the number of integers $a$, $1 \leq a \leq n$, with $(a, n) = 1$.
\end{definition}

\begin{remark}
  Directly from the definition we get that $\phi(p^k) = p^k - p^{k-1}$ for $p$ prime, that $\phi(n)$ is multiplicative and thus $\phi(n) = n \prod_{p \mid n} (1 - 1/p)$.
\end{remark}

\begin{theorem}[Euler-Fermat]
  If $a, n \in \Z$ have $(a, n) = 1$, then $a^{\phi(n)} \equiv 1 \pmod{p}$.
\end{theorem}
\begin{proof}
By Lagrange's theorem (on groups), the order of $a$ in $(\Z/n\Z)^\times$ divides the order of the group, $\phi(n)$.
\end{proof}

\begin{theorem}[Wilson]
$(p - 1)! \equiv -1 \pmod{p}$.
\end{theorem}
\begin{proof}
  We can pair up terms in the product $(p-1)!$ with their inverses which then multiply to one, but this leaves only $1$ and $p - 1$ unpaired.
\end{proof}

\begin{theorem}[Lagrange]
  Let $p$ be a prime and let $f(x)$ be an integer polynomial of degree $n$. Then $f(x) \equiv 0 \pmod{p}$ has at most $n$ solutions modulo $p$.
\end{theorem}

\begin{definition}
  For $a, n \in \Z$ with $n > 0$, the \emph{order} of $a$ modulo $n$ is the least positive integer $d$ such that $a^d \equiv 1 \pmod{n}$. We say that $a$ is a \emph{primitive root} if its order is $\phi(n)$.
\end{definition}

\begin{theorem}
There exists a primitive root $\pmod{n}$ if and only if $n = 2, 4, p^j$ or $2p^j$, where $p$ is an odd prime.
\end{theorem}
\begin{proof}
For $n = 2, 4$ this is easy. Suppose $n = p$, a prime. Let $\psi(n)$ count the number of $1, 2, \dots, p - 1$ of order $d$ modulo $p$. Observe that $\psi(d) = 0$ if $d \nmid p - 1$, so $\sum_{d \mid p -1} \psi(d) = p - 1$. If $a$ has order $d$, $\{a, a^2, \dots, a^d\}$ are solutions to $x^d \equiv 1 \pmod{p}$ and hence are all of them by Lagrange. So $\psi(d) \leq \phi(d)$. But $\sum_{d \mid n} \phi(d) = n$ as $\phi(d)$ counts $\{m \mid (m, n) = n/d\}$. Thus $\psi(d) = \phi(d)$ for all $d$. In particular, there are $\phi(p - 1)$ elements of order $p - 1$. Let $g$ be one of them.

If $n = p^j$, $j > 1$, suppose $g$ has orders $p^ks$ modulo $n$, $s \mid p - 1$. By Fermat's Little Theorem $g^{p^k s} \equiv g^s \pmod{p}$, so $s = p - 1$. $g^{p - 1} \equiv 1 \pmod{p^2}$ implies $(g + p)^{p - 1} \equiv 1 + (p - 1)g^{p - 2}p \not \equiv 1 \pmod{p^2}$, so WLOG $g^{p -1} \not \equiv 1 \pmod{p^2}$, thus $g$ is a primitive root mod $p^2$. $g^{p^{j -2} (p - 1)} = (1 + \ell p)^{p^{j - 2}} \equiv 1 + \ell p ^{j - 1} \not \equiv 1 \pmod{p^j}$, so $g$ is also a primitive root mod $p^j$ for all $j \geq 2$. One of $g, g+p^j$ is odd, and thus also serves as a primitive root mod $2p^j$, noting $\phi(2p^j) = \phi(p^j)$.

For the other direction, for $n$ not a prime power, 
$n = rs$ with $(r, s) = 1$ and $\phi(r), \phi(s)$ both even. $a^{\phi(n)}/2 \equiv a^{\phi(r) \phi(s)/2} \equiv 1$ modulo $r$ and $s$. So $a^{\phi(n)/2} \equiv 1 \pmod{n}$. 
For $n = 2^j$, $j \geq 3$, $a^2 \equiv 1 \pmod{8}$ for all odd $a$, thus $a^{2^{k + 1}} \equiv (a^2)^{2^k} \equiv 1 \pmod{2^{k + 3}}$.
\end{proof}


\section{Quadratic Residues}
\ 
\begin{definition}
We call $a$ a \emph{quadratic residue} modulo $n$ if there exists $x$ such that $x^2 \equiv a \pmod{n}$.
\end{definition}

\begin{lemma}
  Let $p$ be an odd prime. Then there are exactly $(p-1)/2$ quadratic residues modulo $p$.
\end{lemma}
\begin{proof}
  Take $g$ to be a primitive root modulo $p$. Then the set of quadratic residues is exactly $\{g^{2n} \bmod p \mid n \in \Z\}$ which has size $(p-1)/2$.
\end{proof}

\begin{definition}
Let $p$ be an odd prime and $a \in \mathbb{Z}$. The \emph{Legendre symbol} is defined as
  $$
  \left(\frac{a}{p}\right)=\begin{cases}
    0 & \text{if } p \mid a, \\
  1 & \text{if } a \text { is a quadratic residue modulo } p, \\
  -1 & \text{otherwise}.
   \end{cases}
  $$
\end{definition}

\begin{theorem}[Euler's Criterion]
  Let $p$ be an odd prime, and $a \in \mathbb{Z}$, then
  $$
  \left(\frac{a}{p}\right) \equiv a^{\frac{p-1}{2}} \pmod{p}.
  $$
\end{theorem}
\begin{proof}
    If $a$ is a quadratic residue, we can write $x^2 \equiv a \pmod{p}$ and then $a^{(p-1)/2}\equiv x^{p-1} \equiv 1\pmod{p}$, as required. 
    By Lagrange, $a^{(p-1)/2} \equiv 1 \pmod{p}$ has at most $(p-1)/2$ solutions, so if $a$ is not a quadratic residue, we must have $a^{(p-1)/2} \equiv -1 \pmod{p}$. 
\end{proof}

\begin{corollary}
  Let $p$ be an odd prime and let $a, b \in \mathbb{Z}$, then
$$
\left(\frac{a b}{p}\right)=\left(\frac{a}{p}\right)\left(\frac{b}{p}\right).
$$
\end{corollary}
\begin{corollary}
  Let $p$ be an odd prime, then
  $$
  \left(\frac{-1}{p}\right)=\begin{cases}
    1 & \text{if } p \equiv 1 \pmod{4}, \\
  -1 & \text{if } p \equiv 3 \pmod{4}
   \end{cases}
  $$
\end{corollary}

\begin{definition}
    The \emph{numerically least residue} $a'$ of $a \pmod n$ to $a' \equiv a$ with $-n/2 < a' \leq n/2$. 
\end{definition}

\begin{theorem}[Gauss' Lemma]
given an odd prime $p$ and $a$ with $(a, p) = 1$, let $a_j$ denote the numerically least residue of $a_j \pmod{p}$. Then $(a/p) = (-1)^\ell$, where $\ell = |\{j \leq (p-1)/2 \mid a_j < 0\}|$.
\end{theorem}
\begin{proof}
    We have $a_j = \pm a_k$ if and only if $j = \pm k$, so $|a_j|$ takes all values $1, \dots, (p-1)/2$. Hence $\prod_{j \leq (p-1)/2} a_j = (-1)^\ell r!$, so $a^{(p-1)/2} \equiv (-1)^\ell \pmod{p}$. The result then follows from Euler's Criterion.
\end{proof}

\begin{theorem}[Law of Quadratic Reciprocity]
If $p, q$ are distinct odd primes, $(p/q) = (q/p)$ unless $p \equiv q \equiv 3 \pmod{4}$, in which case $(p/q) = -(q/p)$. More concisely, $(p/q)(q/p) = (-1)^{((p-1)/2)((q-1)/2)}$.
\end{theorem}
\begin{proof}
    By Gauss' lemma, $(p/q) = (-1)^\ell$, where $\ell$ is the number of lattice points $(x, y)$ satisfying $0 < x< q/2$, $-q/2 < px - qy < 0$. For such points $q(y - 1/2) < px$, so $y < p/2 + 1/2$/ We therefore lose nothing in imposing the extra symmetrising condition $0 < y < p/2$. Similarly $(q/p) = (-1)^m$ where $m$ is the number of lattice points in the rectangle $0 < x < q/2$. $0 < y < p/2$, satisfying $-p/2 < qx - py < 0$ or equivalently $0 < px - qy < p/2$. Now it suffices to prove that $((p-1)/2)((q-1)/2) - (\ell + m)$ is even. But $((p-1)/2)((q-1)/2)$ is just the number of lattice points in our rectangle and we have a bijection between such points with $px - qy \leq -q/2$ and those with $px - qy \geq p/2$ by means of transformation $x' = (q+1)/2 - x$, $y' = (p + 1)/2 - y$. (This does \emph{not} fix the line $px - qy = 0$, so we cannot say the same for our original sets.
\end{proof}

\begin{definition}
    For $n$ odd, $(m, n) = 1$, we define the \emph{Jacobi symbol} $(m/n)$ by 
    $$
    \left(\frac{m}{p_1^{\alpha_1} \cdots p_k^{\alpha_k}}\right)
    = \left(\frac{m}{p_1}\right)^{\alpha_1} \cdots \left(\frac{m}{p_k}\right)^{\alpha_k}
    $$
    in terms of Legendre symbols.
\end{definition}

\begin{remark}
    All of our previously stated theorems hold for Jacobi symbols, though Jacobi symbols are not a test for quadratic residuality, only a computational technique. 
\end{remark}

\section{Quadratic Forms}
\ 

\begin{definition}
    A \emph{binary quadratic form} is a function $f(x, y) = ax^2 + byx + cy^2$, $a, b, c \in \Z$. This may sometimes be represented more simply as $(a, b, c)$. Note that
    $$
    f(x, y) = 
    \begin{pmatrix}
        x & y
    \end{pmatrix}
    \begin{pmatrix}
        a & b/2 \\ b/2 & c
    \end{pmatrix} 
    \begin{pmatrix}
        x \\ y
    \end{pmatrix}.
    $$
\end{definition}

\begin{definition}
    A \emph{unimodular substitution} is a transformation $X = px + qy$, $Y = rs + sy$ with $ps - qr = 1$. Equivalently, $\begin{pmatrix}
        X & Y
    \end{pmatrix}^\intercal = A \begin{pmatrix}
        x & y
    \end{pmatrix}^\intercal,$ with $A \in \SL_2(\Z)$.
\end{definition}

\begin{definition}
    Two binary quadratic forms $f$ and $f'$ are \emph{equivalent} of they are related by a unimodular substitution. We then write $f \sim f'$.
\end{definition}

\begin{definition}
    $4a f(x, y) = (2ax + by)^2 - dy^2$ where $d = \operatorname{disc}(f) = b^2 - 4ac$ is the \emph{discriminant} of $f$.
\end{definition}

\begin{theorem}
    Equivalent forms have the same discriminant.
\end{theorem}
\begin{proof}
We have
\begin{align*}
    \operatorname{disc}(f) &= -4\left| \begin{pmatrix}
        a & b/2 \\ b/2 & c
    \end{pmatrix}\right|\\ &= -4 \left|\begin{pmatrix}
        p & q \\ r & s
    \end{pmatrix}^{\intercal} \begin{pmatrix}
        a & b/2 \\ b/2 & c
    \end{pmatrix} \begin{pmatrix}
        p & q \\ r & s
    \end{pmatrix}\right|
\end{align*}
as $(ps - qr)^2 = 1$, and this is equal to $\operatorname{disc}(f')$.
\end{proof}

\begin{definition}
    A binary quadratic form with $d \neq 0$ is \emph{positive definite} if $f(x, y) \geq 0$ for all $x, y$.
    It's \emph{negative definite} if $f(x, y) \leq 0$, and \emph{indefinite} otherwise. 
\end{definition}

\begin{theorem}
    A positive definite form $(a, b, c)$ is equivalent to some \emph{reduced} form satisfying $-a < b \leq a < c$ or $0 \leq b \leq a = c$.
\end{theorem}
\begin{proof}
    Using unimodular substitution $S: (a, b, c) \mapsto (c, -b, a)$ and $T_{\pm}: (a, b, c) \rightarrow (a, b \pm 2a, a \pm b + c)$, if $a > c$. i.e. $S$ to decrease $a$ while keeping $|b|$ fixed. If $a < c$ and $|b| > a$, then use $T_+$ or $T_-$ to decrease $|b|$ whilst keeping $a$ fixed, noting all the while that $a + |b|$ is strictly decreasing so this process must stop. Finally if $b = -a$, apply $T_+$ to get $+a$ and if $a = c$, apply $S$ to get $b > 0$.
\end{proof}

\begin{theorem}
    The smallest integer $\neq 0$ represented by a reduced positive definite form $(a, b, c)$ all coprime are $a, c$ and $a - |b| + c$ in that order.
\end{theorem}
\begin{proof}
    $f(0, 0) = 0$. $f(1, 0) = a$, $f(0, 1)= c$, $0 < a \leq c$ since $f$ is reduced. Now, for $|x| \geq |y| > 0$, $f(x, y) \geq a|x|^2 - |b||x|^2 + c|y|^2 = (a - |b|)|x|^2 + c|y|^2 \geq a - |b\ + c$. Similarly if $|y| \geq |x| > 0$, and we can only achieve equality at $(\pm 1, \pm 1)$ and indeed we do.
\end{proof}

\begin{theorem}
    No two reduced forms are equivalent.
\end{theorem}

\begin{proof}
    By our result on the smallest represented integer, $f \sim f'$ implies $a = a'$ and $c = c'$. Then by equivalent forms having the same discriminant, $d = d'$ hence $b = \pm b'$. If $b = 0$ or $b = b'$ we are done, so suppose $b > 0$ and $(a, b, c) \sim (a, -b, c)$. Then both reduced implies $a < c$ and $|b| < a$, or $b = -a$ (which cannot happen). Now $f, f'$ both satisfy $f(x, y) = a$ if and only if $(x, y) = (\pm 1, 0)$ and $f(x, y) = c$ if and only if $(x, y) = (0, \pm 1)$. so if they are equivalent under substitution $\begin{pmatrix}
        p & q \\ r & s
    \end{pmatrix}$ we must have $\begin{pmatrix}
        p & q \\ r & s
    \end{pmatrix}\begin{pmatrix}
        \pm 1 \\ 0
    \end{pmatrix} = \begin{pmatrix}
        \pm 1 \\ 0
    \end{pmatrix}$
    implying $p = \pm 1, r = 0$ and $\begin{pmatrix}
        p & q \\ r & s
    \end{pmatrix}\begin{pmatrix}
       0 \\ \pm 1
    \end{pmatrix} = \begin{pmatrix}
        0 \\\pm 1
    \end{pmatrix}$ implying $q = 0$, $s = \pm 1$. $\det \begin{pmatrix}
        p & q \\ r & s
    \end{pmatrix} = \pm I$ implies $f = f'$ as required.
\end{proof}

\begin{theorem}
    There are finitely many reduced forms with discriminant $d$. Indeed, $|b| \leq a \leq \sqrt{d/2}$.
\end{theorem}
\begin{proof}
    $d = b^2 - 4ac \leq ac - 4ac \leq -3a^2$ which implies our claim. $c = (b^2 - d)/4a$.
\end{proof}

% \begin{definition}
%     The \emph{class number} $h(d)$ of $d$ is the number of reduced forms of discriminan
% \end{definition}

\begin{theorem}
    Given $n \in \N$, $n$ is properly represented by a form $f$, that is $f(x,y) = n$ for some coprime $x, y$
if and only if $f$ is equivalent to a form $(n, w, c)$ with first coefficient $n$.
    \end{theorem} 
\begin{proof}
    If $f$ is equivalent to this form $f \sim f'$, $f'(1, 0) = n$ so $f(x, y) = n$ properly represented as coprimality is preserved under $\SL_2(\Z)$. 
    If $f(x, y) = n$ with $(x, y) = 1$ then there exists $q, s$ such that $xs - qy = 1$. Then
    $$
    \begin{pmatrix}
        x \\ y
    \end{pmatrix} = \begin{pmatrix}
        x & q \\ y & s
    \end{pmatrix} \begin{pmatrix}
        1 \\ 0
    \end{pmatrix},
    $$ 
    so applying this matrix being in $\SL_2(\Z)$ to $f$, we get $f'$ with $f'(1, 0) = n$.
\end{proof}

\begin{theorem}
    Given $n \in \N$, $n$ is properly represented by some form of discriminant $d$ if and only if there is a solution $w$ to $w^2 = d\pmod{4n}$. 
\end{theorem}
\begin{proof}
    By the previous theorem, $w^2 - 4nc = d$. Conversely, $w^2 \equiv d\pmod{4n}$ implies there exists $c$ such that $w^2 = d + 4nc$ implies $(n, w, c)$ has discriminant $d$.
\end{proof}

% primes of the form x^2 + y^2? s

\section{Prime Numbers}
\ 

\begin{definition}
The \emph{prime counting function} is $\pi(x)$, the number of primes $\leq x$.
\end{definition}

\begin{definition}
    The \emph{Möbius function} $\mu(n)$, $n \in \N$ is such that $\mu(n) = 0$ if $n$ is not squarefree, and $\mu(n) = (-1)^r$ if $n$ is a product of $r$ distinct primes. 
\end{definition}

Observe that $\sum_{d \mid n} \mu(d) = 0$ for $n> 1$.

\begin{theorem}[Legendre's Formula]
$\pi(x) - \pi(\sqrt{x}) + 1 = \sum_{d \mid P} \mu(d) \lfloor x/d\rfloor$, where $P$ denotes the product $P = p_1 \cdots p_k$ of primes $\leq \sqrt{x}$.
\end{theorem}
\begin{proof}
    The Eratosthenes sieve up to $\sqrt{x}$ applied to $\{1$, $2$, $\dots$, $x\}$ crosses out all multiples of $p_1, \dots, p_k$ leaving behind only primes $p$ with $\sqrt{x} < p \leq x$ and $1$. The result now follows by the Inclusion-Exclusion principle.
\end{proof}

\begin{theorem}
    $\sum_{i = 1}^{\infty} 1/p_i$ diverges.
\end{theorem}
\begin{proof}
    Otherwise, there exists $k$ such that the truncated series $\sum_{i > k} 1/p_i < 1/2$. Set $P = p_1 \cdots p_k$ and observe that every number equivalent to $1\pmod{p}$ can be factorised as a product of $j$ of the primes $P_{k+1}, p_{k+2}, \dots$ for some $j$. Hence
    $$
    \sum_{n \equiv 1 \bmod{p}} \frac{1}{n} \leq \sum_{j = 1}^{\infty} \left(\sum_{i = k + 1}^{\infty} \frac{1}{p_i}\right)^j \leq \sum_{j = 1}^{\infty} \left(\frac{1}{2}\right)^j = 1,
$$
which is a contradiction.
\end{proof}


\begin{theorem}
    $\prod_{p \leq n} p \leq 4^n$.
\end{theorem}
\begin{proof}
The proof is by induction, observing 
$$
\prod_{n < p \leq 2n} p \mid \binom{2n}{n} \leq 2^{2n} = 4^n.
$$
\end{proof}

\begin{corollary}
    $\pi(x) \leq x \log 4/\log x$.
\end{corollary}

\begin{theorem}[Prime Number Theorem]
$\pi(x) \sim \frac{x}{\log x}$.
\end{theorem}

\begin{definition}
    The \emph{Riemann zeta function} is defined for $s = \sigma + it$ for $\sigma > 1$ by
        $\zeta(s) = \sum_{n = 1}^\infty n^{-s}$.
\end{definition}

\begin{theorem}[Euler Product]
$\zeta(s) = \prod_p (1 - p^{-s})^{-1}$, $\sigma > 1$.
\end{theorem}
\begin{proof}
    For any $N$, $\prod_{p \leq N} (1 - p^{-s})^{-1} = \prod_{p \leq N} (1 + p^{-s}+ p^{-2s} + \cdots) = \sum_m m^{-s}$, where $m$ runes through all integers with no prime factors $> N$. the difference between this and $\sum_{n \leq N} n^{-s}$ is bounded in absolute value by $\sum_{n > N} n^{-\sigma} \rightarrow 0$ as $N \rightarrow \infty$.
\end{proof}

\begin{theorem}
There are infinitely many primes in the arithmetic progression $a$, $a + d$, $\dots$ provided $(a, d) = 1$.   
\end{theorem}

\section{Diophantine Approximation}
\ 

\begin{theorem}[Dirichlet]
Given $\theta \in \R$, $N \in \N$, there exists integers $p, q$ with $q \leq N$ such that $|\theta - p/q| \leq 1/q^N \leq 1/q^2$.
\end{theorem}
\begin{proof}
    Two of $\theta$, $2\theta$, $\dots$, $N\theta$ must differ by $\leq 1/N$ modulo 1. Their difference $q \theta$ differs from some $p$ by $\leq 1/N$.
\end{proof}

\begin{definition}
    Given $\theta \in \R$, define $a_0 = \lfloor \theta \rfloor$, $\theta_1 = \frac{1}{\theta - a_0}$, $a_1 = \lfloor \theta_1 \rfloor$, $\theta_2 = \frac{1}{\theta_1 - a_1}$, etc. terminating if some $\theta_i$ is an integer. Then
    $$
    \theta = a_0 + \frac{1}{a_1 + \frac{1}{a_2 +\cdots}}
    $$
    is the \emph{continued fraction} representation of $\theta$. The integers $a_i$ are called \emph{partial quotients} of $\theta$, and we write $\theta = [a_0, a_1, \dots]$. $p_n/q_n = [a_0, a_1, \dots, a_n]$ are called \emph{convergents}.
\end{definition}

\begin{theorem}
    (i) The $p_n, q_n$ satisfy the recurrence relations
    \begin{align*}
        p_n &= a_n p_{n-1} + p_{n-2}, \\
        q_n &= a_n q_{n - 1} + q_{n-2},
    \end{align*}
    with $p_0 = a_0$ and $q_0 = 1$, $p_1 = a_0 a_1 + 1$ and $q_1 = a_1$. 
    (ii) $|p_{n-1}/q_{n-1} - p_n/q_n| = \frac{1}{q_{n - 1}q_n}$. (iii) $\theta$ lies in $[p_{n-1}/q_{n-1}, p_n/q_n]$ and thus $|\theta - p_n/q_n| \leq 1/q_n^2$.
\end{theorem}
\begin{proof}
    (i) We check this is true for $n = 2$.
    Then supposing it is true for $n = m - 1 \geq 2$, we observe $p_j/q_j = a_0 + q_j'/p_j'$, where $p_j'/q_j' = [a_1, a_2, \dots, a_j]$. Then $p_j = a_0 p_j' + q_j'$ and $q_j = p_j'$ expanded implies our result. (ii) $p_n q_{n +1} - p_{n+1}q_n = (-1)^{n + 1}$ follows from (i) by induction.
    (iii) Observe $\theta = [a_0, a_1, \dots, a_n, \theta_{n + 1}]$ and $0 < 1/\theta_{n + 1} \leq 1/a_{n + 1}$ so $\theta$ lies in the interval $[p_n/q_n$, $p_{n +1}/q_{n+1}]$ and this with the other results gives the stated bound.
\end{proof}

\begin{theorem}
    The continued fractions process terminates if $\theta$ is rational.
\end{theorem}
\begin{proof}
    If $\theta = a/b$ then $1/q_n^2 \geq |\theta - p_n/q_n| \geq 1/q_n b$ for $p_n/q_n \neq a/b$, so $q_n$ may never exceed $b$. The partial quotients $a_0, a_1, \dots$ are in fact the $q_i$ of Euclid's algorithm on $(a, b)$.
\end{proof}

\begin{theorem}
$\theta = \frac{p_n \theta_{n + 1} + p_{n - 1}}{q_n \theta_{n + 1} + q_{n - 1}}$ from $\theta = [a_0$, $\dots$, $a_n$, $\theta_{n + 1}]$. The $p_n/q_n$ give successively better approximations of $\theta$ as $n$ increases.  
\end{theorem}
\begin{proof}
  Multiplying up $|q_n \theta - p_n|$, we find that it equals $(q_n \theta_{n + 1}+q_{n - 1})^{-1}$ and the denominator of the latter exceeds $q_n + q_{n - 1} = (a_n + 1)q_{n - 1} + q_{n - 2} > q_{n - 1} \theta_n + q_{n - 2}$.
\end{proof}

\begin{theorem}
  If $0 < q< q_{n + 1}$, $|q \theta - p| \geq |q_n \theta - p_n|$.
\end{theorem}
\begin{proof}
  Define $u$, $v$ by $p = u p_n + v p_{n + 1}$, $q = u q_n + v q_{n + 1}$. Solving by multiplying the former by $q_{n + 1}$ and the latter by $p_{n + 1}$ and $q_n$, $p_n$ respectively observe that $u, v$ are integers. $|q \theta -  p| = |u(q_n \theta - p_n) + v(q_{n + 1} \theta - p_{n + 1})| \geq |q_n \theta - p_n|$ as $u \neq 0$ and $\operatorname{sgn}(v) \neq \operatorname{sgn}(u)$.
\end{proof}

\begin{theorem}
  If rational $p/q$ satisfies $|\theta - p/q| < \frac{1}{2q^2}$, then it is convergent to $\theta$.
\end{theorem}
\begin{proof}
  Suppose $q_n < q< q_{n + 1}$. Then $|p/q - p_n/q_n| \leq |\theta - p/q| + |\theta - p_n /q_n| \leq 2 |q \theta - p|/q_n < 1/q q_n$, which is a contradiction.
\end{proof}

\begin{definition}
  The equation $x^2 - dy^2 = 1$, where $d$ is a positive integer which is not a square is known as \emph{Pell's equation}.
\end{definition}

\begin{theorem}
  If $(x, y)$ is a solution to the Pell equation $x^2 - dy^2 = 1$, then $x/y$ must be a convergent to $\sqrt{d}$.
\end{theorem}
\begin{proof}
  $(x - y \sqrt{d})(x + y \sqrt{d}) = 1$ implies $x - y \sqrt{d} = \frac{1}{x + y \sqrt{d}}$. Certainly $x > y \sqrt{d}$, whence $x - y \sqrt{d} < \frac{1}{2y \sqrt{d}}$ and thus $|\sqrt{d} - x/y| < \frac{1}{2y^2}$. The previous theorem hen gives us our result.
\end{proof}

\begin{theorem}
  If $(x, y)$ is the solution of $x^2 - dy^2 = 1$ with $x + y \sqrt{d}$ minimal, then every solution is given by $(x_n, y_n)$, where $x_n + y_n \sqrt{d} = (x + y \sqrt{d})^n$ for some $n$.
\end{theorem}
\begin{proof}
  Denoting $x + y \sqrt{d} \in \R$ by $\varepsilon > 1$, suppose to the contrary we have a solution $(a, b)$ with $\varepsilon^k < a + b \sqrt{d} < \varepsilon^{k + 1}$. Define $N(a + b\sqrt{d}) = a^2 - db^2$ and observe that $N(\alpha \beta) = N(\alpha)N(\beta)$, $N(\varepsilon) = 1 = N(\varepsilon^{-k}(a + b \sqrt{d}))$, where $\varepsilon^{-1} = x - y \sqrt{d}$, which gives us a contradiction.
\end{proof}

\begin{definition}
  $\alpha \in \R$ or $\C$ is \emph{algebraic} if it is the root of a polynomial with integer coefficients. If $P(\alpha) = 0$ with $P$ irreducible and $\deg P = n$ then $\alpha$is said to have \emph{degree} $n$. Non-algebraic numbers are called \emph{trancendental}.
\end{definition}

\begin{theorem}[Liouville's Theorem]
If $\alpha \in \R$ is algebraic of degree $n > 1$, there exists $c$ depending on $\alpha$ such that $|\alpha - p/q| > c/q^n$ for all $p/q \in \Q$.
\end{theorem}
\begin{proof}
  Observe $P(\alpha) - P(p/q) = (\alpha - p/q)p'(\xi)$ for some $\xi$ between $\alpha$ and $p/q$. Choose $P$ to be the minimal polynomial of $\alpha$, so $P(\alpha) = 0$ and $P$ irreducible implies $0 \neq |P(p/q)| \geq 1/q^n$. WLOG $|\alpha - p/q| < p$ and choose $C$ so that $|P'(\xi)| < C$ for $|\xi - \alpha| < 1$. Then $|\alpha - p/q| \geq c/q^n$ with $c = 1/C$.
\end{proof}

\section{Primality Testing}
\ 

\begin{definition}
    If $n$ is an odd composite number and $(b, n) = 1$, then $n$ is a \emph{Fermat pseudoprime} to the base $b$ if $b^{n - 1} \equiv 1 \pmod{n}$. If it is a pseudoprime to every $b$, then it is a \emph{Carmichael number}.
\end{definition}

\begin{theorem}
  Let $N >1$. If $N$ is not a Fermat pseudoprime to some base $b_0$, then it is not a Fermat pseudoprime to base $b$ for at least half of $b$ coprime to $N$.
\end{theorem}
\begin{proof}
The set $B$ of integers $1 \leq b < N$ such that $(b, N) = 1$ where $N$ is a Fermat pseudoprime to base $b$ is clearly a subgroup of $(\mathbb{Z} / N Z)^{\times}$. It's proper as $b_0 \in(\mathbb{Z} / N Z)^{\times}$is not in $B$. Consequently $|B| \leq\left|(\mathbb{Z} / N Z)^{\times}\right| / 2$ which concludes the proof.
\end{proof}

\begin{definition}
Let $b \in \mathbb{N}$. An odd composite integer $N>1$ is said to be an Euler pseudoprime to base $b$ if $b^{(N-1) / 2} \equiv\left(\frac{b}{N}\right) \mod{N}$.
\end{definition}

\begin{theorem}
 Let $N >1$. If $N$ is not an Euler pseudoprime to some base $b_0$, then it is not a Euler pseudoprime to base $b$ for at least half of $b$ coprime to $N$. 
\end{theorem}
\begin{proof}
  Same as the corresponding theorem for Fermat pseudoprimes.
\end{proof}

% \begin{theorem}
%     If $n$ is Carmichael, then it is squarefree and for every $p \mid n$, $p-1\mid n-1$.
% \end{theorem}



\end{document}
