\documentclass[a4paper]{scrartcl}

\usepackage[
    fancytheorems, 
    fancyproofs, 
    noindent, 
]{adam}


\title{Geometry}
\author{Adam Kelly (\texttt{ak2316@cam.ac.uk})}
\date{\today}

\allowdisplaybreaks

\begin{document}

\maketitle

% This is a short description of the course. It should give a little flavour of what the course is about, and what will be roughly covered in the notes.

This article constitutes my notes for the `Geometry' course, held in Lent 2022 at Cambridge. These notes are \emph{not a transcription of the lectures}, and differ significantly in quite a few areas. Still, all lectured material should be covered.



\tableofcontents

\section{Topological and Smooth Surfaces}

\subsection{Topological Surfaces}

We will begin immediately with a definition that will occupy us for some time.

\begin{definition}[Topological Surface]
    A topological surface is a topological space $\Sigma$ such that
    \begin{enumerate}[label=(\roman*)]
        \item Each $p \in \Sigma$ has an open neighbourhood $U$ with $p \in U$ such that $U$ is homeomorphic to $\R^2$, with its usual Euclidean topology.
        \item $\Sigma$ is Hausdorff and second countable.
    \end{enumerate}    
\end{definition}

Recall that a space $X$ is Hausdorff if for $p \neq q$ in $X$ there exists disjoint open sets $p \in U$ and $q \in V$ in $X$, and that a space is second countable it's topology has a countable basis.
In some ways, the real nature of topological spaces comes from the condition (a), and the condition (b) is really there for technical honesty.

\subsection{Examples of Topological Surfaces}

Let's now take some to consider some examples of topological surfaces.

\begin{example}[$\R^2$]
    The plane $\R^2$ is a topological surface.
\end{example}

\begin{example}[Subsets of $\R^2$]
    Any open subset of $\R^2$ is a topological surface. For example
    \begin{enumerate}[label=(\roman*)]
        \item $\R^2 \backslash \{0\}$ is a topological surface;
        \item Let $Z = \{(0, 0) \} \cup \{(1, 1/n) \mid n \in \N\}$, then $\R^2 \backslash Z$ is a topological surface.
    \end{enumerate}
\end{example}

\begin{example}[Graphs]
    Let $f: \R^2 \rightarrow \R$ be a continuous function. Then the \vocab{graph} 
    $$\Gamma_f = \{(x, y, f(x, y)) \mid (x, y) \in \R^2\} \subseteq \R^3 \text{ (subspace topology)}.$$
    
    Recall that if $X$ and $Y$ are topological spaces, the product topology on $X \times Y$ has basis open sets $U \times V$ with $U \subseteq X$ and $V \subseteq Y$ both open sets.

    It has the feature that $f: Z \rightarrow X \times Y$ is continuous if and only if $\pi_x \circ f: Z \rightarrow X$ and $\pi_y \circ f: Z \rightarrow Y$ are continuous.

    So if $\Gamma_f \subseteq X \times Y$ and $f: X \rightarrow Y$ is continuous then $\Gamma_f$ is homeomorphic to $X$, with the map $s: x \mapsto (x, f(x))$, so that $\left.\pi\right|_{\Gamma_f}$ and $s$ are inverse homeomorphisms.

    So $\Gamma_f \cong \R^2$ for \emph{any} continuous $f: \R^2 \rightarrow \R$, and $\Gamma_f$ is a topological surface.
\end{example}

As a note, the topological surface $\Gamma_f$ is independent of $f$. Later on as we develop more tools in geometry we will be able to better reflect the structure of the function $f$. 

\begin{example}[Stereographic Projection]
    Consider the sphere
    $$
S^2 = \{(x, y, t) \in \R^3 \mid x^2 + y^2 + z^2 =1 \}.
    $$

    We can consider the stereographic projection
    \begin{align*}
        \pi_+ : S^2 \backslash \{(0, 0, 1)\} &\rightarrow \R^2 (z = 0) \subseteq \R^3 \\
        (x, y, t) &\mapsto \left(\frac{x}{1 - t}, \frac{y}{1 - t}\right).
    \end{align*}
    Such a projection is shown below.

    \begin{center}
        

% Gradient Info
  
\tikzset {_dgh8fcybk/.code = {\pgfsetadditionalshadetransform{ \pgftransformshift{\pgfpoint{89.1 bp } { -128.7 bp }  }  \pgftransformscale{1.32 }  }}}
\pgfdeclareradialshading{_var369p8e}{\pgfpoint{-72bp}{104bp}}{rgb(0bp)=(1,1,1);
rgb(0bp)=(1,1,1);
rgb(25bp)=(0.78,0.88,1);
rgb(400bp)=(0.78,0.88,1)}
\tikzset{every picture/.style={line width=0.75pt}} %set default line width to 0.75pt        

\begin{tikzpicture}[x=0.75pt,y=0.75pt,yscale=-1,xscale=1]
%uncomment if require: \path (0,300); %set diagram left start at 0, and has height of 300

%Shape: Rectangle [id:dp9621550874774156] 
\draw   (119.66,91.2) -- (378.29,91.2) -- (330.63,158.95) -- (72,158.95) -- cycle ;
%Shape: Circle [id:dp7711674031779723] 
\path  [shading=_var369p8e,_dgh8fcybk] (168,120.2) .. controls (168,86.51) and (195.31,59.2) .. (229,59.2) .. controls (262.69,59.2) and (290,86.51) .. (290,120.2) .. controls (290,153.89) and (262.69,181.2) .. (229,181.2) .. controls (195.31,181.2) and (168,153.89) .. (168,120.2) -- cycle ; % for fading 
 \draw   (168,120.2) .. controls (168,86.51) and (195.31,59.2) .. (229,59.2) .. controls (262.69,59.2) and (290,86.51) .. (290,120.2) .. controls (290,153.89) and (262.69,181.2) .. (229,181.2) .. controls (195.31,181.2) and (168,153.89) .. (168,120.2) -- cycle ; % for border 

%Shape: Ellipse [id:dp7010564209713763] 
\draw  [dash pattern={on 0.84pt off 2.51pt}] (168,120.2) .. controls (168,110.55) and (195.31,102.74) .. (229,102.74) .. controls (262.69,102.74) and (290,110.55) .. (290,120.2) .. controls (290,129.85) and (262.69,137.66) .. (229,137.66) .. controls (195.31,137.66) and (168,129.85) .. (168,120.2) -- cycle ;
%Straight Lines [id:da601420156609989] 
\draw [color={rgb, 255:red, 74; green, 144; blue, 226 }  ,draw opacity=1 ] [dash pattern={on 4.5pt off 4.5pt}]  (229,59.2) -- (275,102.03) ;
%Shape: Circle [id:dp9581781475680486] 
\draw  [draw opacity=0][fill={rgb, 255:red, 208; green, 2; blue, 27 }  ,fill opacity=1 ] (224.75,59.2) .. controls (224.75,56.85) and (226.65,54.95) .. (229,54.95) .. controls (231.35,54.95) and (233.25,56.85) .. (233.25,59.2) .. controls (233.25,61.55) and (231.35,63.45) .. (229,63.45) .. controls (226.65,63.45) and (224.75,61.55) .. (224.75,59.2) -- cycle ;
%Straight Lines [id:da4941203681575539] 
\draw [color={rgb, 255:red, 74; green, 144; blue, 226 }  ,draw opacity=1 ]   (275,102.03) -- (322,148.45) ;
%Straight Lines [id:da7315894591096315] 
\draw [color={rgb, 255:red, 74; green, 144; blue, 226 }  ,draw opacity=1 ] [dash pattern={on 4.5pt off 4.5pt}]  (322,148.45) -- (346,172.95) ;
%Shape: Circle [id:dp7248118504557512] 
\draw  [draw opacity=0][fill={rgb, 255:red, 208; green, 2; blue, 27 }  ,fill opacity=1 ] (270.75,102.03) .. controls (270.75,99.69) and (272.65,97.78) .. (275,97.78) .. controls (277.35,97.78) and (279.25,99.69) .. (279.25,102.03) .. controls (279.25,104.38) and (277.35,106.28) .. (275,106.28) .. controls (272.65,106.28) and (270.75,104.38) .. (270.75,102.03) -- cycle ;
%Shape: Circle [id:dp13443566890481096] 
\draw  [draw opacity=0][fill={rgb, 255:red, 208; green, 2; blue, 27 }  ,fill opacity=1 ] (317.75,148.45) .. controls (317.75,146.1) and (319.65,144.2) .. (322,144.2) .. controls (324.35,144.2) and (326.25,146.1) .. (326.25,148.45) .. controls (326.25,150.8) and (324.35,152.7) .. (322,152.7) .. controls (319.65,152.7) and (317.75,150.8) .. (317.75,148.45) -- cycle ;
%Straight Lines [id:da5636421496414195] 
\draw    (72,158.95) -- (330.63,158.95) ;

% Text Node
\draw (312.6,122.5) node [anchor=north west][inner sep=0.75pt]  [color={rgb, 255:red, 208; green, 2; blue, 27 }  ,opacity=1 ]  {$\pi _{+}( p)$};
% Text Node
\draw (221.8,37.3) node [anchor=north west][inner sep=0.75pt]    {$N$};
% Text Node
\draw (254.2,88.5) node [anchor=north west][inner sep=0.75pt]    {$p$};


\end{tikzpicture}

    \end{center}

    Note that $\pi_{+}$ is continuous and has an inverse
    $$
    (u, v) \mapsto \left(\frac{2u}{u^2 + v^2 + 1}, \frac{2v}{u^2 + v^2 + 1}, \frac{u^2 + v^2 - 1}{u^2 + v^2 + 1}\right).
    $$
    So $\pi_+$ is a continuous bijection with continuous inverse and hence a homeomorphism.

    Of course we could also have projected from the south pole, to get a homeomorphism $\pi_-$ from $S^2 \backslash \{0, 0, -1\}$ to $\R^2$, so indeed every point lies in an open set which is homeomorphic through either $\pi_+$ or $\pi_-$ to $\R^2$. So $S^2$ is a topological surface.
\end{example}

\begin{remark}
    $S^2$ is compact as a topological space, since it is a closed bounded set in $\R^3$.
\end{remark}

\begin{example}[Real Projective Plane]
    The group $\Z/2\Z$ acts on $S^2$ by homeomorphisms via the \vocab{antipodal map} $a: S^2 \rightarrow S^2$ with
    $$
    a(x, y, t) = (-x, -y, -t).
    $$
    That is, there exists a homeomorphism $\Z/2\Z \rightarrow \operatorname{Homeo}(S^2)$ sending the non-identity element to $a$.

    The \vocab{real projective plane} is the quotient space of $S^2$ given by identifying every point with its antipodal image: $\R \PP^2 = S^2/(\Z/2\Z) = S^2/\sim$ with $x \sim a(x)$.

    Note that $\sim$ is the equivalence relation of belonging to the same orbit under the given action.

    As a set, $\R \PP^2$ is naturally in bijection with the set of straight lines in $\R^3$ through the origin, with the bijection given by mapping lines with the identified points on the sphere that they pass through.
    \begin{center}
        

% Gradient Info
  
\tikzset {_52kankop7/.code = {\pgfsetadditionalshadetransform{ \pgftransformshift{\pgfpoint{89.1 bp } { -128.7 bp }  }  \pgftransformscale{1.32 }  }}}
\pgfdeclareradialshading{_4ataw6yqb}{\pgfpoint{-72bp}{104bp}}{rgb(0bp)=(1,1,1);
rgb(0bp)=(1,1,1);
rgb(25bp)=(0.78,0.88,1);
rgb(400bp)=(0.78,0.88,1)}
\tikzset{every picture/.style={line width=0.75pt}} %set default line width to 0.75pt        

\begin{tikzpicture}[x=0.75pt,y=0.75pt,yscale=-1,xscale=1]
%uncomment if require: \path (0,202); %set diagram left start at 0, and has height of 202

%Shape: Circle [id:dp9158198273647966] 
\path  [shading=_4ataw6yqb,_52kankop7] (168,73) .. controls (168,39.31) and (195.31,12) .. (229,12) .. controls (262.69,12) and (290,39.31) .. (290,73) .. controls (290,106.69) and (262.69,134) .. (229,134) .. controls (195.31,134) and (168,106.69) .. (168,73) -- cycle ; % for fading 
 \draw   (168,73) .. controls (168,39.31) and (195.31,12) .. (229,12) .. controls (262.69,12) and (290,39.31) .. (290,73) .. controls (290,106.69) and (262.69,134) .. (229,134) .. controls (195.31,134) and (168,106.69) .. (168,73) -- cycle ; % for border 

%Shape: Ellipse [id:dp9820685265727336] 
\draw  [dash pattern={on 0.84pt off 2.51pt}] (168,73) .. controls (168,63.35) and (195.31,55.54) .. (229,55.54) .. controls (262.69,55.54) and (290,63.35) .. (290,73) .. controls (290,82.65) and (262.69,90.46) .. (229,90.46) .. controls (195.31,90.46) and (168,82.65) .. (168,73) -- cycle ;
%Straight Lines [id:da5819287923563754] 
\draw [color={rgb, 255:red, 74; green, 144; blue, 226 }  ,draw opacity=1 ] [dash pattern={on 4.5pt off 4.5pt}]  (177.33,93.25) -- (275,54.83) ;
%Shape: Circle [id:dp5576782083705247] 
\draw  [draw opacity=0][fill={rgb, 255:red, 208; green, 2; blue, 27 }  ,fill opacity=1 ] (224.75,73) .. controls (224.75,70.65) and (226.65,68.75) .. (229,68.75) .. controls (231.35,68.75) and (233.25,70.65) .. (233.25,73) .. controls (233.25,75.35) and (231.35,77.25) .. (229,77.25) .. controls (226.65,77.25) and (224.75,75.35) .. (224.75,73) -- cycle ;
%Straight Lines [id:da1474048195066866] 
\draw [color={rgb, 255:red, 74; green, 144; blue, 226 }  ,draw opacity=1 ]   (275,54.83) -- (372.67,16.42) ;
%Shape: Circle [id:dp7710725956210569] 
\draw  [draw opacity=0][fill={rgb, 255:red, 208; green, 2; blue, 27 }  ,fill opacity=1 ] (270.75,54.83) .. controls (270.75,52.49) and (272.65,50.58) .. (275,50.58) .. controls (277.35,50.58) and (279.25,52.49) .. (279.25,54.83) .. controls (279.25,57.18) and (277.35,59.08) .. (275,59.08) .. controls (272.65,59.08) and (270.75,57.18) .. (270.75,54.83) -- cycle ;
%Straight Lines [id:da21359057500415712] 
\draw [color={rgb, 255:red, 74; green, 144; blue, 226 }  ,draw opacity=1 ]   (79.67,131.67) -- (177.33,93.25) ;
%Shape: Circle [id:dp001468373157834657] 
\draw  [draw opacity=0][fill={rgb, 255:red, 208; green, 2; blue, 27 }  ,fill opacity=1 ] (173.08,93.25) .. controls (173.08,90.9) and (174.99,89) .. (177.33,89) .. controls (179.68,89) and (181.58,90.9) .. (181.58,93.25) .. controls (181.58,95.6) and (179.68,97.5) .. (177.33,97.5) .. controls (174.99,97.5) and (173.08,95.6) .. (173.08,93.25) -- cycle ;

% Text Node
\draw (331.67,11.57) node [anchor=north west][inner sep=0.75pt]    {$\ell $};


\end{tikzpicture}

    \end{center}

    We can also check that $\R\PP^2$ is a topological surface. 
    
    We must first check that it is Hausdorff. Recall that if $X$ is a space and $q: X \rightarrow Y$ is a quotient map, then $V \in Y$ is open if and only if $q^{-1}(V) \in X$ is open.

    If $[p] \neq [q] \in \R\PP^2$ then $\pm p$ and $\pm q \in S^2$ are distinct antipodal points. We can then take small open discs about these in $S^2$ 


\end{example}

% It is not to hard to think up many exotic 'surfaces' with different characteristics: holes, no holes, bounded, unbounded, and so on.

% \begin{center}
    

% \tikzset{every picture/.style={line width=0.75pt}} %set default line width to 0.75pt        

% \begin{tikzpicture}[x=0.75pt,y=0.75pt,yscale=-1,xscale=1]
% %uncomment if require: \path (0,300); %set diagram left start at 0, and has height of 300

% %Shape: Circle [id:dp591684684256957] 
% \draw   (60,110) .. controls (60,87.91) and (77.91,70) .. (100,70) .. controls (122.09,70) and (140,87.91) .. (140,110) .. controls (140,132.09) and (122.09,150) .. (100,150) .. controls (77.91,150) and (60,132.09) .. (60,110) -- cycle ;
% %Shape: Ellipse [id:dp6271436208495527] 
% \draw  [dash pattern={on 4.5pt off 4.5pt}] (60,110) .. controls (60,104.48) and (77.91,100) .. (100,100) .. controls (122.09,100) and (140,104.48) .. (140,110) .. controls (140,115.52) and (122.09,120) .. (100,120) .. controls (77.91,120) and (60,115.52) .. (60,110) -- cycle ;
% %Shape: Polygon Curved [id:ds2919643672856125] 
% \draw   (169.85,112.46) .. controls (170.85,81.46) and (200.85,102.32) .. (219.85,102.46) .. controls (238.85,102.6) and (272.56,83.74) .. (269.85,112.46) .. controls (267.13,141.17) and (235.99,120.6) .. (219.85,122.46) .. controls (203.7,124.32) and (168.85,143.46) .. (169.85,112.46) -- cycle ;
% %Curve Lines [id:da29495862583968235] 
% \draw    (178.85,110.46) .. controls (187.18,118.79) and (203.18,116.13) .. (208.85,110.46) ;
% %Curve Lines [id:da05574670652938796] 
% \draw    (183.85,113.46) .. controls (190.56,108.74) and (198.56,109.6) .. (203.85,113.46) ;
% %Curve Lines [id:da9130401683536076] 
% \draw    (230.85,109.68) .. controls (239.18,118.01) and (255.18,115.35) .. (260.85,109.68) ;
% %Curve Lines [id:da7204510994314768] 
% \draw    (235.85,112.68) .. controls (242.56,107.97) and (250.56,108.82) .. (255.85,112.68) ;
% %Curve Lines [id:da2773823088001599] 
% \draw    (300,70) .. controls (319.67,91) and (319.67,130.33) .. (300,150) ;
% %Curve Lines [id:da9640418131331623] 
% \draw    (354,70) .. controls (335,90.33) and (335.67,130.33) .. (354,150) ;
% %Shape: Ellipse [id:dp22219783359667034] 
% \draw  [dash pattern={on 4.5pt off 4.5pt}] (315,110) .. controls (315,107.24) and (320.6,105) .. (327.5,105) .. controls (334.4,105) and (340,107.24) .. (340,110) .. controls (340,112.76) and (334.4,115) .. (327.5,115) .. controls (320.6,115) and (315,112.76) .. (315,110) -- cycle ;
% %Straight Lines [id:da16456755952599722] 
% \draw  [dash pattern={on 0.84pt off 2.51pt}]  (354,70) -- (362,61) ;
% %Straight Lines [id:da7546245438033814] 
% \draw  [dash pattern={on 0.84pt off 2.51pt}]  (300,70) -- (293,63) ;
% %Straight Lines [id:da9016788212880511] 
% \draw  [dash pattern={on 0.84pt off 2.51pt}]  (300,150) -- (294,156) ;
% %Straight Lines [id:da09916176203257931] 
% \draw  [dash pattern={on 0.84pt off 2.51pt}]  (354,150) -- (360,156) ;
% %Straight Lines [id:da28760276070953084] 
% \draw    (410,100) -- (510,100) ;
% %Straight Lines [id:da6331102333453389] 
% \draw  [dash pattern={on 0.84pt off 2.51pt}]  (510,100) -- (530,100) ;
% %Straight Lines [id:da21402696096677087] 
% \draw  [dash pattern={on 0.84pt off 2.51pt}]  (390,100) -- (410,100) ;
% %Straight Lines [id:da1318444055095227] 
% \draw    (410,130) -- (510,130) ;
% %Straight Lines [id:da3814972937738479] 
% \draw  [dash pattern={on 0.84pt off 2.51pt}]  (510,130) -- (530,130) ;
% %Straight Lines [id:da5990977106686659] 
% \draw  [dash pattern={on 0.84pt off 2.51pt}]  (390,130) -- (410,130) ;
% %Curve Lines [id:da8167058041805688] 
% \draw    (420,112.22) .. controls (425.56,120.56) and (436.22,117.89) .. (440,112.22) ;
% %Curve Lines [id:da07278542081013839] 
% \draw    (423.33,115.22) .. controls (427.81,110.51) and (433.14,111.37) .. (436.67,115.22) ;
% %Curve Lines [id:da23375889947741713] 
% \draw    (448,112.22) .. controls (453.56,120.56) and (464.22,117.89) .. (468,112.22) ;
% %Curve Lines [id:da8125417949179057] 
% \draw    (451.33,115.22) .. controls (455.81,110.51) and (461.14,111.37) .. (464.67,115.22) ;
% %Curve Lines [id:da989215927108835] 
% \draw    (475,111.7) .. controls (480.56,120.04) and (491.22,117.37) .. (495,111.7) ;
% %Curve Lines [id:da7272978688417131] 
% \draw    (478.33,114.7) .. controls (482.81,109.99) and (488.14,110.85) .. (491.67,114.7) ;
% %Straight Lines [id:da2512282339351666] 
% \draw  [dash pattern={on 0.84pt off 2.51pt}]  (505,114) -- (525,114) ;
% %Straight Lines [id:da05590955730268288] 
% \draw  [dash pattern={on 0.84pt off 2.51pt}]  (392,114) -- (412,114) ;




% \end{tikzpicture}

% \end{center}

% In this course we will frequently deal with such surfaces, and they are studied through the lense of topological surfaces.





% We care a significant amount about $X$ random object.

% \begin{definition}[Random Object]
%     We say that an object $X$ is a \vocab{random object} if we literally do not care about what it actually is.
% \end{definition}

% It is trivial to check that all objects you will meet in this course are random objects.


\end{document}
