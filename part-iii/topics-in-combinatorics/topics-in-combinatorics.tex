\documentclass[a4paper]{scrartcl}
% \documentclass{article}

\usepackage[
fancytheorems, 
fancyproofs, 
noindent, 
%  spacingfix,  
]{adam}

\usepackage{tikz}
\usepackage{bbm}
\usepackage{mathtools}

\title{Topics in Combinatorics}
\author{Adam Kelly (\texttt{ak2316@cam.ac.uk})}
\date{\today}


\allowdisplaybreaks

\begin{document}

\maketitle

It is relatively hard to pin down exactly what Combinatorics is. Many different types of problems are studied, usually with some recurring characters: graphs, hypergraphs, sets of numbers, sets of points, subsets of $\{0, 1\}^n$, subsets of a group, and so on. These objects all have a `subset of a structured object' flavour, and it is this type of object that combinatorics usually deals with. In this article, we will be more concerned with various \emph{techniques} used in combinatorics.

This article constitutes my notes for the `Topics in Combinatorics' course, held in Lent 2021 at Cambridge and lectured by Prof. Tim Gowers. These notes are \emph{not a transcription of the lectures}, and differ significantly in quite a few areas. Still, all lectured material should be covered.


\tableofcontents

% \clearpage




\section{Averaging Arguments}

A seemingly obvious fact\footnote{For discrete random variables, this is actually trivial, but needs a little work for continuous random variables.} in probability is that for a random variable $X$ we have
$$
\PP[X \geq \EE X] > 0.
$$
Despite being basic, this fact is incredibly useful in combinatorics. In the discrete case, the way we use it is to say that it is possible for $X$ to be at least its mean. This sort of argument is known as an \emph{averaging argument}, and in this section we are going to look at a few ways to apply it.



\end{document}