\documentclass[]{scrartcl}

\usepackage[
titlingx, 
theoremy, 
links, 
thinx
]{cjquines}
\usepackage{parskip}

%\usepackage[inline]{asymptote}

\begin{document}

\title{Random Problems}
\author{`Euclidean Geometry in Mathematical Olympiads' Chapter 8}
\date{\today}
\maketitle

\begin{problem}[8.25]
Let $A, B, C$ be three collinear points and $P$ be a point not on this line. Prove that the circumcenters of $\triangle P A B, \triangle P B C,$ and $\triangle P C A$ lie on a circle passing through $P$.
\end{problem}

\begin{center}
    \begin{asy}
    import geometry;
    unitsize(50);
    
    pair A = 1*left;
    pair B = 0.25*right;
    pair C = 2*right;
    
    pair P = 0.5*right + 1*down;
    
    line l = line(A, B);
    point O1 = circumcenter(P, A, B);
    point O2 = circumcenter(P, C, B);
    point O3 = circumcenter(P, A, C);
    
    circle omega1 = circle(P, A, B);
    circle omega2 = circle(P, C, B);
    circle omega3 = circle(P, A, C);
    
    label("$A$", A, N); label("$B$", B, N); 
    label("$C$", C, N); label("$P$", P, S);
    
    draw(l); 
    filldraw((path) omega1, opacity (0.05) + blue , blue);
    filldraw((path) omega2, opacity (0.05) + blue , blue);
    filldraw((path) omega3, opacity (0.05) + blue , blue);
    
    draw(circle(O1, O2, O3), red);
    
    dot(A); dot(B); 
    dot(C); dot(P);
    
    dot(O1); dot(O2); dot(O3);
    label("$O_1$", O1, S);
    label("$O_2$", O2, S);
    label("$O_3$", O3, S);
    \end{asy}
    \end{center}
    
\begin{proof}[Solution]
Inverting the diagram about $P$, we obtain the following diagram. It suffices to prove that $O_1^*$, $O_2^*$ and $O_3^*$ are collinear.
\begin{center}
    \begin{asy}
    import geometry;
    unitsize(125);
    
    pair A = 1*left;
    pair B = 0.25*right;
    pair C = 2*right;
    
    pair P = 0.5*right + 1*down;
    
    line l = line(A, B);
    point O1 = circumcenter(P, A, B);
    point O2 = circumcenter(P, C, B);
    point O3 = circumcenter(P, A, C);
    
    circle omega1 = circle(P, A, B);
    circle omega2 = circle(P, C, B);
    circle omega3 = circle(P, A, C);

    inversion iv = inversion(P, 1);

    circle lS = iv*l;

    point X = intersectionpoint(line(P, iv*O1), line(iv*A, iv*B));
    point Y = intersectionpoint(line(P, iv*O2), line(iv*C, iv*B));
    point Z = intersectionpoint(line(P, iv*O3), line(iv*A, iv*C));

    filldraw((path) lS, opacity (0.025) + red , red);

    draw(iv * omega1, dashed); 
    draw(iv * omega2, dashed); 
    draw(iv * omega3, dashed);

    label("$P$", P, S); 
    label("$A^*$", iv*A, NW); label("$B^*$", iv*B, N); label("$C^*$", iv*C, NE);
    label("$O_1^*$", iv*O1, NW); label("$O_2^*$", iv*O2, N); label("$O_3^*$", iv*O3, S);

    label("$X$", X, W);  label("$Y$", Y, W); label("$Z$", Z, W);

    draw(line(iv*O1, iv*O2), blue);
    draw(line(X, Y), green);

    dot(P); dot(iv*A); dot(iv*B); dot(iv*C);
    dot(iv*O1); dot(iv*O2); dot(iv*O3);
    dot(X); dot(Y); dot(Z);
    
    // label("$A$", A, N); label("$B$", B, N); 
    // label("$C$", C, N); label("$P$", P, S);
    
    // draw(l); 
    // filldraw((path) omega1, opacity (0.05) + blue , blue);
    // filldraw((path) omega2, opacity (0.05) + blue , blue);
    // filldraw((path) omega3, opacity (0.05) + blue , blue);
    
    // dot(A); dot(B); 
    // dot(C); dot(P);
    
    // dot(O1); dot(O2); dot(O3);
    // label("$O_1$", O1, S);
    // label("$O_2$", O2, S);
    // label("$O_3$", O3, S);
    \end{asy}
    \end{center}
    It is well known that $O_1^*$ is the reflection of the point $P$ across $A^* B^*$ and so on for points $O_2^*$ and $O_3^*$.
    Let the points $X$, $Y$ and $Z$ be the foot of the perpendiculars from $P$ to $A^* B^*$, $B^* C^*$ and $A^* C^*$ respectively.
    As $P$ lies on $(A^* B^* C^*)$, $XYZ$ is a simson line. Then, the homothety at $P$ with a scale factor of $2$ sends $X$ to $O_1^*$, $Y$ to $O_2^*$ and $Z$ to $O_3^*$, thus $O_1^*$, $O_2^*$ and $O_3^*$ are collinear, as required.
\end{proof}
\clearpage
\begin{problem}[BAMO 2008/6]
A point $D$ lies inside triangle $A B C .$ Let $A_{1}, B_{1}, C_{1}$ be the second intersection points of the lines $A D, B D,$ and $C D$ with the circumcircles of $B D C$, $C D A,$ and $A D B,$ respectively. Prove that
$$
\frac{A D}{A A_{1}}+\frac{B D}{B B_{1}}+\frac{C D}{C C_{1}}=1
$$
\end{problem}
\begin{center}
\begin{asy}
import geometry;
unitsize(45);

pair A = dir(120);
pair B = dir(210);
pair C = dir(330);

pair D = 0.10*down + 0.15*left;

circle omega1 = circle(B, D, C);
circle omega2 = circle(A, D, C);
circle omega3 = circle(B, D, A);

pair A1 = intersectionpoints(line(A, D), omega1)[0];
pair B1 = intersectionpoints(line(B, D), omega2)[1];
pair C1 = intersectionpoints(line(C, D), omega3)[1];

draw(A -- B -- C -- cycle);
draw(omega1, blue+dashed);
draw(omega2, blue+dashed);
draw(omega3, blue+dashed);
draw(segment(A, A1), blue);
draw(segment(B, B1), blue);
draw(segment(C, C1), blue);

label("$A$", A, N);
label("$B$", B, SW);
label("$C$", C, SE);
label("$D$", D, S);

label("$A_1$", A1, SE);
label("$B_1$", B1, E);
label("$C_1$", C1, W);

dot(A); dot(B); dot(C);
dot(D);

dot(A1); dot(B1); dot(C1);
\end{asy}
\end{center}
\begin{proof}[Solution]
We invert with respect to the the unit circle centered at $D$.

\begin{center}
    \begin{asy}
    import geometry;
    unitsize(60);
    
    pair A = dir(120);
    pair B = dir(210);
    pair C = dir(330);
    
    pair D = 0.10*down + 0.15*left;
    
    inversion iv = inversion(D, 0.6);
    
    circle omega1 = circle(B, D, C);
    circle omega2 = circle(A, D, C);
    circle omega3 = circle(B, D, A);
    
    pair A1 = intersectionpoints(line(A, D), omega1)[0];
    pair B1 = intersectionpoints(line(B, D), omega2)[1];
    pair C1 = intersectionpoints(line(C, D), omega3)[1];
     
    // draw((circle) iv, dashed);
    label("$D$", D, S);
    
    draw(iv*A -- iv*B -- iv*C -- cycle, red);
    
    label("$A^*$", iv*A, N);
    label("$B^*$", iv*B, SW);
    label("$C^*$", iv*C, SE);
    
    label("$A_1^*$", iv*A1, SE);
    label("$B_1^*$", iv*B1, E);
    label("$C_1^*$", iv*C1, W);
    
    dot(iv*A); dot(iv*B); dot(iv*C);
    dot(D);
    
    dot(iv*A1); dot(iv*B1); dot(iv*C1);
    \end{asy}
    \end{center}
We note that $A_1^*$, $B_1^*$ and $C_1^*$ lie on the sides of the triangle $A^* B^* C^*$.
Thus we have the well known fact
$$
\frac{D A_1^*}{A^* A_1^*} + \frac{D B_1^*}{B^* B_1^*} + \frac{D C_1^*}{C^* C_1^*} = 1.
$$
As we inverted about a unit circle, we have the following distance formulas.
$$
D X^* = \frac{1}{DX}, \quad \quad X^*Y^* = \frac{XY}{DX \cdot DY}. 
$$
Then, we note that by by the definition of inversion, $1 = AD \cdot A^* D \implies A^* D = \frac{1}{AD}$.
Applying this, we have 
\begin{align*}
    \frac{D A_1^*}{A^* A_1^*} + \frac{D B_1^*}{B^* B_1^*} + \frac{D C_1^*}{C^* C_1^*} 
&= \frac{1/D A_1}{A A_1 / (DA \cdot DA_1)} + \frac{1/D B_1}{B B_1 / (DB \cdot DB_1)} + \frac{1/D C_1}{C C_1 / (DC \cdot DC_1)} \\
&= \frac{DA}{A A_1} + \frac{D B_1}{B B_1} + \frac{D C_1}{C C_1} = 1,
\end{align*}
as required.
\end{proof}


\clearpage

\begin{problem}[Iran Olympiad 1996]
Consider a semicircle with center $O$ and diameter $\overline{A B}$. A line intersects line $A B$ at $M$ and the semicircle at $C$ and $D$ such that $M C>M D$
and $M B<M A$. Suppose $(A O C)$ and $(B O D)$ meet at a point $K$ other than $O$. Prove that $\angle M K O=90^{\circ}$. 
\end{problem}

\begin{center}
\begin{asy}
import geometry;
unitsize(60);

pair O = origin;
pair A = left;
pair B = right;

pair M = right*1.5;
pair C = dir(115);
pair D = intersectionpoints(line(C, M), circle(A, B))[0];

circle omega1 = circle(A, O, C);
circle omega2 = circle(B, O, D);

pair K = intersectionpoints(omega1, omega2)[1];

// draw(arc(circle(A, B), 0, 180));
draw(segment(A, B));
draw(circle(A, B));

draw(segment(M, C));
draw(segment(B, M), dotted);

draw(omega1, dashed);
draw(omega2, dashed);

label("$A$", A, SW);
label("$O$", O, SE);
label("$B$", B, SE);

label("$C$", C, NW);
label("$D$", D, NE);
label("$M$", M, S);

label("$K$", K, W);

dot(O); dot(A); dot(B);
dot(M); dot(C); dot(D);
dot(K);
\end{asy}
\end{center}

Invert about $(ABCD)$. It suffices to show that $AB \perp M^* K^*$.
We have $K^*$ is the intersection of lines $AC$ and $BD$, and $M^*$ is the intersection of $AB$ with $(COD)$. Then, as $O$ is the midpoint of $\overline{AB}$, $C$ is the foot of the perpendicular from $B$ to $AK^*$, and $D$ is the foot of the perpendicular from $A$  to $BK^*$, thus $(COD)$ is the nine-point circle of $ABK^*$, so $M^*K^* \perp AB$, as required. 
% Due to inversion, $OM^* = \frac{R^2}{OM}$
% We have a cyclic quadrilateral $CK^* MB$.
\begin{center}
    \begin{asy}
    import geometry;
    unitsize(60);
    
    pair O = origin;
    pair A = left;
    pair B = right;
    
    pair M = right*1.5;
    pair C = dir(115);
    pair D = intersectionpoints(line(C, M), circle(A, B))[0];
    
    inversion iv = inversion(O, 1);

    circle omega1 = circle(A, O, C);
    circle omega2 = circle(B, O, D);
    
    pair K = intersectionpoint(line(A, C), line(D, B));

    // pair Ks = iv*K;
    // pair Ms = iv*M;
    
    // draw(arc(circle(A, B), 0, 180));
    draw(segment(A, B));
    draw(circle(A, B));

    
    // draw(segment(M, C));
    // draw(segment(B, M), dotted);
    
    // draw(omega1, dashed);
    // draw(omega2, dashed);
    
    label("$A$", A, SW);
    label("$O$", O, SE);
    label("$B$", B, SE);
    
    label("$C$", C, NW);
    label("$D$", D, NE);
    label("$M^*$", iv*M, S);
    
    label("$K^*$", K, W);
    
    dot(O); dot(A); dot(B);
    dot(C); dot(D);
    // dot(M); dot(C); dot(D);
    dot(K);  
    dot(iv*M);
    
    draw(segment(A, K));
    draw(segment(B, K));

    // dot(Ms, red);
    // label("$M^*$", Ms, S, red);
    // dot(Ks, red);
    // label("$K^*$", Ks, W, red);
    // draw(segment(Ks, B), red); draw(segment(Ks, A), red);
    // draw(perpendicular(Ks, line(A, B)), red);
    // //draw(circle(C, K, B), blue);
    // line[] ts = tangents(circle(A, B, C), M);
    // draw(ts[0]);

    draw(circle(O, C, D));
    \end{asy}
    \end{center}

\clearpage

\begin{problem}[EGMO 8.29]
    Let $A B C$ be a triangle with incenter $I$ and circumcenter $O .$ Prove that line $I O$ passes through the centroid $G_{1}$ of the contact triangle.
\end{problem}

\begin{center}
\begin{asy}
    import geometry;
    unitsize(120);

    pair A = dir(120);
    pair B = dir(200);
    pair C = dir(340);

    pair O = origin;
    pair I = incenter(triangle(A, B, C));

    triangle it = intouch(triangle(A, B, C));
    draw(it);
    draw(circumcircle(it));

    pair G1 = centroid(it);

    draw(A -- B -- C -- cycle);

    label("$A$", A, N);
    label("$B$", B, SW);
    label("$C$", C, SE);

    label("$O$", O, S);
    label("$I$", I, S);
    label("$G_1$", G1, N);

    label("$X$", it.A, S);
    label("$Y$", it.B, NE);
    label("$Z$", it.C, W);

    dot(A); dot(B); dot(C);
    dot(O); dot(I); dot(G1);
    dot(it.A); dot(it.B); dot(it.C);
\end{asy}
\end{center}

Invert about the incircle. It is well known that the circumcircle of $\triangle ABC$ inverts to the nine-point circle of the contact triangle. Thus the nine-point center lies on the line $OI$. As the circumcenter of the contact triangle is $I$, $OI$ is the Euler line for the contact triangle. Thus implies $G_1$, the centroid of the contact triangle, lies on this line too.


\begin{center}
\begin{asy}
    import geometry;
    unitsize(120);

    pair A = dir(120);
    pair B = dir(200);
    pair C = dir(340);

    triangle t = triangle(A, B, C);

    pair O = origin;
    pair I = incenter(t);

    triangle it = intouch(t);

    pair X = it.A;
    pair Y = it.B;
    pair Z = it.C;

    // pair G1 = centroid(it);

    inversion iv = (inversion) incircle(t);

    // draw(it);
    // draw(circumcircle(it));
    // draw(A -- B -- C -- cycle);

    // label("$A$", A, N);
    // label("$B$", B, SW);
    // label("$C$", C, SE);

    draw(circle(it.A, it.B, it.C), dashed);
    draw(circle(Z, I), red);
    draw(circle(Y, I), red);
    draw(circle(X, I), red);
    
    draw(segment(X, Z));
    draw(segment(X, Y));
    draw(segment(Z, Y));

    draw(iv*circle(A, B, C), blue);
    // draw(circle(I, ))

    // label("$O$", O, S);
    label("$I$", I, S);
    // label("$G_1$", G1, N);

    label("$X$", X, S);
    label("$Y$", Y, NE);
    label("$Z$", Z, W);

    // dot(A); dot(B); dot(C);
    // dot(O); dot(I); dot(G1);
    dot(I);
    dot(X); dot(Y); dot(Z);
\end{asy}
\end{center}

\clearpage

\begin{problem}[NIMO 2014]
Let $A B C$ be a triangle and let $Q$ be a point such that $\overline{A B} \perp \overline{Q B}$ and $\overline{A C} \perp \overline{Q C} .$ A circle with center $I$ is inscribed in $\triangle A B C,$ and is tangent to $\overline{B C}, \overline{C A},$ and $\overline{A B}$ at points $D, E,$ and $F,$ respectively. If ray $Q I$ intersects $\overline{E F}$ at $P,$ prove that $\overline{D P} \perp \overline{E F} .$ 
\end{problem}

We begin with the following lemma.

\begin{lemma*}
    Let $\Gamma$ be a circle with diameter $AQ$, and let $X$ be some point not on the circle. Then the intersection of the line $QX$ and the circle with diameter $AX$ intersect again on $\Gamma$.
\end{lemma*}

\begin{center}
\begin{asy}
import geometry;
unitsize(60);

pair A = dir(75);
pair Q = dir(255);

pair X = 0.25*down + 0.3*left;

circle omega = circle(A, X);

pair P = intersectionpoints(omega, line(X, Q))[1];

// filldraw((path) circle(A, Q), opacity (0.05) + blue , blue);
// filldraw((path) omega, opacity (0.05) + red , red);
draw(omega, red);
draw(circle(A, Q), blue);
draw(Q -- P -- A); 

label("$A$", A, N);
label("$Q$", Q, S);
label("$X$", X, SW);
label("$K$", P, N);

dot(A); dot(Q); dot(X); dot(P);

// dot(A); dot(B); dot(C);
\end{asy}
\end{center}
\begin{proof}
    Let $K$ be the point of intersection between $QX$ and the circle with diameter $AX$. Then 
    $\angle AKX = \angle AKQ = 90^\circ$, so $K$ lies on $\Gamma$. 
\end{proof}

Now we return to the original problem.

\begin{center}
\begin{asy}
import geometry;
unitsize(60);

pair A = dir(75);
pair B = dir(220);
pair C = dir(320);
pair Q = dir(255);

pair I = incenter(triangle(A, B, C));
pair D = intouch(triangle(A, B, C)).A;
pair E = intouch(triangle(A, B, C)).B;
pair F = intouch(triangle(A, B, C)).C;

pair P = intersectionpoint(line(Q, I), line(E, F));

draw(A -- B -- C -- cycle);
draw(B -- Q -- C);
draw(circle(A, B, C), dashed);
draw(segment(Q, P));
draw(segment(E, F));
draw(incircle(triangle(A, B, C)));
draw(circle(A, I), red);
inversion iv = (inversion) incircle(triangle(A, B, C));
draw(iv * circle(A, B, C), red + dashed);

label("$A$", A, N); 
label("$B$", B, W);
label("$C$", C, E);
label("$Q$", Q, S);
label("$D$", D, S); 
label("$E$", E, E); 
label("$F$", F, W); 
label("$I$", I, SE); 
label("$P$", P, NE); 

dot(A); dot(B); dot(C); dot(Q);
dot(D); dot(E); dot(F); dot(I);
dot(P);

\end{asy}
\end{center}
Inverting about the incircle, we have that $P^*$ is the intersection of $QI$ and $(FIE)$. As $\angle IFA = \angle IEA = 90^\circ$, $(FIE)$ is a circle with diameter $AI$. Similarly, as $\angle ABQ = \angle ACQ = 90^\circ$, $AQ$ is a diameter of $(ABC)$. Thus we can apply our lemma, and $P^*$ lies on $(ABC)$. As $(ABC)$ inverts to the nine-point circle of $DEF$, $P$ is a point of intersection between $EF$ and the nine-point circle of $DEF$, that is, it's the midpoint of $EF$ or the foot of the altitude from $D$. $P$ can only be the midpoint of $EF$ if $I$, $A$ and $Q$ are collinear (in which case the foot of the perpendicular from $D$ is the midpoint of $EF$), thus $DP \perp EF$, as required.

\clearpage

\begin{problem}[EGMO 2013/5] 
Let $\Omega$ be the circumcircle of the triangle $A B C$. The circle $\omega$ is tangent to the sides $A C$ and $B C,$ and it is internally tangent to the circle $\Omega$ at the point
$P$. A line parallel to $A B$ intersecting the interior of triangle $A B C$ is tangent to $\omega$ at $Q$. Prove that $\angle A C P=\angle Q C B$. 
\end{problem}

\begin{center}
\begin{asy}
import geometry;
unitsize(60);

circle Omega = circle(origin, 1);

pair A = dir(115);
pair P = dir(165);

circle omega = circle(-0.25 * P, P);

line[] ts1 = tangents(omega, A);

pair C = intersectionpoints(ts1[1], Omega)[0];

line[] ts2 = tangents(omega, C);

pair B = intersectionpoints(ts2[0], Omega)[0];

pair Op = midpoint(segment(P, -0.25*P));
pair[] omegaips = intersectionpoints(omega, line(A, B));
pair X = midpoint(segment(omegaips[0], omegaips[1]));
pair Q = intersectionpoints(line(X, Op), omega)[0];

// pair B = dir(210);
// pair C = dir(330);

draw(parallel(Q, line(A, B)));

label("$A$", A, N);
label("$B$", B, SW);
label("$C$", C, SE);
label("$P$", P, W);
label("$\Omega$", dir(60), NE);

draw(Omega);
draw(omega);

draw(A -- B -- C -- cycle);

dot(A); dot(B); dot(C);
dot(P);

dot(Q);
\end{asy}
\end{center}

First, take a homothety from $P$ sending $Q$ to some point $Q'$ on $\Gamma$.

\begin{center}
\begin{asy}
    import geometry;
    unitsize(60);
    
    circle Omega = circle(origin, 1);
    
    pair A = dir(120);
    pair P = dir(165);
    
    circle omega = circle(-0.25 * P, P);
    
    line[] ts1 = tangents(omega, A);
    
    pair C = intersectionpoints(ts1[1], Omega)[0];
    
    line[] ts2 = tangents(omega, C);
    
    pair B = intersectionpoints(ts2[0], Omega)[0];
    
    pair Op = midpoint(segment(P, -0.25*P));
    pair[] omegaips = intersectionpoints(omega, line(A, B));
    pair X = midpoint(segment(omegaips[0], omegaips[1]));
    pair Q = intersectionpoints(line(X, Op), omega)[0];
    
    // inversion iv = inversion(P, 0.5);
    real r = length(origin - P) / length(Op - P);
    pair Qp = scale(r, P) * Q;

    // pair B = dir(210);
    // pair C = dir(330);
    
    draw(parallel(Q, line(A, B)));
    
    label("$A$", A, N);
    label("$B$", B, SW);
    label("$C$", C, SE);
    label("$P$", P, W);
    label("$Q'$", Qp, SW);
    label("$\Omega$", dir(60), NE);
    
    draw(Omega);
    draw(omega);

    draw(P -- Qp);
    draw(tangents(Omega, Qp)[0]);
    
    draw(A -- B -- C -- cycle);
    
    dot(A); dot(B); dot(C);
    dot(P);
    
    dot(Q);
    dot(Qp);

    // draw(iv*Omega, red);
    // dot(iv*A, red); dot(iv*B, red); dot(iv*C, red);
    // 282, 449, 255, 142, p273
\end{asy}
\end{center}


\end{document}