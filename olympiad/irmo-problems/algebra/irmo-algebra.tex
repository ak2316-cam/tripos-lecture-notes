\documentclass{pset}

\course{IrMO Algebra}
% \due{Due on \today}
\name{Adam Kelly}

% \blurb{\textbf{Remark.} This is a collection of all functional equation problems that have appeared in the Irish Mathematical Olympiad.
% The questions are ordered chronologically. All problems are due to their respective creators.}

 \blurb{\textbf{Remark.} This is a collection of all algebra problems that have appeared in the Irish Mathematical Olympiad and the Irish EGMO selection test.
 The questions are ordered chronologically. All problems are due to their respective creators.}


\begin{document}


\section*{EGMO Selection Test Problems}

\begin{problems}

\begin{problem}[EGMO TST 2019]
    For any positive integer \(n \geq 1\) denote \(n !=1 \cdot 2 \cdot 3 \cdots n .\) Prove that:
    \begin{enumerate}
        \item for any integer \(n \geq 1\) we have
        $$
        \frac{(2 n)^{2}}{(2 n-1) !(2 n+1) !}<\frac{1}{(2 n-1) !}-\frac{1}{(2 n+1) !}
        $$
        \item We have
        $$
        \frac{2^{2}}{1 ! 3 !}+\frac{4^{2}}{3 ! 5 !}+\frac{6^{2}}{5 ! 7 !}+\cdots+\frac{2018^{2}}{2017 ! 2019 !}<1-\frac{1}{2019 !}
        $$
    \end{enumerate}
    
\end{problem}

\begin{problem}[EGMO TST 2019]
    Finn has 5 distinct real numbers. He takes the sum of each pair of numbers and writes down the 10 sums. The 3 smallest sums are \(30,34\) and \(35,\) while the 2 largest are 46 and \(49 .\)
    Determine, with proof, the largest of Finn's 5 numbers.
\end{problem}

\begin{problem}[EGMO TST 2018]
    The non-zero real numbers \(a, b, c, d\) satisfy the equalities
    $$
    a+b+c+d=0, \quad \frac{1}{a}+\frac{1}{b}+\frac{1}{c}+\frac{1}{d}+\frac{1}{a b c d}=0
    $$
    Find, with proof, all possible values of the product \((a b-c d)(c+d)\)
\end{problem}

\begin{problem}[EGMO TST 2018]
    Let \(\left\{S_{n}: n=0,1,2 \ldots\right\}\) be a sequence defined by \(S_{0}=1\) and \(S_{n}=S_{n-1}+\frac{1}{\sqrt{n}}\) for \(n \geq 1\) Show that \(S_{n} \leq 2 \sqrt{n},\) for all \(n \geq 1\)
\end{problem}

\begin{problem}[EGMO TST 2015]
    \begin{enumerate}
        \item Which is the larger number: \(A=200 !\) or \(B=100^{200} ?\) Justify your answer.
        \item Which is the larger number: \(A=2000 !\) or \(B=100^{2000} ?\) Justify your answer.
    \end{enumerate}
\end{problem}

\begin{problem}[EGMO TST 2015]
    For any positive integer \(k\) define
    $$
    H_{k}=1+\frac{1}{2}+\frac{1}{3}+\cdots+\frac{1}{k}
    $$
    Prove that for \(n \geq 1\)
    $$
    1+\frac{1}{n+1}\left(H_{1}+H_{2}+\cdots+H_{n}\right)=H_{n+1}
    $$
\end{problem}

\begin{problem}[EGMO TST 2014]
    Let \(j, n\) be two integers such that \(n \geq 1\) and \(0 \leq j \leq n .\) Prove that
    $$
    \sum_{k=j}^{n}\left(\begin{array}{l}
    {n} \\
    {k}
    \end{array}\right)\left(\begin{array}{l}
    {k} \\
    {j}
    \end{array}\right)=2^{n-j}\left(\begin{array}{l}
    {n} \\
    {j}
    \end{array}\right)
    $$
\end{problem}

\begin{problem}[EGMO TST 2013]
    Let \(n \geq 1\) be a positive integer. Evaluate in terms of \(n\) the sum
    $$
    1 \cdot 2 \cdot 4+2 \cdot 3 \cdot 5+3 \cdot 4 \cdot 6+\ldots+n(n+1)(n+3)
    $$
\end{problem}

\begin{problem}[EGMO TST 2013]
    Determine with proof the largest and smallest of the three numbers
    $$
    \sqrt{7}, 1+\sqrt[3]{3}, \sqrt[4]{67}
    $$
\end{problem}



\end{problems}


\clearpage

\section*{IrMO Problems}

\begin{problems}

\begin{problem}[IrMO 2020 Q5]
    Let $a, b, c>0 .$ Prove that
$$
\sqrt[7]{\frac{a}{b+c}+\frac{b}{c+a}}+\sqrt[7]{\frac{b}{c+a}+\frac{c}{a+b}}+\sqrt[7]{\frac{c}{a+b}+\frac{a}{b+c}} \geq 3.
$$
\end{problem}

\begin{problem}[IrMO 2019 Q4]
    Find the set of all quadruplets \((x, y, z, w)\) of non-zero real numbers which satisfy
    $$
    1+\frac{1}{x}+\frac{2(x+1)}{x y}+\frac{3(x+1)(y+2)}{x y z}+\frac{4(x+1)(y+2)(z+3)}{x y z w}=0
    $$
\end{problem}

\begin{problem}[IrMO 2019 Q7]
    Three non-zero real numbers \(a, b, c\) satisfy \(a+b+c=0\) and \(a^{4}+b^{4}+c^{4}=128\) Determine all possible values of \(a b+b c+c a\)
\end{problem}


\begin{problem}[IrMO 2018 Q9]
    The sequence of positive integers \(a_{1}, a_{2}, a_{3}, \ldots\) satisfies
    $$
    a_{n+1}=a_{n}^{2}+2018 \quad \text { for } n \geq 1
    $$
    Prove that there exists at most one \(n\) for which \(a_{n}\) is the cube of an integer.
\end{problem}


\begin{problem}[IrMO 2017 Q2]
    Solve the equations
    $$
    a+b+c=0, \quad a^{2}+b^{2}+c^{2}=1, \quad a^{3}+b^{3}+c^{3}=4 a b c
    $$
    for \(a, b,\) and \(c\)
\end{problem}

\begin{problem}[IrMO 2017 Q5]
    The sequence \(a=\left(a_{0}, a_{1}, a_{2}, \ldots\right)\) is defined by \(a_{0}=0, a_{1}=2\) and
    $$
    a_{n+2}=2 a_{n+1}+41 a_{n} \text { for all } n \geq 0
    $$
    Prove that \(a_{2016}\) is divisible by 2017 .
\end{problem}

\begin{problem}[IrMO 2017 Q10]
    Given a positive integer \(m,\) a sequence of real numbers \(a=\left(a_{1}, a_{2}, a_{3}, \ldots\right)\) is called m-powerful if it satisfies \(\left(\sum_{k=1}^{n} a_{k}\right)^{m}=\sum_{k=1}^{n} a_{k}^{m} \quad\) for all positive integers \(n\)
    \begin{enumerate}
        \item Show that a sequence is 30 -powerful if and only if at most one of its terms is non-zero.
        \item Find a sequence none of whose terms is zero but which is 2017 -powerful.
    \end{enumerate}
\end{problem}

\begin{problem}[IrMO 2016 Q3]
    Do there exist four polynomials \(P_{1}(x), P_{2}(x), P_{3}(x), P_{4}(x)\) with real coefficients, such that the sum of any three of them always has a real root, but the sum of any two of them has no real root?
\end{problem}

\begin{problem}[IrMO 2016 Q8]
    Suppose \(a, b, c\) are real numbers such that \(a b c \neq 0 .\) Determine \(x, y, z\) in terms of \(a, b, c\) such that
    $$
    b z+c y=a, c x+a z=b, a y+b x=c
    $$
    Prove also that
    $$
    \frac{1 x^{2}}{a^{2}}=\frac{1 y^{2}}{b^{2}}=\frac{1 z^{2}}{c^{2}}
    $$
\end{problem}

\begin{problem}[IrMO 2016 Q9]
    Show that the number
    $$\left(\frac{251}{\frac{1}{\sqrt[3]{252}-5 \sqrt[3]{2}}-10 \sqrt[3]{63}}+\frac{1}{\frac{251}{\sqrt[3]{252}+5 \sqrt[3]{2}}+10 \sqrt[3]{63}}\right)^{3}$$
    is an integer and find its value.
\end{problem}

\begin{problem}[IrMO 2015 Q5]
    Suppose a doubly infinite sequence of real numbers
    $$
    a_{-2}, a_{-1}, a_{0}, a_{1}, a_{2}, \ldots
    $$
    has the property that 
    $$a_{n+3}=\frac{a_{n}+a_{n+1}+a_{n+2}}{3},$$
    for all integers \(n\).

    Show that if this sequence is bounded (i.e., if there exists a number \(R\) such that \(\left.\left|a_{n}\right| \leq R \text { for all } n\right),\) then \(a_{n}\) has the same value for all \(n .\)
\end{problem}

\begin{problem}[IrMO 2015 Q9]
    Let \(p(x)\) and \(q(x)\) be non-constant polynomial functions with integer coefficients. It is known that the polynomial
    $$
    p(x) q(x)-2015
    $$
    has at least 33 different integer roots. Prove that neither \(p(x)\) nor \(q(x)\) can be a polynomial of degree less than three.
\end{problem}

\begin{problem}[IrMO 2014 Q4]
    Three different nonzero real numbers \(a, b, c\) satisfy the equations
    $$
    a+\frac{2}{b}=b+\frac{2}{c}=c+\frac{2}{a}=p
    $$
    where \(p\) is a real number. Prove that \(a b c+2 p=0\)
\end{problem}


\begin{problem}[IrMO 2014 Q8]
\begin{enumerate}
    \item Let \(a_{0}, a_{1}, a_{2}\) be real numbers and consider the polynomial \(P(x)=a_{0}+a_{1} x+\) \(a_{2} x^{2} .\) Assume that \(P(-1), P(0)\) and \(P(1)\) are integers. Prove that \(P(n)\) is an integer for all integers \(n\)
    \item Let \(a_{0}, a_{1}, a_{2}, a_{3}\) be real numbers and consider the polynomial \(Q(x)=a_{0}+\) \(a_{1} x+a_{2} x^{2}+a_{3} x^{3} .\) Assume that there exists an integer \(i\) such that \(Q(i)\) \(Q(i+1), Q(i+2)\) and \(Q(i+3)\) are integers. Prove that \(Q(n)\) is an integer for all integers \(n\)
\end{enumerate}
\end{problem}

\begin{problem}[IrMO 2014 Q9]
    Let \(n\) be a positive integer and \(a_{1}, \ldots, a_{n}\) be positive real numbers. Let \(g(x)\) denote the product
    $$
    \left(x+a_{1}\right) \cdots\left(x+a_{n}\right)
    $$
    Let \(a_{0}\) be a real number and let
    $$
    f(x)=\left(x-a_{0}\right) g(x)=x^{n+1}+b_{1} x^{n}+b_{2} x^{n-1}+\ldots+b_{n} x+b_{n+1}
    $$
    Prove that all the coefficients \(b_{1}, b_{2}, \ldots, b_{n+1}\) of the polynomial \(f(x)\) are negative if and only if
    $$
    a_{0}>a_{1}+a_{2}+\ldots+a_{n}
    $$
\end{problem}

\begin{problem}[IrMO 2013 Q9]
We say that the doubly infinite sequence
$$
\dots, s_{-2}, s_{-1}, s_0, s_1, s_2, \dots
$$
is \emph{subaveraging} if $s_n = (s_{n - 1} + s_{n + 1})/4$ for all integers $n$.
\begin{enumerate}
    \item Find a subaveraging sequence in which all entries are different from each other. Prove that all entries are indeed distinct.
    \item Show that if \(\left(s_{n}\right)\) is a subaveraging sequence such there exist distinct integers \(m, n\) such that \(s_{m}=s_{n},\) then there are infinitely many pairs of distinct integers \(i, j\) with \(s_{i}=s_{j}\).
\end{enumerate}
\end{problem}

\begin{problem}[IrMO 2013 Q10]
    Let \(a, b, c\) be real numbers and let \(x=a+b+c, y=a^{2}+b^{2}+c^{2}, z=a^{3}+b^{3}+c^{3}\) and \(S=2 x^{3}-9 x y+9 z\)
    \begin{enumerate}
        \item Prove that \(S\) is unchanged when \(a, b, c\) are replaced by \(a+t, b+t, c+t\) respectively, for any real number \(t\)
        \item Prove that \(\left(3 y-x^{2}\right)^{3} \geq 2 S^{2}\)
    \end{enumerate}
\end{problem}

\begin{problem}[IrMO 2011 Q1]
    Suppose \(a b c \neq 0 .\) Express in terms of \(a, b,\) and \(c\) the solutions \(x, y, z, u, v, w\) of the equations
    $$
    x+y=a, \quad z+u=b, \quad v+w=c, \quad a y=b z, \quad u b=c v, \quad w c=a x
    $$
\end{problem}

\begin{problem}[IrMO 2011 Q3]
    The integers \(a_{0}, a_{1}, a_{2}, a_{3}, \ldots\) are defined as follows:
    $$
    a_{0}=1, \quad a_{1}=3, \quad \text { and } a_{n+1}=a_{n}+a_{n-1} \text { for all } n \geq 1
    $$
    Find all integers \(n \geq 1\) for which \(n a_{n+1}+a_{n}\) and \(n a_{n}+a_{n-1}\) share a common factor greater than 1
\end{problem}

\begin{problem}[IrMO 2010 Q7]
    For each odd integer \(p \geq 3\) find the number of real roots of the polynomial
    $$
    f_{p}(x)=(x-1)(x-2) \cdots(x-p+1)+1
    $$
\end{problem}

\begin{problem}[IrMO 2010 Q9]
    Let \(n \geq 3\) be an integer and \(a_{1}, a_{2}, \ldots, a_{n}\) be a finite sequence of positive integers, such that, for \(k=2,3, \ldots, n\)
    $$
    n\left(a_{k}+1\right)-(n-1) a_{k-1}=1
    $$
    Prove that \(a_{n}\) is not divisible by \((n-1)^{2}\)
\end{problem}

\begin{problem}[IrMO 2009 Q6]
    Let \(p(x)\) be a polynomial with rational coefficients. Prove that there exists a positive integer \(n\) such that the polynomial \(q(x)\) defined by
    $$
    q(x)=p(x+n)-p(x)
    $$
    has integer coefficients.
\end{problem}


\begin{problem}[IrMO 2008 Q2]
    For positive real numbers \(a, b, c\) and \(d\) such that \(a^{2}+b^{2}+c^{2}+d^{2}=1\) prove that
    $$
    a^{2} b^{2} c d+a b^{2} c^{2} d+a b c^{2} d^{2}+a^{2} b c d^{2}+a^{2} b c^{2} d+a b^{2} c d^{2} \leq \frac{3}{32}
    $$
    and determine the cases of equality.
\end{problem}

\begin{problem}[IrMO 2008 Q8]
    Find \(a_{3}, a_{4}, \ldots, a_{2008},\) such that \(a_{i}=\pm 1\) for \(i=3, \ldots, 2008\) and
    $$\sum_{i=3}^{2008} a_{i} 2^{i}=2008$$
    and show that the numbers \(a_{3}, a_{4}, \ldots, a_{2008}\) are uniquely determined by these conditions.
\end{problem}

\begin{problem}[IrMO 2007 Q6]
    Let \(r, s\) and \(t\) be the roots of the cubic polynomial
    $$
    p(x)=x^{3}-2007 x+2002
    $$
    Determine the value of
    $$
    \frac{r-1}{r+1}+\frac{s-1}{s+1}+\frac{t-1}{t+1}
    $$
\end{problem}

\begin{problem}[IrMO 2007 Q10]
    Suppose \(a\) and \(b\) are real numbers such that the quadratic polynomial
    $$
    f(x)=x^{2}+a x+b
    $$
    has no nonnegative real roots. Prove that there exist two polynomials \(g, h,\) whose coefficients are nonnegative real numbers, such that
    $$
    f(x)=\frac{g(x)}{h(x)}
    $$
    for all real numbers \(x\).
\end{problem}

\begin{problem}[IrMO 2006 Q4]
    Find the greatest value and the least value of \(x+y,\) where \(x\) and \(y\) are real numbers, with \(x \geq-2, y \geq-3\) and
    $$
    x-2 \sqrt{x+2}=2 \sqrt{y+3}-y
    $$
\end{problem}

\begin{problem}[IrMO 2004 Q4]
    Prove that there are only two real numbers \(x\) such that
    $$
    (x-1)(x-2)(x-3)(x-4)(x-5)(x-6)=720
    $$
\end{problem}

\begin{problem}[IrMO 2004 Q4]
    Define the function \(m\) of the three real variables \(x, y, z\) by
    $$
    m(x, y, z)=\max \left(x^{2}, y^{2}, z^{2}\right), x, y, z \in \mathbb{R}
    $$
    Determine, with proof, the minimum value of \(m\) if \(x, y, z\) vary in \(\mathbb{R}\) subject to the following restrictions:
    $$
    x+y+z=0, \quad x^{2}+y^{2}+z^{2}=1
    $$
\end{problem}

\begin{problem}[IrMO 2003 Q3]
    For each positive integer \(k,\) let \(a_{k}\) be the greatest integer not exceeding \(\sqrt{k}\) and let \(b_{k}\) be the greatest integer not exceeding \(\sqrt[3]{k} .\) Calculate
    $$
    \sum_{k=1}^{2003}\left(a_{k}-b_{k}\right)
    $$
\end{problem}

\begin{problem}[IrMO 2003 Q9]
    Let \(a, b>0 .\) Determine the largest number \(c\) such that
    $$
    c \leq \max \left(a x+\frac{1}{a x}, b x+\frac{1}{b x}\right)
    $$
    for all \(x>0\)
\end{problem}

\begin{problem}[IrMO 2002 Q9]
    For each real number \(x,\) define \(\lfloor x\rfloor\) to be the greatest integer less than or equal to \(x\)
    Let \(\alpha=2+\sqrt{3} .\) Prove that
    $$
    \alpha^{n}-\left\lfloor\alpha^{n}\right\rfloor= 1-\alpha^{-n}, \text { for } n=0,1,2, \ldots
    $$
\end{problem}

\begin{problem}[IrMO 2000 Q5]
    Let \(p(x)=a_{0}+a_{1} x+\cdots+a_{n} x^{n}\) be a polynomial with non-negative real coefficients. Suppose that \(p(4)=2\) and that \(p(16)=8 .\) Prove that \(p(8) \leq 4\) and find, with proof, all such polynomials with \(p(8)=4\).
\end{problem}

\begin{problem}[IrMO 1999 Q5]
    Three real numbers \(a, b, c\) with \(a<b<c,\) are said to be in arithmetic progression
    if \(c-b=b-a\) Define a sequence \(u_{n}, n=0,1,2,3, \ldots\) as follows: \(u_{0}=0, u_{1}=1\) and, for each \(n \geq\)
    \(1, u_{n+1}\) is the smallest positive integer such that \(u_{n+1}>u_{n}\) and \(\left\{u_{0}, u_{1}, \ldots, u_{n}, u_{n+1}\right\}\) contains no three elements that are in arithmetic progression. Find \(u_{100}\)
\end{problem}

\begin{problem}[IrMO 1999 Q6]
    Solve the system of (simultaneous) equations
    $$
    \begin{aligned}
    y^{2} &=(x+8)\left(x^{2}+2\right) \\
    y^{2} &=(8+4 x) y+5 x^{2}-16 x-16
    \end{aligned}
    $$
\end{problem}

\begin{problem}[IrMO 1998 Q9]
    A sequence of real numbers \(x_{n}\) is defined recursively as follows: \(x_{0}, x_{1}\) are arbitrary positive real numbers, and
    $$
    x_{n+2}=\frac{1+x_{n+1}}{x_{n}}, n=0,1,2, \ldots
    $$
    Find \(x_{1998}\)
\end{problem}

\begin{problem}[IrMO 1995 Q7]
    Suppose that \(a, b\) and \(c\) are complex numbers, and that all three roots \(z\) of the equation
    $$
    x^{3}+a x^{2}+b x+c=0
    $$
    satisfy \(|z|=1\) (where || denotes absolute value). Prove that all three roots \(w\) of the equation
    $$
    x^{3}+|a| x^{2}+|b| x+|c|=0
    $$
    also satisfy \(|w|=1\)
\end{problem}


\begin{problem}[IrMO 1994 Q1]
    A sequence \(x_{n}\) is defined by the rules: \(x_{1}=2\) and
    $$
    n x_{n}=2(2 n-1) x_{n-1}, \quad n=2,3, \ldots
    $$
    Prove that \(x_{n}\) is an integer for every positive integer \(n\)
\end{problem}

\begin{problem}[IrMO 1994 Q7]
    Let \(p, q, r\) be distinct real numbers that satisfy the equations
    $$
    \begin{aligned}
    q &=p(4-p) \\
    r &=q(4-q) \\
    p &=r(4-r)
    \end{aligned}
    $$
    Find all possible values of \(p+q+r\)
\end{problem}

\begin{problem}[IrMO 1994 Q9]
    Let \(w, a, b\) and \(c\) be distinct real numbers with the property that there exist real numbers \(x, y\) and \(z\) for which the following equations hold:
    $$
    \begin{aligned}
    x+y+z &=1 \\
    x a^{2}+y b^{2}+z c^{2} &=w^{2} \\
    x a^{3}+y b^{3}+z c^{3} &=w^{3} \\
    x a^{4}+y b^{4}+z c^{4} &=w^{4}
    \end{aligned}
    $$
    Express \(w\) in terms of \(a, b\) and \(c .\)
\end{problem}

\begin{problem}[IrMO 1993 Q1]
    The real numbers \(\alpha, \beta\) satisfy the equations
    $$
    \begin{aligned}
    \alpha^{3}-3 \alpha^{2}+5 \alpha-17 &=0 \\
    \beta^{3}-3 \beta^{2}+5 \beta+11 &=0
    \end{aligned}
    $$
    Find \(\alpha+\beta\)
\end{problem}

\begin{problem}[IrMO 1993 Q4]
    Let \(a_{0}, a_{1}, \ldots, a_{n-1}\) be real numbers, where \(n \geq 1,\) and let the polynomial
    $$
    f(x)=x^{n}+a_{n-1} x^{n-1}+\ldots+a_{0}
    $$
    be such that \(|f(0)|=f(1)\) and each root \(\alpha\) of \(f\) is real and satisfies \(0<\alpha<1\) Prove that the product of the roots does not exceed \(1 / 2^{n}\).
\end{problem}



\begin{problem}[IrMO 1993 Q5]
    Given a complex number \(z=x+i y(x, y \text { real), we denote by } P(z)\) the corresponding point \((x, y)\) in the plane. Suppose \(z_{1}, z_{2}, z_{3}, z_{4}, z_{5}, \alpha\) are nonzero complex numbers such that
    \begin{enumerate}
        \item \(P\left(z_{1}\right), P\left(z_{2}\right), P\left(z_{3}\right), P\left(z_{4}\right), P\left(z_{5}\right)\) are the vertices of a convex pentagon \(\mathbf{Q}\) containing the origin 0 in its interior and
        \item \(P\left(\alpha z_{1}\right), P\left(\alpha z_{2}\right), P\left(\alpha z_{3}\right), P\left(\alpha z_{4}\right)\) and \(P\left(\alpha z_{5}\right)\) are all inside \(\mathbf{Q}\)
    
    \end{enumerate}
    If \(\alpha=p+i q,\) where \(p\) and \(q\) are real, prove that \(p^{2}+q^{2} \leq 1\) and that
    $$
    p+q \tan (\pi / 5) \leq 1
    $$
\end{problem}

\begin{problem}[IrMO 1993 Q7]
    Let \(a_{1}, a_{2}, \ldots, a_{n}, b_{1}, b_{2}, \ldots, b_{n}\) be \(2 n\) real numbers, where \(a_{1}, a_{2}, \ldots, a_{n}\) are distinct, and suppose that there exists a real number \(\alpha\) such that the product
    $$
    \left(a_{i}+b_{1}\right)\left(a_{i}+b_{2}\right) \ldots\left(a_{i}+b_{n}\right)
    $$
    has the value \(\alpha\) for \(i=1,2, \ldots, n .\) Prove that there exists a real number \(\beta\) such that the product
    $$
    \left(a_{1}+b_{j}\right)\left(a_{2}+b_{j}\right) \ldots\left(a_{n}+b_{j}\right)
    $$
    has the value \(\beta\) for \(j=1,2, \ldots, n\)
\end{problem}

\begin{problem}[IrMO 1993 Q9]
    Let \(x\) be a real number with \(0<x<\pi .\) Prove that, for all natural numbers \(n,\) the
    $$
    \sin x+\frac{\sin 3 x}{3}+\frac{\sin 5 x}{5}+\ldots+\frac{\sin (2 n-1) x}{2 n-1}
    $$
    is positive.
\end{problem}



\begin{problem}[IrMO 1992 Q1]
    Describe in geometric terms the set of points \((x, y)\) in the plane such that \(x\) and \(y\) satisfy the condition \(t^{2}+y t+x \geq 0\) for all \(t\) with \(-1 \leq t \leq 1\)
\end{problem}

\begin{problem}[IrMO 1992 Q2]
    How many ordered triples \((x, y, z)\) of real numbers satisfy the system of equations
    $$
    \begin{aligned}
    x^{2}+y^{2}+z^{2} &=9 \\
    x^{4}+y^{4}+z^{4} &=33 \\
    x y z &=-4 ?
    \end{aligned}
    $$
\end{problem}

\begin{problem}[IrMO 1992 Q8]
    Let \(a, b, c\) and \(d\) be real numbers with \(a \neq 0 .\) Prove that if all the roots of the cubic equation
    $$
    a z^{3}+b z^{2}+c z+d=0
    $$
    lie to the left of the imaginary axis in the complex plane, then
    $$
    a b>0, b c-a d>0, a d>0
    $$
\end{problem}

\begin{problem}[IrMO 1991 Q5]
    Find all polynomials
    $$
    f(x)=x^{n}+a_{1} x^{n-1}+\cdots+a_{n}
    $$
    with the following properties:
    \begin{itemize}
        \item all the coefficients \(a_{1}, a_{2}, \ldots, a_{n}\) belong to the set \(\{-1,1\}\)
        \item all the roots are real.
    \end{itemize}
\end{problem}

\begin{problem}[IrMO 1990 Q8]
    Let \(t\) be a real number, and let
    $$
    a_{n}=2 \cos \left(\frac{t}{2^{n}}\right)-1, \quad n=1,2,3, \ldots
    $$
    Let \(b_{n}\) be the product \(a_{1} a_{2} a_{3} \cdots a_{n} .\) Find a formula for \(b_{n}\) that does not involve a product of \(n\) terms, and deduce that
    $$
    \lim _{n \rightarrow \infty} b_{n}=\frac{2 \cos t+1}{3}
    $$
\end{problem}

\begin{problem}[IrMO 1988 Q7]
    A function \(f,\) defined on the set of real numbers \(\mathbb{R}\) is said to have a horizontal chord of length \(a>0\) if there is a real number \(x\) such that \(f(a+x)=f(x) .\) Show that the cubic
    $$f(x)=x^{3}-x$$
    (where \(x \in \mathbb{R}\))
    has a horizontal chord of length \(a\) if, and only if, \(0<a \leq 2\)
\end{problem}

\begin{problem}[IrMO 1988 Q11]
    If facilities for division are not available, it is sometimes convenient in determining the decimal expansion of \(1 / a, a>0,\) to use the iteration
    $$
    x_{k+1}=x_{k}\left(2-a x_{k}\right), \quad k=0,1,2, \ldots
    $$
    where \(x_{0}\) is a selected "starting" value. Find the limitations, if any, on the starting values \(x_{0},\) in order that the above iteration converges to the desired value \(1 / a\)
\end{problem}

\begin{problem}[IrMO 1988 Q12]
    Prove that if \(n\) is a positive integer, then
    $$
    \sum_{k=1}^{n} \cos ^{4}\left(\frac{k \pi}{2 n+1}\right)=\frac{6 n-5}{16}
    $$
\end{problem}






















































































































\end{problems}

\end{document}