\documentclass[a4paper]{scrartcl}

\usepackage[
    fancytheorems, 
    fancyproofs, 
    noindent, 
]{adam}


\title{Complex Analysis}
\author{Adam Kelly (\texttt{ak2316@cam.ac.uk})}
\date{\today}

\allowdisplaybreaks

\begin{document}

\maketitle

% This is a short description of the course. It should give a little flavour of what the course is about, and what will be roughly covered in the notes.

This article constitutes my notes for the `Complex Analysis' course, held in Lent 2022 at Cambridge. These notes are \emph{not a transcription of the lectures}, and differ significantly in quite a few areas. Still, all lectured material should be covered.



\tableofcontents


\section{Basic Notions}

\subsection{Complex Differentiation}

This course is about complex valued functions of a (single) complex variable, that is, functions
$$
f:A \rightarrow \C, \quad \text{where }A \subset C.
$$
In this course we will focus on differentiable functions $f$ (the definition of which we will give soon), but let's start by recalling the notion of continuity.

\begin{definition}[Continuity]
    A function $f: A \rightarrow \C$ is \vocab{continuous} at a point $w \in A$ if for all $\varepsilon > 0$ there exists some $\delta > 0$ such that if $z \in A$ with $|z - w| < \delta$ then $|f(z) - f(w)| < \varepsilon$.
\end{definition}

Note that by identifying $\C$ with $\R^2$ in the usual way, we can write
$
f(z) = u(x, y) + iv(x, y)
$
for $z = x + iy \in A$ with $u, v: A \rightarrow \R$. With this, the triangle inequality implies that $f$ is continuous at $w = c + id$ if and only if $u, v$ are continuous at $(c, d)$.

We can now define the notion of differentiation for complex functions.

\begin{definition}[Complex Differentiation]
    Let $f: U \rightarrow \C$ where $U \subseteq \C$ is open.   
    We say that $f$ is \vocab{differentiable} at $w \in U$ if the limit
    $$
    f'(w) = \lim_{z \to w} \frac{f(z) - f(w)}{z - w}
    $$
    exists as a complex number.
\end{definition}
\begin{definition}[Holomorphic]
    We say that $f$ is \vocab{holomorphic at $w \in U$} if there is $\varepsilon > 0$ such that $D(w, \varepsilon) \subset U$ where $f$ is differentiable at every point in $D(w, \varepsilon)$. We say that $f$ is \vocab{holomorphic} if it is holomorphic at every point in $U$.
\end{definition}

\begin{remark}
    Sometime authors use the word `analytic' instead of the word `holomorphic'.
\end{remark}
% for $z = x + iy \in A$ and a pair of real valued functions $u, v : A \rightarrow \R$.

It's straightforward to check that the same rules of differentiation for real variables hold for complex functions (product rule, quotient rule, chain rule, etc), which makes computing complex derivatives relatively straightforward.

\begin{example}[Complex Differentiable Functions]~
    \vspace*{-1.5\baselineskip}
\begin{enumerate}[label=(\roman*)]
    \item Polynomials $p(z) = \sum_{k = 0}^n a_k z^k$, $a_1, \dots, a_k \in \C$ are holomorphic on all of $\C$.
    \item If $p$ and $q$ are polynomials, then $p/q$ is holomorphic on $\C \backslash \{z \mid q(z) = 0\}$.
\end{enumerate}
\end{example}
    
We saw above that the notion of continuity is basically the same for both real and complex variables. A natural question is then if the same is true for complex differentiability.
Indeed, the answer is no! Complex differentiability is a much stronger condition than asserting real differentiability on the real and imaginary components of a function.

We do however have a criterion for establishing complex differentiability from the real and imaginary parts.

\begin{theorem}[Cauchy-Riemann Equations]
    A function $f = u + iv: U \rightarrow \C$ is differentiable at $w = c + id \in U$ if and only if $u, v : U \rightarrow \R$ are differentiable at $(c, d) \in U$ \emph{and} $u, v$ satisfy the \vocab{Cauchy-Riemann} equations at $(c, d)$:
    \begin{align*}
        \frac{\partial u}{\partial x} &= \frac{\partial v}{\partial y} \\
        \frac{\partial u}{\partial y} &= - \frac{\partial v}{\partial x} \quad \text{at }(c, d).
    \end{align*}
    If $f$ is differentiable at $w = c + id$, then $f'(w) = \frac{\partial u}{\partial x}(c, d) + i \frac{\partial v}{\partial x} (c, d)$.
\end{theorem}
\begin{proof}
    The function $f$ is differentiable at $w$ with derivative $f'(w) = p + iq$ if and only if
    $$
    \lim _{z \rightarrow w} \frac{f(z)-f(w)}{z-w}=p+i q \iff \lim _{z \rightarrow w} \frac{f(z)-f(w)-(z-w)(p+i q)}{|z-w|}=0.
    $$
    Separating real and imaginary parts and writing $f = u + iv$, the above holds if and only if
    $$
    \lim _{(x, y) \rightarrow(c, d)} \frac{u(x, y)-u(c, d)-p(x-c)+q(y-d)}{\sqrt{(x-c)^{2}+(y-d)^{2}}}=0
    $$
    and
    $$
    \lim _{(x, y) \rightarrow(c, d)} \frac{v(x, y)-v(c, d)-q(x-c)-p(y-d)}{\sqrt{(x-c)^{2}+(y-d)^{2}}}=0.
    $$
    That is, if and only if $u$ is differentiable at $(c, d)$ with $Du(c, d)(x, y) = px - qy$, and $v$ is differentiable at $(c, d)$ with $Dv(c, d)(x, y) = qx + py$. 

    Taking partial derivatives, this is true if and only if $u, v$ are differentiable at $(c, d)$ and $u_x(c, d) = p = v_y(c, d)$ and $u_y(c, d) = -q = -v_x(c, d)$, that is, the Cauchy-Riemann equations hold at $(c, d)$.

    We also get from the above that if $f$ is differentiable at $w$, then $f'(w) = p + iq = u_x(c, d) + i v_x(c, d)$.
\end{proof}

\begin{remark}[Warning]
    The real and imaginary parts
    $u, v$ satisfying the Cauchy-Riemann equations at a point does \emph{not} guarantee the differentiability of $f$, and a counterexample is given on the first example sheet.
\end{remark}

We can easily use this criterion to show that many functions are not complex differentiable.

\begin{example}[Conjugation isn't Differentiable]
    Consider the function $f(z) = \overline{z} = x - iy$. For this, $u(x, y) = x$ and $v(x, y) = -y$, so $u_x = 1$ and $u_y = 0$, $v_x = 0$ and $v_y = -1$. So the Cauchy-Riemann equations do not hold anywhere, and $f$ is not differentiable at any point.
\end{example}

% Because partial derivatives are a little easier to work with than the standard derivative of a function on $\R^2$, we can use a straightforward corollary of the Cauchy-Riemann equations for checking complex differentiability.

% \begin{corollary}
    
% \end{corollary}

\end{document}
