\documentclass[a4paper]{scrartcl}

\usepackage[
    fancytheorems, 
    fancyproofs, 
    noindent, 
]{adam}


\title{Analysis and Topology}
\author{Adam Kelly (\texttt{ak2316@cam.ac.uk})}
\date{\today}

\allowdisplaybreaks

\begin{document}

\maketitle

This course is a second course in Analysis and a first course in Topology. 
We will study both concrete results over $\R$ and $\C$ concerning uniform convergence and continuity, and we will also move to more abstract settings to discuss metric and topological spaces, completeness, connectedness and compactness.

This article constitutes my notes for the `Analysis and Topology' course, held in Michaelmas 2021 at Cambridge. These notes are \emph{not a transcription of the lectures}, and differ significantly in quite a few areas. Still, all lectured material should be covered.



\tableofcontents

% \section{Introduction}

% \subsection{Outline of the Course}

% This is a second course in Analysis and a first course in Topology. 

% \begin{enumerate}
%     \item Uniform convergence and uniform continuity
%     \item Metric spaces
%     \item Completeness and the contraction mapping theorem
%     \item Topological spaces
%     \item Connectedness
%     \item Compactness
%     \item Differentiation and the inverse function theorem
% \end{enumerate}

% \subsection{Prerequisites}

% Of course the Analysis I course from Part IA is going to assumed knowledge.

% \subsection{Books}

% Burkill \& Burkill and Sutherland are both good. Sutherland has a good motivation about abstraction.

% \subsection{Example Sheets}

% Plus questions are not exam questions.

\section{Uniform Convergence and Uniform Continuity} 

Recall the notion of convergence for a sequence in $\R$ or $\C$:

\begin{definition}[Convergence of a Sequence]
    A sequence $a_{1}, a_{2}, \cdots \in \mathbb{R}$ is said to \vocab{converge} to the limit $a$ if given any $\epsilon>0$, we can find an integer $N$ such that $\left|a_{n}-a\right|<\epsilon$ for all $n \geq N$. We write $a_{n} \rightarrow a$ as $n \rightarrow \infty$
\end{definition}

That is, given any $\varepsilon$, there is some point in the sequence after which the terms of the sequence are $\varepsilon$ close to $a$. 

Our aim is going to define a similar notion to make sense of $f_n \rightarrow f$, where $f_n$ is a sequence of functions.

% Diagram 1

\begin{definition}[Uniform Convergence]
    A sequence of functions $f_1, f_2, \dots$ with $f_i : S \rightarrow \R$ is said to \vocab{converge uniformly} on $S$ to a function $f:S \rightarrow \R$ if given any $\varepsilon > 0$, we can find an integer $N$ such that $|f_n(x) - f(x)| < \varepsilon$, for any $x \in S$.
\end{definition}

\begin{remark}
    In the above definition, our $N$ can depend only on $\varepsilon$, and must be independent of the particular choice of $x \in S$. This is why we call this \emph{uniform} convergence -- because the property has to hold uniformly across the domain. 
\end{remark}

Equivalently, we could say that for all $\varepsilon > 0$ there's some $N \in \N$ such that for all $n \geq N$ we have $\sup_{x \in S} |f_n(x) - f(x)| < \varepsilon$.

The above definition implies that if we fix some value of $x$ that $f_1(x), f_2(x), \dots$ converges to $f(x)$. This implies that the function $f$ is unique, due to the uniqueness of limits in $\R$ and $\C$. We sometimes call $f$ the \vocab{uniform limit}.

\begin{definition}[Pointwise Convergence]
We say that $f_n$ \vocab{converges pointwise} on $S$ to $f$ if $f_n(x) \rightarrow f(x)$ for every $x \in S$.
\end{definition}
\begin{remark}
    In this definition our `$N$' can depend on $\varepsilon$ \emph{and} $x$! This makes it a much weaker notion than uniform convergence, and clearly uniform convergence implies pointwise convergence.
\end{remark}

\begin{example}[Checking Uniform Convergence]
    Consider the sequence of functions $f_n(x) = x^2 \cdot e^{-n x}$ for $n \in \N$ and $x \in \R^+$. We want to know if this sequence of functions converges uniformly on this domain.

    Since pointwise convergence is implied by uniform convergence, we can first check the pointwise limit exists and use that to specify $f$ in our definition of uniform convergence.

    Fix $x \geq 0$. Then $x^2 e^{-nx} \rightarrow 0$ as $n \rightarrow 0$. So $f_n$ tends to $0$ (the zero function) pointwise on $\R^+$. We can now check if $f_n$ converges uniformly to the zero function.

    We can attempt to compute the quantity
    $$
    \sup_{x \geq 0} |f_n(x) - 0| =  \sup_{x \geq 0} f_n(x).
    $$
    One approach would be to differentiate it, which would need some care. A better way is to find an upper bound on $|f_n(x) - f(x)|$ which does not depend on $x$. In this case we can bound
    $$
    0 \leq x^2 e^{-nx} = \frac{x^2}{1 + nx + n^2x^2/2 + \cdots} \leq \frac{2}{n^2},
    $$
    for all $x \geq 0$. Thus $\sup_{x \geq 0} f_n(x) \leq 2/n^2 \rightarrow 0$ as $n \rightarrow \infty$. So does indeed $f_n \rightarrow 0$ uniformly on $\R^+$.
\end{example}

\begin{example}[Showing Uniform Convergence Doesn't Hold]
    Consider the sequence of functions $f_n(x) = x^n$ for $n \in \N$, over the domain $[0, 1]$.

    We can compute the limit as
    $$
    x^n \rightarrow f(x) = \begin{cases}
        0 &\mbox{if } 0 \leq x < 1, \\
        1 &\mbox{if } x = 1.
       \end{cases}
    $$
    This implies that $\sup_{x \in [0, 1]} |f_n(x) - f(x)| = 1$. So this doesn't tend to zero, and thus the sequence of functions $f_n$ converges pointwise but not uniformly.

    Alternatively, we could compute $\sup_{x \in [0, 1]} f_n(x) \geq f_n((1/2)^{1/n}) = 1/2$, which shows that $f_n$ does not converge uniformly.
\end{example}

\begin{remark}
    The statement ``$f_n \not \rightarrow f$ uniformly on the domain $S$'' means there exists some $\varepsilon$ such that for all $N \in \N$, there's some $n \geq N$ and $x \in S$ such that $|f_n(x) - f(x)| \geq \varepsilon$. In many cases when thinking about this, it's almost easier just to negate the statement symbolically.
\end{remark}

We will now see that the uniform limit function retains certain properties from the original sequence, namely that the uniform limit of continuous functions is continuous.

\begin{theorem}[Continuity of the Uniform Limit]
    Let $S \subseteq \R$ or $\C$. Given a sequence of functions $f_n : S \rightarrow \R$ (or $\C$) and $n \in \N$. If $f_n$ is continuous for all $n$, and $f_n \rightarrow f$ uniformly on $S$, then $f$ is continuous.
\end{theorem}
\begin{proof}
    To be completed in the next lecture
\end{proof}

Informally, the idea of the proof is to pick an $a \in S$. We want $x \approx a$ to imply $f(x) \approx f(a)$. We choose $n$ such that $f_n \approx f$ everywhere. Then as $f_n$ is continuous, $x \approx a$ implies $f_n(x) \approx f_n(a)$, so $f(x) \approx f_n(x) \approx f_n(a) \approx f(a)$.


\end{document}
