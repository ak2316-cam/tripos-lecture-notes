%  \documentclass[DIV=12, a4]{scrartcl}
%\documentclass[12pt, a5]{scrartcl}

% \documentclass[a4paper]{report}
% \usepackage[
% % fancytheorems, 
% noindent, 
% %spacingfix, 
% %noheader
% ]{vanilla}


\documentclass[a4paper]{scrreprt}
\usepackage[
fancytheorems, 
noindent, 
% %spacingfix, 
% %noheader,
fancyproofs
]{adam} 

\usepackage{tikz}

% \usepackage{subfig}

\setcounter{chapter}{-1}

\title{Analysis I}
% \subtitle{Adam Kelly}
\author{Adam Kelly}
% \date{Michaelmas 2020}
\date{\today}

\begin{document}

\maketitle

\begin{abstract}
	
	% \vspace{2\baselineskip}
	% {\color{red} None of the notes here have been reviewed at all, and are just exactly what was taken down live in the lectures. I would turn around now and come back in a few days, when I have gone back, cleaned things up, fixed explanations and added some structure.}
	% \vspace{5\baselineskip}

	This set of notes is a work-in-progress account of the course `Numbers and Sets', originally lectured by Professor Gabriel Paternain in Lent 2020 at Cambridge. These notes are not a transcription of the lectures, but they do roughly follow what was lectured (in content and in structure).

	These notes are my own view of what was taught, and should be somewhat of a superset of what was actually taught. I frequently provide different explanations, proofs, examples, and so on in areas where I feel they are helpful. Because of this, this work is likely to contain errors, which you may assume are my own. If you spot any or have any other feedback, I can be contacted at \href{mailto:ak2316@cam.ac.uk}{ak2316@cam.ac.uk}.


	% {\color{red} Notes written upto lecture 6.}
	% During the creation of this document, I consulted a number of other books and resources. All of these are listed in the bibliography. 

\end{abstract}

\tableofcontents

\clearpage

\chapter{Introduction}



Recommended books
\begin{itemize}
	\item Burkill A first Course in mathematical Analysis
	\item Spival calculus
	\item Tao Analysis I
\end{itemize}


\chapter{Limits and Convergence}

We will review some content from numbers and sets.

We will use the notation $a_n$, $(a_n)_{n = 1}^{\infty}$ and $a_n \in \R$ for sequences.

\begin{definition}
	$a_n \rightarrow a$ as $n \rightarrow \infty$ if given $\epsilon > 0$, there exists $N$ such that $|a_n - a| < \epsilon$ for all $n \geq N$.
\end{definition}

Note $N$ may depend on $\epsilon$.

We define monotone increasing and decreasing (and strinctly increasing etc) sequences. 

The fundamental axiom of the real numbers

If $a_n \in \R$, $\forall n \geq 1$, $A \in \R$, and $a_1 \leq a_2 \leq a_3 \leq \cdots$, $a_n \leq A$ for all $n$, there exists $a \in \R$ such that $a_n \rightarrow a$ as $n \rightarrow \infty$.

Equivalently, an increasing sequence of reals bounded above converges, and a decreasing sequence of reals bounded below converges.

Also equivalent the least upper bound axiom, every non empty set of real numbers bounded above has a supermum.

Recall the definition of a supremum. Let $S \subset \R$ with $S \neq \emptyset$. Then $\sup S = K$ if
\begin{enumerate}[label=(\roman*)]
	\item $x \leq K$ for all $x \in S$.
	\item Given $\epsilon > 0$, there exists $x \in S$ such that $x > K - \epsilon$.
\end{enumerate}

Note that the supremum is unique, and we can talk about the infimum in the same way.

To get used to using these properties, we will prove some lemmas about limits.

\begin{enumerate}
	\item The limit is unique. If $a_n \rightarrow a$ and $b_n \rightarrow b$, then $a = b$.
	\item If $a_n \rightarrow a$ as $n \rightarrow \infty$ and $n_1 < n_2 < n_3 < \cdots$, then $a_{n_j} \rightarrow a$ as $j \rightarrow \infty$ [subsequences converge to the same limit].
	\item If $a_n = c$ for all $n$, then $a_n \rightarrow c$ as $n \rightarrow \infty$.
	\item If $a_n \rightarrow a$ and $b_n \rightarrow b$, then $a_n + b_n \rightarrow a + b$.
	\item If $a_n \rightarrow a$ and $b_n \rightarrow b$ then $a_n b_n \rightarrow ab$.
	\item If $a_n \rightarrow a$, $a_n \neq 0$ for all $n$, then $1/a_n \rightarrow 1/a$.
	\item If $a_n \leq A$ for all $n$ and $a_n \rightarrow a$, then $a \leq A$.
\end{enumerate}

We will prove 1, 2, and 5

\emph{Proof 1}. Given $\epsilon > 0$, there exists $n_1$ such that $|a_n - a| < \epsilon$ for all $n \geq n_1$ and $n_2$ such that $|a_n - b| < \epsilon$ for all $n \geq n_2$.

Then let $N = \max{n_1, n_2}$. Then $\forall n \geq N$, $|a - b| \leq |a_n - a| + |a_n - b| < 2 \epsilon$.

If $a \neq b$, take $\epsilon = |a - b| /3$, which gives $|a - b| < \frac{2}{3}|a - b|$ which is absurd.

\emph{Proof 2.} Given $\epsilon > 0$, there exists $N$ such that $|a_n - a| < \epsilon$ for all $n \geq N$. Since $n_j \geq j$, we have $|a_{n_j} - a| < \epsilon$ for all $j \geq N$, that is, $a_{n_j} \rightarrow a$ as $j \rightarrow \infty$.

\emph{Proof 5}. Note that
$$
|a_n b_n - ab| \leq |a_n b_n - a_n b| + |a_n b - ab| = |a_n||b_n - b| + |b||a_n - a|
$$
by the triangle inequality.

Now $a_n \rightarrow a$, then given $\epsilon > 0$, there exists $N_1$ such that $|a_n - a| < \epsilon$ for all $n \geq N_1$.

And $b_n \rightarrow b$, so there exists $N_2$ such that $|b_n - b| < \epsilon$ for all $n \geq N_2$.

If $n \geq N_1(1)$, $|a_n - a| < 1$, so $|a_n| \leq |a| + 1$.
Thus $|a_n b_n - ab| \leq \epsilon(|a| + 1 | b|)$ for all $n \geq N_3(\epsilon) = \max{N_1(1), N_1(\epsilon), N_2(\epsilon)}$.


Another lemma. $1/n \rightarrow 0$ as $n \rightarrow \infty$.

Proof. $1/n$ is a decreasing sequence bounded below, and by the fundamental axiom it has a limit $a$. We claim $a = 0$. We have
$1/2n = 1/2 \times 1/n \rightarrow a/2$ by the previous lemma (v), but $1/2n$ is a subsequence, so by lemma 1.1 (ii), $1/2a \rightarrow a$. By uniquness of limits, we get $a = a/2$, so $a = 0$.


\textbf{Remark}. The definition of a limit of a sequence makes perfect sense for $a_n \in \C$.

We define $a_n \rightarrow a$ if given $\epsilon > 0$, there exists $N$ such that $\forall n \geq N$
, $|a_n - a| < \epsilon$.

We can think of this as the sequence $a_n$ being in some disk at $a$ of radius $\epsilon$ for all $n \geq N$.

The first six parts of our first lemma work on $\C$, but the last one doesn't make sense because it uses the order of $\R$.


\end{document}
