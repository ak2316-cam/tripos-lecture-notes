\documentclass[11pt, a4paper]{article}


%\usepackage[noasy]{evan}


\usepackage{amsmath}
\usepackage{amssymb}
\usepackage{amsthm}
\usepackage{enumitem}
\usepackage{parskip}

%\usepackage{geometry}

\newtheorem{theorem}{Theorem}[section]
\newtheorem{lemma}[theorem]{Lemma}
\newtheorem{proposition}[theorem]{Proposition}

\theoremstyle{definition}

\newtheorem{definition}[theorem]{Definition}
\newtheorem{example}[theorem]{Example}

\newcommand{\vocab}[1]{\emph{#1}}

\title{\vspace{-2cm}Groups}
\author{Adam Kelly}
\date{\today}

\begin{document}

\maketitle

\section{What is a Group?}

`Groups' is a course which introduces you to the subject of \emph{Abstract Algebra}.
Indeed, while groups are one of the simplest and most basic of all the algebraic structures\footnote{Apart from `magmas' I suppose, but they don't tend to be a particularly useful notion.}, they are immensely useful and appear in almost every area of mathematics. 

\subsection{Definition of a Group}

We will begin our study of the subject by defining formally what a group is.

\begin{definition}
	A \vocab{group} is a set $G$ with a binary operation\footnote{Some texts include an additional \emph{closure} axiom, but this is implied by $*$ being a binary operation on $G$.} $*$ which satisfies the axioms:
	\begin{itemize}
		\item \emph{Identity}. There is an element $e \in G$ such that $g * e = e * g = g$ for every $g \in G$.
		\item \emph{Inverses}. For every element $g \in G$, there is an element $g^{-1} \in G$ such that $g * g^{-1} = g^{-1} * g = e$.
		\item \emph{Associativity}. The operation $*$ is associative.
	\end{itemize}
\end{definition}

We typically refer to a group as defined above by $(G, *)$, which explicitly states that $*$ is the group operation. When the operation being used is clear, we can refer to the group by just $G$. We will also be omitting the group's operation symbol quite often, for example writing $gh = g * h$. 

 In the next section, we will look at some non-trivial examples of groups.

\subsection{Elementary Properties of Groups}

With the notion of a group now defined, we can now consider some basic facts that follow directly from the definition of a group. We will first address whether it is possible for a group to have multiple identity elements, or for an element to have multiple inverses (no).

\begin{proposition}[Uniqueness of the Identity and Inverse]
	Let $(G, *)$ be a group. Then there is a unique identity element, and for every $g \in G$, $g^{-1}$ is unique.
\end{proposition}
\begin{proof}
	To prove that the identity element is unique, let $e$ and $e'$ be identity elements of $G$. Then $e* e' = e$ and $e* e' = e'$ by definition, giving $e = e'$. 
	
	To prove that the inverses are unique, suppose that for some $g, h, k \in G$ we have $g* h = g *k = e$. Then $g^{-1}* g *h = g^{-1} *g * k$, implying $h = k$. The case of $h * g = k * g = e$ follows analogously.
\end{proof}

The next useful fact is the \emph{cancellation law}, whose proof bears a large resemblance to the proof that inverses are unique.

\begin{proposition}[Cancellation Law]
	If $(G, *)$ is a group, and $a, b, c \in G$, then $a*b = a*c$ and $b*a = c*a$ both imply $b = c$.
\end{proposition}
\begin{proof}
	Taking $a * b = a * c$ and left-multiplying by $a^{-1}$ we have $a^{-1} * a * b = a^{-1} * a * c$, that is, $b = c$. The other case follows analogously.
\end{proof}

The last proposition we will prove in this section gives us a useful result about computing inverses.

\begin{proposition}[Computing Inverses]
	Let $(G, *)$ be a group, and let $g, h \in G$. Then the following hold:
	\begin{enumerate}[label=(\roman*)]
		\item $(g*h)^{-1} = h^{-1} * g^{-1}$.
		\item $(g^{-1})^{-1} = g$.
	\end{enumerate}
\end{proposition}
\begin{proof}$ $\phantom{\qedhere}
	\begin{enumerate}[label=(\roman*)]
		\item We have $(g*h) * (h^{-1} * g^{-1}) =  g * (h * h^{-1}) * g^{-1} = g * g^{-1} = e$, so $(g*h)^{-1} = h^{-1} * g^{-1}$.
		\item Similarly, $g^{-1} * g = e$, so $(g^{-1})^{-1} = g$. \hfill \qedsymbol
	\end{enumerate}
\end{proof}

%\begin{proposition}
%	Let $(G, *)$ be a group, and let $g, h \in G$.
%	\begin{enumerate}[label=(\roman*)]
%		\item The identity element is unique.
%		\item The inverse of $g$ is unique.
%		\item $(g*h)^{-1} = h^{-1} * g^{-1}$.
%		\item $(g^{-1})^{-1} = g$.
%	\end{enumerate}
%\end{proposition}
%\begin{proof}$ $
%	\begin{enumerate}[label=(\roman*)]
%		\item Let $e$ and $e'$ be identity elements of $G$. Then $e*e' = e$, and also $e * e' = e'$ by definition, so $e' = e$.
%		\item Suppose for some $h, k \in G$, $g * h = g * k = e$. Then
%		$g^{-1} * g * h = g^{-1} * g * k \implies h = k$.
%		\item We have $(g*h) * (h^{-1} * g^{-1}) =  g * (h * h^{-1}) * g^{-1} = g * g^{-1} = e$, so $(g*h)^{-1} = h^{-1} * g^{-1}$.
%		\item Similarly, $g^{-1} * g = e$, so $(g^{-1})^{-1} = g$.
%	\end{enumerate}
%\end{proof}
%
%Another important rule (important enough that we shall call it a theorem) is the \emph{cancellation law}\footnote{It is worth noting that we cannot (in general) cancel when we have $a * b = c * a$.}.

\begin{theorem}[Cancellation Law]
	If $(G, *)$ is a group, and $a, b, c \in G$, then $a*b = a*c$ and $b*a = c*a$ both imply $b = c$.
\end{theorem}


It is worth noting though that we cannot (in general) cancel $a * b = c * a$.

With the notion of a group now defined, we can consider some non-trivial examples of groups.

\begin{example}[Non-Examples of Groups]$ $
	\begin{itemize}
		\item The pair $(\mathbb{Q}, \cdot)$ is \emph{not} a group. The element $0 \in \mathbb{Q}$ does not have an inverse.
	\end{itemize}
\end{example}

\begin{example}[Examples of Groups]
	The following are all groups.
	\begin{enumerate}
		\item The additive group of integers, $(\mathbb{Z}, +)$.
		\item $(Z, +)$, $(\mathbb{Q}, +)$, $(\mathbb{R}, +)$ and $(\mathbb{C}, +)$
	\end{enumerate}
\end{example}

\end{document}
