\documentclass[a4paper]{scrartcl}

\usepackage[
    fancytheorems, 
    fancyproofs, 
    noindent, 
]{adam}


\title{Linear Algebra}
\author{Adam Kelly (\texttt{ak2316@cam.ac.uk})}
\date{\today}

\allowdisplaybreaks

\begin{document}

\maketitle

% This is a short description of the course. It should give a little flavour of what the course is about, and what will be roughly covered in the notes.

This article constitutes my notes for the `Linear Algebra' course, held in Michaelmas 2021 at Cambridge. These notes are \emph{not a transcription of the lectures}, and differ significantly in quite a few areas. Still, all lectured material should be covered.



\tableofcontents

\section{Vector Spaces}

\subsection{Vector Spaces and Subspaces}

Let $\F$ be an arbitrary field.

\begin{definition}[Vector Space Over $\F$]
    A \vocab{vector space over $\F$} is an abelian group $(V, +)$ equipped with a function $\F \times V \rightarrow V$, $(\lambda, v) \mapsto \lambda v$ such that 
    \begin{enumerate}[label=(\roman*)]
        \item $\lambda(v_1 + v_2) = \lambda v_1 + \lambda v_2$,
        \item $(\lambda_1 + \lambda_2)v = \lambda_1 v + \lambda_2 v$,
        \item $\lambda(\mu v) = (\lambda \mu) v$,
        \item $1v = v$.
    \end{enumerate} 
\end{definition}

\begin{example}[Examples of Vector Spaces]~
    \vspace{-1.5\baselineskip}
    \begin{enumerate}[label=(\roman*)]
        \item $\F^n$ with $n \in \N$, the set of column vectors of size $n$ with entries in $\F$.
        \item Take any set $X$, and define $\R^X = \{f: X \rightarrow \R\}$, the set of real valued functions on $X$. This is a vector space over $\R$.
        \item $\mathcal{M}_{n, m}$, the set of $n \times m$ matrices with entries in $\F$.
    \end{enumerate}
\end{example}

\begin{remark}
    The axioms of scalar multiplication imply that $0v = 0$, for any $v \in V$.
\end{remark}

\begin{definition}[Subspace]
    Let $V$ be a vector space over $\F$. The subset $U$ of $V$ is a \vocab{vector subspace} of $V$, denoted $U \leq V$, if:
    \begin{enumerate}[label=(\roman*)]
        \item $0 \in U$,
        \item $u_1, u_2 \in U$ implies that $u_1 + u_2 \in U$,
        \item $\lambda \in \F$, $u \in U$ implies that $\lambda u \in U$.
    \end{enumerate}
\end{definition}

Clearly if $V$ is an $\F$ vector space and $U \leq V$, then $U$ is an $\F$ vector space.

\begin{example}[Examples of Subspaces]~
    \vspace{-1.5\baselineskip}
    \begin{enumerate}[label=(\roman*)]
        \item If $V$ is the set of functions $\R \rightarrow \R$, then the set of continuous functions $\mathcal{C}(\R) \leq V$ is a subspace.
        \item The set of vectors
        $$
        \left\{\begin{pmatrix}x_1 \\ x_2 \\ x_3\end{pmatrix} \mid x_1, x_2, x_3 \in \R, x_1 + x_2 + x_3 = t\right\}
        $$
        is a subspace of $\R^3$ for $t = 0$ only.
    \end{enumerate}
\end{example}

\begin{proposition}[Intersecting Subspaces]
    Let $U, W \leq V$. Then $U \cap W \leq V$.
\end{proposition}
\begin{proof}
    Since $0 \in U$ and $0 \in W$, we have $0 \in U \cap W$. Now if $\lambda_1, \lambda_2 \in \F$ and $v_1, v_2 \in U \cap W$, then $\lambda_1 v_1 + \lambda_2 v_2 \in U$ and $V$, and thus is in $U \cap V$. Thus $U \cap W \leq V$.
\end{proof}

The union of two subspaces is generally \emph{not} a subspace, as it is typically not closed by addition. In fact, the union is only ever a subspace if one of the subspaces is contained in the other. 

We can however try to `complete' the union so that it becomes a subspace.

\begin{definition}[Sum of Subspaces]
    Let $V$ be a vector space over $\F$, and let $U, W \leq V$. We define the \vocab{sum} of $U$ and $W$ to be the set
    $$
    U + W = \{u + w \mid u \in U, w \in W \}.
    $$
\end{definition}

This definition immediately forces $U + W \leq V$, and indeed it is the minimal such space (in that any subspace of $V$ containing both $U$ and $W$ must also contain $U + W$).

\subsection{Subspaces and Quotient Spaces}

We now try to blah blah motivate this.

\begin{definition}[Quotient Space]
    Let $V$ a vector space over $\F$, and let $U \leq V$. The \vocab{quotient space} $V/U$ is the abelian group $V/U$ equipped with the scalar multiplication $F \times V/U \rightarrow V/U$, $(\lambda, v + u) \mapsto \lambda v + u$.
\end{definition}

With this definition, we need to check that this scalar multiplication operation makes the quotient space into a well-defined vector space.

\begin{proposition}[Quotient Spaces are Vector Spaces]
    $V/U$ is a vector space over $\F$.
\end{proposition}
\begin{proof}
    TODO: Write this proof.
\end{proof}

\end{document}
