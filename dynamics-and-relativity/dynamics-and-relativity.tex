%  \documentclass[DIV=12, a4]{scrartcl}
%\documentclass[12pt, a5]{scrartcl}

% \documentclass[a4paper]{report}
% \usepackage[
% % fancytheorems, 
% noindent, 
% %spacingfix, 
% %noheader
% ]{vanilla}


\documentclass[a4paper]{scrreprt}
\usepackage[
fancytheorems, 
noindent, 
% %spacingfix, 
% %noheader,
fancyproofs
]{adam} 

\usepackage{tikz}
\usetikzlibrary{calc,patterns,angles,quotes,snakes}


% \usepackage{subfig}

% \setcounter{chapter}{-1}

\title{Dynamics and Relativity}
% \subtitle{Adam Kelly}
\author{Adam Kelly}
% \date{Michaelmas 2020}
\date{\today}

\begin{document}

\maketitle

\begin{abstract}
	
	% \vspace{2\baselineskip}
	% {\color{red} None of the notes here have been reviewed at all, and are just exactly what was taken down live in the lectures. I would turn around now and come back in a few days, when I have gone back, cleaned things up, fixed explanations and added some structure.}
	% \vspace{5\baselineskip}

	This set of notes is a work-in-progress account of the course `Dynamics and Relativity', originally lectured by Prof Peter Haynes in Lent 2020 at Cambridge. These notes are not a transcription of the lectures, but they do roughly follow what was lectured (in content and in structure).

	These notes are my own view of what was taught, and should be somewhat of a superset of what was actually taught. I frequently provide different explanations, proofs, examples, and so on in areas where I feel they are helpful. Because of this, this work is likely to contain errors, which you may assume are my own. If you spot any or have any other feedback, I can be contacted at \href{mailto:ak2316@cam.ac.uk}{ak2316@cam.ac.uk}.


	% {\color{red} Notes written upto lecture 6.}
	% During the creation of this document, I consulted a number of other books and resources. All of these are listed in the bibliography. 

\end{abstract}

\tableofcontents

% \clearpage
% \chapter{Introduction}

% Most of the phenomena in everyday lives involve randomness. What we try to do in probability is model this randomness in a mathematical way. It's likely that you have studied some probability before, but the difference in the treatment here is that we will try to be somewhat more rigerous.

% We will define the notion of a probability space, where `our experiments take place'. Then we will discuss discrete and continuous random variables. 
% In the discrete setting, we will find that there is no real subtleties, and we can be quite rigorous. In the continuous setting however we will have to take some things for granted (but rigour will return in the Part II course).

% \begin{quote}
% 	\emph{``Probability theory has a right and a left hand. On the right is the rigorous foundational work using the tools of measure theory. The left hand `thinks probabilistically,' reduces problems to gambling situations, coin-tossing, motions of a physical particle.''}
% \end{quote}

% In this course, we will need both hands.


\chapter{Newtonian Dynamics -- Basic Concepts}

A central aspect of this course is Newtonian dynamics. In this chapter we will develop some of the ideas and definitions needed to discuss this in detail.


\section{Particles}

When dealing with Newtonian dynamics, we will often use and refer to \emph{particles}, as a way of describing phenomina.

\begin{definition}[Particle]
	A \vocab{particle} is an object of negligible size. It has some mass $m > 0$, and can also have other properties such as (perhaps) an electric charge $q$.
\end{definition}

A particle is completely described by a \vocab{position vector}, usually denoted $\vv{r}(t)$ or $\vv{x}(t)$, with respect to some origin $O$. The cartesian coordinates of $\vv{r}$ are $(x, y, z)$, where
$$
\vv{r} = x \vv{i} + y \vv{j} + z \vv{k},
$$
with $\vv{i}, \vv{j}, \vv{k}$ being orthonormal basis vectors.

\begin{center}
    \begin{tikzpicture}
        \draw [->,>=stealth] (0, 0) -- (0, 1.5) node [anchor=south] {$\vv{j}$};
        \draw [->,>=stealth] (0, 0) -- (1.4625, -0.3) node [anchor=west] {$\vv{i}$};
        \draw [->,>=stealth] (0, 0) -- (-1.0875, -0.75) node [anchor=north east] {$\vv{k}$};

        % \draw [dashed, color=blue] (1, 2) -- (0.7, 1.2);

		\draw [->,>=stealth, color=red] (0, 0) -- (0.85, 1.6) node [anchor=south west] {$\vv{r}(t)$};
		

        % \draw [decoration={brace, raise=0.25cm}, decorate, color=blue] (0,0) -- (0, 1.675) node [pos=0.5, anchor=east, xshift=-0.3cm] {\footnotesize $a_3$}; 
    \end{tikzpicture}
\end{center}

The choice of coordinate axes defines a \vocab{frame of reference} $S$.

Of course, we will be considering particles that are moving, so we will define the velocity, momentum and acceleration of the particle.

\begin{definition}[Velocity]
	The \vocab{velocity} of a particle is
	$
	\vv{u}(t) = \frac{\mathrm{d}}{\mathrm{d}t} \vv{r}(t) = \dot{\vv{r}},
	$
	and is tangent to the path (or trajectory) of the particle.
\end{definition}

\begin{definition}[Momentum]
	The \vocab{momentum} of a particle is
	$
	\vv{p} = m \vv{u} = m \dot{\vv{r}},
	$
	where $m$ is the mass of the particle.
\end{definition}

\begin{definition}[Acceleration]
	The \vocab{acceleration} of the particle is $\dot{\vv{u}} = \ddot{\vv{r}} = \frac{\mathrm{d}^2}{\mathrm{d}t^2} \vv{r}(t)$.
\end{definition}

\section{Newton's Laws of Motion}

We can now write down Newton's three laws of motion, which govern the motion of particles. All of these statements about particles can be extended to finite bodies (which are composed of many particles).

\begin{law*}[Newton's First Law/Galileo's Law of Inertia]
There exist inertial frames of reference in which a particle remains at rest or moves at constant velocity unless it is acted on by a force.
\end{law*}

\begin{law*}[Newton's Second Law]
	In an inertial frame the rate of change of momentum of a particle is equal to the force acting on it.
\end{law*}

\begin{law*}[Newton's Third Law]
To every action there is an equal and opposite reaction. That is, forces excreted between two particles are equal in magnitude and opposite in direction.
\end{law*}

\end{document}
