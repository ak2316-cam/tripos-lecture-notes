%  \documentclass[DIV=12, a4]{scrartcl}
%\documentclass[12pt, a5]{scrartcl}
\documentclass[a4]{scrartcl}

\usepackage[
fancytheorems, 
noindent, 
%spacingfix, 
%noheader
]{adam}

% \usepackage{subfig}

\setcounter{section}{-1}

\title{Numbers and Sets}
% \subtitle{Adam Kelly}
\author{Adam Kelly}
\date{Michaelmas 2020}

\begin{document}

\maketitle

\begin{abstract}
	{\color{red} None of the notes here have been reviewed at all, and are just exactly what was taken down live in the lectures. I would turn around now and come back in a few days, when I have gone back, cleaned things up, fixed explanations and added some structure.}
	\vspace{5\baselineskip}



	This set of notes is a work-in-progress account of the course `Numbers and Sets', originally lectured by Professor Imre Leader in Michaelmas 2020 at Cambridge. These notes are not a transcription of the lectures, but they do roughly follow what was lectured (in content and in structure).

	These notes are my own view of what was taught, and should be somewhat of a superset of what was actually taught. I frequently provide different explanations, proofs, examples, and so on in areas where I feel they are helpful. Because of this, this work is likely to contain errors, which you may assume are my own. If you spot any or have any other feedback, I can be contacted at \href{mailto:ak2316@cam.ac.uk}{ak2316@cam.ac.uk}.

	% During the creation of this document, I consulted a number of other books and resources. All of these are listed in the bibliography. 

\end{abstract}

\tableofcontents

\clearpage
\section{Introduction}

Numbers and sets is one of the first course in pure mathematics that you will take
as an undergraduate at Cambridge. In a sense, it is the `starting course', in that it will introduce you to the `pure maths' way of thinking about things. 
This introduction will happen through the lense of thinking about objects, beginning with the natural and real numbers. You will be introduced to the `thoughtful way' of thinking about such objects, that you can carry through to almost every other course in pure mathematics.

\subsection{Structure of the Course}

This course is divided into four chapters.

\begin{enumerate}
	\item \emph{Elementary Number Theory}
	
	This is a chapter that almost everybody enjoys. 
	We deal with number theory first, which is elementary not in the sense that it is easy but in the sense that it is our `first steps' in the subject. The main aim of this chapter is to get used to the additive and multiplicative structure of the natural numbers.

	It is like that some of you will be familiar with this material already, but nothing in this chapter will be assumed, and everything will be built from the ground up.

	\item \emph{The Reals}
	
	This chapter has a different perspective, centering on the questions of \emph{what is a real number} and \emph{what can we assume about them?} This is one of the harder parts of this course, and many of the definitions contain a subtlety that is not present in other chapters. 

	\item \emph{Sets and Functions}
	
	This is a `terminology' chapter. There is no exciting theorems, mostly notation, definitions, and so on. It is a short chapter, but it is somewhat boring in that sense.

	\item \emph{Countability}
	
	This chapter is best described as `fun with infinite sets'. It is to do with the concepts introduced in Chapter 3 (in the sense that we are thinking about sets and functions), but it has a very different flavour. You will find results in this chapter that are both interesting and surprising. Almost everyone likes this chapter.
\end{enumerate}

Everything in the chapters above makes up the `course'. If you are wondering what is examinable, it will be everything in these lecture notes (unless otherwise stated). For a more formal answer to that question, have a look at \href{https://www.maths.cam.ac.uk/undergrad/files/schedules.pdf}{the schedules}.

\subsection{Books}

As with most mathematics courses in Cambridge, you will not need a textbook to follow this course. What is covered in lectures is enough to do both the example sheets and the examinations for this course. Still, you might find that a textbook can provide a different perspective, additional worked examples, and additional material that you may find informative, helpful or fun.

In particular, the following books are quite relevant/good, but there is no expectation that you will look at these.

\begin{itemize}
	\item R. T. Allenby, \emph{Numbers and Proofs}.
	
	This book is readable, easy to understand and clear.

	\item A. G. Hamilton, \emph{Numbers, Sets and Axioms}.
	
	Another readable and clear book, but with a different flavour to the previous book.

	\item H. Davenport, \emph{The Higher Arithmetic}.

	This book can be thought of as showing `where things go next'. It is very interesting, and goes quite a bit beyond this course. It is worth noting however that this book contains no exercises.
\end{itemize}

You should be able to find all of these books in either your college library or the university library.

\subsection{Example Sheets}

As is normal for a 24 lecture course, there will be 4 example sheets. You should be able to have a good go at the first one after lecture 3 or 4.

\subsection{A Brief Note About These Notes}

In the original lecture course, there was two lectures that (informally) introduced the idea of a proof, along with examples and non-examples of what a proof is. This material has been purposefully excluded, and familiarity with proofs (and common logical notation such as $\forall$, $\exists$, and $\implies$) is assumed. 

If you are interested in reading a brief introduction to proofs, I will direct you to this \href{https://math.berkeley.edu/~hutching/teach/proofs.pdf}{quite readable introduction}.


\clearpage

\section{Elementary Number Theory}

This chapter is looking at the properties of the natural numbers. We will begin by defining exactly what they are, in a way that hopefully matches your own intuition.

\subsection{The Peano Axioms}

Intuitively, the natural numbers $\N$ consist of the list of numbers
$$
1, 1 + 1, 1 + 1 + 1, 1 + 1 + 1 + 1, \dots
$$
In this list, every `number' is distinct from the previous, and then it goes on forever with some vague notion of `$\dots$'.
Let's try and make this precise.

Instead of trying to say what a natural number \emph{is}, we will instead define \emph{how they work}, that is, \emph{what we can assume about them}.
We do this by specifying the axioms that the natural numbers satisfy, that (hopefully) define that structure in a way that matches our intuitive idea of the natural numbers.

\begin{definition}[Peano Axioms]
	The natural numbers, written $\N$, is a set containing an element `$1$', and an operation `+1' that satisfies the following axioms.
	\begin{enumerate}[label=(\roman*)]
		\item For all $n \in N$, $n + 1 \neq 1$.
		\item If $n \neq m$, then $n + 1 \neq m + 1$.
		\item For any property $p(n)$, if it is the case that $p(1)$ is true, and for every $n$ we have $p(n) \implies p(n + 1)$, then $p(n)$ is true for all $n \in \N$. This is the \vocab{induction axiom}.
	\end{enumerate} 
\end{definition}

\begin{remark}
	We will write $2$ for $1 + 1$ and so on.
\end{remark}

With the operation $+1$ defined, we can define the operation $+k$ for any $k \in \N$.

\begin{definition}[Addition]
	We define the operation of \emph{addition} so that
	$$
	n + (k + 1) = (n + k) + 1,
	$$
	for every natural number $k \in \N$.
\end{definition}

This is defined for all natural numbers by induction.
In a similar way, we can define multiplication, powers, order, etc, and we can prove the basic properties that they satisfy. Some of these are listed below.

\begin{proposition}
	For all $a, b \in \N$:
	\begin{enumerate}[label=(\roman*)]
		\item $a + b = b + a$.
		\item $ab = ba$.
		\item $a + (b + c) = (a + b) + c$.
		\item $a(bc) = (ab)c$.
		\item $a(b + c) = ab + ac$.
		\item If $a < b$ then
		\begin{itemize}
			\item $a + c < b + c$.
			\item $ac < bc$.
			\item If $b < c$ then $a < c$.
		\end{itemize}
	\end{enumerate}
\end{proposition}

\begin{remark}
	This is the last time that we'll dump a bunch of statements -- that's not what this class is about.
\end{remark}

% \clearpage

% \section{Proofs: An Informal Discussion}

% What is proof, and why do we need them? That's the question that this section will address (in a rather informal manner).
% Unlike in the rest of these notes, we will freely use concepts (such as the natural numbers) without worrying about how they are defined.

% Let us begin by defining what a proof is.
% \begin{definition}
% 	A \vocab{proof} is a logical argument that establishes a conclusion.
% \end{definition}

% So that's all well and good, but why do we bother proving things? There is two main reasons:
% \begin{enumerate}
% 	\item To be \emph{sure} something is true.
% 	\item To understand \emph{why} something is true.
% \end{enumerate}

% Let's begin by considering the first reason. It is likely that this is the reason you immediately thought of, and indeed it is an immensely important one.
% The history of mathematics is littered with statements that are true up to $n = 1,000$ or $n = 1,000,000$ but fail after that. And of course, even one failure implies that a statement is not true for all numbers.

% \begin{example}[Silly]
% 	All natural numbers $n$ satisfy the inequality $n \leq 1,000,000$.
% \end{example}
% \begin{proof}[Counterexample]
% 	Consider the case of $n = 1,000,001$.
% \end{proof}

% This is, of course, rather contrived but there are plenty more actual (and much more subtle) examples.

% The other reason is about understanding. Understanding \emph{why} something is true not only allows you to convince someone else of it's truth, but also allows you to use the ideas of that proof to prove another result. 

% \subsection{Proofs and Non-Proofs}

% Let's have a look at some claims, alongside examples of proofs (and non-proofs).

% \begin{claim*}
% 	For any positive integer $n$, $n^3 - n$ is always a multiple of 3.
% \end{claim*}
% \begin{proof}
% 	For any positive integer $n$, we have
% 	$$
% 	n^3 - n = (n - 1)(n)(n + 1),
% 	$$
% 	but one of $n-1$, $n$ or $n + 1$ is a multiple of 3, as they are three consecutive integers, thus $n^3 - n$ is a multiple of 3.
% \end{proof}

% \begin{claim*}
% 	For any positive integer $n$, if $n^2$ is even, then $n$ is even.
% \end{claim*}
% \begin{proof}[Non-Proof]
% 	Given a positive integer $n$ that is even, we have $n = 2k$ for some integer $k$. Thus $n^2 = (2k)^2 = 4k^2$, which is even, so $n^2$ is even.
% \end{proof}

% This is nonsense! We have shown the truth of the opposite direction, that is, we wanted to show ``if $A$ then $B$'', but what we ended up showing was ``If $B$ then $A$'' which is a completely different claim! Let's try another one using the same proof method.

% \begin{claim*}
% 	For any positive integer $n$, if $n^2$ is a multiple of 9, then so is $n$.
% \end{claim*}
% \begin{proof}[Non-Proof]
% 	For any positive integer $n$ that is a multiple of 9, we h ave $n = 9k$ for some positive integer $k$, so $n^2 = (9k)^2 = 9\cdot(9k^2)$ which is a multiple of $9$, so $n^2$ is a multiple of 9.
% \end{proof}
% Again, this is rubbish for the same reason. In fact, this claim is false!

% \begin{proof}[Counterexample]
% 	Take $n = 6$.
% \end{proof}

% This is an important point: If we have a claim about `every $n$', but then it fails for some $n$, then the statement is \emph{false}. That is, to show ``If $A$ then $B$'' is false, we only need one example where $A$ is true and $b$ is false. We call this a \vocab{counterexample}.

% Let's return to our other claim (if $n^2$ is even then $n$ is even), and look at a different, correct, proof.

% \begin{proof}
% 	Suppose $n$ is odd. Then $n = 2k + 1$ for some integer $k$. Then $n^2 = (2k + 1)^2 = 4k^2 + 4k + 1$, which is odd. This is a contradiction, so $n$ isn't odd and thus must be even.
% \end{proof}

% Recall why we wanted proofs: to \emph{know} that things are true. From this proof, we can see that the thing that makes this claim true is a property of odd numbers. Namely, that all odd numbers square to odd numbers.

% We also may want to reuse a `clever idea' from a proof. In this example, this was `proof by contradiction'. To show ``If $A$ then $B$'', we showed that there is no case when $A$ is true and $B$ is false.

% Another viewpoint on this is that we tried to show $A \implies B$ ($A$ implies $B$), which is the same as showing that $\text{not } A \implies \text{not } B$.



\end{document}
