\documentclass{article}
\usepackage{graphicx} % Required for inserting images
\usepackage{amsfonts}
\usepackage{tikz-cd}

\title{Galois tricks}
\author{Christmos}
\date{Judgement Day}

\begin{document}

\maketitle


\section{Counter-examples}

\begin{itemize}
    \item Inseparable irreducible polynomial: take $K = \mathbb{F}_{p}(t)$ and $L = \mathbb{F}_p(t^{1/p})$ and consider $X^p-t \in K[X]$. Irreducible by Eisenstein and Gauss' lemma (since $t$ is prime in $\mathbb{F}_p[t]$).
    \item Non-primitive finite extension: take $K = \mathbb{F}_{p}(X,Y)$ and $L = \mathbb{F}_p(X^{1/p},Y^{1/p})$. Then every $x \in L$ is a root of $T^p-x^p \in K[T]$, so any primitive extension has degree at most $p$, while $[L:K] = p^2$.
    \item Finite extension $K \subseteq L$ such that $a,b \in L\symbol{92}K$ where $a$ separable over $K$ but $b$ isn't: Take $L=\mathbb{F}_3(a), K=\mathbb{F}_3(a^6), E=\mathbb{F}_3(a^3)$. Take $b=a^3$ Then $[L:K]=6$ so min poly of $a$ is $T^6-a^6$ but $b$ is separable.
        \item Two cycle and p cycle generate $S_p$
        \item If its a quintic almost always mod p
        \item polys of the form $T^p-T+a$ mod p see https://math.stackexchange.com/questions/81583/how-do-i-prove-that-xp-xa-is-irreducible-in-a-field-with-p-elements-when and sheet q ES4 q2 ES3 q13 
        \item know generators of $D_{2n}$
        \item can someone remember how j wilson did Q10 (ii) ES4 and summarise here ty
        \item Finite fields are $F_p$ isomorphic so all polys of degree d split in the finite field of ordr $p^d$
        \item if u know thw GG over $F_p$ and the splitting field has degree with factor r, u get the GG over $F_p^r$ for free.
        \item lets write down solutions to separbility sheet q
        \item ES2 Q6 https://math.stackexchange.com/questions/1314208/perfect-field-of-characteristic-p
        \item ES2 Q7 if insperable the min poly of $x^p$ has strictly less degree than that of $x$ so $K(x) \neq k(x^p)$. If $K(x) \neq k(x^p)$ it is ez to show $x$ is inseparable of $k(x^p)$ and so it is inseprable over $k$ as the min polys divide. From this, by the tower law, we deduce that p divides the index of a inseperable extension.
        \item ES2 Q8 Counting embeddings Lemma
        \item ES2 Q9 Consider min poly 
        \item Irreducible quadratic mod 2 is only $T^2+T+1$. Mod 3 it's $T^2+1, T^2+T-1, T^2-T-1$.
        \item $\Phi_{np}=\frac{\Phi_n(x^p)}{\Phi_n(x)}$
        \item ES3 Q14 (i) important.Show that an irreducible polynomial $f \in Fq[X]$ of degree d divides $G=X^{q^n} - X $ if and only if d divides n. The splitting field of f is $F_{q^d}$ and so each element of the splitting field and in particular the roots of f (separable) satisfy G so $f|G$. Now for the other direction if the roots of f divide G then $F_{q^d}$ is a subfield of $F_{q^n}$ and apply tower law.
        \item Cyclotomic: Automorphism of a cyclic group of order $n$ correspond to unit group $(Z/nZ) ^x$ by considering image of generator. We can view (injectively) automorphism of a cyclotomic extension by consicering theor action on the roots of unity (or a primitive one) which is a cyclic group hence we get an injection into $(Z/nZ) ^x$. This is surjective iff there is a single orbit of all primitive roots of unity because these are exactlly the coprime powers of a generate. (def of primitive is that it has order n). Define cyclo poly. Proof its irreducible over rationals. For prime charisteristuc it will be the orbit of the primitive root under the frob map which is the (multiplicative) order of p in $(Z/nZ) ^x$. 
        \item Kummer: Proof of linear independence of characters. Classification of kummer extensions by looking at what each element rotates the root by. As the galois group is cyclic the order of the GG is the least m st all elements raised to m are the identity. $T^n-a$ reducible iff $a$ a dth power for $d|n$. If $a=y^d$ as we have a primitive dth root of unity we can factor $T^d-y^d$ hence we can factor $(X^m)^d-y^d$. Also $f$ is irreducible iff the GG group above is transitive which happems iff $n=m$. Converse to kummer If Cyclic and root of unit => Kummer (LI of C + Lagrange resolvent produces eigenvalue).

        \item  Trace and norm. Rememeber $P= \sigma_i(x_j))$ to use as a change of basis matric. Also note that ES2 QT gives us trace K(x)/K where $x^p \in k$ is degenerate. Use composition of trace to generalise this degerenacy to all insperabable ectensions. Seperable ectensions have non degernetate trace 
        by linear independence of characters.
        \item A field K is algebraically closed if every non-constant polynomial over K splits into linear factors over K. A field extension L/K is called an algebraic closure of K if it is algebraic and L is algebraically closed. Countable case, enumarte polynomials and union splitting field i.e. inductively define Li to be a splitting field for fi over $L_{i-1}$
        \item any algberaic extension embeds into an algebraic closure. Use a standard zorn arument with poset (extension, embedding). Maximal ideal standard zorn. Existence of alg closure seems too long to come up.
        \item Cubic: Splits trivial GG, Quadratic C2, Irreducible then check disc. disc $= -4p^3-27q^2$ note both negative and the square integer goes with the cube and vice versa
        \item Quartics: Insert transitive subgroup diagram.
        \item Quartics: WLOG depress the cubic. Let $y_{12}=x_1+x_2$ now (as cubic depressed) there is a seperable (check ) cubic with roots $y_{12}^2$,$y_{13}^2$,$y_{14}^2$. And we compute it to be $T^3 +2aT^2 + (a^2 -4c)T - b^2$. If disc not a square and resolvent irreducible tyehn we have a transitive sungroup not in $A_4$ with order divisible by $3$ so we have $S_4$.If it is a square and irre is is $A_4$ by similar reasoning. If reducible and not a square $D_4 $ or $C_4$ as not in $A_4$ and $S_4$ by construction acts tranditively on the roots of the resolvent cubic. Similar for last case $V$
        \item Transitive subgroup lattice for $S_4$\begin{center}\[ \begin{tikzcd}
	& {S_4} \\
	{A_4} && {D_8} \\
	& V & {C_4}
	\arrow[no head, from=3-2, to=2-1]
	\arrow[no head, from=3-2, to=2-3]
	\arrow[no head, from=1-2, to=2-1]
	\arrow[no head, from=1-2, to=2-3]
	\arrow[no head, from=2-3, to=3-3]
\end{tikzcd}\]
\end{center}
        \item Transitive subgroup lattice for $S_5$
        \begin{center}\[\centering \begin{tikzcd}
	& {S_5} \\
	{A_5} && {\langle(12345),(2354)\rangle} \\
	& {D_{10}} \\
	& {C_5}
	\arrow[no head, from=1-2, to=2-1]
	\arrow[no head, from=1-2, to=2-3]
	\arrow[no head, from=2-3, to=3-2]
	\arrow[no head, from=3-2, to=2-1]
	\arrow[no head, from=3-2, to=4-2]
\end{tikzcd}\]\end{center}

\item Artin proof. Prove each element satisfies the poly with roots the orbit thus biunding the degree and showuing sperable ectension. Now take a element with max degree it must  generate everything ow primitive element thm here means we can find a generator with higher degree. (not really sure why we dont just assume primitive element thm in the first place, actually its bc we dont know its finite) Now apply galois corresponse,
\item Fixed field of rational functions is rational functions in the fixed field.Gausses lemma shows they are multiplied by some unit. Order of G in n! so applying n times then the unit is a nth root of unity then $fg^{n!-1}$ and $g^{n!}$ invariant as polys so symmetric sp doen.
\end{itemize}

\end{document}
