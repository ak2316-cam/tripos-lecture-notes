\documentclass[11pt]{article}

\usepackage[
% fancytheorems, 
% fancyproofs,
noindent, 
nokoma
%spacingfix, 
]{adam}

% \usepackage{ulem}
% \DeclareRobustCommand{\hsout}[1]{\texorpdfstring{\sout{#1}}{#1}}

\usepackage{cancel}

% \usepackage{titling}
% \setlength{\droptitle}{-4em}


\title{\vspace{-3\baselineskip}\ \\Recognizing Direct Products}
\author{Adam Kelly (\texttt{ak2316@cam.ac.uk})}
\date{\today}

\begin{document}

\maketitle


The \vocab{direct product} of two groups $G, H$ is the set $G \times H$, along with the operation of component wise composition,
$$
(g_1, h_1) * (g_2, h_2) = (g_1 *_G g_2, h_1 *_H h_2).
$$

When a group can be written as a direct product, it becomes easier to reason about as we can (in many cases) study the groups used in the direct product rather than the group as a whole.

In this article, we will look at some ways of recognizing when a group can be written as some non-trivial direct product. We will look at some concrete examples\footnote{Spoiler warning: some of these examples relate heavily to example sheet content. This is because these questions happen to be pretty good (and quite standard) examples of working with direct products. For other sources, email me.} (and non-examples) and some general techniques.

\section{The Direct Product Theorem}

The following well known theorem (known as the direct product theorem, amongst other names) gives one way of recognizing when a group is isomorphic to the direct product its subgroup.

\begin{theorem}[Direct Product Theorem]
	Let $H, K \leq G$ such that
	\begin{enumerate}[label=(\roman*)]
		\item $H \cap K = \{e\}$
		\item $\forall h \in H$ and $k \in K$, we have $hk = kh$
		\item $\forall g \in G$, there exists $h \in H$, $k \in K$ such that $g = hk$
	\end{enumerate}
	then $G \cong H \times K$.
\end{theorem}
\begin{proof}
	Consider the function $\phi: H \times K \rightarrow G$ with $\phi(h, k) = hk$. This is clearly a homomorphism, so we wish to show that it is bijective (and hence an isomorphism). By (iii), $\phi$ must be surjective and also if $(h, k) \in \operatorname{ker}(\phi)$, then $hk = e$, and $k = h^{-1}$. Thus $h, k \in H \cap K$, and thus $\ker(\phi) = \{e\}$. Thus $\phi$ is also injective, and hence $G \cong H \times K$.
\end{proof}

Let's have a look at how this can be used to recognize some direct products.

\begin{example}
	Is $O(3) \cong SO(3) \times C_2$?
\end{example}
\begin{proof}[Solution]
	I claim that the answer is yes. 
	Consider the subgroup $H = \{I, -I\}$ of $O(3)$, which is isomorphic to $C_2$. We will show that $O(3) \cong SO(3) \times H \cong SO(3) \times C_2$ using the direct product theorem.
	\begin{itemize}
		\item If a matrix $M \in O(3)$, then $\det(M) = \pm 1$, and is equal 1 iff $M \in SO(3)$. But $\det(-I) = -1$, thus $H \cap SO(3) = \{I \}$.
		\item For any matrix $M$, $-I \times M = M \times -I$.
		\item For any matrix $M \in O(3)$, we can write $M = (\pm I) M'$ where $M' \in SO(3)$ (where we use $-I$ if $M \not \in SO(3)$).
	\end{itemize}
	Thus $O(3) \cong SO(3) \times H$ by the direct product theorem, and $O(3) \cong SO(3) \times C_2$ as required.
\end{proof}

\begin{example}\label{ex:dihedral}
	Prove that $D_{4n} \cong D_{2n} \times C_2$, where $n$ is an odd number.
\end{example}
\begin{proof}[Solution]
	Consider the subgroups $H = \langle r^2, s \rangle$ and $K = \langle r^n \rangle$. We have that $H \cong D_{2n}$ and $K \cong C_2$. I claim that $D_{4n} \cong H \times K$. We will show this by the direct product theorem.
	\begin{itemize}
		\item We have that $r^k \in H$ if and only if $2 \mid k$, and thus as $n$ is odd, $r^n \not \in H$, and thus $H \cap K = \{e \}$.
		\item If $h = r^m$ for some $m$, then clearly $r^n r^m = r^m r^n$. Otherwise, we can write $h = r^m s$ for some $m$. Then $r^m s r^n = r^m r^{-n} s = r^{-n} r^m s = r^n r^m s$. Thus $r^n h = h r^n$ for all $h \in H$, and $hk = kh$ for all $k \in K$, $h \in H$.
		\item If $g \in D_{4n}$, then $g = r^m s^t$ for some $m$ and $t$.
		If $m$ is even, then $g \in H$. Otherwise, we can write $g = r^{2m' + 1}s^t$ for some $m'$. Rewriting this,
		\begin{align*}
			g &= r r^{2m'} s^t \\
			  &= r^n r^{-(n - 1)} r^{2m'} s^t  \\
			  &= r^n (r^2)^{m' - \frac{n - 1}{2}} s^t,
		\end{align*}
		where $r^n \in K$ and $(r^2)^{m' - \frac{n - 1}{2}} s^t \in H$. 
		Thus any $g \in D_{4n}$ can be written as $g = hk$ where $h \in H$ and $k \in K$.
	\end{itemize}
	So by the direct product theorem, $D_{4n} \cong H \times K$, and thus $D_{4n} \cong D_{2n} \times C_2$.
\end{proof}

\begin{example}
	Is $\mathbb{Q}^\times$, the group of non-zero rationals with multiplication, the direct product of two non-trivial groups?
\end{example}
\begin{proof}[Solution]
	Let $H = \{q \in \mathbb{Q}^\times \mid q > 0 \}$ and $K = \{1, -1\}$. I claim that $\mathbb{Q}^\times \cong H \times K$. 
	
	First note that $H, K \leq \mathbb{Q}^\times$. Now we will use the direct product theorem. We have $H \cap K = \{1 \}$, and as $\mathbb{Q}^\times$ is abelian, we have $hk = kh$ for all $h \in H$ and $k \in K$. Finally, if $q \in \mathbb{Q}^\times$, we can write $q = \pm 1 \cdot |q|$ where $\pm 1 \in K$ and $|q| \in H$. Thus by the direct product theorem, $\mathbb{Q}^\times \cong H \times K$ as required.
\end{proof}

% \section{\hsout{Seven} Two Deadly Sins}
\section{\xcancel{Seven} Two Deadly Sins}

There are two common mistakes that are made when trying to recognize a direct product using the direct product theorem. Both of them basically boil down to the following:

\begin{center}
	\emph{The direct product theorem is not `if and only if'.}
\end{center}

In short -- just because $H \times K$ isn't isomorphic to $G$ by the direct product theorem doesn't mean that it's not isomorphic at all.
Let's dig into this a bit more.

\subsection{Sin One -- Forgetting Other Isomorphisms}

The direct product theorem takes a group $G$ and two subgroups $H, K \leq G$, forms the direct product $H \times K$, and then tries to use a \emph{specific} isomorphism $\phi(h, k) = hk$ to see if $H \times K \cong G$. A common sin is forgetting that there can be \emph{other isomorphisms} from $H \times K \rightarrow G$. 

In the next example, we will see a solution that is incorrect for precisely this reason.

\begin{example}
	Let $G = \{(g_1, g_2, g_3, \dots) \mid g_i \in \{0, 1\},\  i \in \mathbb{N}\}$ be a group with the binary operation of component-wise addition modulo 2. Let $H = \{g \in G \mid g_1 = 0 \}$ and $K = \{g \in G \mid g_1 = g_2 = 0 \}$. Is $G \cong H \times K$?
\end{example}
\begin{proof}[{\color{Maroon}Wrong Solution}]
	Clearly $(0, 0, 1, 1, 1, \dots) \in H, K$, so $H \cap K \neq \{e \}$. Also $(1, 0, 0, \dots) \in G$ but cannot be written in the form $h+k$ for $h \in H, k \in K$. Thus $H \times K \not \cong G$ by the direct product theorem.
\end{proof}

Okay so that's incorrect, but why? Well all we have shown is that the specific isomorphism used in the direct product theorem proof fails... we forgot that there can be other isomorphisms! 

\begin{proof}[{\color{ForestGreen}Correct Solution}]
	I claim that $H \times K \cong G$. Consider the function $\phi: H \times K \rightarrow G$, with $\phi(h, k) = (h_2, k_3, h_3, k_4, h_4, k_5, \dots)$. This is clearly a homomorphism, and is clearly bijective, so it is an isomorphism and $H \times K \cong G$.
\end{proof}


\begin{example}
	Is $O_4 \cong SO_4 \times C_2$?
\end{example}
\begin{proof}[{\color{Maroon}Wrong Solution}]
	I claim the answer is no. Note that $C_2 \cong \{I , -I \}$, but then $\det(-I) = 1$, and thus $SO_4 \cap \{I , -I \} \neq \{e\}$, and $O_4 \not \cong SO_4 \times C_2$ by the direct product theorem.
\end{proof}

Here the answer is correct but just like before, the proof is completely wrong. In fact, there's even more reasons why it's wrong. First of all, we only consider the subgroup $C_2 \cong \{I, -I \}$. This isn't the only subgroup of $O_4$ that's isomorphic to $C_2$, so we have only shown that \emph{this specific subgroup} doesn't work. For example, if we used the subgroup
$$
H =  \left \{ \begin{pmatrix}
	1 & 0 & 0 & 0 \\
	0 & 1 & 0 & 0 \\
	0 & 0 & 1 & 0 \\
	0 & 0 & 0 & 1
\end{pmatrix}, \begin{pmatrix}
	1 & 0 & 0 & 0 \\
	0 & 1 & 0 & 0 \\
	0 & 0 & 1 & 0 \\
	0 & 0 & 0 & -1
\end{pmatrix}\right\},$$
then $SO_4 \cap H = \{e\}$... but then other parts of the direct product theorem will fail. In reality, the core issue (and why this can't be patched up) is that we have only tried \emph{one specific isomorphism}, just like in the last example.

If you haven't already, I encourage you to think about how to solve this problem correctly.

\subsection{Sin Two -- Forgetting Other Subgroups}

The previous example hinted at another deadly sin: forgetting about other subgroups.

Let's say we had some group $G$, and we wanted to try show that $G \cong H \times K$. It might be tempting to try and see if some subgroups $H', K' \leq G$ with $H' \cong H$ and $K' \cong K$ satisfy the direct product theorem. If you do this, and it doesn't work, you may have been led astray

Let's jump into a slightly contrived but hopefully instructive example.

\begin{example}[Contrived]
	Let $G = \{e, g, h, gh \}$ be a group such that $g^2 = h^2$ and $gh = hg$. Is $G \cong C_2 \times C_2$?
\end{example}
\begin{proof}[{\color{Maroon}Wrong Solution}]
	I claim that the answer is {\color{Maroon} no}. Let $H = \{e, gh \}$, and $K = \{e, gh\}$ Then $H, K \cong C_2$. But $H \cap K \neq \{e\}$ and $g$ cannot be written $g = hk$ for $h \in H$, $k \in K$. Thus by the direct product theorem, $G \not \cong C_2 \times C_2$.   
\end{proof}

The problem with the proof above (apart from the previous sin about how we have only considered one isomorphism) is that we just picked some arbitrary isomorphic subgroup and stuck with it. In reality, using the direct product theorem may require you to use `well chosen' subgroups.

\begin{proof}[{\color{ForestGreen}Correct Solution}]
	Let $H = \{e, h\}$ and $K = \{e, g\}$. Then $H, K \cong C_2$. Also $H \cap K = \{e\}$, $hk = kh$ for all $h \in H, k \in K$, and any element in $G$ can be written as the product of elements in $H$ and $K$. Thus by the direct product theorem, we have $G \cong H \times K \cong C_2 \times C_2$.
\end{proof}

To see the same sort of incorrectness in action, consider a special case of Example 1.3.

\begin{example}
	Is $D_{12} \cong D_6 \times C_2$?
\end{example}
\begin{proof}[{\color{Maroon}Wrong Solution}]
	I claim that the answer is {\color{Maroon} no}. If $H = D_6$ and $K = \langle s \rangle \cong C_2$, then $H \cap K \neq \{e \}$, so by the direct product theorem $D{12} \not \cong D_6 \times C_2$.   
\end{proof}

So to sum it all up, remember this:
\begin{center}
	\emph{We can't use the direct product theorem to conclude that a group is not a direct product.}
\end{center}

\section{Proving We Don't Have A Direct Product}

So the last section showed a bunch of ways that we \emph{can't} show some group is not isomorphic to a direct product. So a natural question is eh... how do we show it then? The answer to this is, naturally, it depends. However, a good strategy is to \emph{look at the properties of the group}.

This is, of course, a meta strategy, but usually finding some difference between two groups is a great way of showing that they aren't isomorphic, and the same applies for direct products.

Some properties that you can look at are things like
\begin{itemize}
	\item properties of all direct products
	\item order of elements
	\item order of the groups
	\item number of subgroups of a certain order
	\item the center or normalizer of the groups
	\item number of generators
\end{itemize}
of course there is many others. Example time!

\begin{example}
	Prove that $A_n \times C_2 \not \cong S_n$ for $n \geq 3$.
\end{example}
\begin{proof}[Solution]
	We consider the center of each group. For $n = 3$, $Z(A_3 \times C_2) = A_n \times C_2$ since $A_n$ is abelian, but $Z(S_3)$ is trivial. For $n \geq 4$, $Z(A_3 \times C_2) = C_2$, but $Z(S_n)$ is again trivial.
	Thus $A_3 \times C_2 \not \cong S_3$ for all $n \geq 3$.
\end{proof}

\begin{example}
	Prove that $\mathbb{Q}$, the additive group of rationals, is not the direct product of two non-trivial groups.
\end{example}
\begin{proof}[Solution]
	If $\mathbb{Q} \cong H \times K$ for nontrivial $H, K$, then there exists two nontrivial subgroups $H', K' \leq \mathbb{Q}$ such that $H' \cap K' = \{ e \}$\footnote{Note that this may seem like one of the previous fallacies, but in fact we are not invoking the direct product theorem. We are merely using properties that come from the definition of the direct product.}. Then, as $H'$ and $K'$ are non-trivial, let $\frac{x}{y} \in H'$ and $\frac{a}{b} \in K'$. Then $y \cdot \frac{x}{y} = x \in H'$, and $b \cdot \frac{a}{b} = a \in K'$ by addition. But then $ax \in H' \cap K'$, which is a contradiction. Thus $\mathbb{Q}$ cannot be written as the direct product of two non-trivial groups.
\end{proof}

\begin{example}
	Prove that $D_{2n} \not \cong C_n \times C_2$ for $n \geq 3$.
\end{example}
\begin{proof}[Solution]
	Note that $C_n \times C_2$ is abelian, but $D_{2n}$ is not.
\end{proof}

\begin{example}
	Is $C_{p^2} \cong C_p \times C_p$?
\end{example}
\begin{proof}[Solution]
	I claim the answer is no. 
	The group $C_p \times C_p$ cannot be generated by a single element, whereas the group $C_{p^2}$ can.
\end{proof}

\clearpage


% To see what I'm talking about, consider the following bogus proof.

% \begin{example}
% 	Prove that if $m$ and $n$ are not relatively prime then $C_{mn} \not \cong C_m \times C_n$.
% \end{example}
% \begin{proof}[Incorrect Proof]
% 	Note that $C_m$ and $C_n$ are subgroups of $C_{mn}$. Then if $m$ and $n$ are not relatively prime and $\langle g \rangle = C_{mn}$, then $g \in C_m, C_n$. Thus $C_m \cap C_n \neq \{e\}$, so $C_{mn} \not \cong C_m \times C_n$ by the direct product theorem. 
% \end{proof}

% \section{Finitely Generated Abelian Groups}

% Test

% \section{Problems}

% \begin{problem}
% 	Prove that $A_n \times C_2 \not \cong S_n$ for $n \geq 3$.
% \end{problem}
% \begin{problem}
% 	Prove that $\mathbb{Q}$ is not the direct product of two groups.
% \end{problem}

\end{document}

% Citations: 

% This page for counterexample 
% https://math.stackexchange.com/questions/2333666/group-is-isomorphic-to-direct-product-of-its-subgroups
