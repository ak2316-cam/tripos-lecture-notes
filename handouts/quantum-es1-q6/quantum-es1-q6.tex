\documentclass[a4paper]{scrartcl}

\usepackage[
  noindent,
  color % Enable/disable this option to get tinted problem boxes.
]{examplesheet}

\title{Quantum Mechanics -- Sheet 1, Question 6}
\author{Cambridge Mathematical Tripos Part IB -- Adam Kelly (\texttt{ak2316})}
\date{\today}


% Add environment definitions if needed.
\newtheorem*{postulate}{Postulate}
\newtheorem*{proposition}{Proposition}
% \newtheorem*{lemma}{Lemma}
% \newtheorem*{claim}{Claim}

\begin{document}
\maketitle

On the example sheet was the question below, which should be straightforward but is made a bit harder by the way quantum mechanics was introduced in lectures. This is my attempt at explaining the solution alongside the `proper' way of thinking.
It isn't very concise though -- apologies.

\begin{problem*}
	A quantum system has Hamiltonian $\hat{H}$ with normalised eigenstates $\chi_{n}$ and corresponding energies $E_{n}$ $(n=1,2,3, \ldots)$. A linear operator $\hat{Q}$ associated with the quantity $Q$ is defined by its action on these states:
$$
\hat{Q} \chi_{1}=\chi_{2}, \quad \hat{Q} \chi_{2}=\chi_{1}, \quad \hat{Q} \chi_{n}=0 \quad n>2
$$
Show that $\hat{Q}$ has eigenvalues $\pm 1$ (in addition to zero) and find the corresponding normalised eigenstates $\chi_{\pm}$, in terms of energy eigenstates. Calculate $\langle\hat{H}\rangle$ in each of the states $\chi_{\pm}$.

A measurement of $Q$ is made at time zero, and the result $+1$ is obtained. The system is then left undisturbed for a time $t$, at which instant another measurement of $Q$ is made. What is the probability
that the result will again be $+1$? Show that the probability is zero if the measurement is made when a time $T=\pi \hbar /\left(E_{2}-E_{1}\right)$ has elapsed (assume $\left.E_{2}-E_{1}>0\right)$.
\end{problem*}

\begin{proof}[Solution]
    We begin by describing the state of our physical system. In quantum mechanics, the way this is done is described by the following postulate.
    
    \begin{postulate}[State]
        the state of a physical system is given by a non-zero \emph{state vector} $\psi$ in a \emph{Hilbert space} $\mathcal{H}$, a complex vector space with a complete inner product $\langle \cdot, \cdot \rangle$.
    \end{postulate}

    In this case, our system is described by the Hamiltonian $\hat{H}$, so it's natural to work using the energy eigenstates $\chi_n$ as our basis.
    With this, we can describe the state $\psi$ of our system with the linear combination
    $$
    \psi = \sum_{n \geq 1} \alpha_n \chi_n.
    $$

    With the state of our system described, we now turn our attention to a physical quantity of interest, $Q$. This quantity $Q$ is an \emph{observable}, something we will want to measure using some physical experiment.
    The nature of observables is also described by a postulate.\footnote{It's worth dwelling on the nature of observables in quantum mechanics, as it's a little unintuitive. In classical mechanics, we are quite familiar with observables -- position, kinetic energy, angular momentum, and everything else. Observables in a classical setting are just functions that take the state of the system and produce some number, something that can be thought of as reading out a value from some measurement device.
    But in quantum mechanics, since we work with states in a vector space, if we had two different states with two different values for an observable, since the sum of states is also a state -- what would the value of that observable be? 
    
    To sort this out, we consider the observable and measurement separately.
    Instead of letting observables be just real valued functions on the state, we consider two things: the states for which the value of the observable is well defined, and what that value is. Mathematically, these are baked into a self-adjoint linear operator whose eigenvectors and corresponding eigenvalues match the states and values for which the observable is well defined. To then get those well defined values of the observable, we describe measurement as a process of the state probabilistically `collapsing' into one of the eigenstates (where the value of the observable is well defined).}

    \begin{postulate}[Observables]
        The observables of a quantum mechanical system $A$ are given by self-adjoint linear operators $\hat{A} : \mathcal{H} \rightarrow \mathcal{H}$.        
    \end{postulate}

    \begin{postulate}[Measurement]
        States for which the value of an observable can be characterized by a well-defined number are the states that are eigenvectors for the corresponding self-adjoint operator. The value of the observable in such a state will be a real number, the eigenvalue of the operator.
    \end{postulate}

    We know that the quantity $Q$ is associated with the self-adjoint linear operator $\hat{Q}$, and we can compute the possible values of $Q$ by computing the eigenvalues of $\hat{Q}$.

    Suppose that $\sum_{n \geq 1} \beta_i \chi_i$ is an eigenvector for $\hat{Q}$, with corresponding eigenvalue $\lambda$.
    Then we would have
    \begin{align*}
       Q\left[\sum_{n \geq 1} \beta_i \chi_i\right] &= \sum_{n \geq 1} \lambda \beta_i \chi_i \\ 
\implies \beta_1 \chi_2 + \beta_2 \chi_1 &= \sum_{n \geq 1} \lambda \beta_i \chi_i.
    \end{align*}
    Comparing coefficients, this implies that 
    $\beta_1 = \lambda \beta_2$ and $\beta_2 = \lambda \beta_1$, with $\beta_i = 0$ for all $i \geq 3$. These equations imply that $\lambda = \pm 1$, and the corresponding eigenstates are
    (once normalized)
    $$
    \chi_+ = \frac{\chi_1 + \chi_2}{\sqrt{2}}, \quad \text{and} \quad \chi_- = \frac{\chi_1 - \chi_2}{\sqrt{2}}.
    $$
    So we have shown that our quantity $Q$ can only take on values in $\{1, -1\}$, and we have found the states of the system that these values correspond to.

    For the rest of the problem, we want to compute various facts about the probabilities of different outcomes, corresponding to the two observables $\hat{Q}$ and $\hat{H}$ (which correspond with quantities $Q$ and the energy of the system). The way we relate the values of observables and the probabilities of measuring them is with another postulate, the \emph{Born rule}\footnote{In a continuous setting, such as when we work using the position basis, we have to replace various sums with integrals (like what was presented in lectures). We won't need that here though, since we work using the discrete energy eigenstates as a basis.}.

    \begin{postulate}[Born Rule]
        If an observable $A$ corresponding to a self-adjoint operator $\hat{A}$ is measured in a system with normalized state vector $\psi$, then the probability of measuring a given eigenvalue $\lambda_i$ will equal
        $
        |\langle \chi_i, \psi \rangle|^2,
        $
        where $\chi_i$ is the normalized eigenvector corresponding to $\lambda_i$.

        If such an eigenvalue $\lambda_i$ is obtained, then the state of the system will be left in the corresponding eigenstate, $\psi = \chi_i$.
    \end{postulate}

    With the Born rule specifying the probabilities of different outcomes, we can use standard tools from probability such as expected value to study our observable.
    
    For example, we can compute the expected value of the energy of the system, $\langle \hat{H} \rangle$, in each of the states $\psi = \chi_{\pm}$. We know that $\chi_+ = (\chi_1 + \chi_2)/\sqrt{2}$, and $\chi_- = (\chi_1 + \chi_2)/\sqrt{2}$, so it's not hard to see that in either case, measuring the energy must result in either $E_1$ or $E_2$, the corresponding eigenvalues to the eigenstates $\chi_1$ and $\chi_2$ of $\hat{H}$.

    So in the state $\chi_+$, the expected value is given by
    \begin{align*}
    \langle \hat{H}\rangle &= E_1 |\langle \chi_1, \chi_+\rangle|^2 + E_2 |\langle \chi_2, \chi_+\rangle|^2 \\
    &= E_1 \left|\left\langle \chi_1, \frac{\chi_1 + \chi_2}{\sqrt{2}} \right\rangle\right|^2 + E_2 \left|\left\langle \chi_2, \frac{\chi_1 - \chi_2}{\sqrt{2}} \right\rangle\right|^2 \\
    &= \frac{1}{2}E_1 + \frac{1}{2}E_2,
    \end{align*}
    where we use the orthonormality of $\chi_i$. Similarly in the state $\chi_-$, the expected value is given by
    \begin{align*}
        \langle \hat{H}\rangle &= E_1 |\langle \chi_1, \chi_-\rangle|^2 + E_2 |\langle \chi_2, \chi_-\rangle|^2 \\
        &= E_1 \left|\left\langle \chi_1, \frac{\chi_1 - \chi_2}{\sqrt{2}} \right\rangle\right|^2 + E_2 \left|\left\langle \chi_2, \frac{\chi_1 - \chi_2}{\sqrt{2}} \right\rangle\right|^2 \\
        &= \frac{1}{2}E_1 + \frac{1}{2}E_2,
        \end{align*}
    so the expected values in these cases are the same\footnote{This isn't surprising, but we won't dwell on that.}.


    Now we are told that $Q$ has been measured at time $t = 0$, and the result $+1$ was obtained. From our measurement postulate, we know that the state of the system is then $\psi(0) = \chi_+$. We are then told that some time $t$ passes.
    We want to be able to describe the new state of the system after some time has passed. As you might have guessed, we have another postulate for this.

    \begin{postulate}[Time Evolution]
        There is a distinguished observable, the Hamiltonian $H$. Time evolution of states $\psi(t) \in \mathcal{H}$ is given by the Schrödinger equation
        $$
        i \hbar \frac{\dd \psi(t)}{\dd t} = \hat{H} \psi(t).
        $$
    \end{postulate}

    Using this equation, we can figure out how a general state in our system evolves over time. Let our state be $\psi(t) = \sum_{n \geq 1} \alpha(t) \chi_n$. Substituting this into the Schrödinger equation describing time evolution gives
    $$
    i \hbar \sum_{n \geq 1}\alpha'(t)\chi_n = \sum_{n \geq 1} E_n \alpha(t) \chi_n.
    $$
    Comparing coefficients, we get a series of first order differential equations
    $$
    i \hbar \alpha_n'(t) = E_n \alpha_n(t),
    $$
    which we can solve to get
    $$
    \alpha_n(t) = \alpha_n(0) e^{-i E_n t/\hbar},
    $$
    so
    $$
    \psi(t) = \sum_{n \geq 1} \alpha_n(0) e^{-iE_n t/\hbar} \chi_n.
    $$

    So if our initial state is $\psi(0) = \chi_+$, we find that $\alpha_1(0) = \alpha_2(0) = 1/\sqrt{2}$, and generally
    $$
    \psi(t) = \frac{1}{\sqrt{2}}\left(e^{-iE_1 t/\hbar} \chi_1+e^{-iE_2 t/\hbar} \chi_2\right).
    $$
    With this computed, we can apply the Born rule to compute the probability of measuring $+1$ again at time $t$, as
    \begin{align*}
    |\langle \chi_+ , \psi(t) \rangle|^2 &= \left|\left\langle \frac{\chi_1 + \chi_2}{\sqrt{2}}, \frac{e^{-iE_1 t/\hbar} \chi_1+e^{-iE_2 t/\hbar} \chi_2}{\sqrt{2}}\right\rangle\right|^2 \\
    &= \left|\frac{1}{2}\left(e^{-iE_1 t/\hbar}+e^{-iE_2 t/\hbar} \right)\right|^2 \\
    &= \left|\frac{1}{2}\cos \left(\frac{E_{1}-E_{2}}{\hbar} t\right) + \frac{1}{2}\right|^2.
    \end{align*}
    This probability is periodic, and indeed if $t = \pi \hbar/(E_2 - E_1)$, then the probability will be zero.
\end{proof}


\end{document}