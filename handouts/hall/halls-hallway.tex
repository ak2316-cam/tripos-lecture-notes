\documentclass[11pt]{article}

\usepackage[
% fancytheorems, 
% fancyproofs,
noindent, 
nokoma
%spacingfix, 
]{adam}

% \usepackage{ulem}
% \DeclareRobustCommand{\hsout}[1]{\texorpdfstring{\sout{#1}}{#1}}

\usepackage{cancel}
\usepackage{tikz}

% \usepackage{titling}
% \setlength{\droptitle}{-4em}


\title{\vspace{-3.25\baselineskip}\ \\Hall's Hallway}
\author{Adam Kelly (\texttt{ak2316@cam.ac.uk})}
\date{\today}

\begin{document}

\maketitle

In a hallway, there is a finite set of \emph{doors} $D$ that you would like to open and a finite set of single-use \emph{keys} $K$. Each key can unlock some subset of the doors, and is destroyed immediately after unlocking. Under what circumstance is it possible to open all of the doors?

\section{Hall's Condition}

In answering this question, we might try and think about reasons why it may \emph{not} be possible. The simplest reason is when there's not enough keys to go around, when $|D| \geq |K|$. Another reason is when there seems to be enough keys overall, but you find that there's not enough keys compatible some subset of the doors to unlock them all. We call these two basic requirements \emph{Hall's condition}.

\begin{definition*}[Hall's Condition]
	We say a collection of doors and keys satisfies \vocab{Hall's condition} if for any $k$ doors there are at least $k$ keys which at one door is opened by. 
\end{definition*}

It turns out that these basic requirements are the \emph{only requirements} that we need to satisfy to open all of the doors.

\section{Hall's Theorem}

\begin{theorem}[Hall's Theorem]
	If is possible to open all of the doors if and only if Hall's condition is satisfied.
\end{theorem}
\begin{proof}
Obviously this is a necessary condition, so we are left to prove that it's sufficient. We are going to induct on the number of doors. 

We begin by noticing that if there's a proper subset of doors $D'$ such that the doors in $D'$ are opened by exactly $|D'|$ keys, then if $K'$ is this set of keys we will have to open the doors in $D'$ using the keys $K'$. Luckily we know this is possible by induction, and afterwards if we ignore the open doors then Hall's condition still holds.

We can repeat this until there's no such subset of doors, but then we can just open the first remaining door with any remaining key, and Hall's condition will still hold for the remaining doors and keys, and we are then done by induction.
\end{proof}

\section{Problems}

\begin{problem}
	Let $S = \left\{ 1,2,\dots,2015 \right\}$.
Prove that there exists an injective function \[ f : \binom{S}{1007} \hookrightarrow \binom{S}{1008} \]
such that $T \subseteq f(T)$ for every $T$.
(Here $\binom Sk$ is the set of $k$-element subsets of $S$.)
\end{problem}

\begin{problem}
	Let $G = A \cup B$ be a bipartite graph on $2n$ vertices
	with minimum degree $n/2$ and $|A|=|B|=n$.
	Show that $G$ has a perfect matching.
\end{problem}

\begin{problem}
	A square sheet of paper of side length $n$ is divided up into $n$ polygons each of area $n$. A second square sheet of paper of side length $n$ is also divided up into $n$ polygons each of area $n$. The first sheet of paper is placed on top of the second sheet of paper, with both sheets aligned in the same way so that the first sheet completely covers the second. Show that it is possible to stick $n$ pins through the sheets in such a way that each of the $2 n$ polygons has a pin through it.
\end{problem}

\begin{problem}
	For what values of $k$ on a $1000 \times 1000$ chessboard is it possible to delete $k$ squares from the board and still place $1000$ non-attacking rooks on the board?
\end{problem}


\begin{problem}
	An $n \times n$ Latin square is an $n \times n$ square array of numbers with each of the numbers $1$, $2$, $\ldots$, $n$ appearing precisely once in each row and precisely once in each column. For $r<n$. an $r \times n$ Latin rectangle is an $r \times n$ rectangular array of numbers $(r$ rows, $n$ columns) with each of the numbers $1$, $2$, $\ldots$, $n$ appearing precisely once in each row and at most once in each column. Prove that every $r \times n$ Latin rectangle may be extended to an $n \times n$ Latin
	square.
\end{problem}

\begin{proof}
	Let $G$ be a bipartite graph on $A \cup B$ with no isolated vertices.
Assume that for each edge $ab$ we have $\deg a \geq \deg b$.
Prove that $G$ contains a matching using all vertices in $A$.
\end{proof}

\begin{problem}
	A round-robin tournament among $2 n$ teams lasted for $2 n-1$ days, as follows. On each day, every team played one game against another team, with one team winning and one team losing in each of the $n$ games. Over the course of the tournament, each team played every other team exactly once. Can one necessarily choose one winning team from each day without choosing any team more than once?
\end{problem}

\begin{problem}
	Let $n \geq 4$ be an integer. A \emph{flag} is a binary string of length $n$. We say that a set of $n$ flags is \emph{diverse} if these flags can be the rows of an $n \times n$ binary matrix with the entries in its main diagonal all equal. Determine the smallest positive integer $M$ such that among any $M$ distinct flags, there exist $n$ flags forming a diverse set.
\end{problem}


\end{document}
