\documentclass[11pt, a4paper, reqno]{amsart}

\usepackage{amsmath}
\usepackage{amssymb}
\usepackage{amsthm}
\usepackage{parskip}
%\usepackage[a4paper, margin=1.3in]{geometry}

\newtheorem*{problem}{Problem}

\newtheorem{theorem}{Theorem}
\newtheorem{lemma}{Lemma}
\newtheorem{corollary}{Corollary}

\theoremstyle{definition}

\newtheorem{definition}{Definition}


\title{Bertrand's Posulate -- Problem Walkthrough}
\author{Adam Kelly (ak2316@cam.ac.uk)}
\date{\today}

\begin{document}
\vspace*{-1cm}
\maketitle

\section*{Problem}

\begin{problem}
Prove that for every $n \geq 1$, there is some prime number $p$ with $n < p \leq 2n$.
\end{problem}


\section*{Walkthrough}

Our proof will be based around various properties of the central binomial coefficient $\binom{2n}{n}$. We will first prove a number of small lemmas, which will then all come together in the final proof of the theorem. 
\begin{enumerate}
	\item Show that Bertrand's Posulate holds for `small' $n$ (say $n \leq 512)$. [You can do this by verifying (by hand!) that $2$, $3$, $5$, $7$, $13$, $23$, $43$, $83$, $163$, $317$, and $521$ are all primes.]
	\item Prove that $\binom{2n}{n} \geq 4^n/2n$. [You may want to employ the binomial theorem.]
	\item If $p$ is a prime such that $p^a \mid \binom{2n}{n}$, prove that $p^a \leq 2n$. [Begin by proving Legendre's formula, and apply it to $\binom{2n}{n}$.]
	\item Prove that if $2n/3 \leq p \leq n$ for some prime $p$, then $p \nmid \binom{2n}{n}$. [This is just casework.]
	\item Define $f(n)$ to be the product of all primes less than or equal to $n$ (with $f(1) = 1$). Prove by induction that $f(n) < 4^n$. [For the prime case, notice that $f(2a + 1)/f(a + 1)$ divides $\binom{2a + 1}{a}$, then use the binomial theorem to get a weak inequality for the size of this.]
	\item Prove Bertrand's Posultate by contradiction.
	\begin{enumerate}
		\item Show that if $n$ was such that there was no prime between $n$ and $2n$, then the prime factors of $\binom{2n}{n}$ are all smaller than $2n/3$.
		\item Noting for $n \geq 5$ we have $\sqrt{2n} < 2n/3$, write $\binom{2n}{n} = P\cdot P'$, where $P$ is the product of primes at most $\sqrt{2n}$ and $P'$ is the product of primes at least $\sqrt{2n}$. Prove that $P \leq (2n)^{\sqrt{2n}}$ and $P' \leq (2n/3)! \leq 4^{2n/3}$, and deduce that  $4^{n/3} \leq (2n)^{\sqrt{2n}+1}$.
		\item Find some $N$ such that the above inequality does not hold for all $n \geq N$. [You may need to compute derivatives to establish this. It may also help to weaken the inequality slightly.] 
		\item Finish off the proof.
	\end{enumerate}
\end{enumerate}
\clearpage
\section*{Proof}

\subsection*{(1)}
We obtain the weak bound
\begin{align*}
	2n\binom{2n}{n} \geq 2 + (2n - 1)\binom{2n}{n} \geq \sum_{j = 0}^{2n}\binom{2n}{j} = (1 + 1)^{2n} = 4^{n},
\end{align*}
and thus $\binom{2n}{n} \geq 4^n/2n$.

\subsection*{(2)} For a prime $p$, we define $\nu_p(n)$ to be the largest integer $\alpha$ such that $p^\alpha \mid n$. By Legendre's formula we then have $\nu_p(n!) = \sum_{j = 1}^{\infty}\lfloor n/p^j \rfloor$.
We can extend our definition of $\nu_p$ to $\mathbb{Q}$, by letting $\nu_p(a/b) = \nu_p(a) - \nu_p(b)$, which is consistent with our previous definition. Applying Legendre's formula we then have
\begin{align*}
	\nu_p \binom{2n}{n} &= \nu_p((2n)!) - \nu_p(n! \cdot n!) \\
	&= \nu_p((2n)!) - 2\nu_p(n!) \\
	&= \sum_{j = 1}^{\infty} \left\lfloor \frac{2n}{p^j}\right\rfloor - 2 \left\lfloor \frac{n}{p^j}\right\rfloor.
\end{align*}
`Taking the integer out', since  $\lfloor 2a \rfloor - 2\lfloor a \rfloor = \lfloor 2 \{a\}\rfloor$ where $\{a\} = a - \lfloor a \rfloor$ and $0 \leq \{a\} < 1$, we have $0 \leq \lfloor 2a\rfloor - 2 \lfloor a \rfloor \leq 1$. Thus if $\alpha$ is the largest integer such that $p^{\alpha} \leq 2n$, for $j > \alpha$ the terms of the sum are zero, and for $j \leq \alpha$ the terms are either 0 or 1, giving
\begin{align*}
	\nu_p \binom{2n}{n} \leq \sum_{j = 1}^{\alpha}1 \leq \alpha.
\end{align*}

Going back to the definition of $\nu_p$, if $p$ is a prime with $p^a$ dividing $\binom{2n}{n}$, then $a \leq \nu_p \binom{2n}{n} \leq \alpha$, and $p^a \leq p^{\alpha} \leq 2n$.

\subsection*{(3)} Suppose $2n/3 < p < n$ for some prime $p$. We wish to show that $p \nmid \binom{2n}{n}$. Since $p< 2n$ and $2p < 2n$ but $2n < 3p$, $p$ must appear as a factor in $(2n)!$ exactly 2 times. But then it also appears as a factor in $n!$ exactly 1 time, and thus in $(n!)^2$ exactly 2 times. Thus it does not appear as a factor of $\frac{(2n)!}{n! \cdot n!} = \binom{2n}{n}$, as desired.

\subsection*{(4)} Defining $f(n)$ as in the walkthrough, we will show that $f(n) < 4^n$ for all $n \geq 1$. The first few values can be seen working by hand: $f(1) = 1 < 4^1$, $f(2) = 2 < 4^2$, $f(3) = 6 < 4^3$, $f(4) = 6 < 4^3$. Now we will use induction. Assuming that $f(k) \leq 4^k$ for all $k < n$, we will show that $f(n) < 4^n$. There are two cases.

\emph{Case 1. $n$ is composite}. In this case, $f(n) = f(n - 1) < 4^{n - 1} < 4^{n}$, and we are done.

\emph{Case 2. $n$ is prime}. We will start by assuming that $n$ is odd (we already showed the case for $n = 2$), and write $n = 2a + 1$. The binomial coefficient $\binom{2a + 1}{a + 1}$ is divisible by all primes between $a + 1$ and $2a + 1$, that is,
$$
	\frac{f(2a + 1)}{f(a)} \mid \binom{2a + 1}{a}.
$$ 
We can then bound the binomial coefficient with
\begin{align*}
	\binom{2a + 1}{a} \leq \sum_{j = 0}^{a} \binom{2a + 1}{a} = \frac{1}{2}\sum_{j = 0}^{2a + 1} \binom{2a + 1}{a} = 2^{2a + 1} = 4^{a},
\end{align*}
and since $f(2a + 1)/f(a) \mid \binom{2a + 1}{a}$ implies that $(2a + 1)/f(a) \leq \binom{2a + 1}{a}$, we have
$$
f(2a + 1) \leq \binom{2a + 1}{a} f(a) \leq 4^{a} 4^{a} < 4^{2a + 1}.
$$

This completes our induction, and $f(n) < 4^n$ for all $n$.

\subsection*{(5)} Using these results, we will now prove Bertrand's postulate.
Assume for a contradiction that there is some $n \geq 1$ such that there is no prime $p$ with $n < p \leq 2n$.

\subsubsection*{(a)} If this is the case, the binomial coefficient $\binom{2n}{n}$ would not be divisible by any prime $p \geq n$. Also, if $2n/3 < p < n$ for some prime $p$ then $p \nmid \binom{2n}{n}$. So the binomial coefficient must be divisible only by primes which are at most $2n/3$.

\subsubsection*{(b)} For $n \geq 5$ we have $\sqrt{2n} < 2n/3$. With this, we write
$$
	\binom{2n}{n} = P\cdot P'
$$
where $P$ is the product of primes which are at most $\sqrt{2n}$, and $P'$ is the product of primes which are at least $\sqrt{2n}$.

Let $P = p_1^{\alpha_1}p_2^{\alpha_2} \cdots p_k^{\alpha_k}$ be the prime factorisation of $P$.  Then $p_i^{\alpha_i} \mid \binom{2n}{n}$ so by our previous result $p_i^{\alpha_i} \leq 2n$. This implies
$
P \leq (2n)^k.
$
Also, there can be at most $\sqrt{2n}$ primes which are at most $\sqrt{2n}$, so $k \leq \sqrt{2n}$ giving
$$
	P \leq (2n)^{\sqrt{2n}}.
$$

Now let $P' = q_1^{\beta_1} q_2^{\beta_2} \cdots q_l^{\beta_l}$ be the prime factorisation of $P'$. Then since $\sqrt{2n} < q_1$, squaring shows $2n < q_1^2$. So to divide $\binom{2n}{n}$, we must have $\beta_1 = \beta_2 = \cdots = \beta_l = 1$.
$P$ is then the product of distinct primes which are at most $2n/3$, so $P \leq f(2n/3) < 4^{2n/3}$.

Combining these inequalities with the lower bound we obtained earlier, 
$$
\frac{4^n}{2n} \leq \binom{2n}{n} = P \cdot P' \leq (2n)^{\sqrt{2n}} 4^{2n/3} \implies 4^{n/3} \leq (2n)^{\sqrt{2n} + 1}.
$$

\subsubsection*{(c)} We will now show that the inequality $4^{n/3} \leq (2n)^{\sqrt{2n} + 1}$ does not hold for suitably large $n$. This will then put an upper bound on the possible counterexamples to Bertrand's postulate, below we can then check by computation.

Considering the opposite inequality and taking the logarithm, we have
\begin{align*}
	(2n)^{\sqrt{2n} + 1} &\leq 4^{n/3} \\
\iff (\sqrt{2n} + 1) \log_2(2n) &\leq 2n/3.
\end{align*}
This is implied by the weaker inequality
\begin{align*}
	(\sqrt{2n} + 1) \log_2(2n) &\leq (2n-1)/3 \\
\iff   0 &\leq (\sqrt{2n}-1)- 3 \log_2(2n)\tag{$*$}
\end{align*}
Computing derivatives, we have
\begin{align*}
	\frac{\mathrm{d}}{\mathrm{d}n}\left[(\sqrt{2n}-1)- 3 \log_2(2n)\right] = \frac{1}{\sqrt{2n}} - \frac{3}{n \log(2)}. \tag{$**$}
\end{align*}
So for $n \geq 18/\log^2(2)$ the derivative is positive. Using a rough bound for $\log(2)$, by taking say the power series, $1/4 < \log^2(2)$ so $72 > 18/\log^2(2)$, and thus taking  $n \geq 72$ will make the derivative positive.

Thus if we can find $N$ such that $(*)$ holds for $N$ with $N \geq 72$, then it will hold for all $n \geq N$. Indeed, taking $N = 512$ the inequality is
$$
0 \leq (32 - 1) - 3 * 10 = 1,
$$
which holds.


\subsubsection*{(d)}
Putting this all together, if $n$ is a counterexample to Bertrand's postulate then $n < 512$, but we can verify by hand that
$$
	\{2, 3, 5, 7, 13, 23, 43, 83, 163, 317, 521\}
$$
are all primes, showing that Bertrand's postulate holds for $n < 512$. Thus there is no counterexamples, and Bertrand's postulate must hold for all $n$.
\end{document}
