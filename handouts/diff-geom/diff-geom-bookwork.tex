% % NOTE: for a more compact, black and white style for printing, use
% % the below.

% \documentclass[a3paper, 10pt]{article}

% \usepackage[nokoma, noindent]{adam}
% \usepackage[landscape,margin=0.5in]{geometry}
% \usepackage{tikz-cd}
% \usepackage{multicol}

% \setlist[enumerate]{leftmargin=8mm}
% \setlist[itemize]{leftmargin=8mm}

% \newcommand{\enumpre}{}

% \newcommand{\enumpre}{\vspace{-1.5\baselineskip}}
% \renewcommand{\vocab}[1]{\emph{#1}}


\documentclass[10pt]{article}

\usepackage[
% fancytheorems, 
% fancyproofs,
noindent, 
nokoma
%spacingfix, 
]{adam}

% \usepackage[margin=0.75in]{geometry}

% \usepackage{ulem}
% \DeclareRobustCommand{\hsout}[1]{\texorpdfstring{\sout{#1}}{#1}}

\usepackage{cancel}

\usepackage{titling}
\setlength{\droptitle}{-4em}


\title{Differential Geometry}
\author{Mathematical Tripos Part II}
\date{\today}

\begin{document}

% \begin{multicols*}{5}

\maketitle


% \tableofcontents

\emph{Note. Knowledge of what a diffeomorphism, homotopy, and isotopy are is assumed.}


\section{Differential Topology}

\subsection{Manifolds}


\begin{definition}[Manifold]
    We say that $X \subseteq \R^n$ is a \vocab{$k$-dimensional manifold} if each $x \in X$ has a neighborhood $V \subseteq X$ diffeomorphic to an open set of $\R^k$. 
\end{definition}

\begin{definition}[Parameterisation and Chart]
    A diffeomorphism $\phi: U \rightarrow V$, where $U$ is an open set of $\R^k$ is a \vocab{parameterisation} of the neighborhood $V$. The inverse diffeomorphism $\phi^{-1}: V \rightarrow U$ is called a \vocab{chart} on $V$.
\end{definition}

If we have manifolds $X$ and $Z$ with $Z \subseteq X$ we say that $Z$ is a \vocab{submanifold} of $X$. In this case, the \vocab{codimension} of $Z$ in $X$ is $\dim X - \dim Z$.

\begin{definition}[Tangent Space of a Manifold]
    Let $X \subseteq \R^n$ be a manifold, $x \in X$. Let $\phi: U \rightarrow X$ be a parameterisation with $\phi(0) = x$. The \vocab{tangent space} $T_x X$ is\footnote{Viewing $\phi$ as a function onto $\R^n$.} $d\phi_0(\R^k)$. 
\end{definition}


Let $f: X \rightarrow Y$ be a smooth map between manifolds. We say that $f$ is a \vocab{local diffeomorphism} at $x$ if $f$ maps a neighbourhood of $x$ diffeomorphically onto a neighbourhood of $f(x)$.

\begin{theorem}[Inverse Function Theorem]
    Suppose that $f: X \rightarrow Y$ is a smooth map whose derivative $d f_{x}$ at the point $x$ is an isomorphism. Then $f$ is a local diffeomorphism at $x$.
\end{theorem}


\subsection{Regular values and Sard's theorem}
Let $f: X \rightarrow Y$ be a smooth map between manifolds. Let $C$ be the set of all points $x \in X$ such that $d f_{x}: T_{x} X \rightarrow T_{f(x)} Y$ is not surjective.

\begin{definition}
    A point in $C$ will be called a \vocab{critical point}. A point in $f(C)$ will be called a \vocab{critical value}. A point in the complement of $f(C)$ will be called a \vocab{regular value}.
\end{definition}

\begin{theorem}[Preimage Theorem]
    Let y be a regular value of $f: X \rightarrow Y$ with $\operatorname{dim} X \geq \operatorname{dim} Y$. Then the set $f^{-1}(y)$ is a submanifold of $X$ with $\operatorname{dim} f^{-1}(y)=$ $\operatorname{dim} X-\operatorname{dim} Y$.
\end{theorem}
\begin{proof}
    Let $x \in f^{-1}(y)$. Since $y$ is a regular value, the derivative $d f_{x}$ maps $T_{x} X$ onto $T_{y} Y$. The kernel of $d f_{x}$ is a subspace $K$ of $T_{x} X$ of dimension $p:=$ $\operatorname{dim} X-\operatorname{dim} Y$. Suppose $X \subset \mathbb{R}^{N}$ and let $T: \mathbb{R}^{N} \rightarrow \mathbb{R}^{p}$ be any linear map such that $\operatorname{Ker}(T) \cap K=\{0\}$. Consider the map $F: X \rightarrow Y \times \mathbb{R}^{p}$ given by
$$
F(z)=(f(z), T(z)).
$$
The derivative of $F$ is given by
$$
d F_{x}(v)=\left(d f_{x}(v), T(v)\right)
$$
which is clearly nonsingular by our choice of $T$. By the inverse function theorem, $F$ is a local diffeomorphism at $x$, i.e. $F$ maps some neighbourhood $U$ of $x$ diffeomorphically onto a neighbourhood $V$ of $(y, T(x))$. Hence $F$ maps $f^{-1}(y) \cap U$ diffeomorphically onto $\left(\{y\} \times \mathbb{R}^{p}\right) \cap V$ which proves that $f^{-1}(y)$ is a manifold with $\operatorname{dim} f^{-1}(y)=p$.
\end{proof}


\begin{theorem}[Stack of Records Theorem]
    Let $f: X \rightarrow Y$ be a smooth map between manifolds of the same dimension with $X$ compact. Let y be a regular value of $f$ and write $f^{-1}(y)=\left\{x_{1}, \ldots, x_{k}\right\}$. Then there exists a neighbourhood $U$ of $y$ in $Y$ such that $f^{-1}(U)$ is a disjoint union $V_{1} \cup \cdots \cup V_{k}$, where $V_{i}$ is an open neighbourhood of $x_{i}$ and $f$ maps each $V_{i}$ diffeomorphically onto $U$.
\end{theorem}
\begin{proof}
    By the inverse function theorem we can pick disjoint neighbourhoods $W_{i}$ of $x_{i}$ such that $f$ maps $W_{i}$ diffeomorphically onto a neighbourhood of $y$. Observe that $f\left(X-\cup_{i} W_{i}\right)$ is a compact set which does not contain $y$. Now take
$
U=\bigcap_{i} f\left(W_{i}\right)-f\left(X-\cup_{i} W_{i}\right).
$
\end{proof}


If we let $\# f^{-1}(y)$ be the cardinality of $f^{-1}(y)$, the theorem implies that the function $y \mapsto \# f^{-1}(y)$ is locally constant as $y$ ranges over regular values of $f$.

\begin{theorem}[Sard's Theorem]
    The set of critical values of a smooth map $f: X \rightarrow Y$ has measure zero.
\end{theorem}

\subsection{Transversality}

\begin{definition}[Transversal]
    A smooth map $f: X \rightarrow Y$ is said to be \vocab{transversal} to a submanifold $Z \subset Y$ if for every $x \in f^{-1}(Z)$ we have
$$
\operatorname{Image}\left(d f_{x}\right)+T_{f(x)} Z=T_{f(x)} Y.
$$
We write $f \pitchfork Z$.
\end{definition}

\begin{theorem}[Transversal Preimage Theorem]
    If the smooth map $f: X \rightarrow Y$ is transversal to a submanifold $Z \subset Y$, then $f^{-1}(Z)$ is submanifold of $X$. Moreover, $f^{-1}(Z)$ and $Z$ have the same codimension.
\end{theorem}


An important special case occurs when $f$ is the inclusion of a submanifold $X$ of $Y$ and $Z$ is another submanifold of $Y$. In this case the condition of transversality reduces to
$$
T_{x} X+T_{x} Z=T_{x} Y
$$
for every $x \in X \cap Z$. If this is the case, then $X \cap Z$ is a submanifold of codimension given by
$$
\operatorname{codim}(X \cap Z)=\operatorname{codim} X+\operatorname{codim} Z.
$$

% \subsection{Manifolds with Boundary}

% Consider the closed half-space
% $$
% \mathbb{H}^{k}:=\left\{\left(x_{1}, \ldots, x_{k}\right) \in \mathbb{R}^{k}: x_{k} \geq 0\right\}.
% $$
% The boundary $\partial \mathbb{H}^{n}$ is defined to be the hyperplane $x_{k}=0$ in $\mathbb{R}^{k}$.

% \begin{definition}
%     A subset $X \subset \mathbb{R}^{N}$ is called a smooth $k$-manifold with boundary if each $x \in X$ has a neighbourhood diffeomorphic to an open set in $\mathbb{H}^{k}$. As before, such a diffeomorphism is called a chart on $X$. The boundary of $X$, denoted $\partial X$, is given by the set of points that belong to the image of $\partial \mathbb{H}^{k}$ under some local parametrization. Its complement is called the interior of $X$, $\operatorname{Int}(X)=X-\partial X$.
% \end{definition}

% \begin{remark}[Warning]
%     Do not confuse the boundary or interior of $X$ as defined above with the topological notions of interior and boundary as a subset of $\mathbb{R}^{N}$.
% \end{remark}


% The tangent space is defined as before, so that $T_{x} X$ is a $k$-dimensional vector space even for points $x \in \partial X$. The interior of $X$ is a $k$-manifold without boundary and $\partial X$ is a manifold without boundary of dimension $k-1$ (this requires a proof!).

% Here is an easy way of generating examples.

% \begin{lemma}
%     Let $X$ be a manifold without boundary and let $f: X \rightarrow \mathbb{R}$ be $a$ smooth function with 0 as a regular value. Then the subset $\{x \in X: f(x) \geq 0\}$ is a smooth manifold with boundary equal to $f^{-1}(0)$.
% \end{lemma}
% \begin{proof}
%     The set where $f>0$ is open in $X$ and is therefore a submanifold of the same dimension as $X$. For a point $x \in X$ with $f(x)=0$, the same proof of the Preimage Theorem $1.13$ shows that $x$ has a neighbourhood diffeomorphic to a neighbourhood of a point in $\mathbb{H}^{k}$.
% \end{proof}

% As an easy application of the lemma, consider the unit ball $B^{k}$ given by all $x \in \mathbb{R}^{k}$ such that $|x| \leq 1$. By considering the function $f(x)=1-|x|^{2}$ it follows that $B^{k}$ is a smooth manifold with boundary $S^{k-1}$.

% \begin{theorem}
%     Let $f: X \rightarrow Y$ be a smooth map from an m-manifold with boundary to an $n$-manifold, where $m>n$. If $y$ is a regular value, both for $f$ and for the restriction of $f$ to $\partial X$, then $f^{-1}(y)$ is a smooth $(m-n)$-manifold with boundary equal to $f^{-1}(y) \cap \partial X$
% \end{theorem}
% \begin{proof}
%     Recall that being a submanifold is a local property so without loss of generality we can suppose that $f: \mathbb{H}^{m} \rightarrow \mathbb{R}^{n}$ with $y \in \mathbb{R}^{n}$ a regular value. Consider $z \in f^{-1}(y)$. If $z$ belongs to the interior of $\mathbb{H}^{m}$, then as in the Preimage theorem $1.13 f^{-1}(y)$ is a smooth manifold near $z$.

% Let $z$ be now in $\partial \mathbb{H}^{m}$. Since $f$ is smooth, there is a neighbourhood $U$ of $z$ in $\mathbb{R}^{m}$ and a smooth map $F: U \rightarrow \mathbb{R}^{n}$ such that $F$ restricted to $U \cap \mathbb{H}^{m}$ is $f$. By shrinking $U$ if necessary we can assume that $F$ has no critical points in $U$ (why?). Hence $F^{-1}(y)$ is a smooth manifold of dimension $m-n$.

% Let $\pi: F^{-1}(y) \rightarrow \mathbb{R}$ be the projection $\left(x_{1}, \ldots, x_{m}\right) \mapsto x_{m}$. Observe that the tangent space of $F^{-1}(y)$ at a point $x \in \pi^{-1}(0)$ is equal to the kernel of
% $$
% d F_{x}=d f_{x}: \mathbb{R}^{m} \rightarrow \mathbb{R}^{n}.
% $$
% Hence 0 must be a regular value of $\pi$ since we are assuming that $y$ is a regular value of $f$ restricted to $\partial \mathbb{H}^{m}$.

% But $F^{-1}(y) \cap \mathbb{H}^{m}=f^{-1}(y) \cap U$ is the set of all $x \in F^{-1}(y)$ with $\pi(x) \geq 0$ and by Lemma $1.25$, is a smooth manifold with boundary equal to $\pi^{-1}(0)$.
% \end{proof}

% It is not hard to guess what is the appropriate version of this theorem for the more general case of a map $f: X \rightarrow Y$ and a submanifold $Z$ of $Y$. Suppose $X$ has boundary, but $Y$ and $Z$ are boundaryless. The next theorem is stated without proof.

% \begin{theorem}
%     Suppose that both $f: X \rightarrow Y$ and $f \mid \partial X: \partial X \rightarrow Y$ are transversal to $Z$. Then $f^{-1}(Z)$ is a manifold with boundary given by $f^{-1}(Z) \cap \partial X$ and codimension equal to the codimension of $Z$.
% \end{theorem}

\subsection{Degree Modulo 2}

\begin{lemma}[Homotopy Lemma]
    Let $f, g: X \rightarrow Y$ be smooth maps which are smoothly homotopic. Suppose $X$ is compact, has the same dimension as $Y$ and $\partial X=\emptyset$. If $y$ is a regular value for both $f$ and $g$, then
$$
\# f^{-1}(y)=\# g^{-1}(y) \pmod{2}.
$$
\end{lemma}

\begin{lemma}[Homogeneity Lemma]
    Let $X$ be a smooth connected manifold, possibly with boundary. Let $y$ and $z$ be points in $\operatorname{Int}(X)$. Then there exists a diffeomorphism $h: X \rightarrow X$ smoothly isotopic to the identity such that $h(y)=z$. 
\end{lemma}

In what follows suppose that $X$ is compact and without boundary and $Y$ is connected and with the same dimension as $X$. Let $f: X \rightarrow Y$ be a smooth map.

\begin{theorem}[Degree Mod 2]
    If $y$ and $z$ are regular values of $f$ then
    $$
    \# f^{-1}(y)=\# f^{-1}(z) \pmod 2.
    $$
    This common residue class is called the \vocab{degree mod 2} of $f$, $ \deg_2 f$, and only depends on the homotopy class of $f$.
\end{theorem}

\begin{corollary}[Smooth Brouwer Fixed Point Theorem]
    Any smooth map $f$ : $B^{k} \rightarrow B^{k}$ has a fixed point.
\end{corollary}
\begin{proof}
    Suppose $f$ has no fixed point. Define $g: B^k \rightarrow S^{k - 1}$ so that $g(x)$ is the point where the line segment starting at $f(x)$ passing through $x$ hits the boundary. This is obviously smooth, and restricts to the identity on $S^{k - 1}$.

    Now the identity map on a compact boundaryless manifold has $\deg_2 = 1$, and the constant map has $\deg_2 = 0$. So they are never homotopic. This implies that there is no smooth map $f: B^{k} \rightarrow S^{k - 1}$ which restricts to the identity on $S^{k - 1}$, as otherwise we could construct a homotopy $H: S^k \times [0, 1] \rightarrow S^k$ between the constant map and the identity given by $H(x, t) = f(t x)$. So $f$ must have a fixed point. 
\end{proof}



\section{Length, Area and Curvature}

\subsection{Curves}

\begin{definition}[Curve]
    Let $I \subset \mathbb{R}$ be an interval and let $X$ be a manifold. A \vocab{curve} in $X$ is a smooth map $\alpha: I \rightarrow X$. The curve is said to be \vocab{regular} if $\alpha$ is an immersion, i.e., if the velocity vector $\dot{\alpha}(t) =d \alpha_t(1) \in T_{\alpha(t)} X$ is never zero.
\end{definition}

By definition, given $t \in I$, the arc-length of $\alpha: I \rightarrow \mathbb{R}^3$ from the point $t_0$ is given by
$$
s(t):=\int_{t_0}^t|\dot{\alpha}(\tau)| d \tau .
$$
If the interval $I$ has endpoints $a$ and $b, a<b$, the length of $\alpha$ is
$$
\ell(\alpha):=\int_a^b|\dot{\alpha}(t)| d t .
$$
The curve is said to be parametrized by arc-length if $|\dot{\alpha}(t)|=1$ for all $t \in$ I. From now on we will assume that curves are parametrized by arc-length.

\begin{definition}
    The \vocab{tangent} at $s\in I$ is $t(s) = \dot{\alpha}(s)$.
    The \vocab{curvature} of $\alpha$ at $s$ is the number $k(s)=|\ddot{\alpha}(s)|$. 
    The \vocab{normal vector} at $s$ is $n(s)$, where $\ddot{\alpha}(s)=k(s) n(s)$. 
    The \vocab{binormal vector} at $s$ is $b(s) = t(s) \wedge n(s)$. We have $\dot{b}(s) = \tau(s) n(s)$, where $\tau(s)$ is the \vocab{torsion} at $s$.
\end{definition}

\begin{proposition}[Frenet Formulas]
    We have
    $$
\dot{t}  =k n, \quad 
\dot{n}  =-k t-\tau b, \quad \text{and} \quad
\dot{b}  =\tau n
$$
\end{proposition}

\begin{theorem}[Fundamental Theorem of the Local Theory of Curves]
    Given smooth functions $k(s)>0$ and $\tau(s), s \in I$, there exists a regular curve $\alpha: I \rightarrow \mathbb{R}^3$ such that $s$ is arc-length, $k(s)$ is the curvature, and $\tau(s)$ is the torsion of $\alpha$. Moreover any other curve $\bar{\alpha}$, satisfying the same conditions, differs from $\alpha$ by an isometry.
\end{theorem}

\subsection{Isoperimetric Inequality}

\begin{lemma}[Wirtinger's Inequality]
    Let $f: \R \rightarrow \R$ be a $C^1$ function which is periodic with period $L$. Suppose $\int_0^L f(t) \dd t = 0$. Then
    $$
    \int_0^L |f'(t)|^2 \dd t \geq \frac{4 \pi ^2}{L^2} \int_0^L |f(t)|^2 \dd t,
    $$
    with equality if and only if there exist constants $a_{\pm 1}$ such that $f(t) = a_{-1} e^{-2 \pi i t/L} + a_1 e^{2 \pi i t/L}$.
\end{lemma}


\begin{theorem}[Isoperimetric Inequality in the Plane]
    Let $\Omega$ be a domain, that is, a connected open set. We assume that $\Omega$ has compact closure and that its boundary $\partial \Omega$ is a connected 1-manifold of class $C^1$. Let $A(\Omega)$ be the area of $\Omega$. Then
$$
    \ell^2(\partial \Omega) \geq 4 \pi A(\Omega)
$$
with equality if and only if $\Omega$ is a disk.
\end{theorem}
\begin{proof}
    Define the vector field $X(x, y) = (x, y)$, and let $n$ be the outward pointing normal vector field along $\partial \Omega$. The divergence theorem gives us that
    $$
    \int_{\Omega} \operatorname{div} X \dd A = \int_{\partial \Omega} \langle X, n \rangle \dd s.
    $$
    But $\operatorname{div}(X) = 2$, so by Cauchy-Schwarz we have
    \begin{align*}
        2A(\Omega) \leq \int_{\partial \Omega} |X| \dd s.
    \end{align*}
    By Cauchy-Schwarz again we have
    \begin{align*}
        2A(\Omega) &\leq \left(\int_{\partial \Omega} |X|^2 \dd s \right)^{1/2}
        \left(\int_{\partial \Omega} \dd s\right)^{1/2} \\
&= \ell(\partial \Omega)^{1/2} \left(\int_{\partial \Omega} |X|^2 \dd s \right)^{1/2}.
    \end{align*}
Since we parameterise $\partial \Omega$ by arc length, $X(s) = (x(s), y(s))$ along $\partial \Omega$ are $C^1$ and periodic with period $L = \ell(\partial \Omega)$. Hence by Wirtinger's inequality we have
\begin{align*}
    \left(\int_{\partial \Omega} |X|^2 \dd s \right)^{1/2} &\leq \left(\frac{\ell(\partial \Omega)^2}{4 \pi^2} \int_{\partial \Omega}\left|X^{\prime}\right|^2 \mathrm{ds}\right)^{1 / 2}\\
    &=\frac{\ell(\partial \Omega)^{3 / 2}}{2 \pi}  
\end{align*}
which gives the desired result. Equality occurs if and only if we have equality in all of the above, in particular in the second $s \mapsto |X(s)|$ is constant, so $\Omega$ is a disk.
\end{proof}

  
\subsection{First Fundamental Form}

\begin{definition}[First Fundamental Form]
Let $S \subset \mathbb{R}^3$ be a surface. The quadratic form $I_p$ on $T_p S$ given by
$$
I_p(w):=\langle w, w\rangle=|w|^2
$$
is called the \vocab{first fundamental form} of the surface at $p$.
\end{definition}

\begin{definition}
    Two surfaces $S_1$ and $S_2$ are said to be \vocab{isometric} if there exists a diffeomorphism $f: S_1 \rightarrow S_2$ such that for all $p \in S_1, d f_p$ is a linear isometry between $T_p S_1$ and $T_{f(p)} S_2$.
\end{definition}


% SHIT ABOUT LOCAL COORDINATES 

Let $\phi: U \subset \mathbb{R}^2 \rightarrow S \subset \mathbb{R}^3$ be a parametrization of a neighbourhood of a point $p \in S$. We will denote by $(u, v)$ points in $U$ and let
$$
\begin{aligned}
& \phi_u(u, v)=\frac{\partial \phi}{\partial u}(u, v) \in T_{\phi(u, v)} S, \\
& \phi_v(u, v)=\frac{\partial \phi}{\partial v}(u, v) \in T_{\phi(u, v)} S .
\end{aligned}
$$
Note these are linearly independent. Set
$$
\begin{aligned}
E&=\left\langle\phi_u, \phi_u\right\rangle_{\phi(u, v)}, \\
F&=\left\langle\phi_u, \phi_v\right\rangle_{\phi(u, v)}, \\
G&=\left\langle\phi_v, \phi_v\right\rangle_{\phi(u, v)} .
\end{aligned}
$$
Since a tangent vector $w \in T_p S$ is the tangent vector of a curve $\alpha(t)=\phi(u(t), v(t))$, $t \in(-\varepsilon, \varepsilon)$, with $p=\alpha(0)=\phi\left(u_0, v_0\right)$ we have
$$
\begin{aligned}
I_p(\dot{\alpha}(0)) & =\langle\dot{\alpha}(0), \dot{\alpha}(0)\rangle_p \\
& =E(\dot{u})^2+2 F \dot{u} \dot{v}+G(\dot{v})^2 .
\end{aligned}
$$

We can compute the length of a curve in $S$ then by integrating $\sqrt{E(\dot{u})^2+2 F \dot{u} \dot{v}+G(\dot{v})^2}$.
Note also that $|\phi_u \wedge \phi_v| = \sqrt{EG - F^2}$.


\begin{definition}[Area]
    Let $\Omega \subset S$ be a bounded domain contained in the image of a parametrization $\phi: U \rightarrow S$. The positive number
    $$
    A(\Omega)=\int_{\phi^{-1}(\Omega)}\left|\phi_u \wedge \phi_v\right| \dd u \dd v
    $$
    is called the area of $\Omega$.
\end{definition}

\subsection{The Gauss Map}

Given a parametrization $\phi: U \subset \mathbb{R}^2 \rightarrow S \subset \mathbb{R}^3$ around a point $p \in S$, we can choose a unit normal vector at each point of $\phi(U)$ by setting
$$
N(q)=\frac{\phi_u \wedge \phi_v}{\left|\phi_u \wedge \phi_v\right|}(q) .
$$

\begin{definition}[Orientable]
    A surface $S \subset \mathbb{R}^3$ is \vocab{orientable} if it admits a smooth field of unit normal vectors. The choice of such a field is called an \vocab{orientation}.
\end{definition}

\begin{definition}[Gauss Map]
    Let $S$ be an oriented surface and $N: S \rightarrow S^2$ the smooth field of unit normal vectors defining the orientation. The map $N$ is called the \vocab{Gauss map} of $S$.
\end{definition}

Since $T_p S$ and $T_{N(p)} S^2$ are parallel planes, we will regard $d N_p$ as a linear map $d N_p: T_p S \rightarrow T_p S$. 

\begin{proposition}
    The linear map $d N_p: T_p S \rightarrow T_p S$ is self-adjoint.
\end{proposition}
\begin{proof}
    Let $\phi: U \rightarrow S$ be a parametrization around $p$. If $\alpha(t)=\phi(u(t), v(t))$ is a curve in $\phi(U)$ with $\alpha(0)=p$ we have
    $$
    \begin{aligned}
    d N_p(\dot{\alpha}(0)) & =d N_p\left(\dot{u}(0) \phi_u+\dot{v}(0) \phi_v\right) \\
    & =\left.\frac{d}{d t}\right|_{t=0} N(u(t), v(t)) \\
    & =\dot{u}(0) N_u+\dot{v}(0) N_v
    \end{aligned}
    $$
    In particular $d N_p\left(\phi_u\right)=N_u$ and $d N_p\left(\phi_v\right)=N_v$ and since $\left\{\phi_u, \phi_v\right\}$ is a basis of the tangent plane, we only have to prove that
    $$
    \left\langle N_u, \phi_v\right\rangle=\left\langle N_v, \phi_u\right\rangle
    $$
    To prove the last equality, observe that $\left\langle N, \phi_u\right\rangle=\left\langle N, \phi_v\right\rangle=0$. Take derivatives with respect to $v$ and $u$ to obtain:
    $$
    \begin{aligned}
    & \left\langle N_v, \phi_u\right\rangle+\left\langle N, \phi_{u v}\right\rangle=0, \\
    & \left\langle N_u, \phi_v\right\rangle+\left\langle N, \phi_{v u}\right\rangle=0
    \end{aligned}
    $$
    which gives the desired equality.
\end{proof}

\begin{definition}[Second Fundamental Form]
    The quadratic form defined on $T_p S$ by $II_p(w)=-\left\langle d N_p(w), w\right\rangle$ is called the \vocab{second fundamental form} of $S$ at $p$.
\end{definition}

\begin{definition}[Normal Curvature]
    Let $\alpha: (-\varepsilon, \varepsilon) \rightarrow S$ be a curve, $\alpha(0) = p$. Then the \vocab{normal curvature} of $\alpha$ at $p$ is defined by $k_n(p) = \langle N, k n\rangle$ where $N$ is the Gauss map, $k$ is the curvature of $\alpha$ and $n$ is the unit normal to $\alpha$ at $p$ (i.e. $kn = \ddot \alpha$). 
\end{definition}

\begin{proposition}
    $k_n(p) = II_p(\dot\alpha(0))$.
\end{proposition}

\begin{definition}[Principal Curvatures and Directions]
    As $dN_p: T_p S \rightarrow T_p S$ is self adjoint, it can be diagonalised. Let $e_1, e_2 \in T_p S$ be such that, with respect to this basis, we have
$$
d N_p=\begin{pmatrix}
-k_1 & 0 \\
0 & -k_2
\end{pmatrix}
$$
where $k_1 \geq k_2$. We call $k_1, k_2$ the \vocab{princial curvatures}, and $e_1, e_2$ the \vocab{principal directions}.
\end{definition}

From standard linear algebra we get that $k_1$ (respectively $k_2$) is the maximum (minimum) value of $II_p$ on the set of unit vectors in $T_p S$. That is, they are the extreme values of the normal curvature at $p$.

\begin{definition}[Gaussian and Mean Curvature]
    The determinant of $d N_p$ is the \vocab{Gaussian curvature} $K(p)$ of $S$ at $p$. Minus half of the trace of $d N_p$ is the \vocab{mean curvature} $H(p)$ of $S$ at $p$.
\end{definition}

Clearly $K=k_1 k_2$ and $H=\frac{k_1+k_2}{2}$. 

A point $p \in S$ of a surface is called \vocab{elliptic} if $K(p)>0$, \vocab{hyperbolic} if $K(p)<0$, \vocab{parabolic} if $K(p)=0$ and $d N_p \neq 0$, and \vocab{planar} if $d N_p=0$.
A point $p \in S$ is called \vocab{umbilical} if $k_1=k_2$.

\subsection{Local Coordinates}

Let $\phi: U \rightarrow S$ be a parametrization around a point $p \in S$. Let us express the second fundamental form in the basis $\left\{\phi_u, \phi_v\right\}$. Since $\left\langle N, \phi_u\right\rangle=\left\langle N, \phi_v\right\rangle=0$ we have
\begin{align*}
 e&=-\left\langle N_u, \phi_u\right\rangle=\left\langle N, \phi_{u u}\right\rangle, \\
 f&=-\left\langle N_v, \phi_u\right\rangle=\left\langle N, \phi_{u v}\right\rangle=-\left\langle N_u, \phi_v\right\rangle, \\
 g&=-\left\langle N_v, \phi_v\right\rangle=\left\langle N, \phi_{v v}\right\rangle .
\end{align*}

If $\alpha$ is a curve passing at $t=0$ through $p$ we can write:
\begin{align*}
    I I_p(\dot{\alpha}(0)) &=-\left\langle d N_p(\dot{\alpha}(0)), \dot{\alpha}(0)\right\rangle\\
    &=e(\dot{u})^2+2 f \dot{u} \dot{v}+g(\dot{v})^2 .
\end{align*}

With respect to the basis $\phi_u, \phi_v$, we can express $\mathrm{d} N_p$ as a matrix, namely
$$
\begin{aligned}
& \mathrm{d} N_p\left(\phi_u\right)=N_u=a_{11} \phi_u+a_{21} \phi_v \\
& \mathrm{~d} N_p\left(\phi_v\right)=N_v=a_{12} \phi_u+a_{22} \phi_v
\end{aligned}
$$
Taking inner products of the above equations with $\phi_u, \phi_v$ we get
$$
\begin{pmatrix}
E & F \\
F & G
\end{pmatrix}\begin{pmatrix}
a_{11} & a_{12} \\
a_{21} & a_{22}
\end{pmatrix}=-\begin{pmatrix}
e & f \\
f & g
\end{pmatrix}
$$
But with respect to the basis $\phi_u, \phi_v, \mathrm{~d} N_p$ has matrix $\begin{pmatrix}a_{11} & a_{12} \\ a_{21} & a_{22}\end{pmatrix}$. Linear algebra then gives
\begin{corollary}
    We can write
    $$
    K=\frac{e g-f^2}{E G-F^2}, \text { and } H=\frac{e G-2 f F+g E}{2\left(E G-F^2\right)}.
    $$
\end{corollary}

\subsection{Theorema Egregium}

\begin{theorem}[Theorema Egregium]
    The Gaussian curvature $K$ of a surface is invariant under isometries.
\end{theorem}
\begin{proof}
    It suffices to write $K$ in terms only of the coefficients $E, F, G$ of the first fundamental form and their derivatives. Let $\phi: U \rightarrow S$ be a parameterisation. Then at each point in the image we have a basis of $\R^3$ given by $\{\phi_u, \phi_v, N\}$. We can then express the derivatives of $\phi_u$ and $\phi_v$ in this basis:
    \begin{align*}
            \phi_{u u} & =\Gamma_{11}^1 \phi_u+\Gamma_{11}^2 \phi_v+e N, \\
            \phi_{u v} & =\Gamma_{12}^1 \phi_u+\Gamma_{12}^2 \phi_v+f N, \\
            \phi_{v u} & =\Gamma_{21}^1 \phi_u+\Gamma_{21}^2 \phi_v+f N, \\
            \phi_{v v} & =\Gamma_{22}^1 \phi_u+\Gamma_{22}^2 \phi_v+g N,
    \end{align*}
    where the $\Gamma_{ij}^k$ are the \vocab{Christoffel symbols}.
    Taking inner products with $\phi_u$ and $\phi_v$, we can see that we can solve for the Christoffel symbols in terms of $E$, $F$, $G$ and their derivatives. So Christoffel symbols are invariant under isometries.

    Consider $\phi_{uuv} = \phi_{uvu}$, and differentiating our previous expressions and substituting in gives (after some manipulation)
    \begin{align*}
            \left(\Gamma_{12}^2\right)_u  -\left(\Gamma_{11}^2\right)_v+
            \Gamma_{12}^1 \Gamma_{11}^2+\\
            \Gamma_{12}^2 \Gamma_{12}^2-\Gamma_{11}^2 \Gamma_{22}^2-\Gamma_{11}^1 \Gamma_{12}^2 \\
             =-f a_{21}+e a_{22}=-E \frac{e g-f^2}{E G-F^2}=-E K .
    \end{align*}
    Thus $K$ can be expressed solely in terms of the coefficients of the first fundamental form and their derivatives as required.
\end{proof}


\begin{definition}[Isothermal Parameterisation]
    A parameterization is \vocab{isothermal} if $E = G  = \lambda^2(u, v)$ and $F = 0$.
\end{definition}

\begin{proposition}
    For isothermal parameterization, $K = -\frac{1}{\lambda^2} \Delta(\log \lambda)$, where $\Delta$ is the Laplacian in $(u, v)$-coordinates.
\end{proposition}

\section{Geodesics \& Minimal Surfaces}

\subsection{Geodesics}

Let $S \subseteq \R^3$ be a surface with $p, q \in S$. Let $\Omega(p, q)$ be the set of all curves $\alpha: [0, 1] \rightarrow S$ with $\alpha(0) = p$ and $\alpha(1) = q$.

\begin{definition}[Energy Functional]
The \vocab{energy} $E: \Omega(p, q) \rightarrow \R$ is given by
$$
E(\alpha) = \frac{1}{2} \int_0^1 |\dot\alpha|^2 \dd t.
$$
\end{definition}

Let $\alpha_s \in \Omega(p, q)$ be a smooth one parameter family of curves, with $s \in(-\varepsilon, \varepsilon)$. Let $E(s)=E\left(\alpha_s\right)$. Then we have that
$$
\frac{\mathrm{d} E}{\mathrm{~d} s}=\int_0^1\left\langle\frac{\partial}{\partial s} \frac{\partial \alpha_s}{\partial t}, \frac{\partial \alpha_s}{\partial t}\right\rangle \mathrm{d} t
$$
Integrating by parts we get
\begin{align*}
\left.\frac{\mathrm{d} E}{\mathrm{~d} s}\right|_{s=0}&=\langle J(1), \dot{\alpha}(1)\rangle-\langle J(0), \dot{\alpha}(0)\rangle\\&-\int_0^1\langle J(t), \ddot{\alpha}(t)\rangle \mathrm{d} t
\end{align*}
where
$$
J(t)=\left.\frac{\partial \alpha_s(t)}{\partial s}\right|_{s=0}
$$
Since $\alpha_s \in \Omega(p, q), J(0)=J(1)=0$. So we get that
$$
\left.\frac{\mathrm{d} E}{\mathrm{ds}}\right|_{s=0}=-\int_0^1\langle J(t), \ddot{\alpha}(t)\rangle \mathrm{d} t
$$
Now notice that for each $t \in[0,1], J(t) \in T_{\alpha(t)} S$, since $s \mapsto \alpha_s(t)$ is a curve in $s$. So if $\alpha$ is such that $\ddot{\alpha} \perp T_{\alpha(t)} S$ for all $t$, then $\alpha$ extremises $E$.

\begin{definition}[Geodesic]
    A curve $\alpha : I \rightarrow S$ is a \vocab{geodesic} if for all $t \in I$, $\ddot{\alpha}(t)$ is orthogonal to $T_{\alpha(t)} S$.
\end{definition}

\subsection{Covariant Derivative}

\begin{definition}[Vector Field]
    Let $\alpha: I \rightarrow S$ be a curve. A \vocab{vector field} along $\alpha$ is a smooth map $V: I \rightarrow \mathbb{R}^3$ such that for all $t$, $V(t) \in T_{\alpha(t)} S$.
\end{definition}

\begin{definition}[Covariant Derivative]
    The \vocab{covariant derivative} of a vector field $V$ along $\alpha$ is
$$
\frac{\mathrm{D} V}{\mathrm{~d} t}(t)=\operatorname{proj}_{T_{a|t\rangle} S}\left(\frac{\mathrm{d} V}{\mathrm{~d} t}\right)
$$
where $\operatorname{proj}_{T_{\alpha(t)}}$ is the orthogonal projection onto $T_{\alpha(t)} S$.
\end{definition}

\begin{proposition}
    A curve $\alpha$ is a geodesic if and only if $\frac{\mathrm{D} \dot{\alpha}}{\mathrm{d} t}=0$ for all $t$.
\end{proposition}

\begin{definition}[Parallel]
    A vector field $V$ along $\alpha$ is \vocab{parallel} if $\frac{\mathrm{D} V}{\mathrm{~d} t}=0$.
\end{definition}

\begin{proposition}
    Let $V, W$ be parallel vector fields along $\alpha$. Then $\langle V(t), W(t)\rangle$ is constant.
\end{proposition}

\begin{corollary}
If $\alpha$ is a geodesic, then $|\dot{\alpha}|$ is constant. So geodesics are parametrised proportional to arc length.
\end{corollary}

\subsection{Local Coordinates}

Let $\phi: U \rightarrow S$ be a parametrisation, $\alpha: I \rightarrow S$ a curve, with $\alpha(/) \subseteq \phi(U)$. Write $\alpha(t)=\phi(u(t), v(t))$. Let $V$ be a vector field along $\alpha$. Then there are functions $a(t), b(t)$ such that
$$
V(t)=a(t) \phi_u+b(t) \phi_v
$$
Differentiating this, we get that
$$
\frac{\mathrm{d} V}{\mathrm{~d} t}=a\left(\phi_{u v} \dot{u}+\phi_{u v} \dot{v}\right)+b\left(\phi_{v u} \dot{u}+\phi_{v v} \dot{v}\right)+a \phi_u+b \phi_v
$$
The covariant derivative is just the $\phi_j$ and $\phi_v$ components of this, since $N$ is orthogonal to $T_{a(t)} S$. Therefore, in terms of Christoffel symbols, we have that
\begin{align*}
\frac{\mathrm{DV}}{\mathrm{d} t}= &\left(\dot{a}+a \dot{u} \Gamma_{11}^1+a \dot{v} \Gamma_{12}^1+b \dot{u} \Gamma_{12}^1+b \dot{v} \Gamma_{22}^1\right) \phi_u\\+&\left(b+a \dot{u} \Gamma_{11}^2+a \dot{v} \Gamma_{12}^2+b \dot{u} \Gamma_{12}^2+b \dot{v} \Gamma_{22}^2\right) \phi_v
\end{align*}
From this expression, we see that the covariant derivative only depends on the first fundamental form.

\begin{proposition}[Geodesic Equations]
$\alpha(t)=\phi(u(t), v(t))$ is a geodesic if and only if
$$
\begin{aligned}
\ddot{u}+\Gamma_{11}^1 \dot{u}^2+2 \Gamma_{12}^1 \dot{u} \dot{v}+\Gamma_{22}^1 \dot{v}^2&=0 \\
 \dot{v}+\Gamma_{11}^2 \dot{u}^2+2 \Gamma_{12}^2 \dot{u} \dot{v}+\Gamma_{22}^2 \dot{v}^2&=0
\end{aligned}
$$
\end{proposition}

\begin{proposition}[Parallel Transport]
    Given $v_0 \in T_{\alpha\left(t_0\right)} S$, there exists a unique parallel vector field $V(t)$ along $\alpha(t)$, with $V\left(t_0\right)=v_0$. We call $V\left(t_1\right)$ the \vocab{parallel transport} of $v_0$ along $\alpha$ at $t_1$.
\end{proposition}

\begin{corollary}[Geodesic Existence]
    Given $p \in S, v \in T_\rho S$, there exists $\varepsilon>0$, and a unique geodesic $\gamma:(-\varepsilon, \varepsilon) \rightarrow S$ such that $\gamma(0)=p$ and $\dot{\gamma}(0)=v$
\end{corollary}

\begin{definition}
    Let $\alpha \in \Omega(p, q)$. Define $P: T_p S \rightarrow T_q S$ the map sending $v \in T_p S$ to the \vocab{parallel transport} of $v$ along $\alpha$ at $q$.
\end{definition}

\subsection{Exponential Map}

\begin{proposition}
Given $p \in S, v \in T_P S$, let $\gamma_v:(-\varepsilon, \varepsilon) \rightarrow S$ by the unique geodesic with $\gamma(0)=p$ and $\dot{\gamma}(0)=v$. Then $\gamma_{\lambda v}$ is defined on $(-\varepsilon / \lambda, \varepsilon / \lambda)$. Furthermore, $\gamma_{\lambda v}(t)=\gamma_v(\lambda t)$.
\end{proposition}

\begin{definition}[Exponential Map]
    Suppose $v \in T_p S$ nonzero is such that $\gamma_v(1)$ is defined, we define the \vocab{exponential map} $\exp_p(v) = \gamma_v(1)$.
\end{definition}

We note exists $\varepsilon>0$ such that $\exp _p: B_{\varepsilon}(0) \rightarrow S$ is well defined and smooth.

\begin{proposition}
    If $S$ is closed, then $\exp _p$ is defined on all of $T_p S$.
\end{proposition}

\begin{proposition}
    $\exp _p: B_{\varepsilon}(0) \rightarrow S$ is a diffeomorphism onto its image in a neighbourhood $U \subseteq B_{\varepsilon}(0)$ of $0 \in T_p S$
\end{proposition}
\begin{proof}
    By the inverse function theorem, suffices to show $d\left(\exp _p\right)_0$ is nonsingular. Let $\alpha(t)=t v$ for some fixed $v \in T_p S$. Then $\exp _p(t v)=\gamma_v(t)$ at $t=0$ has tangent vector $v$. So $d\left(\exp _p\right)_0(v)=v$.
\end{proof}

\begin{definition}[Normal Neighbourhood]
    Let $U$ be as in the previous proposition. Then $V=\exp _p(U)$ is called a \vocab{normal neighbourhood} of $p$.
\end{definition}

\begin{corollary}
    $\exp _p: U \rightarrow V$ is a parametrisation.
\end{corollary}

\begin{proposition}
If we choose cartesian coordinates on $T_p S$, then with the $\exp _p$ parametrisation, we have the first fundamental form
$$
E(p)=G(p)=1 \text { and } F(p)=0
$$
\end{proposition}

\begin{definition}[Geodesic Polars]
    If we choose polar coordinates $(r, \theta)$ for $T_p S$, then we have the \vocab{geodesic polar coordinates}. That is,
\begin{align*}
    \phi(r, \theta)&=\exp _p\left(r\left(\cos (\theta) e_1+\sin (\theta) e_2\right)\right)\\&=\exp _p(r v(\theta))=\gamma_{v(\theta)}(t)
\end{align*}
where $v(\theta)=\cos (\theta) e_1+\sin (\theta) e_2$.
\end{definition}

\begin{definition}[Geodesic Circles, Radial Geodesics]
    The images of circles centred in the origin under the map $\phi$ are called \vocab{geodesic circles} (i.e. $r=$ const). Similarly, the images of lines through the origin (i.e. $\theta=$ const) are called \vocab{radial geodesics}.
\end{definition}

\begin{proposition}
For geodesic polars we have
\begin{align*}
E=1, \quad F=0, \quad G(0, \theta)=0 \\
\text {and } \quad(\sqrt{G})_r(0, \theta)=1
\end{align*}
Moreover, the Gaussian curvature can be written as
$$
K=-\frac{(\sqrt{G})_{r r}}{\sqrt{G}}
$$
\end{proposition}
\begin{proof}
By definition of $\phi$, we have that $\phi_r=\dot{\gamma}_{v(\theta)}(r)$, so $E=1$ as $v(\theta)$ is a unit vector and geodesics travel at constant speed. Now let $w=\frac{d v}{d \theta}$. Then by chain rule, we have that
$$
\phi_\theta=\mathrm{d}\left(\exp _p\right)_{r v}(r w)=r \mathrm{~d}\left(\exp _p\right)_{r v}(w)
$$
So we have that
$$
\begin{aligned}
& F=r\left\langle\dot{\gamma}_v(r), \mathrm{d}\left(\exp _p\right)_{r v}(w)\right\rangle \\
& G=r^2\left|\mathrm{~d}\left(\exp _p\right)_{r v}(w)\right|^2
\end{aligned}
$$
Clearly $F(0, \theta)=0$, and from the last equality, we find that
$$
(\sqrt{G})_r(0, \theta)=\left|\mathrm{d}\left(\exp _p\right)_0(w)\right|=|w|=1
$$
Finally, we can compute
$$
\begin{aligned}
F_r & =\left\langle\phi_{r r}, \phi_\theta\right\rangle+\left\langle\phi_r, \phi_{\theta r}\right\rangle \\
& =\left\langle\phi_r, \phi_{\theta r}\right\rangle \\
& =\frac{1}{2} \frac{\partial}{\partial \theta}\left\langle\phi_r, \phi_r\right\rangle \\
& =\frac{1}{2} E_\theta \\
& =0
\end{aligned}
$$
where we used the fact that $\phi(\cdot, \theta)=\gamma_v$ is a geodesic, so $\phi_{r r}=\ddot{\gamma}_v$ is normal to $T_p S$. So $F=0$ identically. We omit the computation for $K$, and note that it can be computed using Christoffel symbols.
\end{proof}

\subsection{Geodesic Curvature}

\begin{definition}[Algebraic Value of the Covariant Derivative]
    Let $W$ be a differentiable field fo unit vectors along a curve $\alpha: I \rightarrow S$ along an oriented surface $S$. Then
$$
\left[\frac{\mathrm{D} W}{\mathrm{d} t}\right]=\left\langle\frac{\mathrm{d} W}{\mathrm{d} t}, N \wedge W\right\rangle
$$
\end{definition}

\begin{proposition}
Let $W$ be a field of unit vectors along $\alpha$. Then $\frac{\mathrm{D} W}{\mathrm{d} t}$ is parallel to $N \wedge W$, and we have that
$$
\frac{\mathrm{D} W}{\mathrm{d} t}=\left[\frac{\mathrm{D} W}{\mathrm{d} t}\right](N \wedge W)
$$
\end{proposition}

\begin{definition}[Geodesic Curvature]
    Let $\alpha: I \rightarrow S$ be a regular curve parametrised by arc length. The algebraic value of the covariant derivative
$$
\kappa_g(s)=\left[\frac{\mathrm{D} \dot{\alpha}}{\mathrm{d} t}\right]=\langle\ddot{\alpha}, N \wedge \dot{\alpha}\rangle
$$
is called the \vocab{geodesic curvature} of $\alpha$ at $\alpha(\mathrm{s})$.
\end{definition}

\begin{proposition}
    $\alpha$ is a geodesic if and only if its geodesic curvature is identically zero.
\end{proposition}

\begin{proposition}
    Let $k$ and $n$ be the curvature and unit normal for $\alpha$. Then we have that
$$
\ddot{\alpha}=k_n N+k_g(N \wedge \dot{\alpha})
$$
where $\kappa_n, \kappa_g$ are the normal and geodesic curvatures respectively.
\end{proposition}
\begin{proof}
    Since $W$ has norm 1, we have that $\langle W, W\rangle=0$, so $\left\langle\frac{\mathrm{d} W}{\mathrm{~d} t}, W\right\rangle=0$. Hence $\frac{\mathrm{d} W}{\mathrm{~d} t}$ is perpendicular to $W$. Thus, $\frac{\mathrm{D} W}{\mathrm{~d} t}$ must be perpendicular to both $W$ and $N$, so it is parallel to $N \wedge W$.
\end{proof}

\begin{definition}[Perpendicular Vector Field]
    Let $V$ be a unit vector field along $\alpha: I \rightarrow S$. Let $i V(t)$ be the unique vector field along $\alpha$ such that for every $t \in I, V(t), i V(t), N(t)$ forms a positively oriented orthonormal basis of $\mathbb{R}^3$. That is,
$$
V(t) \wedge i V(t)=N(t)
$$
\end{definition}

\begin{proposition}
    Let $V, W$ be unit vector fields along $\alpha: I \rightarrow S$. Then there exists smooth functions $a, b$, such that
$$
W(t)=a(t) V(t)+b(t) i V(t)
$$
with $a^2+b^2=1$. Furthermore, if we fix $t_0 \in I$, and let $\varphi_0$ be such that
$$
a\left(t_0\right)=\cos \left(\varphi_0\right) \quad \text { and } \quad b\left(t_0\right)=\sin \left(\varphi_0\right)
$$
then there exists a smooth function $\varphi: l \rightarrow S$ such that
$$
a(t)=\cos (\varphi(t)), \quad b(t)=\sin (\varphi(t)) \quad \text { and } \quad \varphi\left(t_0\right)=\varphi_0
$$
\end{proposition}
\begin{proof}
    $V(t), i V(t)$ is an orthonormal basis of $T_{\alpha(t)} S$. The construction of $\varphi$ is as in the construction of the winding number in Complex Analysis.
\end{proof}

\begin{definition}[Smooth Determinantion of Angle]
    $\varphi$ from the previous proposition is called a \vocab{smooth determination of the angle} from $V$ to $W$.
\end{definition}

\begin{proposition}
Let $V, W$ be unit vector fields along $\alpha: I \rightarrow S$ and $\varphi$ by a smooth determination of angle from $V$ to $W$. Then
$$
\left[\frac{\mathrm{D} W}{\mathrm{~d} t}\right]-\left[\frac{\mathrm{D} V}{\mathrm{~d} t}\right]=\frac{\mathrm{d} \varphi}{\mathrm{d} t}
$$
\end{proposition}

\begin{proposition}
 Let $\alpha: I \rightarrow S$ be a curve parametrised by arc length, $V(s)$ a parallel unit vector field along $\alpha, \varphi$ a smooth determination of angle from $V$ to $\dot{\alpha}$. Then
$$
\kappa_g(s)=\frac{\mathrm{d} \varphi}{\mathrm{d} s}
$$
\end{proposition}
\begin{proof}
    $\left[\frac{\mathrm{D} V}{d t}\right]=0$ as $V$ is parallel.
\end{proof}

\section{Gauss-Bonnet}

\begin{theorem}[Gauss's Theorem for Geodesic Triangles]
    Let $T$ be a geodesic triangle on a surface $S$. Suppose $T$ is small enough so that it is contained in a normal neighbourhood of one of its vertices, then
$$
\int_T K \dd A=\alpha_1+\alpha_2+\alpha_3-\pi
$$
where $K$ is the Gaussian curvature of $S$, and $0<\alpha_i<\pi$ are the internal angles of $T$.
\end{theorem}
\begin{proof}
We can assume without loss of generality that we have geodesic polar coordinates centred at one of the vertices of $T$, one of the edges corresponds to $\theta=0$ and another corresponds to $\theta=\theta_0$. The remaining edge is a geodesic segment $\gamma$.

First notice that $\gamma$ can be written in the form $r=h(\theta)$. Suppose not, then there exists such that $\dot{\gamma}(s)$ is parallel to $\phi_r$. But radial segments are geodesics, so this means that $\gamma$ is radial. Contradiction. Hence we can write $\gamma$ as $r=h(\theta)$. Then
\begin{align*}
\int_T K \mathrm{~d} A&=\int_T K \sqrt{G} \mathrm{~d} r \mathrm{~d} \theta\\
&=\int_0^{\theta_0}\left(\lim _{\varepsilon \rightarrow 0} \int_{\varepsilon}^{h(\theta)} K \sqrt{G} \mathrm{~d} r\right) \mathrm{d} \theta
\end{align*}
But in geodesic polar coordinates, we have $K \sqrt{G}=-(\sqrt{G})_{r r}$, and $\lim _{r \rightarrow 0}(\sqrt{G})_r=1$, so
$$
\lim _{\varepsilon \rightarrow 0} \int_{\varepsilon}^{h(\theta)} K \sqrt{G} \mathrm{~d} r=1-(\sqrt{G})_r(h(\theta), \theta)
$$
Now suppose $\gamma(s)=\phi(r(s), \theta(s))$ makes an angle $\varphi(s)$ with $\phi_r$, that is, the curves $\theta=$ const. Then the previous corollary $(u=r, v=\theta)$ gives that
$$
(\sqrt{G})_r \frac{\mathrm{d} \theta}{\mathrm{d} s}+\frac{\mathrm{d} \varphi}{\mathrm{d} s}=0
$$
as $\gamma$ is a geodesic. Therefore, we have that
$$
\begin{aligned}
\int_T K \mathrm{~d} A & =\int_0^{\theta_0}\left(1-(\sqrt{G})_r(h(\theta), \theta)\right) \mathrm{d} \theta \\
& =\int_0^{\theta_0} \mathrm{~d} \theta-\int_0^{s_0}(\sqrt{G})_r \frac{\mathrm{d} \theta}{\mathrm{ds}} \mathrm{d} s \\
& =\theta_0+\int_0^{s_0} \frac{\mathrm{d} \varphi}{\mathrm{d} s} \mathrm{~d} s \\
& =\theta_0+\int_{\varphi(0)}^{\varphi\left(s_0\right)} \mathrm{d} \varphi \\
& =\theta_0+\varphi\left(s_0\right)-\varphi(0)
\end{aligned}
$$
Finally, by the orientations, we have the result.
\end{proof}

\begin{definition}[Triangulation]
    Let $S$ be a compact surface. A \vocab{triangulation} of $S$ is a finite number of closed subsets $T_1, \ldots, T_n$ which cover $S$, each $T_i$ is homeomorphic to a Euclidean triangle in the plane. Moreover, any two distinct triangles are either disjoint, share a vertex, or share an edge.
\end{definition}

\begin{theorem}
  Triangulations always exist. Furthermore, we can choose it so that each $T_i$ is diffeomorphic to a Euclidean triangle, and each edge is a geodesic segment.
\end{theorem}

\begin{definition}[Euler Characteristic]
    Given a triangulation of $S$, let $F$ be the number of faces, $E$ the number of edges, $V$ the number of vertices. Then
$$
x(S)=V-E+F
$$
is the \vocab{Euler characteristic} of $S$.
\end{definition}

This is independent of the choice of triangulation.

\begin{proposition}[Classification of Compact Orientable Surfaces]
All compact orientable surfaces are diffeomorphic to some $\Sigma_g$ where $g$ is a $g$-holed torus. $g$ is called the \vocab{genus} of $\Sigma_g$. Furthermore,
$$
\chi\left(\Sigma_g\right)=2-2 g
$$
\end{proposition}


\begin{theorem}[Global Gauss-Bonnet]
    Let $S$ be a compact surface without boundary. Then
$$
\int_S K d A=2 \pi \chi(S)
$$
\end{theorem}
\begin{proof}
    Consider a triangulation by geodesic triangles $T_1, \ldots, T_F$. We can assume wlog that each $T_i$ is contained in a normal neighbourhood of one of its vertices.
Let $\alpha_i, \beta_i, \gamma_i$ be the interior angles of $T_i$. Then by Gauss's theorem for triangles, we have that
$$
\int_{T_i} K \mathrm{~d} A=\alpha_i+\beta_i+\gamma_i-\pi
$$
Summing over all $i$, we have that
$$
\int_S K \mathrm{~d} A=\sum_{i=1}^F\left(\alpha_i+\beta_i+\gamma_i\right)-\pi F
$$
Now notice that the sum of the angles at every vertex is $2 \pi$, so
$$
\sum_{i=1}^F\left(\alpha_i+\beta_i+\gamma_i\right)=2 \pi V
$$
Finally, for a triangulation, every edge belongs to two triangles, so $2 E=3 F$. Putting this all together we get that
$$
\int_S K \mathrm{~d} A=\pi(2 V-F)=2 \pi \chi(S).
$$
\end{proof}


\begin{theorem}[Local Gauss-Bonnet]
Let $\phi: U \rightarrow S$ be an orthogonal parametrisation of an oriented surface $S, U$ is a disc in $\mathbb{R}^2$, and $\phi$ is compatible with the orientation of $S$. Let $\alpha: I \rightarrow \phi(U)$ be a smooth simple closed curve enclosing a domain $R$. Suppose $\alpha$ is positively oriented and parametrised by arc length. Then
$$
\int_l k_g(s) d s+\int_R K d A=2 \pi
$$
where $k_g$ is the geodesic curvature of $\alpha$.
\end{theorem}

\begin{theorem}[Gauss-Bonnet with Boundary]
Let $R \subseteq S$ be a connected open relatively compact\footnote{That is, the closure is compact.} subset. $^2$ Suppose $\partial R$ contains of $n$ piecewise smooth simple closed curves $\alpha_i: I_i \rightarrow S$, where the images do not intersect. Suppose the $\alpha_i$ are parametrised by arc length, and are positively oriented. Let $\theta_i$ be the external angles of the vertices of these curves. Then
    $$
    \sum_{i=1}^n \int_{l_i} k_g(s) \mathrm{d} s+\int_R K \mathrm{~d} A+\sum_i \theta_i=2 \pi \chi(R)
    $$
\end{theorem}

\begin{theorem}
    Suppose $S$ is a compact orientable surface with $K>0$. Then $S$ is diffeomorphic to $S^2$. Moreover, if $\alpha, \beta$ are simple closed geodesics on $S$, then they must intersect.
\end{theorem}
\begin{proof}
    Gauss-Bonnet gives us that $\chi(S)>0$, so $S$ is diffeomorphic to $S^2$. Now suppose $\alpha, \beta$ do not intersect. Then they bound a domain $R$ with $\chi(R)=0$. But then Gauss-Bonnet means that $R$ must in fact have measure zero. Contradiction.
\end{proof}

\subsection{Minimal Surfaces}

\begin{definition}[Minimal Surface]
    A surface $S$ is \vocab{minimal} if its mean curvature vanishes everywhere.
\end{definition}

\begin{definition}[Normal Variation]
    Let $\phi: U \rightarrow S$ be a parametrisation, $D \subseteq U$ bounded open connected, with $\bar{D} \subseteq U$. Let $h: \bar{D} \rightarrow \mathbb{R}$ be smooth. Then the \vocab{normal variation} of $\phi(\bar{D})$ determined by $h$ is the map $\rho: \bar{D} \times(-\varepsilon, \varepsilon) \rightarrow \mathbb{R}^3$ given by
$$
\rho(u, v, t)=\phi(u, v)+t h(u, v) N(u, v)
$$
\end{definition}


% \end{multicols*}


\end{document}
