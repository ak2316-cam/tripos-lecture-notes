\documentclass[a4paper]{article}

\usepackage[
    % fancytheorems, 
    % fancyproofs, 
    % noindent, 
    nokoma
]{adam}


\title{Topology}
\author{Adam Kelly (\texttt{ak2316@cam.ac.uk})}
\date{\today}

\allowdisplaybreaks

\begin{document}

\maketitle

This is a short description of the course. It should give a little flavour of what the course is about, and what will be roughly covered in the notes.

This article constitutes my notes for the `Non-Existent' course, held in Hillary 2052 at Cambridge. These notes are \emph{not a transcription of the lectures}, and differ significantly in quite a few areas. Still, all lectured material should be covered.


\section{Topological Spaces}

The idea of a topological space is to provide just enough structure to a set so that the notion of continuity makes sense. Let's first think about what continuity looks like in a metric space.

Let $f: M \rightarrow M'$ be a function between metric spaces. Then the definition of continuity for $f$ looks like this:

\begin{center}\color{blue}
    We say that $f$ is continuous at $x \in M$ if given $\varepsilon > 0$, \\there exists some $\delta > 0$ such that $d_M(x, y)< \delta$ \\implies that $d_{M'}(f(x), f(y)) < \varepsilon$.
\end{center}

This definition really centers around an idea of `the functions values being arbitrarily close for  sufficiently close points'. Here, the notion of `closeness' is specified using the metric.
But this isn't the only way to specify closeness.


\begin{definition}
    
\end{definition}


We care a significant amount about $X$ random object.

\begin{definition}[Random Object]
    We say that an object $X$ is a \vocab{random object} if we literally do not care about what it actually is.
\end{definition}

It is trivial to check that all objects you will meet in this course are random objects.


\end{document}
