\documentclass[a4paper, twocolumn]{article}

\usepackage[
    % fancytheorems, 
    % fancyproofs, 
    % noindent, 
    nokoma
]{adam}

% \documentclass[a4paper]{scrartcl}

% \usepackage[
%     fancytheorems, 
%     fancyproofs, 
%     noindent, 
% ]{adam}

\title{Geometry}
\author{Adam Kelly (\texttt{ak2316@cam.ac.uk})}
\date{\today}

\allowdisplaybreaks

\begin{document}

\maketitle

% This is a short description of the course. It should give a little flavour of what the course is about, and what will be roughly covered in the notes.

% This article constitutes my notes for the `Geometry' course, held in Lent 2022 at Cambridge. These notes are \emph{not a transcription of the lectures}, and differ significantly in quite a few areas. 
% In particular, many of the examples given in lectures (including the extended ones) are \emph{excluded}, and instead just the outer structure of the subject has been left.
% It is assumed that the reader is very familiar with the content from the `Analysis and Topology' course.

% %Still, all lectured material should be covered.
% %In 

% \tableofcontents

\section{Surfaces}

\subsection{Topological Surfaces}

\begin{definition}
  A \vocab{topological surface} is a topological space $\Sigma$ such that
  \begin{enumerate}[label=(\roman*)]
    \item for all points $p \in \Sigma$, there is an open neighborhood $p \in U \subset \Sigma$ such that $U$ is homeomorphic to $\R^2$ with its usual Euclidean topology.
    \item $\Sigma$ is Hausdorff and second countable.
  \end{enumerate}
\end{definition}

\subsection{Examples of Topological Surfaces}

\begin{itemize}
  \item The graph of a function
  \item The sphere
  \item The real projective plane
  \item Torus
  \item General polygons with identified edges
  \item Connect sum
\end{itemize}
The graph of a function

\subsection{Subdivisions and Triangulations}

\begin{definition}
  A \vocab{subdivision} of a compact topological surface $\Sigma$ comprises of a finite subset $V \subseteq \Sigma$ of \vocab{vertices}, and a finite collection of continuous embeddings $\{e_i: [0, 1] \rightarrow \Sigma\}$ called \vocab{edges}, each of which has endpoints in $C$ and any two of which are disjoint except perhaps at endpoints. Both of these are such that each connected component of the complement of $V \cup E$ in $\Sigma$ is homeomorphic to an open disk. We call each component a \vocab{face}. 
  
  A subdivision is a \vocab{triangulation} if the closure of each face contains exactly three edges, and two closed faces meet each-other at exactly one edge, or they don't meet.
\end{definition}

\begin{definition}
  The \vocab{Euler characteristic} of a subdivision is $V - E + F$.
\end{definition}

\begin{theorem}
  Every compact topological surface has a subdivision, and the Euler characteristic is invariant under choice of subdivision, and is topologically invariant.
\end{theorem}

\subsection{Smooth Surfaces}

Test

\end{document}
