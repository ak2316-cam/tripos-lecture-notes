\documentclass[11pt]{article}

\usepackage[
% fancytheorems, 
% fancyproofs,
noindent, 
nokoma
%spacingfix, 
]{adam}


\usepackage{cancel}


\title{Algebraic Number Theory}
\author{Adam Kelly (\texttt{ak2316@cam.ac.uk})}
\date{\today}

\begin{document}

\maketitle

\section{Algebraic Numbers}

\begin{definition}
	Let $\alpha \in \C$. We say that $\alpha$ is an \vocab{algebraic number} with $\alpha \in \overline{\Q}$ if it is the root of a monic polynomial with rational coefficients. If this polynomial has integer coefficients, we say that $\alpha$ is a \vocab{algebraic integer} with $\alpha \in \overline{\Z}$.
\end{definition}


\begin{proposition}
	The only rational algebraic integers are regular integers. 
\end{proposition}

\begin{proof}
	Clearly regular integers are algebraic integers, so we just check the other direction. Let $p/q$ be the root of a monic polynomial $f = \sum_{i = 0}^n a_i X^i$ with $a_i \in \Z$ and $p, q$ coprime. Then since $f(p/q) = 0$, we have
	$
	q^n f(p/q) = \sum_{i = 0}^n a_i p^i q^{n - 1},
	$
	and so $q^n f(p/q) \equiv p^n \equiv 0 \pmod{q}$, and since $p$ and $q$ are coprime, we must have $q = \pm 1$, and thus $p/q$ is an integer.
\end{proof}

\end{document}