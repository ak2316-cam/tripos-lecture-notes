\documentclass[a4paper]{scrartcl}

\usepackage[
    fancytheorems, 
    % fancyproofs, 
    noindent, 
    % nokoma
]{adam}


\title{Baire Category Theorem}
\author{Adam Kelly (\texttt{ak2316@cam.ac.uk})}
\date{\today}

\allowdisplaybreaks

\begin{document}

\maketitle

\begin{abstract}
    The Baire category theorem is a lovely result that is missing from the Part IB `Analysis and Topology' course. This article is a presentation of this result, along with a number of applications of it.
\end{abstract}

\vspace*{\baselineskip}
% Let $(X, \tau)$ be a topological space, and let $Y \subseteq X$. We say that $Y$ is \vocab{dense} in $X$ if $\overline{Y}$ (the closure of $Y$) is $X$.
% Equivalently, $Y$ is dense if $Y \cap U \neq \emptyset$ for every $U \in \tau$.

\begin{theorem*}[Baire Category Theorem]
    Let $U_1, U_2, \dots$ be a sequence of dense open sets in a complete metric space $X$. Then $U = \cap_{n = 1}^{\infty} U_n$ is dense in $X$.    
\end{theorem*}
\begin{proof}
    Since $U_n$ is a dense open subset of $X$ for all $x \in X$ and $r > 0$ the open ball $D_r(x)$ intersects $U_n$ in a non-empty open set, so there is $y \in X$ and $s > 0$ such that
    $$
    B_s(y) \subset D_{2s}(y) \subset Y_n \cap D_r(x).
    $$
\end{proof}

\end{document}
