\documentclass[a4paper]{scrartcl}

\usepackage[
  noindent,
  %color % Enable/disable this option to get tinted problem boxes.
]{examplesheet}

\usepackage{physics}
\usepackage{enumitem}
% \setlist[enumerate]{topsep=0pt,before=\leavevmode\vspace{-1.5em}}
% \setlist[description]{style=nextline}

\title{PQM Alimo -- Hurdles to Greatness}
\author{Cambridge Mathematical Tripos Part II -- Adam Kelly (\texttt{ak2316})}
\date{\today}



\begin{document}
\maketitle

\emph{Note}. Likelihood ratings are between 1 and 5, and are conditioning on interaction picture and angular momentum questions being the most likely to come up.

\vspace*{\baselineskip}


\begin{description}
    \item[2010 Paper 1] \emph{Likelihood Rating: 2 {\color{gray}(Probably too straightforward and doesn't require much more than Paper 3 question).}}

    $\ket{j_1 + j_2 \ j_1 + j_2} = \ket{j_1 \ j_1} \ket{j_2 \ j_2}$. Other combined angular momentum states can be found by applying the lowering operator $J_- = J_-^{(1)} + J_-^{(2)}$ to find lowered $M$ states, and by finding orthogonal vectors to get lowered $J$ states.
    
    If $j_1 = j_2=j$, then states with $M = 0$ are the sum of terms of the form $\ket{j\ m}\ket{j\ {-}m}$ for $m \in \{-j, \dots, j\}$. Noting that $J_+ \ket{0\ 0} = 0$, applying on the right and noting the coefficients of basis states must sum to zero, we get $a_m = -a_{m+1}$. So by normalisation, we have $2j + 1$ states and hence $\alpha_m = \frac{1}{\sqrt{2j + 1}}(-1)^m$.

    The probability of measuring combined total angular momentum zero is $\abs{\bra{0\ 0}(\ket{j\ j}\ket{j\ {-}j})}^2 = \norm{\alpha_j}^2 = \frac{1}{2j + 1}$ by the Born rule.

    \item[2010 Paper 2] \emph{Likelihood Rating: 5 {\color{gray}(Disjoint enough and hard enough).}}

        $U(n, \theta) = e^{-i \theta n \cdot J / \hbar}$. A state with zero orbital angular momentum would transform the only the spin component of the state by the same rotation (about the origin with respect to spin).

    When $j = 1/2$, we ge that $J = \frac{1}{2}\hbar \sigma$. We can explicitly calculate 
    \begin{align*}
        J \cdot a &= \frac{h}{2} \begin{pmatrix}
            a_z & a_x - i a_y \\ a_x + i a_y & - az
        \end{pmatrix},
    \end{align*}
    and so $(J \cdot a)^2 = \frac{h^2}{4}\norm{a}^2 I$. The eigenvalues of $J \cdot a$ are $\pm \frac{h}{2} \norm{a}$.  For $j = 1/2$, we can write
    $$
    U(n, \theta) = e^{-i \theta n \cdot J / \hbar} = e^{-i \theta n \cdot \sigma/2} = \cos (- \theta n \cdot \sigma/2) + i \sin (- \theta n \cdot \sigma /2),
    $$
    which, by linearity and considering the action on the eigenstates, gives $U(n, \theta) = \cos \theta/2 - i n \cdot \sigma  \sin \theta/2$.

    We have $n' \cdot J \ket{\frac{1}{2}\ m}_\theta = \hbar m'$. So the probability of measuring $m' \hbar$ along the direction $n'$ is 
    $$
    \abs{\ip{\frac{1}{2}\ m}{\frac{1}{2}\ m'}_\theta}^2 = \abs{\mel{\frac{1}{2}\ m}{U(y, \theta)}{\frac{1}{2}\  m'}}^2,
    $$
    using $\ket{\frac{1}{2}\ m'}_\theta = U(y, \theta) \ket{\frac{1}{2}\ m'}$, with $y$ the $y$-axis unit vector. So if $m = \frac{1}{2}$, $m' = -\frac{1}{2}$, then substituting in gives our probability as $\abs{\bra{\frac{1}{2} \ \frac{1}{2}} \left(\sin \frac{\theta}{2} \ket{\frac{1}{2}\ \frac{1}{2}} + \cos \frac{\theta}{2} \ket{\frac{1}{2} \ \frac{1}{2}}\right)}^2 = \sin^2 \frac{\theta}{2} = 1 - \cos^2 \frac{\theta}{2}$.
        
    \item[2010 Paper 4] \emph{Likelihood Rating: 4 {\color{gray}(Exact probability calculation and exact condition for approximation to be valid similar to Paper 2 qn).}}
    
    Our time evolution equation is
    $
     i \hbar \frac{d}{dt }\ket{\psi_I(t)}= V_I(t) \ket{\psi_I(t)}
    $. We want to compute $\abs{\ip{b}{\psi(t)}}^2$ where $\ket{\psi(0)} = \ket{a}$. We can integrate our time evolution equation to get
    \begin{align*}
    \ket{\psi_I(t)} &= -\frac{i}{\hbar}\ket{a} + \frac{i}{\hbar} \int_0^{t} V_I(t') \ket{\psi_I(t')} \dd t' \\
    \implies \abs{\ip{b}{\psi_I(t)}}^2 &= \frac{1}{\hbar^2} \abs{\int_0^{t} \bra{b} V_I(t') \ket{\psi_I(t')} \dd t}^2 \\
    &= \frac{1}{\hbar^2} \abs{\int_0^{t} \bra{b} V_I(t') \ket{a} \dd t}^2\\
    &= \frac{1}{\hbar^2}\left|\int_0^t d t^{\prime}\mel{b}{V(t')}{a} e^{i\left(E_b-E_a\right) t^{\prime} / \hbar}\right|^2,
    \end{align*}
    to order $V(t)^2$ as required.

    In the case of $H_0$ and $V$ given, we can compute that $\mel{2}{vt \sigma_1}{1} = vt$, and hence our transition probability is given by
    $$
    \frac{1}{\hbar^2}\abs{\int_0^t vt' \dd t'}^2 = \frac{v^2t^4}{4\hbar^2}.
    $$
    We can clearly see that the given state is a solution to the TDSE, so letting $\ket{\psi(0)} = \ket{1}$, we can substitute in and take inner products with $\ket{2}$. Then (after rewriting the exponential to get only the $\sigma_1$ terms and rearranging the sum) we get that the exact transition probability is $\sin^2(\frac{vt^2}{2\hbar})$. So our approximation is valid if $vt^2/2\hbar$ is small.
    
    \item [2011 Paper 1] \emph{Likelihood Rating: 1} (Requires same angular momentum calculation skills).
    \item [2011 Paper 2] \emph{Likelihood Rating: 3} (Not on a specific topic but he seems to like the `from the start' questions).
    \item [2011 Paper 3] \emph{Likelihood Rating: 5 {\color{gray}(Disjoint enough and covers end part of course).}}

\begin{enumerate}[label=(\roman*)]
    \item $\ket{\uparrow} = \begin{pmatrix}
        1 \\ 0
    \end{pmatrix}$, $\ket{\downarrow} = \begin{pmatrix}
        0 \\ 1
    \end{pmatrix}$. $s = \frac{1}{2} \hbar \sigma$. We can then manually check using the given matrices that $[s_i, s_j] = \varepsilon_{ijk} s_k$ and $[s^2, s_3] = 0$. Lastly we check that $(n \cdot s)^2 = I$ which gives us that the eigenvalues are as desired. We also have
    $$
    s_x\begin{pmatrix}
        1 \\ 0
    \end{pmatrix} = \frac{\hbar}{2}\begin{pmatrix}
        0 \\ 1
    \end{pmatrix}, \quad 
    s_x\begin{pmatrix}
        0 \\ 1
    \end{pmatrix} = \frac{\hbar}{2}\begin{pmatrix}
        1 \\ 0
    \end{pmatrix}, \quad 
    s_y\begin{pmatrix}
        1 \\ 0
    \end{pmatrix} = \frac{\hbar}{2}\begin{pmatrix}
        0 \\ i
    \end{pmatrix}, \quad 
    s_y\begin{pmatrix}
        0 \\ 1
    \end{pmatrix} = \frac{\hbar}{2}\begin{pmatrix}
        -i \\ 0
    \end{pmatrix}.
    $$
    These can be rewritten in terms of $\ket{\uparrow}$ and $\ket{\downarrow}$ as desired.

    \item It's easy to check that the eigenvalues are $-\frac{\hbar^3}{8}, -\frac{\hbar^3}{8}, -\frac{\hbar^3}{8}, \frac{\hbar^3}{8}$.
    \item Let $\tilde{s}_x^{(i)}$ take on value $\frac{\hbar}{2}$ with probability $p_{x}^{(i)}$, and $-\frac{\hbar}{2}$ otherwise. Define $\tilde{s}_y^{(i)}$ similarly. Then from the previously calculated eigenvalues we have that $\PP(\tilde{s}_x^{(1)})\tilde{s}_y^{(2)}\tilde{s}_y^{(2)} = -\frac{\hbar^3}{8}) = 1$ and so on. Multiplying gives that the probability they are all $-1$ is $1$, and hence $\PP(\tilde{s}_x^{(1)})\tilde{s}_x^{(2)})\tilde{s}_x^{(2)}) = -\frac{\hbar^3}{8}) = 1$ (since the $\tilde{s}_y^{(i)}$ terms all square to 1). Hence there is a classical unique possibility for this of $-\frac{\hbar^3}{8}$ with probability 1. 
    
    This lets us test quantum mechanics experimentally as we can produce this state and measure to get -1 which contradicts this `hidden classical variables' theory.
\end{enumerate}



    \item [2012 Paper 2] \emph{Likelihood Rating: 3} (Heisenberg picture plus dirac formalism not really examined yet).
    \item [2012 Paper 4] \emph{Likelihood Rating: 3.5} (Disjoint enough but still quite easy, covers the angular momentum content not tested in Paper 3).
    \item [2013 Paper 2] \emph{Likelihood Rating: 1} (Easy spin question, same year as a previous repeat, calculates commutators like in Paper 1 and simple addition of angular momentum which was examined in Paper 3)
    \item [2014 Paper 1] \emph{Likelihood Rating: 2} (Dirac notation question, basically just position and momentum space with not much else going on)
    \item [2014 Paper 3] \emph{Likelihood Rating: 3.5} (Angular momentum question disjoint to the previously asked stuff)

    For states $\ket{\psi_S}$ and operators $A_S$ in the Schr\"odinger picture, we define the interaction picture states and operators by
    $$
    \left|\psi_I(t)\right\rangle=e^{i H_0 t / \hbar}\left|\psi_S(t)\right\rangle, \quad A_I(t)=e^{i H_0 t / \hbar} A_S e^{-i H_0 t / \hbar}.
    $$
    We can then check that matrix elements are preserved as then they give the same physical predictions, and indeed if $\ket{\psi}$, $\ket{\phi}$ and $A$ are operators then
    $$
    \left\langle\phi_S\left|A_S\right| \psi_S\right\rangle=\left\langle\phi_S\left|e^{-i H_0 t / \hbar} e^{i H_0 t / \hbar} A_S e^{-i H_0 t / \hbar} e^{i H_0 t / \hbar}\right| \psi_S\right\rangle=\left\langle\phi_I\left|A_I(t)\right| \psi_I\right\rangle,
    $$
    so our theory is the same. We can then just differentiate to obtain the equation of motion 
    $$
    i \hbar \frac{\partial}{\partial t}\left|\psi_I(t)\right\rangle=V_I(t)\left|\psi_I(t)\right\rangle.
    $$
    
    The rest isn't PQM content it's just DEs so you just follow the instructions and you get what you're told.

    \item [2014 Paper 4] \emph{Likelihood Rating: 4} (Interaction Picture question, pretty disjoint from previous stuff)
    \item [2015 Paper 2] \emph{Likelihood Rating: 2} (Simple spin half stuff combined with already done calculations for addition of angular momentum)
    \item [2016 Paper 4] \emph{Likelihood Rating: 5} (Interaction picture with the part of spin we didn't discuss).
    
    \begin{enumerate}[label=(\alph*)]
        \item Interaction picture bookwork as above, except we get out the operator derivative $i \hbar \frac{d A_I}{dt} = [A_I(t), H_0]$.
        \item Define $S = \frac{1}{2} \hbar \sigma$. Then we can check $(n \cdot S)^2 = \hbar^2/4$ for any unit vector $n$, and also $[S_i, S_j] = i \hbar \varepsilon_{ijk} S_k$ so we have the correct spin algebra.
        \item In general, to check that $U$ acts correctly as the transformation $T$ with infinitesimal parameter, we need to check
        $$U(\delta \theta) Q U(\delta \theta)^{-1}= T(\delta \theta) Q$$
        The operators $x, p, \sigma$ all transform as vectors, so we can just check for a `vector transforming object' $V$ that this holds. 
        We work to first order in $\delta \theta$ to get
        $$
        \left(I + \frac{i \delta \theta}{\hbar} n \cdot J\right) V \left(I - \frac{i \delta \theta}{\hbar} n \cdot J\right) + O(\delta \theta^2) = V + \frac{i \delta \theta}{\hbar} [n \cdot J, V] + O(\delta \theta^2)
        $$
        then we use that $[J_i, V_j] = i \hbar \varepsilon_{ijk} V_k$ to get the above is
        $$
        V + \delta \theta n \times V + O(\delta \theta^2),
        $$
        which is the correct form of our transformation for infinitesimal rotations.
        \item Define $J_i^I(t)$ and $J_i$ in the obvious way. Then (a) gives
        $$
        i \hbar \frac{d J_i^I(t)}{d t}=\left[J_i^I(t), H_0\right]=e^{i H_0 t / \hbar}\left[J_i, H_0\right] e^{-i H_0 t / \hbar}
        $$
        We can evaluate $\left[J_i, H_0\right]$ with (using index notation):
        \begin{align*}
            \left[J_i, H_0\right]&=\frac{1}{2 m}\left[J_i, p_k p_k\right]+\frac{\alpha}{m \hbar}\left[J_i, L_k S_k\right]\\
            &=\frac{1}{2 m}\left(\left[J_i, p_k\right] p_k+p_k\left[J_i, p_k\right]\right)+\frac{\alpha}{m \hbar}\left(\left[J_i, L_k\right] S_k+L_k\left[J_i, S_k\right]\right)
        \end{align*}
        Using the commutation relation for $J_i$ with a vector operator, we have
        $$
        \left[J_i, H_0\right]=\frac{1}{2 m}\left(\epsilon_{i k j} p_j p_k+\epsilon_{i k j} p_k p_j\right)+\frac{\alpha}{m \hbar}\left(\epsilon_{i k j} L_j S_k+\epsilon_{i k j} L_k S_j\right)=0
        $$
        since both terms are the product of something antisymmetric on $k, j$ and something symmetric on $k, j$. We deduce that
        $
        \frac{d J_i^I(t)}{d t}=0
        $
        as required.

        In the Heisenberg picture, we instead must use the Heisenberg equation of motion:
$$
i \hbar \frac{d J_i^H(t)}{d t}=\left[J_i^H(t), H\right]=e^{i H t / \hbar}\left[J_i, H\right] e^{-i H t / \hbar}.
$$
We have already seen that $\left[J_i, H_0\right]=0$. So we are left to evaluate the commutator:
$$
\left[J_i, B \sigma_3\right]=\frac{2 B}{\hbar}\left[J_i, S_3\right]=\frac{2 B}{\hbar} i \hbar \epsilon_{i 3 k} S_k=i \hbar B \epsilon_{i 3 k} \sigma_k
$$
Hence we're left with:
$$
\frac{d J_i^H(t)}{d t}=B \epsilon_{i 3 k} e^{i H t / \hbar} \sigma_k e^{-i H t / \hbar} \neq 0,
$$
unless $i=3$. So the vector $J^H(t)$ in the Heisenberg picture is not independent of time.
    \end{enumerate}

\end{description}







% \begin{description}
%     \item[2010 Paper 1] \emph{Likelihood Rating: 2} (Probably too straightforward and doesn't require much more than Paper 3 question).
    
%     % $\ket{j_1 + j_2 \ j_1 + j_2} = \ket{j_1 \ j_1} \ket{j_2 \ j_2}$. Other combined angular momentum states can be found by applying the lowering operator $J_- = J_-^{(1)} + J_-^{(2)}$ to find lowered $M$ states, and by finding orthogonal vectors to get lowered $J$ states.
    
%     % If $j_1 = j_2=j$, then states with $M = 0$ are the sum of terms of the form $\ket{j\ m}\ket{j\ {-}m}$ for $m \in \{-j, \dots, j\}$. Noting that $J_+ \ket{0\ 0} = 0$, applying on the right and noting the coefficients of basis states must sum to zero, we get $a_m = -a_{m+1}$. So by normalisation, we have $2j + 1$ states and hence $\alpha_m = \frac{1}{\sqrt{2j + 1}}(-1)^m$.

%     % The probability of measuring combined total angular momentum zero is $\abs{\bra{0\ 0}(\ket{j\ j}\ket{j\ {-}j})}^2 = \norm{\alpha_j}^2 = \frac{1}{2j + 1}$ by the Born rule.

% % \vspace*{\baselineskip}

%     \item[2010 Paper 2] \emph{Likelihood Rating: 5} (Disjoint enough and hard enough).
    
%     % $U(n, \theta) = e^{-i \theta n \cdot J / \hbar}$. A state with zero orbital angular momentum would transform the only the spin component of the state by the same rotation (about the origin with respect to spin).

%     % When $j = 1/2$, we ge that $J = \frac{1}{2}\hbar \sigma$. We can explicitly calculate 
%     % \begin{align*}
%     %     J \cdot a &= \frac{h}{2} \begin{pmatrix}
%     %         a_z & a_x - i a_y \\ a_x + i a_y & - az
%     %     \end{pmatrix},
%     % \end{align*}
%     % and so $(J \cdot a)^2 = \frac{h^2}{4}\norm{a}^2 I$. The eigenvalues of $J \cdot a$ are $\pm \frac{h}{2} \norm{a}$.  For $j = 1/2$, we can write
%     % $$
%     % U(n, \theta) = e^{-i \theta n \cdot J / \hbar} = e^{-i \theta n \cdot \sigma/2} = \cos (- \theta n \cdot \sigma/2) + i \sin (- \theta n \cdot \sigma /2),
%     % $$
%     % which, by linearity and considering the action on the eigenstates, gives $U(n, \theta) = \cos \theta/2 - i n \cdot \sigma  \sin \theta/2$.

%     \item[2010 Paper 4] \emph{Likelihood Rating: 4} (Exact probability calculation and exact condition for approximation to be valid similar to Paper 2 qn).

    
%     \item [2011 Paper 1] \emph{Likelihood Rating: 1} (Requires same angular momentum calculation skills).
%     \item [2011 Paper 2] \emph{Likelihood Rating: 3} (Not on a specific topic but he seems to like the `from the start' questions).
%     \item [2011 Paper 3] \emph{Likelihood Rating: 5} (Disjoint enough and covers end part of course).
    
%     \item [2012 Paper 2] \emph{Likelihood Rating: 3} (Heisenberg picture plus dirac formalism not really examined yet).
%     \item [2012 Paper 4] \emph{Likelihood Rating: 3.5} (Disjoint enough but still quite easy, covers the angular momentum content not tested in Paper 3).
%     \item [2013 Paper 2] \emph{Likelihood Rating: 1} (Easy spin question, same year as a previous repeat, calculates commutators like in Paper 1 and simple addition of angular momentum which was examined in Paper 3)
%     \item [2014 Paper 1] \emph{Likelihood Rating: 2} (Dirac notation question, basically just position and momentum space with not much else going on)
%     \item [2014 Paper 3] \emph{Likelihood Rating: 5} (Angular momentum question disjoint to the previously asked stuff)
    
%     \item [2014 Paper 4] \emph{Likelihood Rating: 4} (Interaction Picture question, pretty disjoint from previous stuff)
%     \item [2015 Paper 2] \emph{Likelihood Rating: 2} (Simple spin half stuff combined with already done calculations for addition of angular momentum)
%     \item [2016 Paper 4] \emph{Likelihood Rating: 5} (Interaction picture with the part of spin we didn't discuss).
    
% \end{description}


% % \begin{description}
% %     \item[2016 Paper 4] \emph{Likelihood Rating: 5}
% %     \begin{enumerate}[label=(\alph*)]
% %         \item Interaction picture bookwork
% %         \item Define $S = \frac{1}{2} \hbar \sigma$. Then we can check $(n \cdot \sigma)^2 = \hbar^2/4$ for any unit vector $n$, and also $[S_i, S_j] = i \hbar \varepsilon_{ijk} S_k$ so we have the correct spin algebra.
% %         \item In general, to check that $U$ acts correctly as the transformation $T$ with infinitesimal parameter, we need to check
% %         $$U(\delta \theta) Q U(\delta \theta)^{-1}= T(\delta \theta) Q$$
% %         The operators $x, p, \sigma$ all transform as vectors, so we can just check for a `vector transforming object' $V$ that this holds. 
% %     \end{enumerate}
% % \end{description}

% % If you want to space out problems, add an extra `\clearpage'
% % command.
% % \clearpage



\end{document}