\documentclass[a4paper]{amsart}

\usepackage[
    % fancytheorems, 
    % fancyproofs, 
    % noindent, 
    nokoma
]{adam}

% \documentclass[a4paper]{scrartcl}

% \usepackage[
%     fancytheorems, 
%     fancyproofs, 
%     noindent, 
% ]{adam}

\title{The Simplicity of $A_n$}
\author{Adam Kelly (\texttt{ak2316@cam.ac.uk})}
\date{\today}

\allowdisplaybreaks

\begin{document}

\maketitle

Here we establish the simplicity of $A_n$.

\begin{lemma*}
    $A_n$ is generated by 3-cycles.
\end{lemma*}

\begin{proof}
    All elements of $A_n$ are, by definition, generated by an even number of transposition. It thus suffices to show that a product of two transpositions can be written as a product of 3-cycles. Explicitly,
$$
(a\ b)(c\ d) = (a\ c\ b)(a\ c\ d); \quad (a\ b)(b\ c) = (a\ b\ c).
$$
\end{proof}



\begin{lemma*}
    If $n \geq 5$, all 3-cycles in $A_n$ are conjugate (in $A_n$).
\end{lemma*}
\begin{proof}
    We claim that every 3-cycle is conjugate to $(1\ 2\ 3)$. If $(a\ b\ c)$ is a 3-cycle, we have $(a\ b\ c) = \sigma (1\ 2\ 3)\sigma^{-1}$ for some $\sigma \in S_n$. If $\sigma \in A_n$, then the proof is finished. Otherwise $\sigma \mapsto \sigma(4\ 5) \in A_n$ suffices, since $(4\ 5)$ commutes with $(1\ 2\ 3)$.
\end{proof}

\begin{theorem}
    $A_n$ is simple for $n \geq 5$.
\end{theorem}
\begin{proof}
    Suppose $1 \neq N \triangleleft A_n$. To disprove normality, it suffices to show that $N$ contains a 3-cycle by the lemmas above, since the normality of $N$ would imply $N$ contains all 3-cycles and hence all elements of $A_n$.

Let $1 \neq \sigma \in N$, writing $\sigma$ as the product of disjoint cycles.
\begin{enumerate}
    \item 
Suppose $\sigma$ contains a cycle of length $r \geq 4$. Without loss of generality, let $\sigma = (1\ 2\ 3\ \dots\ r) \tau$, where $\tau$ fixes $1, \dots, r$. Now let $\delta = (1\ 2\ 3)$. We have
$$
\underbrace{\sigma^{-1} }_{\in N} \underbrace{\delta^{-1} \sigma \delta}_{\in N} = (r\ \dots\ 2\ 1)(1\ 3\ 2)(1\ 2\ \dots\ r) = (2\ 3\ r)
$$
So $N$ contains a 3-cycle.
\item
Suppose $\sigma$ contains two 3-cycles, which can be written without loss of generality as $(1\ 2\ 3)(4\ 5\ 6)\tau$. Then let $\delta = (1\ 2\ 4)$, and then
$$
\sigma^{-1}\delta^{-1}\sigma \delta = (1\ 3\ 2)(4\ 6\ 5)(1\ 4\ 2)(1\ 2\ 3)(4\ 5\ 6)(1\ 2\ 3) = (1\ 2\ 4\ 3\ 6).
$$
Therefore, there existst an element of $N$ which contains a cycle of length $5 \geq 4$, which reduces our problem to the previous case.
\item
Finally, suppose that $\sigma$ contains two 2-cycles, which will be written $(1\ 2)(3\ 4)\tau$. Then let $\delta = (1\ 2\ 3)$ and
$$
\sigma^{-1}\delta^{-1}\sigma\delta = (1\ 2)(3\ 4)(1\ 3\ 2)(1\ 2)(3\ 4)(1\ 2\ 3) = (1\ 4)(2\ 3) = \pi.
$$
Let $\varepsilon = (2\ 3\ 5)$. Then
$$
\underbrace{\pi^{-1} }_{\in N}\underbrace{\varepsilon^{-1} \pi \varepsilon}_{\in N} = (1\ 4)(2\ 3)(2\ 5\ 3)(1\ 4)(2\ 3)(2\ 3\ 5) = (2\ 3\ 5),
$$
So $N$ contains a 3-cycle.
\end{enumerate}
There are now three remaining cases, where $\sigma$ is a transposition, a 3-cycle, or a transposition composed with a 3-cycle. Note that the remaining cases containing transpositions cannot be elements of $A_n$. If $\sigma$ is a 3-cycle, we already know $A_n$ contains a 3-cycle, namely $\sigma$ itself.
\end{proof}



\end{document}
