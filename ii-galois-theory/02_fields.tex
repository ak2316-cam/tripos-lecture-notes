\subsection{Field extensions}

We introduce our fundamental object of study.

\begin{definition}[Field Extension]
	Let \( K \subset L \) be fields\footnote{Implicitly assuming that the field operations and identity elements on \( K \) and \( L \) are the same.}.
	We say \( K \) is a subfield of \( L \), and \( L \) is a \vocab{field extension} of \( K \), denoted \( L / K \) (read `\( L \) over \( K \)').
\end{definition}

\begin{definition}[Adjoining Elements]
	Let \( K \subset L \), and \( x \in L \).
	We define 
	$$
	K[x] = \{p(x) \mid p \in K[T]\}
	$$ to be the ring where we \vocab{adjoin} $x$ to $K$.
	
	We further define 
	$$
	K(x) =\left\{\frac{p(x)}{q(x)} \mid p,q \in K[T], q(x) \neq 0\right\}
	$$
	to be the field of fractions of \( K[x] \).
\end{definition}

\subsection{Field Extensions as Vector Spaces}
 
	A field extension \( L / K \) turns \( L \) into a \( K \)-vector space by forgetting the multiplication between elements of \( L \).

\begin{definition}[Finite/Infinite Extensions]
	A field extension \( L / K \) is called a \vocab{finite extension} if \( L \) is a finite-dimensional \( K \)-vector space. We write $[L:K] = \dim_K L$ for the \vocab{degree} of the extension.
	
	Otherwise, we say \( L / K \) is an \vocab{infinite extension}, and write \( [L:K] = \infty \).
\end{definition}

We particularly care about finite field extensions.

\begin{definition}[Quadratic/Cubic/$\dots$ Extensions]
	An extension is \vocab{quadratic}, \vocab{cubic}, etc.\ if its degree is 2, 3, etc.
\end{definition}

\begin{proposition}[Sizes of Finite Fields]
	Suppose \( K \) is a finite field (necessarily of prime characteristic \( p \)).
	Then \( \abs{K} \) is a power of \( p \).
\end{proposition}
\begin{proof}
	Note that \( K / \mathbb F_p \) is a finite extension, and so \( K \cong \mathbb F_p^n \), giving \( \abs{K} = p^n \).
\end{proof}
We will later show that for all prime powers \( q = p^n \), there exists a finite field \( \mathbb F_q \) with \( q \) elements.

\begin{lemma}
	Let \( L / K \) and \( V \) is an \( L \)-vector space.
	Then \( V \) is a \( K \)-vector space, and\footnote{Take the obvious meaning if any of these expressions are infinite.} \( \dim_K V = [L:K] \dim_L V \).
\end{lemma}
\begin{proof}
	Let \( \dim_L V = d < \infty \).
	Then \( V \cong L \oplus \dots \oplus L = L^d \) as an \( L \)-vector space, so this also holds as a \( K \)-vector space.
	But since \( L \cong K^{[L:K]} \) as a \( K \)-vector space, we have \( V \cong (K^{[L:K]})^d \cong K^{d[L:K]} \) as a \( K \)-vector space.

	If \( V \) is finite-dimensional over \( K \), then a \( K \)-basis for \( V \) will span \( V \) over \( L \), so \( V \) is finite-dimensional over \( L \).
	Thus if \( V \) is infinite-dimensional over \( L \), it is infinite-dimensional over \( K \).

	Likewise, if \( [L:K] = \infty \) and \( V \neq 0 \), then \( V \) has an infinite set of linearly independent vectors as a \( K \)-vector space, so \( \dim_K V = \infty \).
\end{proof}

\begin{theorem}[Tower Law]
	Let \( M / L \), \( L / K \) be a pair of field extensions.
	Then \( M / K \) is a finite extension if and only if \( M / L \) and \( L / K \) are finite.
	If so, we have \( [M:L][L:K] = [M:K] \).
\end{theorem}
\begin{proof}
Take \( V = M \) in the previous lemma.
\end{proof}

\begin{proposition}[Multiplicative Group Structure]
	Let \( K \) be a field, and \( G \subset K^\times \) be a finite subgroup of the multiplicative group.
	Then \( G \) is cyclic.
	In particular, if \( K \) is finite, \( K^\times \) is cyclic.
\end{proposition}
\begin{proof}
	We can find \( m_i \) such that
	\[ G \cong \faktor{\mathbb Z}{m_1\mathbb Z} \times \dots \times \faktor{\mathbb Z}{m_k \mathbb Z} \]
	where \( 1 < m_1 \mid m_2 \mid \dots \mid m_k = m \) by the structure theorem for abelian groups.
	Then, every element of \( G \) satisfies \( x^m = 1 \).
	Since \( K \) is a field, the polynomial \( T^m - 1 \) has at most \( m \) roots.
	Every element of \( G \) is a root of this polynomial, so \( \abs{G} \leq m \).
	This can only happen when \( k = 1 \), so \( G = \faktor{\mathbb Z}{m\mathbb Z} \).
\end{proof}


\begin{proposition}[Frobenius Endomorphism]
	Let \( R \) be a ring, with finite prime characteristic \( p \).
	%  be a prime such that \( p \cdot 1_R = 0_R \) (for instance, \( R \) could be a field of characteristic \( p \)).
	Then, the map \( \varphi_p \colon R \to R \) given by \( \varphi_p(x) = x^p \) is a homomorphism, known as the \vocab{Frobenius endomorphism}.
\end{proposition}
\begin{proof}
	First, \( \varphi_p(1) = 1^p = 1 \) and \( \varphi_p(x)\varphi_p(y) = x^p y^p = (xy)^p = \varphi_p(xy) \).
	For \( x,y \in R \),
	\begin{align*}
		\varphi_p(x + y) &= \sum_{k = 0}^p \binom{p}{k} x^{p-k} y^p = x^p + y^p 
		\\&= \varphi_p(x) + \varphi_p(y)
	\end{align*}
	since \( p \mid \binom p k \) for \( k \in {1, \dots, p-1} \) as $p$ is prime.
\end{proof}


\subsection{Algebraic Elements and Minimal Polynomials}
We now want to consider field elements that satisfy some polynomial over the field.

\begin{definition}[Algebraic and Trancendental]
	Let \( L / K \) be an extension and \( x \in L \).
	\( x \) is \vocab{algebraic} over \( K \) if there exists a nonzero polynomial \( f \in K[T] \) such that \( f(x) = 0 \).
	Otherwise, we say \( x \) is \vocab{transcendental} over \( K \).
\end{definition}
For \( f \in K[T] \), we have \( f(x) \in L \).
Varying \( f \), this gives a map \( \mathrm{ev}_x \colon K[T] \to L \) defined by \( f \mapsto f(x) \).
This is a ring homomorphism.

The kernel \( I = \ker(\mathrm{ev}_x) \subset K[T] \) is an ideal, the set of polynomials which vanish at \( x \).
As \( \Im(\mathrm{ev}_x) \) is a subring of \( L \) which is a field, it is an integral domain.
In particular, \( I \) is a prime ideal, so either \( I = 0 \), in which case \( x \) is transcendental over \( K \), or there exists a unique monic irreducible polynomial \( 0 \neq g \in K[T] \) such that \( I = (g) \), in which case \( x \) is algebraic and we say \( g \) is the \emph{minimal polynomial} of \( x \) over \( K \).
In this case, \( f(x) = 0 \) if and only if \( g \mid f \).
We write \( m_{x,K} \) for the minimal polynomial of \( x \) over \( K \).
Note that \( m_{x,K} \) is the monic polynomial in \( K \) of least degree with \( x \) as a root.

\begin{example}
	If \( x \in K \), \( m_{x,K} = T - x \).
	If \( p \) is prime and \( d \geq 1 \), \( T^d - p \in \mathbb Q[T] \) is irreducible by Eisenstein's criterion, so it is the minimal polynomial of \( \sqrt[d]{p} \in \mathbb R \) over \( \mathbb Q \).
	If \( p \) is prime, \( z = e^{\frac{2\pi i}{p}} \) is a root of \( T^p - 1 = (T-1)(T^{p-1} + T^{p-2} + \dots + 1) = (T-1)g(T) \), which is a product of irreducibles as
	\[ g(T+1) = \binom p p T^{p-1} + \binom p {p-1} T^{p-2} + \dots + \binom p 2 T + \binom p 1 \]
	This is irreducible by Eisenstein's criterion, so \( g \) is minimal for \( z \) over \( \mathbb Q \).
\end{example}

We say the degree of an algebraic element \( x \) over \( K \) is the degree of its minimal polynomial, written \( \deg_K x = \deg(x/K) \).

\begin{proposition}
	Let \( L / K \) and \( x \in L \).
	Then, the following are equivalent.
	\begin{enumerate}
		\item \( x \) is algebraic over \( K \).
		\item \( [K(x):K] \) is finite.
		\item \( K[x] \) is finite-dimensional as a \( K \)-vector space.
		\item \( K[x] = K(x) \).
		\item \( K[x] \) is a field.
	\end{enumerate}
	If these hold, \( \deg x = [K(x):K] \).
\end{proposition}
\begin{proof}
	\emph{(ii) implies (iii).} This follows since \( K[x] \subseteq K(x) \).

	\emph{(iv) is equivalent to (v)} is trivial.

	\emph{(iii) implies (v) and (ii).}
	Let \( 0 \neq y = g(x) \in K[x] \).
	Consider the map \( K[x] \to K[x] \) given by \( z \mapsto yz \).
	This is a \( K \)-linear transformation, and since \( y \neq 0 \) this is injective.
	Because \( \dim K[x] \) is finite, this injective map must be a bijection.
	Therefore there exists \( z \) such that \( yz = 1 \), so \( y \) is invertible.
	Hence (v) holds.
	Since (v) implies (iv), \( [K(x):K] = \dim_K K[x] < \infty \) as required for (ii).

	\emph{(v) implies (i).}
	If \( x = 0 \), the proof is complete, so assume \( x \neq 0 \).
	Then \( x^{-1} = a_0 + a_1 x + \dots + a_n x^n \in K[x] \).
	Therefore, \( a_n x^{n+1} + \dots + a_0 x - 1 = 0 \), so \( x \) is algebraic over \( K \).

	\emph{(i) implies (v), (iii), and the degree formula.}
	The image of \( \mathrm{ev}_x \colon K[T] \to L \) is the subring \( K[x] \subset L \).
	If \( x \) is algebraic over \( K \), \( \ker(\mathrm{ev}_x) = (m_{x,K}) \) is a maximal ideal by irreducibility of \( m_{x,K} \).
	By the first isomorphism theorem, \( \faktor{K[T]}{(m_{x,K})} \cong K[x] \).
	But quotients by maximal ideals are fields, so \( K[x] \) is a field, proving (v).
	This polynomial is monic of degree \( d = \deg_K x \).
	Hence \( \faktor{K[T]}{(m_{x,K})} \) has a \( K \)-basis \( 1, T, \dots, T^{d-1} \).
	Thus, \( \dim_K K[x] = d = [K(x):K] < \infty \), proving (iii) and the degree formula.
\end{proof}
\begin{corollary}
	\( x_1, \dots, x_n \) are algebraic over \( K \) if and only if \( L = K(x_1, \dots, x_n) \) is finite over \( K \).
	If so, every element of \( K(x_1, \dots, x_n) \) is algebraic over \( K \).

	If \( x, y \) are algebraic over \( K \), then so are \( x \pm y \), \( xy \), and \( x^{-1} \) if \( x \) is nonzero.
	If \( L / K \) is a field extension, the set of algebraic elements of \( L \) forms a subfield of \( L \).
\end{corollary}
\begin{proof}
	If \( x_n \) is algebraic over \( K \), then it is also algebraic over \( K(x_1, \dots, x_{n-1}) \).
	Hence \( L / K(x_1, \dots, x_{n-1}) \) is finite.
	By induction on \( n \), the tower law gives the required result.
	Conversely, if \( L \) is finite over \( K \), the subfield \( K(y) \) is finite over \( K \) for all \( y \in L \), so \( y \) is algebraic over \( K \).

	Suppose \( x, y \) are algebraic over \( K \).
	Then \( x \pm y, xy, x^{-1} \in K(x,y) \), which is finite over \( K \) as required.
\end{proof}
\begin{example}
	Consider \( z = e^{2\pi i/p} \in \mathbb C \) where \( p \) is an odd prime.
	This has degree \( p - 1 \) as discussed above.
	Now consider \( x = 2\cos \frac{2\pi}{p} \), so \( x = z + \frac 1z \in \mathbb Q(z) \).
	This is algebraic over \( \mathbb Q \) because it belongs to this finite extension.
	Note that \( \mathbb Q(z) \supset \mathbb Q(x) \supset \mathbb Q \), and \( z^2 - xz + 1 = 0 \).
	Hence the degree of \( z \) over \( \mathbb Q(x) \) is at most 2.
	But \( [\mathbb Q(z):\mathbb Q(x)] \neq 1 \) because \( z \in \mathbb C \setminus \mathbb R \).
	By the tower law, we must have \( [\mathbb Q(z):\mathbb Q] = \frac{p-1}{2} \).

	We can now derive the minimal polynomial by considering \( z^{\frac{p-1}{2}} + z^{\frac{p-3}{2}} + \dots + z^{\frac{-p-1}{2}} = 0 \).
	Since \( z + z^{-1} = x \), we can express this as a polynomial in \( x \) of degree \( \frac{p-1}{2} \).
\end{example}
\begin{example}
	Let \( x = \sqrt m + \sqrt n \) where \( m, n \) are integers, and \( m, n, mn \) are not squares.
	We know that \( n = (x-\sqrt m)^2 = x^2 - 2x\sqrt m + m \), so \( [\mathbb Q(x):\mathbb Q(\sqrt m)] \leq 2 \).
	By symmetry, \( [\mathbb Q(x):\mathbb Q(\sqrt n)] \leq 2 \).
	Note that \( \sqrt m \in \mathbb Q(x) \) because \( \frac{x^2 + m - n}{2x} = \sqrt m \).

	\( m, n \) are not squares, so \( [\mathbb Q(\sqrt m):\mathbb Q] = 2 \).
	By the tower law we have \( [\mathbb Q(x):\mathbb Q] \in {2,4} \).
	If \( [\mathbb Q(x):\mathbb Q] = 2 \), we have \( \mathbb Q(x) = \mathbb Q(\sqrt m) = \mathbb Q(\sqrt n) \).
	In this case, \( \sqrt m = a + b \sqrt n \implies m = a^2 + b^2 n + 2ab \sqrt n \), but \( n \) is not a square, so by rationality, \( ab = 0 \).
	But if \( b = 0 \), \( m \) is a square, and if \( a = 0 \), \( mn = b^2 n^2 \) is a square.
	Hence the degree of the field extension is 4.
\end{example}
\begin{definition}
	An extension \( L / K \) is \emph{algebraic} if all elements of \( L \) are algebraic over \( K \).
\end{definition}
\begin{lemma}
	Let \( M / L / K \), where \( L / K \) is algebraic.
	Suppose \( x \) is algebraic over \( L \).
	Then \( x \) is algebraic over \( K \).
\end{lemma}
\begin{proof}
	There exists \( f = T^n + a_{n-1} T^{n-1} + \dots + a_0 \in L[T] \) where \( f \neq 0 \) and \( f(x) = 0 \).
	Let \( L_0 = K(a_0, \dots, a_{n-1}) \).
	As each \( a_i \in L \) is algebraic over \( K \), we must have that \( [L_0 : K] \) is finite.
	As \( f \in L_0[T] \), \( x \) is algebraic over \( L_0 \).
	So \( [L_0(x):L_0] < \infty \implies [L_0(x):K] < \infty \).
	Hence \( [K(x):K] < \infty \), so \( x \) is algebraic over \( K \).
\end{proof}
\begin{proposition}
	\begin{enumerate}
		\item Finite extensions are algebraic.
		\item \( K(x) \) is algebraic over \( K \) if and only if \( x \) is algebraic over \( K \).
		\item If \( M / L / K \), we have \( M / K \) is algebraic if and only if \( M / L \) and \( L / K \) are algebraic.
	\end{enumerate}
\end{proposition}
\begin{proof}
	\begin{enumerate}
		\item \( [L : K] < \infty \), so for all \( x \in L \), \( [K(x) : K] < \infty \), so \( x \) is algebraic.
		\item Certainly if \( K(x) \) is algebraic over \( K \), we have that \( x \) is algebraic over \( K \).
			Conversely, if \( x \) is algebraic over \( K \), \( [K(x) : K] \) is finite, so it is algebraic by part (i).
		\item Suppose \( M / K \) is algebraic.
			Then for all \( x \in M \), we have that \( x \) is algebraic over \( K \), so it satisfies a polynomial \( f \in K[T] \).
			Hence \( f \in L[T] \) is another polynomial that \( x \) satisfies, so \( M / L \) is algebraic.
			\( L / K \) is clearly algebraic because it is contained within \( M \).

			Conversely, suppose \( M / L \) and \( L / K \) are algebraic.
			Let \( x \in M \).
			Then by the previous lemma, \( x \) is algebraic over \( K \) as required.
	\end{enumerate}
\end{proof}
\begin{example}
	Let \( K = \mathbb Q \) and \( L = {x \in \mathbb C \mid x \text{ is algebraic over } \mathbb Q} = \overline{\mathbb Q} \).
	This extension \( \overline{\mathbb Q}/\mathbb Q \) is algebraic, but not finite.
	Indeed, for every \( n \geq 1 \), \( \sqrt[n]{2} \in L \), and \( [\mathbb Q(\sqrt[n]{2}):\mathbb Q] = n \) by irreducibility of \( T^n - 2 \).
	In particular, \( L \) contains subfields of arbitrarily large degree, so cannot be a finite extension.
\end{example}

\subsection{Algebraic numbers in the real line and complex plane}
Traditionally, we call \( x \in \mathbb C \) algebraic if it is algebraic over \( \mathbb Q \), otherwise it is transcendental.
\( \overline{\mathbb Q} = {x \mid x \text{ algebraic}} \) is a proper subfield of \( \mathbb C \).
Indeed, \( \mathbb Q[T] \) is a countable set, and \( \mathbb C \) is uncountable.
However, it is difficult to explicitly find an element of \( \mathbb C \setminus \overline{\mathbb Q} \), or to show that a given number is transcendental.
\begin{example}
	Liouville's constant \( c = \sum_{n \geq 1} 10^{-n!} \) is transcendental, as proven in IA Numbers and Sets.
	This can be proven by showing that algebraic numbers cannot be `well approximated' by rationals.
\end{example}
\begin{example}
	Hermite and Lindemann showed that \( e \) and \( \pi \) are transcendental.
\end{example}
\begin{example}
	Let \( x, y \) be algebraic, and \( x \neq 0,1 \).
	Gelfond and Schneider showed that \( x^y \) is algebraic if and only if \( y \) is rational.
	In particular, \( e^\pi = (-1)^{-i} \) is transcendental.
\end{example}

\subsection{Ruler and compass constructions}
\begin{definition}
	A \emph{ruler and compass construction} in plane geometry is a drawing constructed with the following methods.
	\begin{enumerate}
		\item Given \( P_1, P_2, Q_1, Q_2 \) in the plane and \( P_i \neq Q_i \), we can construct the point of intersection of the lines \( P_1Q_1 \) and \( P_2Q_2 \), if indeed they do intersect.
		\item Given \( P_1, P_2, Q_1, Q_2 \) in the plane and \( P_i \neq Q_i \), we can construct the points of intersection of the circles with centres \( P_i \) that pass through the \( Q_i \), if they intersect.
		\item Similarly we can construct the points of intersection of a line and a circle.
	\end{enumerate}
	A point \( (x,y) \in \mathbb R^2 \) is \emph{constructible} from a set \( {(x_1, y_1), \dots, (x_n, y_n)} \) if it can be obtained by finitely many expansions of the set under applications of the above operations.
	A real number \( x \in \mathbb R \) is \emph{constructible} if \( (x,0) \) is constructible from \( {(0,0), (1,0)} \).
\end{definition}
\begin{remark}
	Every rational is constructible.
	Square roots of constructible numbers are constructible.
\end{remark}
\begin{definition}
	Let \( K \subseteq \mathbb R \) be a subfield of the reals.
	We say \( K \) is \emph{constructible} if there exists \( n \in \mathbb N \) and fields \( Q = F_0 \subset F_1 \subset \dots \subset F_n \subseteq \mathbb R \) and \( a_i \in F_i \) for \( 1 \leq i \leq n \) such that
	\begin{enumerate}
		\item \( K \subseteq F_n \);
		\item \( F_i = F_{i-1}(a_i) \);
		\item \( a_i^2 \in F_{i-1} \).
	\end{enumerate}
\end{definition}
\begin{remark}
	By conditions (ii) and (iii), \( F_i / F_{i-1} \) is at most a quadratic extension.
	Then, by the tower law, \( F_n / \mathbb Q \) has degree a power of two, so \( K / \mathbb Q \) is a finite extension with degree a power of two.
\end{remark}
\begin{theorem}
	If \( x \) is constructible, \( \mathbb Q(x) \) is constructible.
\end{theorem}
\begin{proof}
	Let \( K = \mathbb Q(x) \).
	We show that if \( (x,y) \) can be constructed with \( k \) steps, \( \mathbb Q(x,y) \) is a constructible extension of \( \mathbb Q \).
	By induction, suppose \( \mathbb Q = F_0 \subset \dots \subset F_n \) satisfy conditions (ii) and (iii) such that the coordinates of the points obtained after \( k-1 \) constructions lie in \( F_n \).

	The intersection point of two lines has coordinates given by rational functions of the coordinates of the points \( P_i, Q_i \) with rational coefficients.
	In particular, if the \( k \)th construction is of this type, the intersection point has coordinates in \( F_n \).
	We can similarly see that the intersection points of two circles and the intersection points of a line and a circle have coordinates given by quadratic equations \( a \pm b \sqrt e, c \pm d \sqrt e \), where \( a, b, c, d, e \) are rational functions of the coordinates \( P_i, Q_i \).
	Thus the new points have coordinates which lie in \( F_n(\sqrt e) \), a constructible extension of \( \mathbb Q \) as required.
\end{proof}
\begin{corollary}
	If \( x \) is constructible, \( x \) is algebraic over \( \mathbb Q \) and the degree of the minimal polynomial is a power of two.
\end{corollary}
\begin{remark}
	One can show that if \( \mathbb Q(x) \) is constructible, we also have \( x \) is constructible, so the above theorem is a bi-implication.
	However, this will not be required for our purposes in this course.
\end{remark}

\subsection{Classical problems}
\begin{theorem}
	It is impossible to square the circle.
\end{theorem}
\begin{proof}
	The statement is to construct a square with area equal to that of a given circle.
	In particular, we must construct \( \sqrt \pi \).
	Suppose such a construction can occur.
	Then \( \pi \) is also constructible.
	But \( \pi \) is transcendental and hence inconstructible.
\end{proof}
\begin{theorem}
	It is impossible to duplicate the cube.
\end{theorem}
\begin{proof}
	To duplicate the cube, one must be able to construct \( \sqrt[3]{2} \).
	The minimal polynomial of \( \sqrt[3]{2} \) is \( X^3 - 2 \).
	This can be easily checked with Eisenstein's criterion.
	Since the minimal polynomial is of degree not a power of two, \( \sqrt[3]{2} \) is inconstructible.
\end{proof}
\begin{theorem}
	It is impossible to trisect a given angle.
\end{theorem}
\begin{proof}
	If we can trisect any constructible angle, we can in particular trisect the (constructible) angle \( \frac{2\pi}{3} \), for example to construct a regular nonagon.
	Then the angle \( \frac{2\pi}{9} \) would be constructible, so its sine and cosine would be constructible.
	By the triple angle formula for cosine,
	\[ \cos 3\theta = 4\cos^3 \theta - 3 \cos\theta \implies 4\cos(\frac{2\pi}{9})^3 - 3\cos(\frac{2\pi}{9}) = \cos(\frac{2\pi}{3}) \]
	Hence \( \cos(\frac{2\pi}{9}) \) is a root of \( 8X^3 - 6X + 1 \).
	In particular, \( 2\cos(\frac{2\pi}{9}) - 2 \) is a root of \( X^3 + 6X^2 + 9X + 3 \), which can be shown to be irreducible by Eisenstein's criterion.
	But this has degree 3, so \( \deg_{\mathbb Q} \cos (\frac{2\pi}{9}) = 3 \), so this is inconstructible.
	In particular, the regular nonagon is inconstructible.
\end{proof}
We will later prove the following theorem.
\begin{theorem}[Gauss]
	A regular \( n \)-gon is is constructible if and only if \( n \) is the product of a power of two and distinct \emph{Fermat primes}, which are the primes of the form \( 2^{2^k} + 1 \).
\end{theorem}
