
\chapter{Introduction}

Vectors and Matrices covers topics in both algebra and geometry, and the way in which they relate to one another. The course uses approaches that are quite varied in their nature (can be abstract or more concrete, conceptual or more computational, and so on). You will need to be able to fluently switch between these approaches.

The course assumes that you are vaguely familiar with Euclidean and coordinate geometry, along with the idea of geometric transformations.

\section{Course Structure}

This course is divided into a number of chapters.

\begin{enumerate}
	\item \emph{Complex Numbers}

	This chapter takes the point of view of thinking of points in the plane as pairs of real numbers, and defining `multiplication' on it to turn it into the complex numbers.

	\item \emph{Vectors in Three Dimensions}
	
	Here, we will recap on the relationship between three dimensional vectors and some of their geometrical applications, and we will discuss things like the dot and cross product. Towards the end of that discussion, we will introduce the `index notation', a powerful and helpful notation for dealing with vectors. We will also introduce the `summation convention', which is also incredibly useful.

	\item \emph{Vectors in a General Setting}
	
	This chapter will discuss what vectors are in general, and different ways of looking at them. We will be particularly concerned with vectors in $\mathbb{R}^n$ and $\mathbb{C}^n$, that is, vectors who's entries are in $\mathbb{R}$ and $\mathbb{C}$ respectively.

	\item \emph{Matrices and Linear Maps}
	
	Picking up on the idea of generalizing vectors, this chapter will consider the idea of a `linear map', an abstraction of matrices. 

	\item \emph{Determinants and Inverses}
	
	This chapter will detail how to define and compute determinants of general $n \times n$ matrices. This will take two points of view, in that we need to be able to compute them but we also must understand what they mean. The relation between determinants and finding inverses of matrices will also be considered.

	\item \emph{Eigenvalues and Eigenvectors}
	
	This chapter also involves both geometry and algebra. The core question of this chapter is: given a linear map or matrix, what does it act on in a very straightforward way?

	\item \emph{Changing Basis, Canonical Forms and Symmetries}
	
	In this final chapter, we will consider a set of far reaching results by trying to describing an arbitrary linear map. These ideas are far reaching, and immensely useful.
\end{enumerate}

\section{Differences to the Lecture Course}

This set of notes may diverge slightly from the lectures. If this occurs, I will attempt to describe the differences in this section.
For now, while these notes are incomplete, I will leave it up to the reader to check themselves what is included or missing.  
