\documentclass{scrartcl}

\usepackage[noindent]{handout}
\usepackage{multicol}
\usepackage{amsmath}
\usepackage{bm}
\newcommand{\vv}[1]{\boldsymbol{\mathbf{#1}}}

\newtheorem*{lemma}{Lemma}
\theoremstyle{definition}
\newtheorem*{definition}{Definition}

% \newcommand{\vocab}[1]{\textbf{\color{blue} #1}} % Coloured vocab
\newcommand{\vocab}[1]{\emph{#1}}
\newcommand{\hh}[1]{\hat{\vv{#1}}}

\title{Orbits}
\author{Adam Kelly (\texttt{ak2316})}
\date{\today} 
 
\begin{document}

\maketitle  

\subsection*{Central Forces}

If a force is given by a potential $V$, depending only on the distance from the origin, then
$$
\vv F(\vv x) = - \nabla V(|\vv x|) = - \frac{dV}{dr} \hh r,
$$
where $\hh r = \vv x/|\vv x|$. Such forces are known as \emph{central forces}.

\subsection*{Angular Momentum}

An important result when dealing with central forces is the existence of another conserved quantity, \emph{angular momentum}. We define angular momentum to be
$$
\vv L = \vv x \times \vv p = m \vv x \times \dot{\vv x}.
$$
With a general force $\vv F$, we have $d \vv L/dt = \vv x \times \vv F = \vv \tau$, a quantity named \emph{torque}. If a force is central, then $\vv F \parallel \vv x$, and thus $\vv \tau = \vv 0$, and $\vv L$ is conserved.

\end{document}