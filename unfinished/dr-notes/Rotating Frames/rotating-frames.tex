\documentclass{scrartcl}

\usepackage[noindent]{handout}
\usepackage{multicol}
\usepackage{amsmath}
\usepackage{bm}
\newcommand{\vv}[1]{\boldsymbol{\mathbf{#1}}}

\newtheorem*{lemma}{Lemma}
\theoremstyle{definition}
\newtheorem*{definition}{Definition}

% \newcommand{\vocab}[1]{\textbf{\color{blue} #1}} % Coloured vocab
\newcommand{\vocab}[1]{\emph{#1}}
\newcommand{\hh}[1]{\hat{\vv{#1}}}

\title{Rotating Frames}
\author{Adam Kelly (\texttt{ak2316})}
\date{\today} 
 
\begin{document}

\maketitle  

\subsection*{Motion in Rotating Frames}

Suppose that $\FF$ is an inertial frame, and $\FF'$ is rotating about the $z$ axis with angular velocity $\vv \omega = \omega \vv e_z$ with respect to $\FF$.

Suppose we have basis vectors $\{\vv e_i\}$ and $\{\vv e_i'\}$ in $\FF$ and $\FF'$ respectively. If a particle is at rest in $\FF'$, then in $\FF$ its velocity is given by
	$$
	\left(\frac{d \vv r}{d t}\right)_{\FF} = \vv \omega \times \vv r.
	$$
	Of course, this also applies to the basis vectors in $\FF'$, with
	$$
	\left(\frac{d \vv e_i'}{d t}\right)_{\FF} =\vv \omega \times \vv e_i'.
	$$

	Now for some vector $\vv a$, we can write it in the $\{\vv e_i'\}$ basis as
	$$
\vv a = \sum_i a_i'(t) \vv e_i'.
$$
In the frame $\FF'$, the basis vectors $\vv e_i'$ are constant, and thus the derivative of $\vv a$ is given by
$$
\left(\frac{d \vv a}{d t}\right)_{\FF'} = \sum_i \frac{d a_i'(t)}{dt} \vv{e}_i'.
$$
In the frame $\FF'$ however, the basis vectors $\{\vv e_i'\}$ are not constant, and we have
$$
\left(\frac{d \vv a}{d t}\right)_{\FF} = \sum_i \frac{d a_i'(t)}{dt} \vv{e}_i' + \sum_i a_i'(t) \vv \omega \times \vv e_i' = \left(\frac{d \vv a}{d t}\right)_{\FF'} +\vv \omega \times \vv a
$$

\subsection*{Change of Frame Operator}

Let $\FF$ be an inertial frame, and $\FF'$ be rotating relative to $\FF$ with angular velocity $\vv \omega$.
Then we have
$$
\left(\frac{d}{d t}\right)_{\FF} = \left(\frac{d}{d t}\right)_{\FF'} +\vv \omega \times 
$$

\subsection*{Velocity and Acceleration}

Using the change of frame operator, we can see that
$$
\left(\frac{d \vv r}{d t}\right)_{\FF} = \left(\frac{d \vv r}{d t}\right)_{\FF'} +\vv \omega \times \vv r,
$$
and applying the operator again (and noting that $\dot{\vv \omega}$ is the same in both frames), we have
$$
\left(\frac{d^2 \vv r}{d t^2}\right)_{\FF} = \left(\frac{d^2 \vv r}{d t^2}\right)_{\FF'} + 2 \vv \omega \times \left(\frac{d \vv r}{d t}\right)_{\FF'} + \dot{\vv \omega} \times \vv r + \vv \omega \times (\vv \omega \times \vv r).
$$

\subsection*{Force in the Rotating Frame}

Since $\FF$ is an inertial frame, we have $m\left(d^2 \vv r/ dt^2\right)_{\FF} = F$ by Newton's laws. This allows us to write down what the force appears to be in the frame $\FF'$ (as if the observer in the frame was trying to apply Newton's laws).

$$
m \left(\frac{d^2 \vv r}{dt^2}\right)_{\FF'} = \vv F - 2m \vv \omega \times \left(\frac{d \vv r}{d t}\right)_{\FF'} - m\dot{\vv \omega} \times \vv r - m\vv \omega \times (\vv \omega \times \vv r).
$$

The additional terms on the right are known as \emph{fictitious forces}, each with a different name.

\begin{enumerate}
	\item \emph{Coriolis force}. $- 2m \vv \omega \times \left(\frac{d \vv r}{d t}\right)_{\FF'}$.
	\item \emph{Euler force}. $- m\dot{\vv \omega} \times \vv r$.
	\item \emph{Centrifugal force}. $- m\vv \omega \times (\vv \omega \times \vv r)$.
\end{enumerate}

\end{document}