\documentclass{scrartcl}

\usepackage[noindent]{handout}
\usepackage{multicol}
\usepackage{amsmath}
\usepackage{bm}
\newcommand{\vv}[1]{\boldsymbol{\mathbf{#1}}}

\newtheorem*{lemma}{Lemma}
\theoremstyle{definition}
\newtheorem*{definition}{Definition}

% \newcommand{\vocab}[1]{\textbf{\color{blue} #1}} % Coloured vocab
\newcommand{\vocab}[1]{\emph{#1}}
\newcommand{\hh}[1]{\hat{\vv{#1}}}

\title{Rigid Bodies}
\author{Adam Kelly (\texttt{ak2316})}
\date{\today} 
 
\begin{document}

\maketitle  

% \emph{Topics: the $u(\theta)$ equation; escape velocity; Kepler's laws; stability of orbits; motion in a repulsive potential (Rutherford scattering). }

In many cases, we want to study a system of particles such that the particles are constrained such that their relative positions are fixed. Such systems correspond to solid objects that cannot deform, break or bend. 
Such systems are called \emph{rigid bodies}.

\subsection*{Angular Velocity}

Suppose a rigid body $\BB$ is rotating about a fixed axis with angular speed $\omega$. If $\hh n$ is the unit vector parallel to the rotation axis, we define the \emph{vector angular velocity} of $\BB$ to be
$$
\vv \omega = \omega \hh n,
$$
where we choose the sign such that a positive $\omega$ corresponds to a right-handed rotation about the axis in the direction $\hh n$.

If $P$ is a particle in the rigid body $\BB$, then its velocity due to the rotation of the body is given by
$$
\vv v = \vv \omega \times (\vv r - \vv b),
$$
where $\vv b$ is the position vector corresponding to any fixed point $B$ on the rotation axis.

\subsection*{Moment of Inertia}

Now suppose we have a rigid body made up of $N$ particles, rotating with angular velocity $\vv \omega$. Then we could write down the kinetic energy for the rigid body as
$$
T = \frac{1}{2} \sum_{i} m_i  \dot{\vv x_i} \cdot \dot{\vv x_i} = \frac{1}{2}I \omega^2,
$$
where
$$
I = \sum_{i = 1}^N m_i d_i^2
$$
is the \vocab{moment of inertia}, and $d_i$ is the perpendicular distance from particle $i$ to the axis of rotation.

The intuition behind moments of inertia is that they are to rotations what mass is to translations: the larger $I$ is, the more energy you need to supply to the body to make it spin.

\subsection*{Angular Momentum and Moments of Inertia}

Recall that the angular momentum of a body is
$$
\vv L = \sum_i m_i \vv x_i \times  \dot{\vv x_i}.
$$
Suppose that the body is rotating with angular velocity $\omega \hh n$ about an axis through the origin. Then the component of angular momentum in the $\hh n$ direction is given by
\begin{align*}
	\vv L \cdot \hh n &= \sum_i m_i \vv x_i \times (\vv \omega \times \vv x) \cdot \hh n \\
	&= \omega \sum_i m_i (\vv x_i \times \hh n) \cdot (\vv x_i \times \hh n) \\
	&= I \omega.
\end{align*}
If there's a torque $\tau$ acting on the rigid body in the same direction as the angular velocity, so $\vv \tau = \tau \hh n$, then $I \dot{\omega} = \tau$.

\subsection*{Calculating Moment of Inertia}

If we consider the rigid body as a continuous object with density distribution $\rho (\vv x)$, then the moment of inertia is given by
$$
I = \int \rho (\vv x) x_{\perp}^2 \; dV
$$

Some examples of common moments of inertia are given below.
\begin{enumerate}
	\item \emph{Thin Rod}. With mass $M$ and length $2a$ about an axis perpendicular to the rod through its center, we have $I = \frac{1}{3}M a^2$.
	\item \emph{Circular Hoop}. With mass $M$ and radius $a$, through an axis perpendicular to the plane of the hoop through its center, we have $I = Ma^2$.
	\item \emph{Circular Disk}. A disk with mass $M$ and radius $A$, through an axis perpendicular to the plane of the disk through its center, we have $I = \frac{1}{2}Ma^2$.
	\item \emph{Solid Sphere}. A sphere with mass $M$ and radius $a$, through an axis through its center, we have $I = \frac{2}{5}Ma^2$. 
\end{enumerate}

We also have the \emph{parallel axis theorem}, which says if $I_G$ is the moment of inertia about an axis through the centre of mass, and $I$ is the moment of inertia about a parallel axis a distance $a$ away, then
$$
I= I_G + Ma^2.
$$

\end{document}