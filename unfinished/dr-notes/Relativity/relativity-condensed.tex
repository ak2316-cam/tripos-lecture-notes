\documentclass{scrartcl}

\usepackage[noindent]{handout}
\usepackage{multicol}
\usepackage{amsmath}

\newtheorem*{lemma}{Lemma}
\theoremstyle{definition}
\newtheorem*{definition}{Definition}

% \newcommand{\vocab}[1]{\textbf{\color{blue} #1}} % Coloured vocab
\newcommand{\vocab}[1]{\emph{#1}}
\newcommand{\hh}[1]{\hat{\vv{#1}}}

\title{Special Relativity}
\author{Adam Kelly (\texttt{ak2316})}
\date{\today} 
 
\begin{document}

\maketitle  

% \emph{Topics: the $u(\theta)$ equation; escape velocity; Kepler's laws; stability of orbits; motion in a repulsive potential (Rutherford scattering). }

\subsection*{Axioms of Special Relativity}

\begin{enumerate}
	\item \emph{Galilean relativity}. The laws of physics are the same in all inertial reference frames.
	\item \emph{Speed of light}. The speed of light in a vacuum is the same in all inertial reference frames.
\end{enumerate}


\subsection*{Lorentz Transformations}

In special relativity we think in terms of events: instantaneous point-like occurrences. These are specified by four coordinates, one of time and three of position, like $(t, x, y, z)$.
These coordinates will be measured differently in different inertial frames, and to make our axioms hold we need to use a new set of transformation laws.

If we have two inertial frames $\FF$ and $\FF'$, and $\FF'$ is moving at speed $v$ relative to $\FF$ in the $x$ direction, then we have
\begin{align*}
	x' &= \gamma \left(x - \frac{v}{c} ct\right) \\
	y' &= y \\
	z' &= z \\
	ct' &= \gamma\left(ct - \frac{v}{c}x\right),
\end{align*}  
where
$$
\gamma = \sqrt{\frac{1}{1 - v^2/c^2}}.
$$


\subsection*{Relativistic Physics}

\begin{enumerate}
	\item \emph{Relativity of simultaneity}. Events with the same $t$ no longer correspond to events with equal $t'$, so what is simultaneous in one frame is not necessarily simultaneous in another.
	\item \emph{Causality}. While observers can disagree about the temporal ordering of events, if an event is within the `light cone' of an event $P$ (within the region of a space-time diagram traced out by light passing through $P$) then all observers will agree on a causal ordering.  
	\item \emph{Time Dilation}. Consider a clock sitting stationary at the origin of the frame $\FF'$, ticking at intervals of $T'$. The tick events in frame $\FF'$ will occur at $(t'_1, 0), (t'_1 + T', 0), \dots$. 
	
	In the frame $\FF$, using the Lorentz transformations, we see that the time interval between ticks is $T = \gamma T'$. So the ticks are longer in the stationary frame. 
	\item \emph{Length Contraction}. Consider a rod of length $L'$, stationary in the frame $\FF'$. 
	The endpoints of the rod are given by $x' = 0$ and $x' = L'$, which are then mapped into $x = vt$ and $x = vt + L'/\gamma$. 
	So in $\FF$, the length of the rod is $L'/\gamma$, and thus lengths of moving objects are contracted in the direction of motion. To deal with this, we define \emph{proper length} to be the length measured in an objects rest frame.
	\item \emph{Composition of Velocities}. Suppose a particle moves with constant velocity $u'$ in frame $\FF'$, which moves with velocity $v$ relative to $\FF$. We want to find its velocity $u$ in the frame $\FF$.

	In $\FF'$, for the particle we have $x' = u't'$. Substituting this into the Lorentz transformation laws, we have
	$$
	u = \frac{x}{t} = \frac{\gamma(x' + vt')}{\gamma(t' + vx'/c^2)} = \frac{u' + v}{1 + u' v/c}.
	$$
	\item \emph{Newtonian Limit}. When $v/c$ is very small, the Lorentz transformations approximate the Galilean transformations that we use in Newtonian mechanics.
\end{enumerate}


\subsection*{Geometry of Minkowski Space}

Consider two events $P_1$ and $P_2$ have coordinates $(t_1, x_1)$ and $(t_2, x_2)$ in the frame $\FF$. These events are separated by $\Delta t = t_1 - t_2$ in time and $\Delta x = x_1 - x_2$.

We define the \emph{invariant interval} between $P_1$ and $P_2$ to be
$$
\Delta s^2 = c^2 \Delta t^2 - \Delta x^2.
$$
We say it is invariant because it is the same in all inertial reference frames, that is, it is invariant under Lorentz transformations\footnote{Feel free to check this.}.

It is possible for $\Delta s^2$ to be either positive or negative. If it is positive, we say the events are \emph{timelike} separated, and if it is negative we say they are \emph{spacelike} separated, and if it is zero, we say they are \emph{lightlike} separated. Events that are in each other's light-cones are timelike, and can influence one another.

\subsection*{4-Vectors}

We can view Minkowski space as a vector space equipped with the \emph{Minkowski inner product}\footnote{This is not an inner product in the normal sense since it's not positive definite}. The coordinates of some event $P$ in the frame $\FF$ can be written as a \emph{4-vector}
$$	
X= \begin{pmatrix}
	ct \\ x \\ y \\ z
\end{pmatrix},
$$
and then the Minkowski inner product is given by
$$
X \cdot Y = X^T \eta Y, \quad \quad \text{where} \quad \eta = \begin{pmatrix}
	1 & 0 & 0 & 0 \\
	0 & -1 & 0 & 0 \\
	0 & 0 & -1 & 0 \\
	0 & 0 & 0 & -1
\end{pmatrix}.
$$
Taking $X \cdot X$ gives the invariant interval between the origin and $P$, and is known as the \emph{Minkowski metric}\footnote{Again, this isn't actually a metric in the normal sense since it's not positive definite}.

\subsection*{The Lorentz Group}

Using $4$-vectors, we can view Lorentz transformations as linear transformations from the coordinates of one inertial frame $\FF$ to another $\FF'$. This would be represented by a $4 \times 4$ matrix $\Lambda$, with
$$
X' = \Lambda X.
$$
Since such transformations must preserve the invariant interval, they must preserve the Minkowski inner product\footnote{These are the analogue of orthogonal matrices for Minkowski space} and hence satisfy
$$
\Lambda^T \eta \Lambda = \eta.
$$
The group of all such matrices is the \emph{Lorentz group}, and is generated by 
$$
\left(\begin{array}{c|ccc}
	1 & 0 & 0 & 0 \\
	\hline 0 & & & \\
	0 & & \mathrm{R} & \\
	0 & & &
	\end{array}\right) 
	\quad \quad \text{and} \quad \quad \left(\begin{array}{cc|cc}
		\gamma & -\gamma v / c & 0 & 0 \\
		-\gamma v / c & \gamma & 0 & 0 \\
		\hline 0 & 0 & 1 & 0 \\
		0 & 0 & 0 & 1
		\end{array}\right),
$$
where the first corresponds to a rotation of space, leaving time intact (so we need $R$ to be orthogonal), and the second corresponds to a Lorentz boost along the $x$ direction.

\subsection*{Relativistic Kinematics}

\begin{enumerate}
	\item \emph{Proper Time}. To do kinematics, it helps to have a consistant notion of time. We do this by defining \emph{proper time} $\tau$, which is the time experienced by a particle in its own reference frame. It is given by
	$$
	\Delta \tau = \frac{\Delta s}{c},
	$$
	in all reference frames.
	\item \emph{Relating Proper Time to Measured Time}. By considering small changes, if $\vv u$ is the velocity of the particle in the frame $\FF$, we get that
	$$
	d \tau = \frac{1}{c} d s = \frac{1}{c} \sqrt{c^2 d t^2 - |d\vv x|^2} = \sqrt{1 - u^2/c^2} d t, \quad \text{so} \quad \frac{dt}{d \tau} = \gamma_u.
$$
We can then get the total time experienced by a particle as
$$
T = \int d \tau = \int \frac{1}{\gamma_u} d t.
$$
\item \emph{$4$-Position}. Using proper time, we can parametrise the trajectory of a particle using a $4$-vector
$$
X(\tau) = \begin{pmatrix}
	ct(\tau) \\
	\vv x(\tau)
\end{pmatrix}.
$$
\item \emph{$4$-Velocity}. Then we can define the \emph{$4$-velocity} as
$$
U = \frac{dX}{d \tau} = \begin{pmatrix}
	c dt/d\tau \\
	d \vv x / d \tau
\end{pmatrix} \quad \quad \text{that is,} \quad \quad U = \frac{dt}{d\tau} \begin{pmatrix}
	c \\ \vv u
\end{pmatrix} = \gamma_u \begin{pmatrix}
	c \\ \vv u
\end{pmatrix},
$$
where $\vv u = d \vv x/ dt$. Since this is a $4$-vector, it transforms like $U' = \Lambda U$. 
\item \emph{$4$-Momentum}. For a particle of mass $m$, the \emph{$4$-momentum} is defined to be
$$
P = mU = \begin{pmatrix}
	mc \gamma_u \\
	m \gamma_u \vv u
\end{pmatrix}.
$$
The spacial components of $P$ give us the \emph{relativistic $3$-momentum}, $\vv p = m \gamma_u \vv u$. $4$-momentum is conserved in the absence of external forces, and for a system of particles the total $4$-momentum is the sum of the $4$-momenta of the particles.
\item \emph{Relativistic Energy}. We define the \emph{relativistic energy} of a particle to be $E = P^0 c$, so that
$$
E = m \gamma c^2 = mc^2 + \frac{1}{2} m |\vv u|^2 + \cdots,
$$
and for a stationary particle we have $E = mc^2$. 

Since we can calculate the Lorentz invariant quantity $P \cdot P$ in the particles rest frame, we have
$$
P \cdot P = \frac{E^2}{c^2} - |\vv p|^2 = m^2 c^2,
$$
so generally we have 
$$
E^2 = |\vv p|^2 c^2 + m^2 c^4.
$$
Relativistic energy is a conserved quantity, which includes mass (just another form of energy, it's not conserved separately).
\item \emph{Massless Particles}. For particles that have zero mass (like photons), they can still have momentum and energy. However, $P \cdot P = 0$, and we can't define proper time for the particles. We still have $E^2 = |\vv p|^2 c^2$, and thus
$$
P = \frac{E}{c} \begin{pmatrix}
	1 \\ \vv n
\end{pmatrix},
$$
where $\vv n$ is a unit vector in the direction of propagation. We also have $E = hc/\lambda = hf$, where $h$ is Planck's constant, $\lambda$ is the wavelength, and $f$ is the frequency of the photon/particle. 
\end{enumerate}

\subsection*{Rapidities}

Consider a $2 \times 2$ matrix corresponding to a Lorentz boost by $v$ in the $x$ direction. Then if $\beta = v/c$, we have
$$
\Lambda[\beta] = \begin{pmatrix}
	\gamma & -\gamma \beta \\
	-\gamma \beta & \gamma
\end{pmatrix}, \quad \text{and the composition law} \quad \Lambda[\beta_1] \Lambda[\beta_2] = \Lambda\left[\frac{\beta_1 + \beta_2}{1 + \beta_1 \beta_2}\right],
$$
exactly our composition law from before.

A slightly nicer way to deal with this is with \emph{rapidities}.
We define the \emph{rapidity} of a Lorentz boost $\phi$ such that
$$
\beta = \tanh \phi, \quad \gamma = \cosh \phi,\quad \gamma \beta = \sinh \phi.
$$
Then
$$
\Lambda[\beta] = \begin{pmatrix}
	\cosh \phi & - \sinh \phi \\
	-\sinh \phi & \cosh \phi 
\end{pmatrix} = \Lambda(\phi), \quad \text{and also} \quad \Lambda(\phi_1) \Lambda(\phi_2) = \Lambda(\phi_1 + \phi_2),
$$
which is much nicer. This also shows that Lorentz boosts correspond to hyperbolic rotations in spacetime.



% In special relativity, we work not in Euclidean space but \emph{Minkowski space}, typically either $(1 + 1)$-dimensional or $(1 + 3)$-dimensional. 

% To make our axioms true, we need a new way to transform between different inertial frames.

%  We normally think of 


% \section{Newtonian Dynamics}

% \subsection{Particles and Reference Frames}

% Newtonain mechanics uses the idea of a force to build up a mathematical framework which we can use to describe the motion of objects, typically particles.

% \begin{definition}[Particle] 
% 	An object of insignificant size to which we assign physical attributes.
% \end{definition}

% We think of particles as moving through space and time, which are treated (for the time being) as separate entities. The position of a particle can be described by its vector displacement $\vv r$ from an origin $O$, fixed in some rigid\footnote{Rigid as in there is a well defined notion of direction relative to it.} reference frame $\mathcal{F}$. 

% We define the \vocab{velocity} $\vv v$ and \vocab{acceleration} $\vv a$ to be
% $$
% \vv v = \frac{d \vv r}{\dd t} = \dot{\vv r}, \quad \text{and} \quad \vv a = \frac{d \vv v}{d t} = \frac{d^2 \vv r}{d t^2}  = \ddot{\vv r}.
% $$

% In Newtonian mechanics, we particularity care about \vocab{inertial reference frames}, 
% \section{Basic Concepts}

% Space and time, frames of reference, Galilean transformations. Newton's laws. Dimensional analysis. Examples of forces, including gravity, friction and Lorentz.

% \section{Newtonian Dynamics of a Single Particle}

% Equation of motion in Cartesian and plane polar coordinates. Work, conservative forces and potential energy, motion and the shape of the potential energy function; stable equilibria and small oscillations; effect of damping.

% Angular velocity, angular momentum, torque. 

% Orbits: the $u(\theta)$ equation; escape velocity; Kepler's laws; stability of orbits; motion in a repulsive potential (Rutherford scattering). 

% Rotating frames: centrifugal and Coriolis forces.


% \section{Newtonian Dynamics of Systems of Particles}

% Momentum, angular momentum, energy. Motion relative to the centre of mass; the two body problem. Variable mass problems; the rocket equation.

% \section{Rigid Bodies}

% Moments of inertia, angular momentum and energy of a rigid body. Parallel axis theorem. Simple examples of motion involving both rotation and translation (e.g. rolling).

% \section{Special Relativity}

% The principle of relativity. Relativity and simultaneity. The invariant interval. Lorentz transformations in (1+1)-dimensional spacetime. Time dilation and length contraction. The Minkowski metric for (1+1)-dimensional spacetime. 

% Lorentz transformations in $(3+1)$ dimensions. 4 -vectors and Lorentz invariants. Proper time. 4 velocity and 4-momentum. Conservation of 4-momentum in particle decay. Collisions. The Newtonian limit.


\end{document}