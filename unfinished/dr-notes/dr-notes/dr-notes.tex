% \documentclass[10pt, a4paper, twocolumn]{scrartcl}
\documentclass[11pt, a4paper]{scrartcl}

\usepackage[noindent]{handout}
\usepackage{multicol}
\usepackage{amsmath}
\usepackage{bm}
% \usepackage[margin=0.35in]{geometry}
\newcommand{\vv}[1]{\boldsymbol{\mathbf{#1}}}

\newtheorem*{lemma}{Lemma}
\theoremstyle{definition}
\newtheorem*{definition}{Definition}

% \newcommand{\vocab}[1]{\textbf{\color{blue} #1}} % Coloured vocab
\newcommand{\vocab}[1]{\emph{#1}}
\newcommand{\hh}[1]{\hat{\vv{#1}}}

\title{Dynamics and Relativity}
\author{Adam Kelly (\texttt{ak2316})}
\date{\today} 
 
\begin{document}

\maketitle  


\section{Newton's Laws and Forces}


In Newtonian mechanics, we use the idea of a force to build up a mathematical framework in which we can describe the motion of objects, typically particles.

\subsection*{Particles \& Inertial Frames}

A \emph{particle} is an idealised object of negligible size and no internal structure to which we assign physical properties such as charge or mass.
To describe the positions of particles over time, we use a \emph{reference frame}, which consists of a choice of origin and some coordinate system which can be used to consistently describe position.

For a particle with position vector $\vv x(t)$, a displacement vector from the origin of our reference frame, we then define the following quantities.
\begin{itemize}
	\item \emph{Velocity}. The velocity $\vv v$ of the particle is $\vv v = \dot{\vv x} = d \vv x(t) / dt$. The speed is then $v = |\vv v|$.
	\item \emph{Acceleration}. The acceleration $\vv a$ of a particle is $\vv a = \ddot{\vv x} = d^2 \vv x(t)/dt^2$.
	\item \emph{Momentum}. The momentum $\vv p$ is then $\vv p = m \vv v = m \dot{\vv x}$.
\end{itemize}


\subsection*{Newton's Laws}

\begin{enumerate}
	\item \emph{Galileo's Law of Inertia}. There exists inertial frames, reference frames in which particles travel at constant velocity when the force acting on them vanishes.
	\item \emph{Newton's Second Law}. If a force $\vv F$ acts on a particle with momentum $\vv p$, then $d\vv p/dt = \vv F$.
	\item \emph{Newton's Third Law}. Every action has an equal and opposite reaction.
\end{enumerate}

All of these are tied together with the \emph{principle of Galilean relativity}, which is that the laws of physics are the same in every inertial reference frame.

\subsection*{Galilean Transformations}

A natural question is how one can transform between different inertial frames. That is, under what transformations do the laws of physics remain unchanged. It is straightforward to see that if we have some $(\vv x, t)$ in an inertial frame $\FF$, then we can construct another inertial frame $\FF'$ by combining:
\begin{itemize}
	\item \emph{Translations}. $\vv x' = \vv x + \vv a$ for a constant vector $\vv a$.
	\item \emph{Rotations}. $\vv x' = R \vv x$, where $R$ is some orthogonal matrix.
	\item \emph{Boosts}. $\vv x' = \vv x + \vv t$, for a constant velocity vector $\vv v$.
	\item \emph{Rescaling}. $\vv x' = \lambda \vv x$ for a constant $\lambda$. 
\end{itemize}
This set of transformations forms the \emph{Galilean group}, the group of symmetries of Newton's laws.

When working with Newton's laws, we assume that there is some `absolute time', $t$, $t'$ in the inertial frames which differ by at most a constant in the two reference frames\footnote{Once we fix the units, of course.}

\subsection*{Examples of Forces}

The \emph{Lorentz force} describes the force on a particle moving in an electromagnetic field. If a particle has position $\vv x$ and charge $q$, and it moves in an electric field $\vv E$ and magnetic field $\vv B$, then the force on the particle is given by
$$
\vv F = q(\vv E + \dot{\vv x} \times \vv B).
$$


The \emph{gravitational force} describes the force due to gravity on a particle. If two particles have masses $m_1$ and $m_2$ and positions $\vv x_1$ and $\vv x_2$, then the force of gravity between them is given by
$$
\vv F_1 = - \frac{G m_1 m_2}{| \vv x_1 - \vv x_2|^3} (\vv x_1 - \vv x_2) = - \vv F_2,
$$
where $G$ is the \vocab{gravitational constant}.


\subsection*{Potentials in One Dimension}

Roughly, the framework of Newtonian mechanics takes in particles and forces, and from that we get the dynamics at play. However, in many cases we think of forces as coming from a more primitive idea, one of \emph{potentials}. We will work in one dimension for the time being.

For a force field $F(x)$, we define the \emph{potential energy} to be a function $V(x)$ such that
$$
F = - \frac{dV}{dx},
$$
so $V = -\int F \; dx$, defined up to a constant. With this, Newton's second law becomes $\dot{p} = -dV/dx$.

With forces in one dimension depending on position only, we have a naturally conserved quantity. We define the \emph{kinetic energy} of a particle to be $T = \frac{1}{2}m \dot{x}^2$. Then if the kinetic and potential energies of a particle in a force field are $T, V$ and $E = T + V$, then $E$, the \emph{total energy}, is conserved\footnote{This can be checked by differentiating.}. One nice thing about having such a conservation quantity is that
$$
E = \frac{1}{2}m \dot{x}^2 + V(x) \implies  \dot{x} = \pm \sqrt{\frac{2}{m} (E - V(x))},
$$
which gives us a first order differential equation for position. We can also get a lot of information about the motion of a particle by sketching the graph of the potential versus position.

\subsection*{Potentials in Three Dimensions}

The concept of potentials generalises into three dimensions, but there is some groundwork needed to be done first. We first will need to redefine kinetic energy to make sense in multiple dimensions. This is done naturally with 
$$
T = \frac{1}{2} m \dot{\vv x} \cdot \dot{\vv x}.
$$

Now consider a force field $\vv F(\vv x)$, depending only on position.  If the force $\vv F$ acts on a particle, moving it from $\vv x(t_1)$ to $\vv x(t_2)$ along a trajectory $\mathcal{C}$, we define the \emph{work done} $W$ to be
$$
W = \int_\mathcal{C} \vv F \cdot d \vv x = \int_{t_1}^{t_2} \vv F \cdot \frac{d \vv x}{d t} d t.
$$
Note that if $F = m \ddot{\vv x}$ (as in there is no change in mass), then the work done is
$$
W = m \int_{t_1}^{t_2} \ddot{\vv x} \cdot \dot{\vv x} \; dt = T(t_2) - T(t_1).
$$

From this, we can see that there is a conserved energy if and only if $\vv F = - \nabla V$, for some potential $V(\vv x)$. If this is the case, then the conserved quantity is $E = T + V$, as before.

\subsection*{Gravity}

Gravity is an example of a conserved force. For a particle of mass $m$, under the influence of gravity from a particle of mass $M$ at the origin, we have the potential energy
$$
V(r) = -\frac{GMm}{r}.
$$
The force of gravity is then given by
$$
\vv F = - \nabla V = -\frac{G Mm}{r^2} \hh r,
$$
with $\hh r$ in the direction of the particle. Note that this force points towards the origin.
It is sometimes convenient to remove the dependence on the mass of the particle $m$, which we do by defining the \emph{gravitational potential} $\phi(r) = -GM/r$, with $V = \phi m$.

If there is many particles say of masses $M_i$ and positions $\vv r_i$, the gravitational potential just adds, with 
\begin{align*}
	\phi(\vv r) &= -G \sum_i \frac{M_i}{|\vv r - \vv r_i|} \\
\implies \vv F &= -Gm \sum_i \frac{M_i}{|\vv r - \vv r_i|^3} (\vv r - \vv r_i).
\end{align*}

\subsection*{Electromagnetism}

If we have time-independent electric and magnetic fields $\vv E$ and $\vv B$, then $\vv E = - \nabla \phi$ for some $\phi(\vv x)$, the \emph{electric potential}. We have conservation of energy for time-independent fields, with the total energy\footnote{Again, this can be checked by differentiating.}
$$
E = \frac{1}{2}m \dot{\vv x} \cdot \dot{\vv x} + q \phi(\vv x). 
$$

Just as objects with mass both feel the effect of and product gravitational force, the same is true of objects with charge. At the origin, a particle of charge $Q$ creates an electric field as
$$
E = - \nabla \left(\frac{Q}{4 \pi \varepsilon_0 r}\right) = \frac{Q}{4 \pi \varepsilon_0} \frac{\hh r}{r^2},
$$
where $r^2 = \vv x \cdot \vv x$. The force between objects of charge $Q$ and $q$ is $\vv F = q \vv E$, or
$$
\vv F = \frac{Qq}{4 \pi \varepsilon_0} \frac{\hh r}{r^2}.
$$
This is the \emph{Coulomb force}, and has the same force form as gravity.

\subsection*{Friction}

In many physical scenarios we deal with forces that do not come from potentials. Friction is one such force, and it is a force that is used to approximate the complex relationship between objects in contact.

One common form of friction is \emph{dry friction}, defined by $\vv F = \mu \vv R$ where $\vv R$ is the reaction force normal to the floor, in the direction that opposes motion. Note that friction never causes motion. Another common type of friction is \emph{fluid drag}. Fluid drag is typically either \emph{linear}, with $\vv F = - \gamma \vv v$ (suitable for slow moving objects in viscous fluids), or \emph{quadratic}, with $\vv F = - \gamma |\vv v| \vv v$ (suitable for fast moving objects in less viscous fluids).


\clearpage

\section{Orbits}


\subsection*{Central Forces}

If a force is given by a potential $V$, depending only on the distance from the origin, then
$$
\vv F(\vv x) = - \nabla V(|\vv x|) = - \frac{dV}{dr} \hh r,
$$
where $\hh r = \vv x/|\vv x|$. Such forces are known as \emph{central forces}.

\subsection*{Angular Momentum}

An important result when dealing with central forces is the existence of another conserved quantity, \emph{angular momentum}. We define angular momentum to be
$$
\vv L = \vv x \times \vv p = m \vv x \times \dot{\vv x}.
$$
With a general force $\vv F$, we have $d \vv L/dt = \vv x \times \vv F = \vv \tau$, a quantity named \emph{torque}. If a force is central, then $\vv F \parallel \vv x$, and thus $\vv \tau = \vv 0$, and $\vv L$ is conserved.

\subsection*{Reduction to One Dimension}

Suppose we had a potential $V(r)$, with $\vv F = - \nabla V = - dV/dr \hh r$. Newton's second law in the plane with plane polar coordinates $(r, \theta)$ is (for constant mass)
\begin{align*}
	m \ddot{\vv x} &= - \frac{dV}{dr} \hh r \\
\implies m (\ddot{r} - r \dot{\theta}^2) \hh r + m (r \ddot{\theta} + 2 \dot{r} + \dot{\theta}) \hh \theta &= -\frac{dV}{dr} \hh r.
\end{align*}
Then since the components are orthogonal, we can then obtain two separate equations. The first (taking the $\hh \theta$ component) is
\begin{align*}
	r \ddot{\theta} + 2 \dot{r} \dot{\theta} = 0 \implies \frac{1}{r} \frac{d}{dt}(r^2 \dot{\theta}) = 0.
\end{align*}
The angular momentum of the system is $\vv L = \vv x \times m \dot{\vv x} = m r^2 \dot{\theta} (\hh r \times \hh \theta)$, and thus $|\vv L| = m r^2 \dot{\theta} = m \ell$, where $\ell$ is the angular momentum per unit mass. So, since angular momentum is a conserved quantity, the equation will always hold.

The other component then gives us the equation
$$
m(\ddot{r} - r \dot{\theta}^2) = - \frac{dV}{dr} \implies m \ddot{r} = -\frac{dV}{dr} + r \dot{\theta^2} = - \frac{dV}{dr} + \frac{m\ell^2}{r^3}.
$$
Thus we have reduced our three dimensional problem of a central force into a one dimensional problem involving the radial component. 

\subsection*{The Effective Potential}

Using our reduced problem, we can `forget' about the $\theta$ component, and consider a one dimensional system, under the influence of the \emph{effective potential}
$$
V_{\text{eff}}(r) = V(r) + \frac{m \ell^2}{2r^2}.
$$
The rightmost term is the \emph{angular momentum barrier}, and it stops the particle from getting to close to the origin.

We can even use this viewpoint in energy conservation, as
$$
E = \frac{1}{2}m \dot{\vv x} \cdot \dot{\vv x} + V(r) = \frac{1}{2}m \dot{r}^2 + V_{\text{eff}}(r).
$$

\subsection*{The $u(\theta)$ Equation}

It turns out that the substitution $u = 1/r$ just seems to work really well when dealing with the system we have setup. Performing this substitution, we have
\begin{align*}
	\dot{r} &= \frac{dr}{d \theta} \dot{\theta} = \frac{dr}{d\theta} \frac{\ell}{r^2} = \ell \frac{du}{d \theta}, \\
	\ddot{r} &= \frac{d}{dt}\left(- \ell \frac{du}{d\theta}\right) = \ell \frac{d^2 u}{d\theta^2}\dot{\theta} = -\ell^2 u^2 \frac{d^2 u}{d \theta^2}.
\end{align*}
Then our equation of motion goes from $m \ddot{r} - m\ell^2/r^3 = F(r)$ to \emph{Binet's equation}:
$$
-mh^2 u^2 \left(\frac{d^2 u }{d \theta^2} + u\right) = F\left(\frac{1}{u}\right).
$$
For inverse square laws (such as gravity), this is a linear equation! After it is solved, we can get back the time dependence with
$\dot{\theta} = \ell u^2$.


\subsection*{The Kepler Problem}

Binet's equation can be used to find the shapes of orbits under an inverse square law. Let
$$
V(r) = \frac{mk}{r}, \quad F = -\frac{mk}{r^2}.
$$
Then Binet's equation is
$$
\frac{d^2 u}{d \theta^2} + u = \frac{k}{u^2}.
$$
The general solution to this equation is
$$
u = \frac{k}{\ell^2} + A \cos(\theta - \theta_0).
$$
Taking $\theta_0 = 0$, we can conclude that for a planet in orbit about the origin(which we can take to be say a fixed planet), the shape of the orbit is given by
$$
r = \frac{p}{1 + e \cos \theta},
$$
with $p = \ell^2/k$ and $e = A \ell^2/k$ is the \emph{eccentricity}. This is always a conic, and if $0 \leq e < 1$ we have an ellipse, if $e > 1$ we have a hyperbola, and if $e = 1$ we have a parabola.

\subsection*{Kepler's Laws}

\begin{enumerate}
	\item The orbit of a planet is an ellipse, with the sun at its focus.
	\item The line between the planet and the sun sweeps out equal areas in equal times.
	\item The square of the orbital period is proportional to the cube of the semi-major axis, so $T^3 \propto a^3$.
\end{enumerate}

The first law follows from our discussion before. The second law follows from conservation of angular momentum. Moving by $d \theta$ gives $dA = \frac{1}{2}r^2 d \theta$ (approximating by a circle), and thus $\frac{dA}{dt} = \frac{\ell}{2}$, which is constant. The third law follows from the area of an ellipse being $A = \pi ab = \frac{\ell}{2}T$ by the second law, but then $b^2 = a^2(1 - e^2)$, and we are done by our definition of $e$.


%\clearpage



\clearpage

\section{Rotating Frames}


\subsection*{Motion in Rotating Frames}

Suppose that $\FF$ is an inertial frame, and $\FF'$ is rotating about the $z$ axis with angular velocity $\vv \omega = \omega \vv e_z$ with respect to $\FF$.

Suppose we have basis vectors $\{\vv e_i\}$ and $\{\vv e_i'\}$ in $\FF$ and $\FF'$ respectively. If a particle is at rest in $\FF'$, then in $\FF$ its velocity is given by
	$$
	\left(\frac{d \vv r}{d t}\right)_{\FF} = \vv \omega \times \vv r.
	$$
	Of course, this also applies to the basis vectors in $\FF'$, with
	$$
	\left(\frac{d \vv e_i'}{d t}\right)_{\FF} =\vv \omega \times \vv e_i'.
	$$

	Now for some vector $\vv a$, we can write it in the $\{\vv e_i'\}$ basis as
	$$
\vv a = \sum_i a_i'(t) \vv e_i'.
$$
In the frame $\FF'$, the basis vectors $\vv e_i'$ are constant, and thus the derivative of $\vv a$ is given by
$$
\left(\frac{d \vv a}{d t}\right)_{\FF'} = \sum_i \frac{d a_i'(t)}{dt} \vv{e}_i'.
$$
In the frame $\FF'$ however, the basis vectors $\{\vv e_i'\}$ are not constant, and we have
$$
\left(\frac{d \vv a}{d t}\right)_{\FF} = \sum_i \frac{d a_i'(t)}{dt} \vv{e}_i' + \sum_i a_i'(t) \vv \omega \times \vv e_i' = \left(\frac{d \vv a}{d t}\right)_{\FF'} +\vv \omega \times \vv a
$$

\subsection*{Change of Frame Operator}

Let $\FF$ be an inertial frame, and $\FF'$ be rotating relative to $\FF$ with angular velocity $\vv \omega$.
Then we have
$$
\left(\frac{d}{d t}\right)_{\FF} = \left(\frac{d}{d t}\right)_{\FF'} +\vv \omega \times 
$$

\subsection*{Velocity and Acceleration}

Using the change of frame operator, we can see that
$$
\left(\frac{d \vv r}{d t}\right)_{\FF} = \left(\frac{d \vv r}{d t}\right)_{\FF'} +\vv \omega \times \vv r,
$$
and applying the operator again (and noting that $\dot{\vv \omega}$ is the same in both frames), we have
$$
\left(\frac{d^2 \vv r}{d t^2}\right)_{\FF} = \left(\frac{d^2 \vv r}{d t^2}\right)_{\FF'} + 2 \vv \omega \times \left(\frac{d \vv r}{d t}\right)_{\FF'} + \dot{\vv \omega} \times \vv r + \vv \omega \times (\vv \omega \times \vv r).
$$

\subsection*{Force in the Rotating Frame}

Since $\FF$ is an inertial frame, we have $m\left(d^2 \vv r/ dt^2\right)_{\FF} = F$ by Newton's laws. This allows us to write down what the force appears to be in the frame $\FF'$ (as if the observer in the frame was trying to apply Newton's laws).

$$
m \left(\frac{d^2 \vv r}{dt^2}\right)_{\FF'} = \vv F - 2m \vv \omega \times \left(\frac{d \vv r}{d t}\right)_{\FF'} - m\dot{\vv \omega} \times \vv r - m\vv \omega \times (\vv \omega \times \vv r).
$$

The additional terms on the right are known as \emph{fictitious forces}, each with a different name.

\begin{enumerate}
	\item \emph{Coriolis force}. $- 2m \vv \omega \times \left(\frac{d \vv r}{d t}\right)_{\FF'}$.
	\item \emph{Euler force}. $- m\dot{\vv \omega} \times \vv r$.
	\item \emph{Centrifugal force}. $- m\vv \omega \times (\vv \omega \times \vv r)$.
\end{enumerate}

\clearpage

\section{Systems of Particles}

Now lets have a look at some systems involving multiple particles.

\subsection*{Centre of Mass \& Conservation of Momentum}

Suppose we have particles with position $\vv x_i$, masses $m_i$, momentum $\vv p_i$ and so on. Newton's second law then becomes
$$
\dot{\vv p_i} = \vv F_i,
$$
and we have
$$
\vv F = \vv F_{\text{ext}} + \sum_{i \neq j} \vv F_{ij},
$$
where $\vv F_{\text{ext}}$ is the external system and $\vv F_{ij}$ is the force on particle $i$ from particle $j$. By Newton's third law, we have $\vv F_{ij} = - \vv F_{ji}$.

So if we define the \emph{center of mass} to be
$$
\vv R = \frac{1}{M} \sum_i m_i \vv x_i,
$$
where $M = \sum_i m_i$, then the total momentum is then $\vv P = M \dot{\vv R} = \sum_{i} \vv p_i$. We then have that
$$
\dot{\vv P} = \sum_i \vv F_i^{\text{ext}}.
$$
Thus the motion of a group of particles center of mass is not influenced by internal forces. If there is no external force, we then have \emph{conservation of momentum}, where $\dot{\vv P} = \vv 0$.

%\clearpage

\clearpage

\section{Rigid Bodies}


In many cases, we want to study a system of particles such that the particles are constrained such that their relative positions are fixed. Such systems correspond to solid objects that cannot deform, break or bend. 
Such systems are called \emph{rigid bodies}.

\subsection*{Angular Velocity}

Suppose a rigid body $\BB$ is rotating about a fixed axis with angular speed $\omega$. If $\hh n$ is the unit vector parallel to the rotation axis, we define the \emph{vector angular velocity} of $\BB$ to be
$$
\vv \omega = \omega \hh n,
$$
where we choose the sign such that a positive $\omega$ corresponds to a right-handed rotation about the axis in the direction $\hh n$.

If $P$ is a particle in the rigid body $\BB$, then its velocity due to the rotation of the body is given by
$$
\vv v = \vv \omega \times (\vv r - \vv b),
$$
where $\vv b$ is the position vector corresponding to any fixed point $B$ on the rotation axis.

\subsection*{Moment of Inertia}

Now suppose we have a rigid body made up of $N$ particles, rotating with angular velocity $\vv \omega$. Then we could write down the kinetic energy for the rigid body as
$$
T = \frac{1}{2} \sum_{i} m_i  \dot{\vv x_i} \cdot \dot{\vv x_i} = \frac{1}{2}I \omega^2,
$$
where
$$
I = \sum_{i = 1}^N m_i d_i^2
$$
is the \vocab{moment of inertia}, and $d_i$ is the perpendicular distance from particle $i$ to the axis of rotation.

The intuition behind moments of inertia is that they are to rotations what mass is to translations: the larger $I$ is, the more energy you need to supply to the body to make it spin.

\subsection*{Angular Momentum and Moments of Inertia}

Recall that the angular momentum of a body is
$$
\vv L = \sum_i m_i \vv x_i \times  \dot{\vv x_i}.
$$
Suppose that the body is rotating with angular velocity $\omega \hh n$ about an axis through the origin. Then the component of angular momentum in the $\hh n$ direction is given by
\begin{align*}
	\vv L \cdot \hh n &= \sum_i m_i \vv x_i \times (\vv \omega \times \vv x) \cdot \hh n \\
	&= \omega \sum_i m_i (\vv x_i \times \hh n) \cdot (\vv x_i \times \hh n) \\
	&= I \omega.
\end{align*}
If there's a torque $\tau$ acting on the rigid body in the same direction as the angular velocity, so $\vv \tau = \tau \hh n$, then $I \dot{\omega} = \tau$.

\subsection*{Calculating Moment of Inertia}

If we consider the rigid body as a continuous object with density distribution $\rho (\vv x)$, then the moment of inertia is given by
$$
I = \int \rho (\vv x) x_{\perp}^2 \; dV
$$

Some examples of common moments of inertia are given below.
\begin{enumerate}
	\item \emph{Thin Rod}. With mass $M$ and length $2a$ about an axis perpendicular to the rod through its center, we have $I = \frac{1}{3}M a^2$.
	\item \emph{Circular Hoop}. With mass $M$ and radius $a$, through an axis perpendicular to the plane of the hoop through its center, we have $I = Ma^2$.
	\item \emph{Circular Disk}. A disk with mass $M$ and radius $A$, through an axis perpendicular to the plane of the disk through its center, we have $I = \frac{1}{2}Ma^2$.
	\item \emph{Solid Sphere}. A sphere with mass $M$ and radius $a$, through an axis through its center, we have $I = \frac{2}{5}Ma^2$. 
\end{enumerate}

We also have the \emph{parallel axis theorem}, which says if $I_G$ is the moment of inertia about an axis through the centre of mass, and $I$ is the moment of inertia about a parallel axis a distance $a$ away, then
$$
I= I_G + Ma^2.
$$


%\clearpage

\clearpage

\section{Special Relativity}


\subsection*{Axioms of Special Relativity}

\begin{enumerate}
	\item \emph{Galilean relativity}. The laws of physics are the same in all inertial reference frames.
	\item \emph{Speed of light}. The speed of light in a vacuum is the same in all inertial reference frames.
\end{enumerate}


\subsection*{Lorentz Transformations}

In special relativity we think in terms of events: instantaneous point-like occurrences. These are specified by four coordinates, one of time and three of position, like $(t, x, y, z)$.
These coordinates will be measured differently in different inertial frames, and to make our axioms hold we need to use a new set of transformation laws.

If we have two inertial frames $\FF$ and $\FF'$, and $\FF'$ is moving at speed $v$ relative to $\FF$ in the $x$ direction, then we have
\begin{align*}
	x' &= \gamma \left(x - \frac{v}{c} ct\right) \\
	y' &= y \\
	z' &= z \\
	ct' &= \gamma\left(ct - \frac{v}{c}x\right),
\end{align*}  
where
$$
\gamma = \sqrt{\frac{1}{1 - v^2/c^2}}.
$$


\subsection*{Relativistic Physics}

\begin{enumerate}
	\item \emph{Relativity of simultaneity}. Events with the same $t$ no longer correspond to events with equal $t'$, so what is simultaneous in one frame is not necessarily simultaneous in another.
	\item \emph{Causality}. While observers can disagree about the temporal ordering of events, if an event is within the `light cone' of an event $P$ (within the region of a space-time diagram traced out by light passing through $P$) then all observers will agree on a causal ordering.  
	\item \emph{Time Dilation}. Consider a clock sitting stationary at the origin of the frame $\FF'$, ticking at intervals of $T'$. The tick events in frame $\FF'$ will occur at $(t'_1, 0), (t'_1 + T', 0), \dots$. 
	
	In the frame $\FF$, using the Lorentz transformations, we see that the time interval between ticks is $T = \gamma T'$. So the ticks are longer in the stationary frame. 
	\item \emph{Length Contraction}. Consider a rod of length $L'$, stationary in the frame $\FF'$. 
	The endpoints of the rod are given by $x' = 0$ and $x' = L'$, which are then mapped into $x = vt$ and $x = vt + L'/\gamma$. 
	So in $\FF$, the length of the rod is $L'/\gamma$, and thus lengths of moving objects are contracted in the direction of motion. To deal with this, we define \emph{proper length} to be the length measured in an objects rest frame.
	\item \emph{Composition of Velocities}. Suppose a particle moves with constant velocity $u'$ in frame $\FF'$, which moves with velocity $v$ relative to $\FF$. We want to find its velocity $u$ in the frame $\FF$.

	In $\FF'$, for the particle we have $x' = u't'$. Substituting this into the Lorentz transformation laws, we have
	$$
	u = \frac{x}{t} = \frac{\gamma(x' + vt')}{\gamma(t' + vx'/c^2)} = \frac{u' + v}{1 + u' v/c}.
	$$
	\item \emph{Newtonian Limit}. When $v/c$ is very small, the Lorentz transformations approximate the Galilean transformations that we use in Newtonian mechanics.
\end{enumerate}


\subsection*{Geometry of Minkowski Space}

Consider two events $P_1$ and $P_2$ have coordinates $(t_1, x_1)$ and $(t_2, x_2)$ in the frame $\FF$. These events are separated by $\Delta t = t_1 - t_2$ in time and $\Delta x = x_1 - x_2$.

We define the \emph{invariant interval} between $P_1$ and $P_2$ to be
$$
\Delta s^2 = c^2 \Delta t^2 - \Delta x^2.
$$
We say it is invariant because it is the same in all inertial reference frames, that is, it is invariant under Lorentz transformations\footnote{Feel free to check this.}.

It is possible for $\Delta s^2$ to be either positive or negative. If it is positive, we say the events are \emph{timelike} separated, and if it is negative we say they are \emph{spacelike} separated, and if it is zero, we say they are \emph{lightlike} separated. Events that are in each other's light-cones are timelike, and can influence one another.

\subsection*{4-Vectors}

We can view Minkowski space as a vector space equipped with the \emph{Minkowski inner product}\footnote{This is not an inner product in the normal sense since it's not positive definite}. The coordinates of some event $P$ in the frame $\FF$ can be written as a \emph{4-vector}
$$	
X= \begin{pmatrix}
	ct \\ x \\ y \\ z
\end{pmatrix},
$$
and then the Minkowski inner product is given by
$$
X \cdot Y = X^T \eta Y, \quad \quad \text{where} \quad \eta = \begin{pmatrix}
	1 & 0 & 0 & 0 \\
	0 & -1 & 0 & 0 \\
	0 & 0 & -1 & 0 \\
	0 & 0 & 0 & -1
\end{pmatrix}.
$$
Taking $X \cdot X$ gives the invariant interval between the origin and $P$, and is known as the \emph{Minkowski metric}\footnote{Again, this isn't actually a metric in the normal sense since it's not positive definite}.

\subsection*{The Lorentz Group}

Using $4$-vectors, we can view Lorentz transformations as linear transformations from the coordinates of one inertial frame $\FF$ to another $\FF'$. This would be represented by a $4 \times 4$ matrix $\Lambda$, with
$$
X' = \Lambda X.
$$
Since such transformations must preserve the invariant interval, they must preserve the Minkowski inner product\footnote{These are the analogue of orthogonal matrices for Minkowski space} and hence satisfy
$$
\Lambda^T \eta \Lambda = \eta.
$$
The group of all such matrices is the \emph{Lorentz group}, and is generated by 
$$
\left(\begin{array}{c|ccc}
	1 & 0 & 0 & 0 \\
	\hline 0 & & & \\
	0 & & \mathrm{R} & \\
	0 & & &
	\end{array}\right) 
	\quad \text{and} \quad \left(\begin{array}{cc|cc}
		\gamma & -\gamma v / c & 0 & 0 \\
		-\gamma v / c & \gamma & 0 & 0 \\
		\hline 0 & 0 & 1 & 0 \\
		0 & 0 & 0 & 1
		\end{array}\right),
$$
where the first corresponds to a rotation of space, leaving time intact (so we need $R$ to be orthogonal), and the second corresponds to a Lorentz boost along the $x$ direction.

\subsection*{Relativistic Kinematics}

\begin{enumerate}
	\item \emph{Proper Time}. To do kinematics, it helps to have a consistant notion of time. We do this by defining \emph{proper time} $\tau$, which is the time experienced by a particle in its own reference frame. It is given by
	$$
	\Delta \tau = \frac{\Delta s}{c},
	$$
	in all reference frames.
	\item \emph{Relating Proper Time to Measured Time}. By considering small changes, if $\vv u$ is the velocity of the particle in the frame $\FF$, we get that
	$$
	d \tau = \frac{1}{c} d s = \frac{1}{c} \sqrt{c^2 d t^2 - |d\vv x|^2} = \sqrt{1 - u^2/c^2} d t,
	$$
	so
	$$ \frac{dt}{d \tau} = \gamma_u.
$$
We can then get the total time experienced by a particle as
$$
T = \int d \tau = \int \frac{1}{\gamma_u} d t.
$$
\item \emph{$4$-Position}. Using proper time, we can parametrise the trajectory of a particle using a $4$-vector
$$
X(\tau) = \begin{pmatrix}
	ct(\tau) \\
	\vv x(\tau)
\end{pmatrix}.
$$
\item \emph{$4$-Velocity}. Then we can define the \emph{$4$-velocity} as
$$
U = \frac{dX}{d \tau} = \begin{pmatrix}
	c dt/d\tau \\
	d \vv x / d \tau
\end{pmatrix}
$$
that is,
$$  U = \frac{dt}{d\tau} \begin{pmatrix}
	c \\ \vv u
\end{pmatrix} = \gamma_u \begin{pmatrix}
	c \\ \vv u
\end{pmatrix},
$$
where $\vv u = d \vv x/ dt$. Since this is a $4$-vector, it transforms like $U' = \Lambda U$. 
\item \emph{$4$-Momentum}. For a particle of mass $m$, the \emph{$4$-momentum} is defined to be
$$
P = mU = \begin{pmatrix}
	mc \gamma_u \\
	m \gamma_u \vv u
\end{pmatrix}.
$$
The spacial components of $P$ give us the \emph{relativistic $3$-momentum}, $\vv p = m \gamma_u \vv u$. $4$-momentum is conserved in the absence of external forces, and for a system of particles the total $4$-momentum is the sum of the $4$-momenta of the particles.
\item \emph{Relativistic Energy}. We define the \emph{relativistic energy} of a particle to be $E = P^0 c$, so that
$$
E = m \gamma c^2 = mc^2 + \frac{1}{2} m |\vv u|^2 + \cdots,
$$
and for a stationary particle we have $E = mc^2$. 

Since we can calculate the Lorentz invariant quantity $P \cdot P$ in the particles rest frame, we have
$$
P \cdot P = \frac{E^2}{c^2} - |\vv p|^2 = m^2 c^2,
$$
so generally we have 
$$
E^2 = |\vv p|^2 c^2 + m^2 c^4.
$$
Relativistic energy is a conserved quantity, which includes mass (just another form of energy, it's not conserved separately).
\item \emph{Massless Particles}. For particles that have zero mass (like photons), they can still have momentum and energy. However, $P \cdot P = 0$, and we can't define proper time for the particles. We still have $E^2 = |\vv p|^2 c^2$, and thus
$$
P = \frac{E}{c} \begin{pmatrix}
	1 \\ \vv n
\end{pmatrix},
$$
where $\vv n$ is a unit vector in the direction of propagation. We also have $E = hc/\lambda = hf$, where $h$ is Planck's constant, $\lambda$ is the wavelength, and $f$ is the frequency of the photon/particle. 
\end{enumerate}

\subsection*{Rapidities}

Consider a $2 \times 2$ matrix corresponding to a Lorentz boost by $v$ in the $x$ direction. Then if $\beta = v/c$, we have
$$
\Lambda[\beta] = \begin{pmatrix}
	\gamma & -\gamma \beta \\
	-\gamma \beta & \gamma
\end{pmatrix}
$$
and the composition law $\Lambda[\beta_1] \Lambda[\beta_2] = \Lambda\left[\frac{\beta_1 + \beta_2}{1 + \beta_1 \beta_2}\right]$,
which is exactly our composition law from before.

A slightly nicer way to deal with this is with \emph{rapidities}.
We define the \emph{rapidity} of a Lorentz boost $\phi$ such that
$$
\beta = \tanh \phi, \quad \gamma = \cosh \phi,\quad \gamma \beta = \sinh \phi.
$$
Then
$$
\Lambda[\beta] = \begin{pmatrix}
	\cosh \phi & - \sinh \phi \\
	-\sinh \phi & \cosh \phi 
\end{pmatrix} = \Lambda(\phi),
$$
and also
$$
\Lambda(\phi_1) \Lambda(\phi_2) = \Lambda(\phi_1 + \phi_2),
$$
which is much nicer. This also shows that Lorentz boosts correspond to hyperbolic rotations in spacetime.




\end{document}