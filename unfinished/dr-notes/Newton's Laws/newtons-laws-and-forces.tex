\documentclass{scrartcl}

\usepackage[noindent]{handout}
\usepackage{multicol}
\usepackage{amsmath}
\usepackage{bm}
\newcommand{\vv}[1]{\boldsymbol{\mathbf{#1}}}

\newtheorem*{lemma}{Lemma}
\theoremstyle{definition}
\newtheorem*{definition}{Definition}

% \newcommand{\vocab}[1]{\textbf{\color{blue} #1}} % Coloured vocab
\newcommand{\vocab}[1]{\emph{#1}}
\newcommand{\hh}[1]{\hat{\vv{#1}}}

\title{Newton's Laws \& Forces}
\author{Adam Kelly (\texttt{ak2316})}
\date{\today} 
 
\begin{document}

\maketitle  

In Newtonian mechanics, we use the idea of a force to build up a mathematical framework in which we can describe the motion of objects, typically particles.

\subsection*{Particles \& Inertial Frames}

A \emph{particle} is an idealised object of negligible size and no internal structure to which we assign physical properties such as charge or mass.
To describe the positions of particles over time, we use a \emph{reference frame}, which consists of a choice of origin and some coordinate system which can be used to consistently describe position.

For a particle with position vector $\vv x(t)$, a displacement vector from the origin of our reference frame, we then define the following quantities.
\begin{itemize}
	\item \emph{Velocity}. The velocity $\vv v$ of the particle is $\vv v = \dot{\vv x} = d \vv x(t) / dt$. The speed is then $v = |\vv v|$.
	\item \emph{Acceleration}. The acceleration $\vv a$ of a particle is $\vv a = \ddot{\vv x} = d^2 \vv x(t)/dt^2$.
	\item \emph{Momentum}. The momentum $\vv p$ is then $\vv p = m \vv v = m \dot{\vv x}$.
\end{itemize}


\subsection*{Newton's Laws}

\begin{enumerate}
	\item \emph{Galileo's Law of Inertia}. There exists inertial frames, reference frames in which particles travel at constant velocity when the force acting on them vanishes.
	\item \emph{Newton's Second Law}. If a force $\vv F$ acts on a particle with momentum $\vv p$, then $d\vv p/dt = \vv F$.
	\item \emph{Newton's Third Law}. Every action has an equal and opposite reaction.
\end{enumerate}

All of these are tied together with the \emph{principle of Galilean relativity}, which is that the laws of physics are the same in every inertial reference frame.

\subsection*{Galilean Transformations}

A natural question is how one can transform between different inertial frames. That is, under what transformations do the laws of physics remain unchanged. It is straightforward to see that if we have some $(\vv x, t)$ in an inertial frame $\FF$, then we can construct another inertial frame $\FF'$ by combining:
\begin{itemize}
	\item \emph{Translations}. $\vv x' = \vv x + \vv a$ for a constant vector $\vv a$.
	\item \emph{Rotations}. $\vv x' = R \vv x$, where $R$ is some orthogonal matrix.
	\item \emph{Boosts}. $\vv x' = \vv x + \vv t$, for a constant velocity vector $\vv v$.
	\item \emph{Rescaling}. $\vv x' = \lambda \vv x$ for a constant $\lambda$. 
\end{itemize}
This set of transformations forms the \emph{Galilean group}, the group of symmetries of Newton's laws.

When working with Newton's laws, we assume that there is some `absolute time', $t$, $t'$ in the inertial frames which differ by at most a constant in the two reference frames\footnote{Once we fix the units, of course.}

\subsection*{Examples of Forces}

The \emph{Lorentz force} describes the force on a particle moving in an electromagnetic field. If a particle has position $\vv x$ and charge $q$, and it moves in an electric field $\vv E$ and magnetic field $\vv B$, then the force on the particle is given by
$$
\vv F = q(\vv E + \dot{\vv x} \times \vv B).
$$


The \emph{gravitational force} describes the force due to gravity on a particle. If two particles have masses $m_1$ and $m_2$ and positions $\vv x_1$ and $\vv x_2$, then the force of gravity between them is given by
$$
\vv F_1 = - \frac{G m_1 m_2}{| \vv x_1 - \vv x_2|^3} (\vv x_1 - \vv x_2) = - \vv F_2,
$$
where $G$ is the \vocab{gravitational constant}.


\subsection*{Potentials in One Dimension}

Roughly, the framework of Newtonian mechanics takes in particles and forces, and from that we get the dynamics at play. However, in many cases we think of forces as coming from a more primitive idea, one of \emph{potentials}. We will work in one dimension for the time being.

For a force field $F(x)$, we define the \emph{potential energy} to be a function $V(x)$ such that
$$
F = - \frac{dV}{dx},
$$
so $V = -\int F \; dx$, defined up to a constant. With this, Newton's second law becomes $\dot{p} = -dV/dx$.

With forces in one dimension depending on position only, we have a naturally conserved quantity. We define the \emph{kinetic energy} of a particle to be $T = \frac{1}{2}m \dot{x}^2$. Then if the kinetic and potential energies of a particle in a force field are $T, V$ and $E = T + V$, then $E$, the \emph{total energy}, is conserved\footnote{This can be checked by differentiating.}. One nice thing about having such a conservation quantity is that
$$
E = \frac{1}{2}m \dot{x}^2 + V(x) \implies  \dot{x} = \pm \sqrt{\frac{2}{m} (E - V(x))},
$$
which gives us a first order differential equation for position. We can also get a lot of information about the motion of a particle by sketching the graph of the potential versus position.

\subsection*{Potentials in Three Dimensions}

The concept of potentials generalises into three dimensions, but there is some groundwork needed to be done first. We first will need to redefine kinetic energy to make sense in multiple dimensions. This is done naturally with 
$$
T = \frac{1}{2} m \dot{\vv x} \cdot \dot{\vv x}.
$$

Now consider a force field $\vv F(\vv x)$, depending only on position.  If the force $\vv F$ acts on a particle, moving it from $\vv x(t_1)$ to $\vv x(t_2)$ along a trajectory $\mathcal{C}$, we define the \emph{work done} $W$ to be
$$
W = \int_\mathcal{C} \vv F \cdot d \vv x = \int_{t_1}^{t_2} \vv F \cdot \frac{d \vv x}{d t} d t.
$$
Note that if $F = m \ddot{\vv x}$ (as in there is no change in mass), then the work done is
$$
W = m \int_{t_1}^{t_2} \ddot{\vv x} \cdot \dot{\vv x} \; dt = T(t_2) - T(t_1).
$$

From this, we can see that there is a conserved energy if and only if $\vv F = - \nabla V$, for some potential $V(\vv x)$. If this is the case, then the conserved quantity is $E = T + V$, as before.

\subsection*{Gravity}

Gravity is an example of a conserved force. For a particle of mass $m$, under teh influence of gravity from a particle of mass $M$ at the origin, we have the potential energy
$$
V(r) = -\frac{GMm}{r}.
$$
The force of gravity is then given by
$$
\vv F = - \nabla V = -\frac{G Mm}{r^2} \hh r,
$$
with $\hh r$ in the direction of the particle. Note that this force points towards the origin.
It is sometimes convenient to remove the dependence on the mass of the particle $m$, which we do by defining the \emph{gravitational potential} $\phi(r) = -GM/r$, with $V = \phi m$.

If there is many particles say of masses $M_i$ and positions $\vv r_i$, the gravitational potential just adds, with 
\begin{align*}
	\phi(\vv r) &= -G \sum_i \frac{M_i}{|\vv r - \vv r_i|} \\
\implies \vv F &= -Gm \sum_i \frac{M_i}{|\vv r - \vv r_i|^3} (\vv r - \vv r_i).
\end{align*}

\subsection*{Electromagnetism}

If we have time-independent electric and magnetic fields $\vv E$ and $\vv B$, then $\vv E = - \nabla \phi$ for some $\phi(\vv x)$, the \emph{electric potential}. We have conservation of energy for time-independent fields, with the total energy\footnote{Again, this can be checked by differentiating.}
$$
E = \frac{1}{2}m \dot{\vv x} \cdot \dot{\vv x} + q \phi(\vv x). 
$$

Just as objects with mass both feel the effect of and product gravitational force, the same is true of objects with charge. At the origin, a particle of charge $Q$ creates an electric field as
$$
E = - \nabla \left(\frac{Q}{4 \pi \varepsilon_0 r}\right) = \frac{Q}{4 \pi \varepsilon_0} \frac{\hh r}{r^2},
$$
where $r^2 = \vv x \cdot \vv x$. The force between objects of charge $Q$ and $q$ is $\vv F = q \vv E$, or
$$
\vv F = \frac{Qq}{4 \pi \varepsilon_0} \frac{\hh r}{r^2}.
$$
This is the \emph{Coulomb force}, and has the same force form as gravity.

\subsection*{Friction}

In many physical scenarios we deal with forces that do not come from potentials. Friction is one such force, and it is a force that is used to approximate the complex relationship between objects in contact.

One common form of friction is \emph{dry friction}, defined by $\vv F = \mu \vv R$ where $\vv R$ is the reaction force normal to the floor, in the direction that opposes motion. Note that friction never causes motion. Another common type of friction is \emph{fluid drag}. Fluid drag is typically either \emph{linear}, with $\vv F = - \gamma \vv v$ (suitable for slow moving objects in viscous fluids), or \emph{quadratic}, with $\vv F = - \gamma |\vv v| \vv v$ (suitable for fast moving objects in less viscous fluids).

\end{document}