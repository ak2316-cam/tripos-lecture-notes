\documentclass[a4paper, 10pt, twocolumn]{amsart}

\usepackage[nokoma,noindent,fancytheorems]{adam}
\usepackage[margin=0.75in]{geometry}

\setlist[enumerate]{leftmargin=8mm}
\setlist[itemize]{leftmargin=8mm}

\newcommand{\enumpre}{\vspace{-1.5\baselineskip}}



% NOTE: for a more compact, black and white style for printing, use
% the below.

% \documentclass[a3paper, 10pt]{article}

% \usepackage[nokoma, noindent]{adam}
% \usepackage[landscape,margin=0.5in]{geometry}
% \usepackage{multicol}

% \setlist[enumerate]{leftmargin=8mm}
% \setlist[itemize]{leftmargin=8mm}

% \newcommand{\enumpre}{}
% % \newcommand{\enumpre}{\vspace{-1.5\baselineskip}}
% \renewcommand{\vocab}[1]{\emph{#1}}

\usepackage{quiver}

\newcommand{\id}{\operatorname{id}}
\newcommand{\rel}{\operatorname{rel}}

\title{Algebraic Topology}
\author{Adam Kelly -- Part II}
\date{\today. Email \texttt{ak2316@srcf.net}}

\begin{document}
% \begin{multicols*}{5}
\maketitle

\section{Covering Spaces}

\subsection{Definitions and Lifting}

We now start to develop some machinery which will allow us to compute fundamental groups.

\begin{definition}[Evenly Covered Set]
Suppose $p: \hat{X} \rightarrow X$ is continuous, we say that $U \subseteq X$ is evenly covered if $p^{-1}(U)=\sqcup_\alpha V_\alpha$, where $\left.p\right|_{v_\alpha}: V_\alpha \rightarrow U$ is a homeomorphism.
\end{definition}

\begin{definition}[Covering Map]
  A map $p: \hat{X} \rightarrow X$ is a covering map if for all $x \in X$, there exists an open neighbourhood $U_x$ which is evenly covered. In this case, we call $\hat{X}$ a covering space for $X$.
\end{definition}

\begin{definition}[Lift]
  Suppose $p: \hat{X} \rightarrow X$ is a covering map, $f: Z \rightarrow X$ continuous. Then we say that $\hat{f}: Z \rightarrow \hat{X}$ is a lift of $f$ if $p \circ \hat{f}=f$, that is,
  {\color{red} DIAGRAM}
  commutes.
\end{definition}

\begin{lemma}[Lebesgue Covering]
Suppose $X$ is a compact metric space, $\left\{U_\alpha\right\}_\alpha$ is an open cover of $X$. Then there exists $\delta>0$ such that for all $x \in X, B_\delta(x) \subseteq U_\alpha$ for some $\alpha$.
\end{lemma}
\begin{proof}
  Given $x \in X$, let $\alpha(x)$ and $\delta(x)>0$ be such that $B_{2 \delta(x)}(x) \subseteq U_{\alpha(x)}$. Then $\left\{B_{\delta(x)}\right\}_{x \in X}$ is an open cover of $X$. Therefore, by compactness there exists a finite subcover $\left\{B_{\delta\left(x_i\right)}\left(x_i\right)\right\}_{i=1}^n$. Let $\delta=\min \{\delta\left(x_1\right),$ $\ldots, \delta\left(x_n\right)\}$. Then for all $y \in X, y \in B_{\delta\left(x_i\right)}\left(x_i\right)$ for all $i$. Then
  $$
  B_\delta(y) \subseteq B_{2 \delta\left(x_i\right)}\left(x_i\right) \subseteq U_{\alpha\left(x_i\right)}
  $$
\end{proof}

\begin{notation}
 We say a path $\gamma$ with $\gamma(0)=x_0$ has the (unique) lifting property if for all $\hat{x}_0 \in p^{-1}\left(x_0\right)$, there exists a (unique) lift $\hat{\gamma}$ of $\gamma$ with $\hat{\gamma}(0)=\hat{x}_0$.
\end{notation}

\begin{lemma}
If $f: Z \rightarrow U, Z$ connected, $\operatorname{im}(f) \subseteq U$, where $U$ is evenly covered, then $\gamma$ has the unique lifting property.
\end{lemma}
\begin{proof}
Since $U$ is evenly covered, $p^{-1}(U)=\sqcup_\alpha V_\alpha$. Then $\hat{x} \in V_{\alpha_0}$ for some $\alpha_0$. Then $p^{\prime}=\left(p \mid v_{\alpha_0}\right)^{-1}: U \rightarrow \hat{X}$ is continuous, with $p^{\prime}\left(x_0\right)=\hat{x}_0$, so $\hat{f}=p^{\prime} \circ f$ is a lift of $f$.

For uniqueness, notice that $p^{-1}(U)=U_{\alpha_0} \sqcup\left(\sqcup_{\alpha \neq \alpha_0} V_\alpha\right)$, which disconnects $p^{-1}(U)$, and as $\mathrm{im}(f)$ is connected, $\operatorname{im}(\hat{f}) \subseteq V_{\alpha_0}$. But $p^{\prime}$ above is a homeomorphism, so $\hat{\gamma}$ is unique.
\end{proof}


\begin{lemma}
  Suppose $\gamma:[a, b] \rightarrow X$ with $a^{\prime} \in[a, b]$, if $\left.\gamma\right|_{\left[a, a^{\prime}\right]}$ has the ULP at $a$ and $\left.\gamma\right|_{\left[a^{\prime}, b\right]}$ has the ULP at $a^{\prime}$, then $\gamma$ has the ULP at $a$.
\end{lemma}
\begin{proof}
  We have a lift $\hat{\gamma}_1:\left[a, a^{\prime}\right] \rightarrow \hat{X}$ of $\left.\gamma\right|_{\left[a, a^{\prime}\right]}$ at $a$, and a lift $\hat{\gamma}_2:\left[a^{\prime}, b\right] \rightarrow \hat{X}$ of $\left.\gamma\right|_{\left[a^{\prime}, b\right]}$ at $a^{\prime}$, such that $\hat{\gamma}_1\left(a^{\prime}\right)=\hat{\gamma}_2\left(a^{\prime}\right)$. So $\hat{\gamma}_1 \hat{\gamma}_2$ is a lift of $\gamma$ at $a$.

For uniqueness, suppose $\hat{\eta}$ is any other lift. Then $\left.\hat{\eta}\right|_{\left[a, a^{\prime}\right]}$ is a lift of $\left.\gamma\right|_{\left[a^{\prime}, a\right]}$, so $\left.\hat{\hat{n}}\right|_{\left[a, a^{\prime}\right]}=\hat{\gamma}_1$. This means that
\end{proof}


\begin{theorem}[Path Lifting]
  Any $\gamma: I \rightarrow X$ has the ULP.
\end{theorem}
\begin{proof}
  $p: \hat{X} \rightarrow X$ is a covering map, so every $x \in X$ has an evenly covered neighbourhood $U_x$. Then $\left\{U_x \mid x \in X\right\}$, so $\left\{\gamma^{-1}\left(U_x\right) \mid x \in X\right\}$ is an open cover of $I$. Thus, by the Lebesgue covering lemma, there exists $\delta>0$ such that $B_\delta(t) \subseteq \gamma^{-1}\left(U_{x(t)}\right)$ for any $t$

Choose $n$ such that $1 / n<\delta, a_i=i / n \in I$. Then $\left[a_i, a_{i+1}\right] \subseteq B_\delta\left(a_i\right)$, so $\gamma\left(\left[a_i, a_{i+1}\right]\right) \subseteq U_{x_i}$, where $a_i=\gamma\left(a_i\right)$. As $U_{x_i}$ is evenly covered, $\left.\gamma\right|_{\left[a_i, a_{i+1}\right]}$ has the ULP at $a_i$. By induction and the previous lemma, $\gamma$ has the ULP.
\end{proof}

\begin{theorem}[Homotopy Lifting]
   Suppose $p: \hat{X} \rightarrow X$ is a covering map, $H: I \times I \rightarrow X$ is a homotopy, then $H$ has the lifting property at $(0,0)$.
\end{theorem}
\begin{proof}
  Suppose $\left\{U_x \mid x \in X\right\}$ is an open cover of $X$ by evenly covered neighbourhoods. Since $l^2$ is compact, by the Lebesgue covering lemma, there exists $\delta>0$ such that $B_\delta(v) \subseteq H^{-1}\left(U_{H(v)}\right)$ for each $v \in R^2$.

Choose $n$ such that $\sqrt{2} / n<\delta$. Then divide $R^2$ into squares with side lengths $1 / n$. Enumerate them $A_1$, $A_2$, $\ldots$, $A_{n^2}$, starting from the bottom left and going right then up. Label the bottom left corner of $A_i$ as $v_i$. Now note that $H\left(A_i\right) \subseteq H\left(B_\delta\left(v_i\right)\right) \subseteq U_{x_i}$ is evenly covered. Thus, $H_{A_i}$ has the ULP at $v_i$, as $l^2$ is connected. Let $B_k=\bigcup_{i=1}^k A_i$

We will prove by induction that $\left.H\right|_{B_k}$ has $L P$ at $(0,0)$. For $k=1, B_1=A_1$, so we are done. Now suppose $\left.H\right|_{B_k}$ has a lift $\hat{H}: B_k \rightarrow X$ with $\hat{H}_k(0,0)=\hat{X}$. Now as $\left.H\right|_{A_k}$ has the lifting property at $v_{k+1}$. So choose a lift $\hat{h}_k: A_{k+1} \rightarrow \hat{X}$ with $\hat{h}_k\left(v_{k+1}\right)=\hat{H}^k\left(v_{k+1}\right)$.

Now note that $B_k \cap A_{k+1}$ is either one or two edges of $A_{k+1}$, both coming from $v_{k+1}$. By uniqueness of path lifting, $\left.\hat{H}_k\right|_{A_{k+1} \cap B_k}=\hat{h}_k \mid A_{k+1} \cap B_k$, so by the gluing lemma we have a well defined lift $\hat{H}_{k+1}$ of $H$ on $B_{k+1}$.
\end{proof}

\begin{proposition}Suppose $\gamma_0, \gamma_1 \in \Omega\left(X, x_0, x_1\right), \gamma_0 \sim_e \gamma_1$. Suppose $\hat{\gamma}_i$ is a lift of $\hat{X}$ with $\hat{\gamma}_i(0)=\hat{x}_0$. Then $\hat{\gamma}_0 \sim_e \hat{\gamma}_1$. In particular, $\hat{\gamma}_0(1)=\hat{\gamma}_1(1)$.
\end{proposition}
\begin{proof}
  Suppose $H: R^2 \rightarrow X$ is a homotopy between $\gamma_0$ and $\gamma_1$.

  {\color{red} DIAGRAM}

  By homotopy lifting, we have a lift $\hat{H}: I^2 \rightarrow \hat{X}$ with $\hat{H}(0,0)=\hat{x}_0$. Let $\alpha_i(t)=\hat{H}(t, i)$ and $\beta_i(t)=\hat{H}(i, t)$. By uniqueness of path lifting.

  {\color{red} DIAGRAM}

  That is, $\hat{\gamma}_0 \sim e \hat{\gamma}_1$ via $\hat{H}$.
\end{proof}

\begin{corollary}
  $p_*: \pi_1(\hat{X}, \hat{x}_0) \rightarrow \pi_1(X, x_0)$ is injective.
\end{corollary}
\begin{proof}
  $$
\begin{aligned}
p_*\left[\gamma_0\right]=p_*\left[\gamma_1\right] & \implies p \circ \gamma_0 \sim_e p \circ \gamma_1 \\
& \implies p \circ \gamma_0 \sim_e \widehat{\rho \circ \gamma_1} \\
& \implies \gamma_0 \sim_e \gamma_1
\end{aligned}
$$
\end{proof}


\end{document}
