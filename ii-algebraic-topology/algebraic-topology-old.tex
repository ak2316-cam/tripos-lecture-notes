\documentclass[a4paper, 10pt, twocolumn]{amsart}

\usepackage[nokoma,noindent,fancytheorems]{adam}
\usepackage[margin=0.75in]{geometry}

\setlist[enumerate]{leftmargin=8mm}
\setlist[itemize]{leftmargin=8mm}

\newcommand{\enumpre}{\vspace{-1.5\baselineskip}}



% NOTE: for a more compact, black and white style for printing, use
% the below.

% \documentclass[a3paper, 10pt]{article}

% \usepackage[nokoma, noindent]{adam}
% \usepackage[landscape,margin=0.5in]{geometry}
% \usepackage{multicol}

% \setlist[enumerate]{leftmargin=8mm}
% \setlist[itemize]{leftmargin=8mm}

% \newcommand{\enumpre}{}
% % \newcommand{\enumpre}{\vspace{-1.5\baselineskip}}
% \renewcommand{\vocab}[1]{\emph{#1}}

\usepackage{quiver}

\newcommand{\id}{\operatorname{id}}
\newcommand{\rel}{\operatorname{rel}}

\title{Algebraic Topology}
\author{Adam Kelly -- Part II}
\date{\today. Email \texttt{ak2316@srcf.net}}

\begin{document}
% \begin{multicols*}{5}
\maketitle

\section{The Fundamental Group}

\subsection{Homotopy}\ 
% We first define a notion of one function on a space being deformed into another 

\begin{definition}[Homotopy]
Let $f, g: X \rightarrow Y$ be maps. A \vocab{homotopy} from $f$ to $g$ is a map
  $
  H: X \times I \rightarrow Y
  $
  such that
  $
  H(x, 0)=f(x)$, and $H(x, 1)=g(x)
  $.

  If such a $H$ exists, we say $f$ is \vocab{homotopic} to $g$, and write $f \simeq h$ or $f \simeq_H g$.
\end{definition}

\begin{definition}[Homotopy Relative to a Subspace]
  We say $f$ is homotopic to $g$ \vocab{relative to $A$}, written $f \simeq g \rel A$ if for all $a \in A \subseteq X$, we have $H(a, t) = f(a) = g(a)$.
\end{definition}


\begin{proposition}
  For spaces $X, Y$ and $A \subseteq X$, `homotopic $\rel A$' is an equivalence relation. In particular, when $A = \emptyset$, homotopy is an equivalence relation.
\end{proposition}
\begin{proof}
  We check
  \begin{enumerate}
    \item \emph{Reflexivity}: $f \simeq f$ since $H(x, t)=f(x)$ is a homotopy.
    \item \emph{Symmetry}: if $H(x, t)$ is a homotopy from $f$ to $g$, then $H(x, 1-t)$ is a homotopy from $g$ to $f$.
    \item \emph{Transitivity}: Suppose $f, g, h: X \rightarrow Y$ and $f \simeq_H g \operatorname{rel} A, g \simeq_{H^{\prime}} h \operatorname{rel} A$. We want to show that $f \simeq h \operatorname{rel} A$. The idea is to "glue" the two maps together.
    
    We know how to continuously deform $f$ to $g$, and from $g$ to $h$. So we just do these one after another. We define $H^{\prime \prime}: X \times I \rightarrow Y$ by
    $$
    H^{\prime \prime}(x, t)= \begin{cases}H(x, 2 t) & 0 \leq t \leq \frac{1}{2} \\ H^{\prime}(x, 2 t-1) & \frac{1}{2} \leq t \leq 1\end{cases}
    $$
    It's well-defined since $H(x, 1)=g(x)=H^{\prime}(x, 0)$. This is also continuous by the gluing lemma. It is easy to check that $H^{\prime \prime}$ is a homotopy $\operatorname{rel} A$.
  \end{enumerate}
  \vspace{-1pc}
\end{proof}

\begin{definition}[Homotopy Equivalence]
  A map $f: X \rightarrow Y$ is a \vocab{homotopy equivalence} if there exists a $g:Y \rightarrow X$ such that $f \circ g \simeq \id_Y$ and $g \circ f \simeq \id_X$. We call $g$ a \vocab{homotopy inverse} for $f$.

  If a homotopy equivalence $f: X \rightarrow Y$ exists, we say $X$ and $Y$ are \vocab{homotopy equivalent}, and write $X \simeq Y$.
\end{definition}


\begin{lemma}
  Consider the spaces and maps
  % https://q.uiver.app/#q=WzAsMyxbMCwwLCJYIl0sWzEsMCwiWSJdLFsyLDAsIloiXSxbMCwxLCJmXzAiLDAseyJjdXJ2ZSI6LTJ9XSxbMSwyLCJnXzAiLDAseyJjdXJ2ZSI6LTJ9XSxbMCwxLCJmXzEiLDIseyJjdXJ2ZSI6Mn1dLFsxLDIsImdfMSIsMix7ImN1cnZlIjoyfV1d
\[\begin{tikzcd}
	X & Y & Z
	\arrow["{f_0}", curve={height=-12pt}, from=1-1, to=1-2]
	\arrow["{g_0}", curve={height=-12pt}, from=1-2, to=1-3]
	\arrow["{f_1}"', curve={height=12pt}, from=1-1, to=1-2]
	\arrow["{g_1}"', curve={height=12pt}, from=1-2, to=1-3]
\end{tikzcd}\]
If $f_0 \simeq_H f_1$ and $g_0 \simeq_{H'} g_1$, then $g_0 \circ f_0 \simeq g_1 \circ f_1$.
\end{lemma}
\begin{proof}
  We will show that $g_0 \circ f_0 \simeq g_0 \circ f_1 \simeq g_1 \circ f_1$, and will then be done since homotopy is an equivalence relation. We write down two homotopies.
  \begin{enumerate}
    \item Consider the composition
    % https://q.uiver.app/#q=WzAsMyxbMCwwLCJYXFx0aW1lcyBJIl0sWzEsMCwiWSJdLFsyLDAsIloiXSxbMCwxLCJIIl0sWzEsMiwiZ18wIl1d
\[\begin{tikzcd}
	{X\times I} & Y & Z
	\arrow["H", from=1-1, to=1-2]
	\arrow["{g_0}", from=1-2, to=1-3]
\end{tikzcd}\]
It is easy to check that this is teh first homotopy we need to show $g_0 \circ f_0 \simeq g_0 f_1$.
\item The following composition is a homotopy from $g_0 \circ f_1$ to $g_1 \circ f_1$:
% https://q.uiver.app/#q=WzAsMyxbMCwwLCJYXFx0aW1lcyBJIl0sWzEsMCwiWVxcdGltZXMgSSJdLFsyLDAsIloiXSxbMCwxLCJmXzEgXFx0aW1lcyBcXGlkX0kiXSxbMSwyLCJIJyJdXQ==
\[\begin{tikzcd}
	{X\times I} & {Y\times I} & Z
	\arrow["{f_1 \times \id_I}", from=1-1, to=1-2]
	\arrow["{H'}", from=1-2, to=1-3]
\end{tikzcd}\]
  \end{enumerate}
  \vspace{-1pc}
\end{proof}

\begin{proposition}
  Homotopy equivalence of spaces is an equivalence relation.
\end{proposition}
\begin{proof}
  Symmetry and reflexivity are trivial. To show transitivity, let $f:X \rightarrow Y$ and $h: Y \rightarrow Z$ be homotopy equivalences, and $g: Y \rightarrow X$ and $k: Z \rightarrow Y$ be their homotopy inverses. We will show that $h \circ f : X \rightarrow Z$ is a homotopy equivalence with homotopy inverse $g \circ k$. We have
  \begin{align*}
  (h \circ f) \circ (g \circ k) = h \circ (f \circ g) \circ k \\
  \simeq h \circ \id_y \circ k = h \circ k \simeq \id_Z.
  \end{align*}
  Similarily,
  \begin{align*}
(g \circ k) \circ(h \circ f)=g \circ(k \circ h) \circ f \\
\simeq g \circ \operatorname{id}_Y \circ f=g \circ f \simeq \operatorname{id}_X,
  \end{align*}
  as required.
\end{proof}

\begin{definition}[Contractible]
  If $X \simeq *$, the one point space, then $X$ is \vocab{contractible}.
\end{definition}

\begin{definition}[Retraction]
  Let $A \subseteq X$ be a subspace. A \vocab{retraction} $r:X \rightarrow A$ is a map such that $r \circ \iota = \id_A$, where $\iota: A \hookrightarrow X$ is the inclusion. If such an $r$ exists, we call $A$ a \vocab{retract} of $X$.

  The retraction $r$ is a \vocab{deformation retraction} if $\iota \circ r \simeq \id_X$. A deformation retraction is \vocab{strong} if we require this homotopy to be a homotopy $\rel A$.
\end{definition}

\subsection{Paths}\ 

\begin{definition}[Path]
  A \vocab{path} in a space $X$ is a map $\gamma: I \rightarrow X$. 
  If $\gamma(0) = x_0$ and $\gamma(1) = x_1$, we say $\gamma$ is a path from $x_0$ to $x_1$, and write $\gamma: x_0 \leadsto x_1$. 
  
  If $\gamma(0) = \gamma(1)$, we say $\gamma$ is a \vocab{loop} (based at $x_0$).
\end{definition}

\begin{definition}[Concatenation and Inverses]
  If we have two paths $\gamma_1$ from $x_0$ to $x_1$ and $\gamma_2$ from $x_1$ to $x_2$, we define the \vocab{concatenation} to be
  $$
  (\gamma_1 \cdot \gamma_2)(t) = \begin{cases}
    \gamma_1(2t) & 0 \leq t \leq \frac{1}{2}\\
    \gamma_2(2t - 1) & \frac{1}{2} \leq t \leq 1.
  \end{cases}
  $$
  This is continuous by the gluing lemma.

  The \vocab{inverse} of a path $\gamma: I \rightarrow X$ is defined by
  $$
  \gamma^{-1}(t) = \gamma(1 -t).
  $$

  The \vocab{constant path} at a point $x \in X$ is given by $c_x(t) = x$.
\end{definition}

\begin{definition}[Homotopy of Paths]
  Paths $\gamma, \gamma': I \rightarrow X$ are \vocab{homotopic as paths} if they are homotopic $\rel\; \{0, 1 \} \subset I$, i.e. the end points are fixed. We write $\gamma \simeq \gamma'$.
\end{definition}

\begin{lemma}[Group Operation is Well-Defined]
  Let $\gamma_1, \gamma_2: I \rightarrow X$ be paths with $\gamma_1(1) = \gamma_2(0)$. Then if $\gamma_1 \simeq \gamma_1'$ and $\gamma_2 \simeq \gamma_2'$, then $\gamma_1 \cdot \gamma_2 \simeq \gamma_1' \cdot \gamma_2'$.
\end{lemma}
\begin{proof}
  Suppose that $\gamma_1 \simeq_{H_1} \gamma_1'$ and $\gamma_2 \simeq_{H_2} \gamma_2'$. Then we can construct the homotopy
  $$
  H(s, t)= \begin{cases}H_1(s, 2 t) & 0 \leq t \leq \frac{1}{2} \\ H_2(s, 2 t-1) & \frac{1}{2} \leq t \leq 1\end{cases}.
  $$
\end{proof}

\begin{lemma}[Indeed a Group Operation]
Let $\gamma_0: x_0 \leadsto x_1$, $\gamma_1: x_1 \leadsto x_2$, $\gamma_2: x_2 \leadsto x_3$ be paths.
Then 
\begin{enumerate}
  \item $(\gamma_0 \cdot \gamma_1) \cdot \gamma_2 \simeq \gamma_0 \cdot (\gamma_1 \cdot \gamma_2)$.
  \item $\gamma_0 \cdot c_{x_1} \simeq \gamma_0 \simeq c_{x_0} \cdot \gamma_0$.
  \item $\gamma_0 \cdot \gamma_0^{-1} \simeq c_{x_0}$ and $\gamma_0^{-1} \cdot \gamma_0 \simeq c_{x_1}$.
\end{enumerate}
\end{lemma}
\begin{proof}
  We first note that the composition of any path $\gamma$ and an order-preserving surjection $\phi: I \rightarrow I$ is homotopic to the original path via $H(s, t) = \gamma(t\phi(s) + (1 - t)s)$. Such a $\phi$ is a \vocab{reparameterisation}. 
  \begin{enumerate}
    \item Take $\phi$ to be the function
    $$
    \phi(t)=\begin{cases}
      \frac{t}{2} & 0 \leq t \leq \frac{1}{2} \\
      t - \frac{1}{4} & \frac{1}{2} \leq t \leq \frac{3}{4} \\
      2t - 1 & \frac{3}{4} t \leq 1.
    \end{cases}
    $$
    Then we have $(\gamma_0 \cdot \gamma_1) \cdot \gamma_2 \simeq ((\gamma_0 \cdot \gamma_1) \cdot \gamma_2) \circ \phi = \gamma_0 \cdot (\gamma_1 \cdot \gamma_2)$.
    \item To see $\gamma_0 \cdot c_{x_1} \simeq \gamma_0$, reparameterise $\gamma_0$ with
    $$
t \mapsto \begin{cases}
  2t & 0 \leq t \leq \frac{1}{2} \\
  1 & \frac{1}{2} \leq t \leq 1,
\end{cases}
    $$
    and the other side is analogous.
    \item We have that $c_{x_0} \simeq \gamma_0 \cdot \gamma_0^{-1}$ via
    $$
H(s, t) = \begin{cases}
  \gamma_0(2s) & 0 \leq s \leq \frac{t}{2} \\
  \gamma_0(t) & \frac{t}{2} \leq s \leq 1  - \frac{t}{2} \\
  \gamma_0^{-1}(2s - 1) & 1 - \frac{t}{2} \leq s \leq 1.
\end{cases}
    $$
  \end{enumerate}
  \vspace{-1pc} 
\end{proof}

\subsection{The Fundamental Group}

We can now define one of our central objects of study.

\begin{definition}[Fundamental Group]
  Let $X$ be a space and $x_0 \in X$. The \vocab{fundamental group} of $X$ (based at $x_0$), denoted $\pi_1(X, x_0)$, is the set of homotopy classes of loops in $X$ based at $x_0$. 
  
  The group operation is defined to be $[\gamma_0][\gamma_1] = [\gamma_0 \cdot \gamma_1]$, and $[\gamma]^{-1} = [\gamma^{-1}]$ with the identity as teh constant path $e = [c_{x_0}]$.
\end{definition}

\begin{theorem}
  The fundamental group is a group.
\end{theorem}
\begin{proof}
  Immediate from our previous lemmas.
\end{proof}


\begin{definition}[Based Space]
A \vocab{based space} is a pair $(X, x_0)$ of a space $X$ and a point $x_0 \in X$, the \vocab{basepoint}. A \vocab{map of based spaces} $f: (X, x_0) \rightarrow (Y, y_0)$ is a continuous map $f: X \rightarrow Y$ such that $f(x_0) = y_0$. A \vocab{based homotopy} is a homotopy $\rel$ $\{x_0\}$.
\end{definition}

\begin{proposition}[$\pi_1$ is a Functor]
  To a based map $f: (X, x_0) \rightarrow (Y, y_0)$, there is an associated function
  $f_* = \pi_1(f): \pi_1(X, x_0) \rightarrow \pi_1(Y, y_0)$, defined by $[\gamma] \mapsto [f \circ \gamma]$. Moreover it satisfies
  \begin{enumerate}
    \item $\pi_1(f)$ is a homomorphism of groups.
    \item If $f \simeq f'$, then $\pi_1(f) = \pi_1(f')$. 
    \item For any maps $(A, a) \stackrel{h}{\longrightarrow}(B, b) \stackrel{k}{\longrightarrow}(C, c)$, we have $\pi_1(k \circ h)=$ $\pi_1(k) \circ \pi_1(h)$.
    \item $\pi_1(\id_X) = \id_{\pi_1(X, x_0)}$.
  \end{enumerate}
\end{proposition}


We want to remove the basepoint dependence of the fundamental group. We will assume that $X$ is path connected.

\begin{proposition}
  A path $u: x_0 \leadsto x_1$ induces a group isomorphism $u_\#: \pi_1(X, x_0) \rightarrow \pi_1(X, x_1)$ by $[\gamma] \mapsto [u^{-1} \cdot \gamma \cdot u]$. This satisfies
  \begin{enumerate}
    \item If $u \simeq y'$, then $u_\# = u'_\#$.
    \item $(c_{x_0})_\# = \id_{\pi_1(X, x_0)}$. 
    \item If $v: x_1 \leadsto x_2$, then $(u \cdot v)_\# = v_\# \circ u_\#$.
    \item If $f: X \rightarrow Y$ with $f(x_0) = y_0$ and $f(x_1) = y_1$, then
    $$
    (f\circ u)_\# \circ f_* = f_* \circ u_\#: \pi_1(X, x_0) \rightarrow \pi_1(Y, y_1).
    $$
    \item If $x_1 = x_0$, then $u_\#$ is an automorphism of $\pi_1(X, x_0)$ given by conjugation by $u$.
  \end{enumerate}
\end{proposition}

So if $x_0$ and $x_1$ are in the same path connected component, then $\pi_1(X, x_0) \cong \pi_1(X, x_1)$.

\begin{lemma}
  The following diagram commutes.
  % https://q.uiver.app/#q=WzAsMyxbMCwxLCJcXHBpXzEoWCwgeF8wKSJdLFsxLDAsIlxccGlfMShZLCBmKHhfMCkpIl0sWzEsMiwiXFxwaV8xKFksIGcoeF8wKSkiXSxbMCwxLCJmXyoiXSxbMSwyLCJ1X1xcIyJdLFswLDIsImdfKiIsMl1d
\[\begin{tikzcd}
	& {\pi_1(Y, f(x_0))} \\
	{\pi_1(X, x_0)} \\
	& {\pi_1(Y, g(x_0))}
	\arrow["{f_*}", from=2-1, to=1-2]
	\arrow["{u_\#}", from=1-2, to=3-2]
	\arrow["{g_*}"', from=2-1, to=3-2]
\end{tikzcd}\]
That is, $g_* = u_\# \circ f_*$.
\end{lemma}
\begin{proof}
Suppose we have a loop $\gamma: I \rightarrow X$ based at $x_0$.

We need to check that
$$
g_*([\gamma])=u_{\#} \circ f_*([\gamma])
$$
In other words, we want to show that
$$
g \circ \gamma \simeq u^{-1} \cdot(f \circ \gamma) \cdot u
$$
To prove this result, we want to build a homotopy. 

Consider the composition:
$$
F: I \times I \stackrel{\gamma \times \mathrm{id}_I}{\longrightarrow} X \times I \stackrel{H}{\longrightarrow} Y
$$
Our plan is to exhibit two homotopic paths $\ell^{+}$ and $\ell^{-}$ in $I \times I$ such that
$$
F \circ \ell^{+}=g \circ \gamma, \quad F \circ \ell^{-}=u^{-1} \cdot(f \circ \gamma) \cdot u
$$
This is in general a good strategy $-X$ is a complicated and horrible space we don't understand. So to construct a homotopy, we make ourselves work in a much nicer space $I \times I$.

Our $\ell^{+}$ and $\ell^{-}$ are defined in a rather simple way. Precisely, $\ell^{+}$ is the path $s \mapsto(s, 1)$, and $\ell^{-}$ is the concatenation of the paths $s \mapsto(0,1-s), s \mapsto(s, 0)$ and $s \mapsto(1, s)$

Note that $\ell^{+}$ and $\ell^{-}$ are homotopic as paths. If this is not obvious, we can manually check the homotopy
$$
L(s, t)=t \ell^{+}(s)+(1-t) \ell^{-}(s)
$$
This works because $I \times I$ is convex. Hence $F \circ \ell^{+} \simeq{ }_{F \circ L} F \circ \ell^{-}$as paths.

Now we check that the compositions $F \circ \ell^{ \pm}$ are indeed what we want. We have
$$
F \circ \ell^{+}(s)=H(\gamma(s), 1)=g \circ \gamma(s)
$$
Similarly, we can show that
$$
F \circ \ell^{-}(s)=u^{-1} \cdot(f \circ \gamma) \cdot u(s)
$$
So done.
\end{proof}

\begin{theorem}
  If $f: X \rightarrow Y$ is a homotopy equivalence, and $x_0 \in X$, then the induced map
  $f_*: \pi_1(X, x_0) \rightarrow \pi_1(Y, f(x_0))$ is an isomorphism.
\end{theorem}
\begin{proof}
  Let $g: Y \rightarrow X$ be a homotopy inverse. So $f \circ g \simeq_H \operatorname{id}_Y$ and $g \circ f \simeq_{H^{\prime}} \operatorname{id}_X$.

  We have no guarantee that $g \circ f\left(x_0\right)=x_0$, but we know that our homotopy $H^{\prime}$ gives us $u^{\prime}=H^{\prime}\left(x_0, \cdot\right): x_0 \leadsto g \circ f\left(x_0\right)$.

Applying our previous lemma with $\operatorname{id}_X$ for ``$f$'' and $g \circ f$ for ``$g$'', we get
$$
u_{\#}^{\prime} \circ\left(\mathrm{id}_X\right)_*=(g \circ f)_*
$$
Using the properties of the $*$ operation, we get that
$$
g_* \circ f_*=u_{\#}^{\prime}
$$
However, we know that $u_{\#}^{\prime}$ is an isomorphism. So $f_*$ is injective and $g_*$ is surjective.

Doing it the other way round with $f \circ g$ instead of $g \circ f$, we know that $g_*$ is injective and $f_*$ is surjective. So both of them are isomorphisms.
\end{proof}

\begin{definition}[Simply Connected Space]
  A space is \vocab{simply connected} if it is path connected and $\pi_1(X, x_0) \equiv 1$ for some (any) choice of $x_0 \in X$.
\end{definition}

\begin{lemma}
  A path-connected space $X$ is simply connected if and only if for any $x_0, x_1 \in X$, there exists a unique homotopy class of paths $x_0 \leadsto x_1$.
\end{lemma}
\begin{proof}
Suppose $X$ is simply connected, and let $u, v: x_0 \leadsto x_1$ be paths. Now note that $u \cdot v^{-1}$ is a loop based at $x_0$, it is homotopic to the constant path, and $v^{-1} \cdot v$ is trivially homotopic to the constant path. So we have
$$
u \simeq u \cdot v^{-1} \cdot v \simeq v
$$
On the other hand, suppose there is a unique homotopy class of paths $x_0 \leadsto x_1$ for all $x_0, x_1 \in X$. Then in particular there is a unique homotopy class of loops based at $x_0$. So $\pi_1\left(X, x_0\right)$ is trivial.
\end{proof}

% \end{multicols*}
\end{document}
