%  \documentclass[DIV=12, a4]{scrartcl}
%\documentclass[12pt, a5]{scrartcl}

% \documentclass[a4paper]{report}
% \usepackage[
% % fancytheorems, 
% noindent, 
% %spacingfix, 
% %noheader
% ]{vanilla}


\documentclass[a4paper]{scrreprt}
\usepackage[
fancytheorems, 
noindent, 
% %spacingfix, 
% %noheader,
fancyproofs
]{adam} 

\usepackage{tikz}

\newcommand{\dd}{\mathrm{d}}

% \usepackage{subfig}

% \setcounter{chapter}{-1}

\title{Differential Equations}
% \subtitle{Adam Kelly}
\author{Adam Kelly}
% \date{Michaelmas 2020}
\date{\today}

\begin{document}

\maketitle

\begin{abstract}
	
	% \vspace{2\baselineskip}
	% {\color{red} None of the notes here have been reviewed at all, and are just exactly what was taken down live in the lectures. I would turn around now and come back in a few days, when I have gone back, cleaned things up, fixed explanations and added some structure.}
	% \vspace{5\baselineskip}

	This set of notes is a work-in-progress account of the course `Differential Equations', originally lectured by Dr. John Taylor in Michaelmas 2020 at Cambridge. These notes are not a transcription of the lectures, but they do roughly follow what was lectured (in content and in structure).

	These notes are my own view of what was taught, and should be somewhat of a superset of what was actually taught. I frequently provide different explanations, proofs, examples, and so on in areas where I feel they are helpful. Because of this, this work is likely to contain errors, which you may assume are my own. If you spot any or have any other feedback, I can be contacted at \href{mailto:ak2316@cam.ac.uk}{ak2316@cam.ac.uk}.


	% {\color{red} Notes written upto lecture 6.}
	% During the creation of this document, I consulted a number of other books and resources. All of these are listed in the bibliography. 

\end{abstract}

\tableofcontents

\clearpage

\chapter{Introduction}

Differential equations is a foundational course in applied mathematics, which many courses in the future use as a foundation. This subject studies equations involving \emph{derivatives}, and towards the end, \emph{partial derivatives}. A working knowledge of calculus is essential, but will be briefly reviewed at the start of the course. 


\section{Structure of the Course}

This course is divided into five sections.

\begin{enumerate}
	\item \emph{Calculus}
	
	% This section is a brief review of Calculus, which you should be familiar with.
	% We will treat (informally) derivatives, integrals, Taylor's theorem and the fundamental theorem of calculus. This review will be done through the introduction of $O$ and $o$ notation (for which familiarity is \emph{not} assumed).

	% After this, we take a brief look at functions of several parameters, and the consequential notion of partial derivatives.

	\item \emph{First-Order Linear Differential Equations}
	
	% This section has a different perspective, centering on the questions of \emph{what is a real number} and \emph{what can we assume about them?} This is one of the harder parts of this course, and many of the definitions contain a subtlety that is not present in other sections. 

	\item \emph{Nonlinear First-Order Differential Equations}
	\item \emph{Higher-Order Linear Differential Equations}
	\item \emph{Multivariate Functions and Applications}
	
	% This is a `terminology' section. There is no exciting theorems, mostly notation, definitions, and so on. It is a short section, but it is somewhat boring in that sense.

\end{enumerate}

Everything in the sections above makes up the `course'. If you are wondering what is examinable, it will be everything that was lectured. It is possible that this set of `lecture notes' will contain additional content that was both not in lectures and not examinable. If you want to be sure whether something you are reading here or elsewhere is examinable, you can get a more formal answer in \href{https://www.maths.cam.ac.uk/undergrad/files/schedules.pdf}{the schedules}.

\section{Books}

As with most mathematics courses in Cambridge, you will not need a textbook to follow this course. What is covered in lectures is enough to do both the example sheets and the examinations for this course. Still, you might find that a textbook can provide a different perspective, additional worked examples, and additional material that you may find informative, helpful or fun.

The following books were recommended in the schedules
\begin{itemize}
	\item J. C. Robinson, \emph{An Introduction to Ordinary Differential Equations}.

	\item R. T. Allenby, \emph{Numbers and Proofs}.
	
	This book is readable, easy to understand and clear.

	\item A. G. Hamilton, \emph{Numbers, Sets and Axioms}.
	
	Another readable and clear book, but with a different flavour to the previous book.

	\item H. Davenport, \emph{The Higher Arithmetic}.

	This book can be thought of as showing `where things go next'. It is very interesting, and goes quite a bit beyond this course. It is worth noting however that this book contains no exercises.
\end{itemize}

You should be able to find all of these books in either your college library or the university library.

% \section{Example Sheets}

% As is normal for a 24 lecture course, there will be 4 example sheets. You should be able to have a good go at the first one after lecture 3 or 4.

\section{A Brief Note About These Notes}

This set of notes differs from what was lectured in a number of areas. I have attempted to briefly outline these changes below.

In this set of notes, the review of calculus in the first chapter is done with slightly more rigour than was present in the original lecture course (however, it is still an informal account by any means).


\clearpage


\chapter{Calculus}

To study differential equations, one needs to have a working knowledge of calculus. The goal of this chapter is to review some of the ideas that you (should) have learned previously, but through the introduction of $O$ and $o$ notation. Following this, we will briefly introduce the idea of `functions of several variables', along with partial differentiation.


\section{Differentiation}

We will begin our review of calculus by defining the notion of the \emph{derivative} of a function. 

\begin{definition}[Derivative]
	Let $f: I \rightarrow \R$ be a function defined on some interval $I \subseteq \R$. The \vocab{derivative} of $f$ 
	is the limit
	$$
	f'(x) = \frac{\dd f}{\dd x} = \lim_{h \to 0} \frac{f(x + h) - f(x)}{h},
	$$
	provided it exists.
	If this limit exists for all $x \in I$, then we say $f$ is \vocab{differentiable} over $I$. 
\end{definition}

\begin{example}
	The function $f(x) = |x|$ is not differentiable at $x = 0$ as the limit does not exist. 
\end{example}


When we have a function $f$ that is sufficiently well behaved\footnote{The intricacies of limits and differentiation will be considered with more rigour in `Analysis', so they need not be the primary focus of this course.}, we can repeatedly take it's derivative, to obtain higher-order derivatives. They are defined recursively as follows.

\begin{definition}[Higher-Order Derivatives]
	For a function $f$, we define the \vocab{$n$th derivative of $f$} by
	$$
	\frac{\dd^n f}{\dd x^n} = \frac{\dd}{\dd x} \left(
		\frac{\dd^{n- 1} f}{\dd x^{n-1}}
	\right),
	$$
	with $\frac{\dd^0 f}{\dd x^0} = f$.
\end{definition}

Throughout these notes, we will use three different notations for the derivative of a function $f$: 
$$
\frac{\dd f}{\dd x} = f'(x) = \dot{f}(x),
$$
which are the Leibniz, Lagrange and Newton notation respectively. Typically we will only use the $\dot{f}(x)$ notation when the independent variable is time.

\subsection{Rules for Differentiating}

There are a number of well known properties of derivatives that make them easier to compute.

\begin{theorem}[Chain Rule]
	Let $f(x) = g(h(x))$ for functions $g(x)$ and $h(x)$. Then
$$
f'(x) = g'(h(x))\cdot h'(x) = \frac{\dd g}{\dd h} \frac{\dd h}{\dd x}.
$$
\end{theorem}

\begin{theorem}[Product Rule]
	Let $f(x) = u(x)v(x)$ for functions $u(x)v(x)$. Then
	$$
	f'(x) = u'(x) v(x) + u(x)v'(x) = \frac{\dd u}{\dd x} v + u\frac{\dd v}{\dd x}.
	$$
\end{theorem}

The product rule can be generalized to higher-order derivatives in the following way.\footnote{This should be reminiscent of the Binomial formula, and indeed both can be proved by the same combinatorial argument.}

\begin{theorem}[Leibniz Rule]
	Let $f(x) = u(x)v(x)$ for functions $u(x)$ and $v(x)$. Then
	\begin{align*}
		f^{(n)}(x) &= \sum_{k = 0}^n \binom{n}{k} u^{(k)} (x) v^{(n - k)}(x)\\
		&= u^{(n)}(x) v(x) + n u^{(n)}(x) v'(x) + \cdots + u(x)v^{(n)}(x).
	\end{align*}
\end{theorem}

\subsection{Orders of Magnitude}

Before we continue our discussion of differentiation, we will look at two ways of comparing the `size' of functions -- big-$O$ and little-$o$ notation. Before we proceed, it is important to note that these are ways of comparing the \emph{local} behavior of functions. 


\begin{definition}[Little-$o$ Notation]
	We say $f(x) \in o(g(x))$ as $x \rightarrow x_0$ for some $x_0 \in \R\cup \{\infty\}$ if
	$$
	\lim_{x \to x_0} \frac{f(x)}{g(x)} = 0.
	$$
\end{definition}
\begin{remark}
	Informally, this can be thought of as `$f(x)$ is much smaller than $g(x)$ near $x_0$'. We will often abuse the notation $f(x) = o(g(x))$ to mean $f(x) \in o(g(x))$.
\end{remark}

\begin{example}
	$x^2 \in o(x)$ as $x \rightarrow 0$, as
	$$
	\lim_{x \rightarrow 0} \frac{x^2}{x} = \lim_{x \rightarrow 0} x = 0.
	$$
\end{example}

\begin{definition}[Big-$O$ Notation]
	We say that $f(x) \in O(g(x))$ as $x \rightarrow x_0$ for $x_0 \in \R \cup \{\infty\}$ if there exists a positive constant $M$ such that
	$$
	\lim_{x \to x_0} \left|\frac{f(x)}{g(x)}\right| = M.
	$$
\end{definition}
\begin{remark}
	Informally, this can be thought of as `$f(x)$ can be locally bounded by $g(x)$'
\end{remark}

\begin{example}~
	    \vspace*{-\baselineskip}
	\begin{enumerate}[label=(\roman*)]
		\item $x^2 \in O(x)$ as $x \rightarrow 0$
		\item $x \not\in O(x^2)$ as $x \rightarrow 0$.
		\item $x^2 \in O(x^2)$ for all $x$.
		\item $2x^3 + 4x + 12 \in O(x^3)$ as $x \rightarrow \infty$.
	\end{enumerate}

\end{example}



\end{document}
