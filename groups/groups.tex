%  \documentclass[DIV=12, a4]{scrartcl}
%\documentclass[12pt, a5]{scrartcl}
\documentclass[a4]{scrartcl}

\usepackage[
fancytheorems, 
noindent, 
%spacingfix, 
%noheader
]{adam}

% \usepackage{subfig}

% \setcounter{section}{-1}

\title{Groups}
% \subtitle{Adam Kelly}
\author{Adam Kelly, Lectured by Dr. A. Khukhro}
\date{Michaelmas 2020}

\begin{document}

\maketitle

\begin{abstract}
	This document is an account of the Cambridge Mathematical Tripos course `Groups', lectured by Dr. Ana Khukhro. in Michaelmas 2020.
	This is a work in progress, and is likely to to contain errors, which you may assume to be my own.
	% This document is a rather brief summary of the first three chapters of H. S. M. Coxeter and S. L. Greitzer's `Geometry Revisited'.
	% In no ways is this fleshed out, and in most cases just contains the important results and diagrams.
	% Specifically, it's purpose is to be somewhat of a reference which one can consult whilst attempting olympiad geometry problems.
\end{abstract}

\tableofcontents

\clearpage

\section{Groups}

\subsection{Definition}

In this section we will formally introduce the notion of a group, and we will consider some examples of groups along with their basic properties.

\begin{definition}
	A \vocab{group} is a set $G$ with a binary operation $*$ on $G$ such that:
	\begin{itemize}
		\item \emph{Identity}. $G$ has an \vocab{identity element} $e$ such that $e * g = g * e = g$ for all $g \in G$.
		\item \emph{Inverses}. Each element $g \in G$ has an \vocab{inverse}, that is, an element $g^{-1} \in G$ such that $g * g^{-1} = g^{-1} * g = e$.
		\item \emph{Associativity}. The operation $*$ is associative, that is $(g * h) * k = g * (h * k)$ for any $g, h, k \in G$. 
	\end{itemize}
\end{definition}

\begin{remark}[A pedantic point]
	In some cases, people will add an additional `closure' axiom, stating that if $g, h \in G$ then $g * h \in G$. However, this is redundant as it is implied by stating that $*$ is a binary operation on $G$. You must keep it in mind however when checking if something is a group.
\end{remark}
\begin{remark}[Bracketing]
	The `associativity' axiom means that we can write $g * h * k$ without specifying what order it should be done first.
\end{remark}

\begin{notation}
	It's proper to state that `$(G, *)$ is a group', but this is regularily abbreviated to saying `$G$ is a group', whenever the operation being used is clear.
\end{notation}

So that's what a group is, let's dive straight into some examples.

\begin{example}[Examples of Groups]
	The following are all examples of groups.
	\begin{enumerate}
		\item $G = \{e\}$, along with the binary operation $*$ satisfying $e * e = e$ (the `trivial group').
		\item $G$ being the set of symmetries of a shape, along with $g*h$ defined to be `performing $h$ followed by $g$' where $g, h \in G$ is a group.
		\item $(\mathbb{Z}, +)$, $(\mathbb{Q}, +)$ , $(\mathbb{R}, +)$  and $(\mathbb{C}, +)$ are all groups.
		\item The nonzero\footnote{You should consider why we need to exclude zero for $\mathbb{R}$ to be a group.} real numbers $\mathbb{R}\backslash \{0\}$ with multiplication is a group.
		\item $(\mathbb{R}, *)$ where $r * s = r + s + 5$ for any $r, s \in \mathbb{R}$ is a group.
		\item $\mathbb{Z}_n = \{0, 1, \dots, n - 1\}$ with addition modulo $n$ is a group.
		\item A vector space with vector addition is a group.
		\item The set of invertible $2 \times 2$ matrices with real coefficients, $GL_2(\mathbb{R})$ is a group with respect to matrix multiplication.
	\end{enumerate}
\end{example}
\begin{proof}[Proof Sketch]
	Check that each construction satisfies all of the axioms stated in the definition of a group.
\end{proof}

Let's also look at some structures that are \emph{not} groups.

\begin{example}[Non-Examples of Groups]
	The following are all \emph{not} groups.
	\begin{enumerate}
		\item $G = \{0, 1, 2, \dots, n - 1\}$ with addition.
		\item $(\mathbb{Z}, \times)$.
		\item $(\mathbb{R}, *)$ where $r * s = r^2 s$ for $r, s \in \mathbb{R}$.
		\item $G = \{0, 1, 2, \dots\}$ and the operation $*$ such that $m * n = |n - m|$ for $m, n \in G$.
	\end{enumerate}
\end{example}


% \begin{example}[Trivial Groups]
% 	The set $G = \{ e \}$ with the operation $*$ satsifying $e * e = e$ is the `trivial group'. 
% \end{example}

% \begin{example}[Symmetries of a Triangle]
% 	Let $G$ be the set of symmetries of a triangle, and define the operation $*$ on the symmetries $g, h \in G$ so that $g * h$ means doing $h$ followed by $g$. Then $G$ is a group. 
% \end{example}



\end{document}
