\documentclass[a4paper]{scrartcl}

\usepackage[
    fancytheorems, 
    fancyproofs, 
    noindent, 
]{adam}


\title{Combinatorics}
\author{Adam Kelly (\texttt{ak2316@cam.ac.uk})}
\date{\today}

\allowdisplaybreaks

\begin{document}

\maketitle

% This is a short description of the course. It should give a little flavour of what the course is about, and what will be roughly covered in the notes.

This article constitutes my notes for the `Combinatorics' course, held in Michaelmas 2021 at Cambridge. These notes are \emph{not a transcription of the lectures}, and differ significantly in quite a few areas. Still, all lectured material should be covered.



\tableofcontents

% Set Systems
% Isopermetric Inequalities
% Intersecting families

% Prereqs is graph theory and general maths stuff

\section{Set Systems}
 
We will begin our study of combinatorics by considering \emph{set systems} -- collections of subsets of a set (which will typically be $X = [n] = \{1, 2, \dots, n\}$).

\begin{definition}[Set Systems]
Let $X$ be a set. A \vocab{set system} on $X$ or a \vocab{family of subsets} of $X$ is a family $\mathcal{A} \subset \mathcal{P}(X)$.
\end{definition}

It's often useful to think about the power set of a set $X$, $\mathcal{P}(X)$, as a graph. We can do this by joining two elements $A$ and $B$ if $|A \triangle B| = 1$, where $\triangle$ is the symmetric difference.
This graph is the \vocab{discrete cube $Q_n$}\footnote{This is the same graph as the boolean hypercube, in the obvious way.}.


\subsection{Chains and Antichains}

We are first going to look at what happens when sets are contained or not contained in each-other. If you know anything about posets, this will likely be familiar. 

\begin{definition}[Chain and Antichain]
We say that $\mathcal{A} \subset \mathcal{P}(X)$ is a vocab{chain} if for all $A, B \in \mathcal{A}$ we have $A \subset B$ or $B \subset A$.

We say that $\mathcal{A}$ is a \vocab{antichain} if for all $A, B \in \mathcal{A}$ with $A \neq B$ we have $A \not \subset B$ and $B \not \subset A$.
\end{definition}

\begin{example}]
    $\{\{1, 4\}, \{1, 4, 7, 8\}, \{1, 2, 4, 7, 8\}\}$ is a chain, and  $\{\{1, 4\}, \{1, 7, 8\}, \{32, 3, 8\}\}$ is an antichain.
\end{example}

A natural first question is how big can a chain be? We can easily get $|\mathcal{A}| = n - 1$ by taking
$$
\mathcal{A} = \{\emptyset, \{1\}, \{1, 2, \}, \dots, [n]\}.
$$
Can we beat this? No, since $\mathcal{A}$ must meet $X^{(r)}$ (the set of $r$ element subsets of $X$) at at most one point.

How about antichains? We can achieve $|\mathcal{A}| = n$ by taking all singleton sets, but can we do any better? Well with the same idea we can take each $\lfloor n/2\rfloor$-element subset of $[n]$, giving $|\mathcal{A}| = \binom{n}{\lfloor n/2\rfloor}$. Can we do better than \emph{this}? It's not quite obvious (and it's this type of question that we will come across frequently in this course...).
\end{document}
