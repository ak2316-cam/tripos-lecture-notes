\documentclass[a4paper]{amsart}

\usepackage[
    nokoma,
    %fancytheorems, 
    %fancyproofs, 
    noindent, 
]{adam}


% \newtheorem{method}[theorem]{Method}

\title{Groups}
\author{Adam Kelly (\texttt{ak2316@cam.ac.uk})}
\date{\today}

\allowdisplaybreaks

\begin{document}

\maketitle

\section{Basic Results}

\begin{theorem}[Lagrange]
	Let $G$ be a finite group, and let $H \leq G$. Then $|G| = |H| |G:H|$, where $|G:H|$ is the \emph{index} of $H$ in $G$, the number of left cosets.
\end{theorem}

\begin{theorem}[First Isomorphism Theorem]
	Let $\phi: G \rightarrow H$ be a homomorphism. Then $G/\ker \phi \cong \img \phi$.
\end{theorem}

\begin{theorem}[Second Isomorphism Theorem]
	Let $H \leq G$ and $N \normal G$. Then $H \cap N \normal H$ and $H/(H \cap N) \cong HN/N$.
\end{theorem}

\begin{theorem}[Third Isomorphism Theorem]
	Let $N \leq M \leq G$ such that $N \normal G$ and $M \normal G$. Then $M/N \normal G/N$ and $(G/N) / (M/N) \cong G/M$.
\end{theorem}

\begin{theorem}[Correspondence Theorem]
	Let $N \normal G$. Then the subgroups of $G/N$ are in bijective correspondence with subgroups of $G$ containing $N$.
\end{theorem}
% \begin{proof}
% 	Let $H \leq G$ with $N \subset H$, and consider the mapping $H \mapsto H/N$.
% \end{proof}

\section{Simple Groups}

\begin{definition}
	A group is \emph{simple} if $\{e\}$ and $G$ are its only normal subgroups.
\end{definition}

\begin{lemma}
	The only simple abelian group if $C_p$ for a prime $p$.
\end{lemma}

\begin{lemma}
	If $G$ is a finite group, then $G$ has a \emph{composition series}
	$$
	1 \cong G_{0} \triangleleft G_{1} \triangleleft \cdots \triangleleft G_{n}=G
	$$
	where each quotient $G_{i + 1}/G_i$ is simple.
\end{lemma}
\begin{proof}[Proof Sketch]
	Induct on $|G|$. Let $G_{n - 1}$ be a normal subgroup of largest possible order not equal to $|G|$. Then $G/G_{n - 1}$ exists, and it's simple by the correspondence theorem.
\end{proof}

\section{Simplicity of the Alternating Group}

\begin{lemma}
	$A_n$ is generated by $3$-cycles.
\end{lemma}

\begin{lemma}
	If $n \geq 5$, all $3$-cycles are conjugate in $A_n$.
\end{lemma}

\begin{theorem}
	$A_n$ is simple for $n \geq 5$.
\end{theorem}
\begin{proof}[Proof Sketch]
		Let $N$ be normal and nontrivial, and let $\sigma \in N$. We then consider $\sigma^{-1}\delta^{-1}\sigma\delta$ for a given $\delta$.
	\begin{itemize}
		\item \emph{Case 1}. $\sigma$ contains a cycle of length $r \geq 4$.
		
		Write $\sigma = (1\ 2\ \cdots\ r) \tau$, and let $\delta = (1\ 2\ 3)$. Then we get $(2\ 3\ r) \in N$.

		\item \emph{Case 2}. $\sigma$ contains two 3-cycles. 
		
		Again write $\sigma = (1\ 2\ 3)(4\ 5\ 6)\tau$. Then let $\delta = (1\ 2\ 4)$, and we get $(1\ 2\ 4\ 3\ 6)$, which is the first case.

		\item \emph{Case 3}. $\sigma$ contains two 2-cycles.
		
		Write $\sigma = (1\ 2)(3\ 4) \tau$. We then repeatedly perform our process for  $\delta$ in $\{(1\ 2\ 3), (1\ 2\ 4), (1\ 2\ 3),$ $(2\ 3\ 5)\}$ (in order), operating on our result. We then get $(2\ 5\ 3)$.
	\end{itemize}
	In each case we get a 3 cycle, and in the last case, $\sigma$ contains a 3 cycle but then (by considering powers of $\sigma$) we get a 3-cycle as required.
\end{proof}

\section{$p$-Groups}

\begin{definition}
	A finite group $G$ is a $p$-group if $|G| = p^n$ where $p$ is a prime.
\end{definition}

\begin{theorem}
	If $G$ is a $p$-group then $Z(G)$ is non-trivial.
\end{theorem}
\begin{proof}
For $g \in G$, we have $\left|\operatorname{ccl}_{G}(g)\right| \cdot\left|C_{G}(g)\right|=|G|=p^{n}$. So each conjugacy class has size that is a power of $p$. Since $G$ is a union of it's conjugacy classes,
$$
\begin{aligned}
|G| & \equiv \#(\text { conj. classes of size } 1) \quad(\bmod p) \\
\Longrightarrow 0 & \equiv|Z(G)| \quad(\bmod p) .
\end{aligned}
$$
In particular, $|Z(g)|>1$.
\end{proof}

\begin{theorem}
	Let $G$ be a $p$-group of order $p^n$. Then $G$ has a subgroup of order $p^r$ for all $0 \leq r \leq n$.
\end{theorem}
\begin{proof}
	We have a composition series $1 \equiv G_0 \triangleleft \cdots G_m = G$ with $G_{i}/G_{i-1}$ simple. Each of these is a $p$-group and is thus isomorphic to $C_p$.
\end{proof}

% \begin{theorem}[Cayley's Theorem]
% 	Any finite group $G$ is isomorphic to a subgroup of $S_{|G|}$.
% \end{theorem}
% \begin{theorem}
% 	Let $G$ be non-abelian and simple, and let $H \leq G$ with index $n > 1$. Then $G$ is isomorphic to a subgroup of $A_n$.
% \end{theorem}
% \begin{proof}
% 	Let $G$ act on $G/H$ be left multiplication. This group action has a permutation representation
% \end{proof}

\end{document}
