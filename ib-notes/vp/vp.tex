\documentclass[a4paper]{amsart}

\usepackage[
    nokoma,
    %fancytheorems, 
    %fancyproofs, 
    noindent, 
]{adam}


% \newtheorem{method}[theorem]{Method}

\title{Variational Principles}
\author{Adam Kelly (\texttt{ak2316@cam.ac.uk})}
\date{\today}

\allowdisplaybreaks

\begin{document}

\maketitle

\emph{To-do: we should probably write actual proofs for the examinable bookwork...}

\section{The Euler Lagrange Equations}

\begin{method}[Lagrange Multipliers]
	To extremize $f: \R^n \rightarrow \R$ subject to $g: \R^n \rightarrow \R$ with $g(x) = 0$, define $h(x, \lambda) = f(x) - \lambda g(x)$, then extremize $h$ without constraints by solving $\nabla h = 0$.
\end{method}

\begin{lemma}[Fundamental Lemma of the Calculus of Variations]
		If $g: [\alpha, \beta] \rightarrow \R$ is continuous and for all $\nabla$ continuous with $\nabla(\alpha) = \nabla(\beta) = 0$ we have $\int_{\alpha}^{\beta}g(x)\nabla(x) \dd x = 0$, then $g(x) = 0$ for all $x$.
\end{lemma}
\begin{proof}[Proof (Sketch)]
If there's a $c$ with $g(c) \neq 0$, then take an interval $[a, b]$ on which $g$ is non-zero (exists by continuity) and consider $\nabla(x) = (x - a)(x - b)$ on $[a, b]$ with $\nabla(x) = 0$ elsewhere. Then the integral is non-zero, which is a contradiction.
\end{proof}

\begin{theorem}[Euler-Lagrange]
	Suppose $y \in C^2_{[\alpha, \beta]}(\R)$ is a function with fixed endpoints that extremizes the functional
	$$
	F[y] = \int_{\alpha}^{\beta} f(x, y(x), y'(x)) \dd x.
	$$
	Then $y$ satisfies the \emph{Euler-Lagrange equations},
	$$
	\frac{\partial f}{\partial y} - \frac{\dd}{\dd x} \left(\frac{\partial f}{\partial y'}\right) = 0.
	$$
\end{theorem}
\begin{proof}[Proof (Sketch)]
Let $y$ be an extreme point and consider a pertubation $y + \varepsilon \nabla$, where $\nabla$ is zero at the endpoints. Then Taylor expanding $F[y + \varepsilon \nabla]$ in $\varepsilon$ gives a first order term that we want to be zero. We can then integrate by parts to get an equation that we can apply our Lemma to.
\end{proof}

\subsection*{First Integrals}

If $f$ does not depend explicitly on $y$, then the Euler Lagrange equations simplify to $\partial f/\partial y' = c$, for some constant $c$, which we can solve.

If $f$ does not depend explicitly on $x$, then noting by computation that
$$
\frac{\dd}{\dd x} \left(f - y' \frac{\partial f}{\partial y'}\right) = \frac{\partial f}{\partial x},
$$
we get a different first integral condition, $f - y' \frac{\partial f}{\partial y'} = c$, for some constant $c$, which we can solve.

\subsection*{Multiple Dependent Variables}

If we had a function $y: \R \rightarrow \R^m$ that we wanted to extremize as before, following the same derivation as the single variable Euler Lagrange yields that the Euler-Lagrange equations hold in each component. We can also obtain the first integrals as before.

\subsection*{Multiple Independent Variables}

Repeating the derivation of the Euler Lagrange equations in the case that we have multiple independent variables, we can obtain (using the divergence theorem) the Euler-Lagrange equation for multiple independent variables,
$$
\frac{\dd f}{\dd \phi} - \nabla \cdot \left(\frac{\partial f}{\partial \phi_x}, \frac{\partial f}{\partial \phi_y}, \frac{\partial f}{\partial \phi_z}\right) = 0.
$$

\subsection*{Higher Derivatives}

If we want to work with higher derivatives of $y$, then we obtain (with the standard derivation of the Euler-Lagrange equations along with more integration by parts) a variant of the Euler-Lagrange equations of the form
$$
\frac{\partial f}{\partial y} - \frac{\dd }{\dd y} \frac{\partial f}{\partial y'} + \cdots + (-1)^n \frac{\dd^n}{\dd x^n} \frac{\partial f}{\partial y^(n)} = 0.
$$

\section{Variational Principles}

We will now look at some examples of laws of nature that arise in the form of variational principles.

\subsection*{Fermat's Principle} 

As light travels between two points, it takes the path of least time.

\subsection*{Principle of Least Action} 

Consider a particle moving in $\R^3$ with kinetic energy $T$ and potential energy $V$. We define the \emph{Lagrangian} to be
$
L(x, \dot{x}, t) = T - V.
$
We then define the \emph{action} to be
$
S[x] = \int_{t_1}^{t_2} L \dd t.
$
The principle of least (or stationary) action then says that on the path of motion of a particle this functional is extremized.

\subsection*{Noether's Theorem}

Consider a functional
$
F[y] = \int_{\alpha}^{\beta} f(y_i, y'_i, x) \dd x$ with $i = 1, \dots, n$.

Suppose there was a one-parameter family of transformations $y_i(x)\mapsto \mathcal{Y}_i(x, s)$ with $\mathcal{Y}_i(x, 0) = y_i(x)$. This family is a \emph{continuous symmetry} of the Lagrangian $f$ if
$$
\frac{\dd}{\dd x} f(\mathcal{Y}_i(x, s), \mathcal{Y}_i'(x, s), x) = 0.
$$

\begin{theorem}[Noether's Theorem]
	Given a continuous symmetry $\mathcal{Y}_i(x, s)$ of $f$,
	$$
		\sum_{i = 1}^n \left.\frac{\partial f}{\partial y_i'} \frac{\partial \mathcal{Y}_i}{\partial s} \right|_{s = 0}
	$$
	is a first integral of the Euler-Lagrange equation.
\end{theorem}
\begin{proof}[Proof (Sketch)]
	Start with $\left.\dd f/\dd s \right|_{s = 0} = 0$ and expand.
\end{proof}

\section{The Legendre Transform}

\begin{definition}[Legendre Transform]
	The Legendre transform of a function $f: \R^n \rightarrow \R$ is a function $f^*$ given by
	$$
		f^*(p) = \sup_{x}(p \cdot x - f(x)).
	$$
	We take the domain of $f^*$ to be such that the supremum above is finite.
\end{definition}


In one dimension, $f^*(p)$ can be thought of as the maximum vertical distance between $y = f(x)$ and $y = px$.

\begin{proposition}
	If the domain of $f^*$ is non-empty, it is a convex-set, and $f^*$ is convex.
\end{proposition}


\section{Second Variations}

\begin{theorem}[Condition on Local Minima]
Let $y$ be a solution to the Euler-Lagrange equation and define
$$
P = \frac{\partial^2 f}{\partial (y')^2} \quad \text{and} \quad Q = \frac{\partial^2 f}{\partial y^2} - \frac{\dd}{\dd x} \frac{\partial^2 f}{\partial y \partial y'}.
$$
If $Q \eta^2 + P(\eta')^2 > 0$ for all $\eta$ which vanish at $\alpha$ and $\beta$, then $y$ is a local minimizer of $F[y]$.
\end{theorem}
\begin{proof}[Proof (Sketch)]
Expand to the second order in $\varepsilon$ and integrate the last term by parts to get $\delta^2 F[y] > 0$.
\end{proof}


\begin{theorem}[Legendre Condition]
	If $y_0$ is a local minimum then $P =\left. \frac{\partial^2 f}{\partial (y')2} \right|_{y_0} \geq 0$.
\end{theorem}

\begin{theorem}
	If $-(Pu')' + Qu = 0$ has a solution with $u \neq 0$ on $[\alpha, \beta]$, then $F[y]$ is a local minima.
\end{theorem}


\end{document}
