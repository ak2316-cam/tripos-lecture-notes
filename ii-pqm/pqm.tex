% \documentclass[a4, 10pt]{scrartcl}
\documentclass[a4paper, 10pt, twocolumn]{amsart}

\usepackage[
% fancytheorems,
% fancyproofs
nokoma
]{adam}

\usepackage[margin=0.75in]{geometry}

\title{Principles of Quantum Mechanics}
\author{Adam Kelly -- Mathematical Tripos Part II}
\date{\today. Email \texttt{ak2316@cam.ac.uk}}

\begin{document}

\maketitle

\section{Uncertainty Principle}

If two operators do not commute, we cannot expect to measure both exactly. This is quantified in an \emph{uncertainty principle}: Let $A, B$ be Hermitian. Then taking $C=A+i \lambda B, \lambda \in \mathbb{R}$,
$$
C^{\dagger} C=A^2+\lambda^2 B^2+\lambda i[A, B] .
$$
The first three are Hermitian, so $i[A, B]$ is also Hermitian. We have
$$
\begin{aligned}
0 & \leqslant\langle C \psi, C \psi\rangle=\left\langle\psi\lvert C^{\dagger} C \lvert \psi\right\rangle \\
& =\left\langle A^2\right\rangle_\psi+\lambda^2\left\langle B^2\right\rangle_\psi+\lambda\langle i[A, B]\rangle_\psi .
\end{aligned}
$$
For this to always be nonnegative, can have at most one real root, so discriminant gives
$$
\left\langle A^2\right\rangle_\psi\left\langle B^2\right\rangle_\psi \geqslant \frac{1}{4}\left(\langle i[A, B]\rangle_\psi\right)^2 .
$$
This works for any $A, B$, so if we apply it to $\tilde{A}=A-\langle A\rangle_\psi$ and $\tilde{B}=B-\langle B\rangle_\psi$, we find $[\tilde{A}, \tilde{B}]=[A, B]$ and hence
$$
(\Delta A)_\psi(\Delta B)_\psi \geqslant \frac{1}{2}\left|\langle[A, B]\rangle_\psi\right| \text {. }
$$
Most famous is Heisenberg's uncertainty principle from applying this to $(38)$,
$$
(\Delta x)(\Delta p) \geqslant \frac{1}{2} \hbar
$$

\end{document}
