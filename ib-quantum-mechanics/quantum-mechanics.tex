\documentclass[a4paper]{scrartcl}

\usepackage[
    fancytheorems, 
    fancyproofs, 
    noindent, 
]{adam}


\usepackage{physics}

\title{Quantum Mechanics}
\author{Adam Kelly (\texttt{ak2316@cam.ac.uk})}
\date{\today}


\allowdisplaybreaks

\begin{document}

\maketitle


Quantum mechanics is the mathematical framework used to describe nature at the scale of subatomic particles. This course will primarily focus on building up this mathematical framework, and not so much on the actual applications or physical background to it.

% This is a short description of the course. It should give a little flavour of what the course is about, and what will be roughly covered in the notes.

This article constitutes my notes for the `Quantum Mechanics' course, held in Michaelmas 2021 at Cambridge. These notes are \emph{not a transcription of the lectures}, and differ significantly in quite a few areas (more so than some of my other notes). Still, all lectured material should be covered.
In particular, these notes are heavily influenced by the `Principles of Quantum Mechanics' course, along with numerous books.



\tableofcontents

% \section{Historical Introduction}

% Historical introduction has been omitted intentionally.

% \section{Historical Introduction}

% \begin{enumerate}
%     \item \emph{1801-03}. Interference/diffraction experiments (by Young), found that light was a wave.
%     \item \emph{1862-64}. The development of electromagnetism (by Maxwell), found that light was an electromagnetic wave.
%     \item \emph{1897}. The discovery of the electron (by Thomson).
%     \item \emph{1900}. The Plack law, explaining black body radiation. 
%     \item \emph{1905}. The photo-electric effect (by Einstein).
%     \item \emph{1909}. Wave-light interference patterns with one photon recorded at a time (by G. I. Taylor).
%     \item \emph{1911}. Rutherford's experiment giving an atomic model.
%     \item \emph{1913}. Bohr model of the atom.
%     \item \emph{1923}. The Compton experiment of x-ray scattering off electrons.
%     \item \emph{1923-24}. Wave-particle duality.
%     \item \emph{1925-30}. The emergence of Quantum Mechanics (by Heisenberg, Bohr, Jordan, Dirac, Pauli, Schrödinger, \dots).
%     \item \emph{1927-28}. Diffraction experiment with electrons (by Davisson, Germer, and Thomson).
% \end{enumerate}


% \subsection{Particles and Waves in Classical Mechanics}

% We will begin by describing the physical background to particles and waves, as studied in classical mechanics.

% % \begin{definition}[Particle]
    
% % \end{definition}

% A \vocab{particle} is an object carrying energy and momentum in an point-like position in space.
% Particles are determined by two vectors, $\vv x$ (position) and $\vv v = \vv{\dot{x}}$ (velocity). Newton's second law gave us that
% $$
%     m \ddot{\vv x} = \vv{F}(\vv{x}(t), \vv{\dot{x}}(t)),
% $$
% and solving this equation would then describe the position totally, so long as the initial conditions are known.

% A \vocab{wave} is any real or complex valued function with periodicity in time and/or space. If the period of the wave is $T$, then it has frequency $1/T$, and we define the \vocab{angular frequency} as $\omega = 2 \pi / T$. If we have a function of $x$ in one dimension, $f(x + \lambda) = f(x)$, we call $\lambda$ the \vocab{wavelength} and $k = 2\pi/\lambda$ the \vocab{wavenumber}.
% In three dimensions, we have $f(x) = \exp(i \vv k \cdot \vv x)$, which we call \vocab{plane waves}. We call $\vv k$ the \vocab{wavevector}, and $\lambda = 2\pi/|\vv k|$.

% One dimensional waves obey the wave equation
% $$
%     \frac{\partial^2 f(x, t)}{\partial t^2} - c^2 \frac{\partial^2 f(x, t)}{\partial x^2} = 0,
% $$
% with $c \in \R$. The solutions are given by
% $$
%     f(x, t) = A \exp(\pm i kx - i \omega t), 
% $$
% with $\omega = ck$ and $\lambda = cT$ (the dispersion relations). We call $A$ the amplitude.

% The three dimensional wave equation is given by
% $$
% \frac{\partial^2 f(\vv x, t)}{\partial t^2} - c^2 \nabla^2 f(\vv x, t) = 0,
% $$
% and the solutions are given by
% $$
% f(\vv x, t) = A \exp(i \vv k \cdot \vv x - i \omega t),
% $$
% with $\omega = c |\vv k|$ and $\lambda = cT$.

% Other kinds of waves arise as solutions to other governing wave equations, provided other dispersion relations are given.
% Also for any governing equation that is linear in $f$, we always have the superposition principle, saying that if $f_1, f_2$ are solutions to the equation, then so is $f = f_1 + f_2$.

% % \subsection{Particle-like Behavior of Light-waves}

% % % Blackbody radiation

% \subsection{Particle-Like Behavior of Light-Waves}

% If you heat a body to a temperature $T$, it will emit radiation.
% The simplest system which this behavior can be studied in is a \emph{black-body}, a totally absorbing surface. In studying such a system, Kirchoff looked at the intensity of light emitted by the black body as a function of the radiation frequency. He observed a %insertimagehere



% \section{Foundations of Quantum Mechanics}

% \subsection{Wavefunctions and Probabilistic Interpretation}

% How do we describe a point particle? 
% In classical mechanics, we had a vector $\vv x$ and a vector $\vv p$ which gave the position and momentum of the particle. 

% The corresponding description in quantum mechanics is the state $\psi$, described\footnote{The state $\psi$ should be thought of as an abstract entity, with the wavefunction $\psi(\vv x, t)$ being the complex coefficients of $\psi$ in a continuous basis.} of  by the complex valued \emph{wavefunction} $\psi(\vv x, t)$. 

% \begin{definition}[Wavefunction]
%     A \vocab{wavefunction} $\psi(\vv x, t)$ is a complex valued function $\R^3 \times \R \rightarrow \C$ that satisfies the mathematical properties dictated by its physical interpretation.
% \end{definition}

% To see how the wavefunction related to the underlying physical object, we need to have some sort of interpretation of it. This is given by the \emph{Born rule}.

% \begin{law}[The Born Rule]
% The probability density for a particle to be at $\vv x$ at a time $t$ is given by
% $$
% \rho(\vv x, t) = \abs{\psi(\vv x, t)}^2.
% $$
% \end{law}

% The Born rule immediately imposes some restrictions on the wavefunction. Because the particle has to be \emph{somewhere} with probability 1, the wavefunction has normalized such that 
% $$
%     \int_{\R^3} \abs{\psi(\vv x, t)}^2 \dd V = 1.
% $$

% It's sometimes useful to work with unnormalized wavefunctions (where we just impose the restriction that this integral is finite), but it must be remembered that the underlying physical interpretation requires this to be normalized.

% \subsection{Hilbert Space and Inner Products}

% The natural language of quantum mechanics is Linear Algebra.
% In a finite dimensional vector space, we have abstract objects of vectors $v$, which can represented concretely by considering some basis and writing say $v = (v_1, v_2, \dots, v_n)$.

% In quantum mechanics, we deal with the abstract object of a \emph{state} $\psi$. This too can be represented concretely using a \emph{wavefunction} $\psi(\vv x, t)$.

% \begin{remark}[For Pedants]
%     If we have two wavefunctions which differ by a complex phase $\tilde{\psi}(\vv x, t) = e^{i \theta} \psi(\vv x, t)$ for some $\theta \in \R$, 
%     then this does not change the probability density $\rho(\vv x, t)$. 
%     Because of this, we treat the two wavefunctions $\tilde{\psi}$ and $\tilde{\psi}$ as corresponding to the same state.
%     So states really correspond to rays in the vector space of wavefunctions.
% \end{remark}

% In linear algebra we work in \vocab{Hilbert space}, the vector space of square integrable functions $L^2(\R^3)$, denoted $\mathcal{H}$.

% Working over a vector space brings with it various facts. One such fact is the \vocab{superposition principle}, which says that if $\psi_1, \psi_2 \in \mathcal{H}$ then $\psi = \lambda_1 \psi_1 + \lambda_2 \psi_2 \in \mathcal{H}$.
% This does correspond to the physics of quantum mechanics, and it is inherently non-classical.

% Of key importance is how the wavefunction evolves over time. This is given by the time-dependent Schrodinger equation (which should be treated as a postulate).

% $$
% i \hbar \frac{\partial \psi}{\partial t}(\mathbf{x}, t)=-\frac{\hbar^{2}}{2 m} \nabla^{2} \psi(\mathbf{x}, t)+U(\mathbf{x}) \psi(\mathbf{x}, t)
% $$


% % The lecturer then checks that hilbert space is indeed a vector space, which is trivial.

% % Mention completeness

% AT THIS POINT IN THE NOTES, I BASICALLY WROTE MY OWN INTRO TO THE COURSE

\section{Foundations of Quantum Mechanics}

\subsection{Classical Mechanics}

In classical mechanics, particle motion is described by Newton's laws. These are a set of second order differential equations, which when coupled with two initial conditions then specify the future motion of the particle for all time. These initial conditions are typically taken to be position and momentum $(x(t_0), p(t_0)).$ So Newton's laws really say that the entire state of a particle is encoded by a vector $(x(t), p(t))$ in \emph{phase space} $M$ of position and momentum. As time passes, the particles state traces out a path in phase space, but the whole state is indeed determined by initial vector (along with the differential equation given by Newton's laws).

But of course a particle is really corresponding to something physical, it's not really an element of an abstract vector space $M$. We need some way to go from this Newtonian framework of phase space to physical space. This is done by introducing \emph{Observables}, which are functions $O: M \rightarrow \R$.

Observables can be all sorts of things -- position, kinetic energy, angular momentum, and everything else. These observables are defined for vectors in $M$, but we typically want to evaluate them at the vector corresponding to our particles state at some time $t$. Still, we don't factor time into our definition of observables, it's just baked into the point at which we evaluate them at.

Classical mechanics is a game of juggling initial conditions, inputs to Newton's laws (such as forces and potentials), and observables. We will see that in the abstract, quantum and classical mechanics are similar in this way.

But generally, classical mechanics and quantum mechanics are fundamentally different - you can't derive quantum mechanics from classical mechanics, and attempts to introduce it like that will fail. In reality, you should look at the mathematical framework of quantum mechanics and use that to build intuition. To help with this, looking at the \emph{mathematical} framework of classical mechanics may also help, but it shouldn't be where you derive your intuition.

\subsection{Postulates of Quantum Mechanics}

We begin by clearly setting out the mathematical framework that quantum mechanics uses.

Instead of phase space, everything we want to know about a quantum mechanical system is contained in a vector\footnote{This notation is Dirac notation, and it has a lot of nice features which are useful when doing quantum mechanics. This notation wasn't used in the IB course, but since it's so ubiquitous I just decided to use it from the beginning.} $\ket{\psi}$ in `Hilbert space'.

\begin{axiom}[State]
    The state of a quantum mechanical system is given by a non-zero \vocab{state vector} $\ket{\psi}$ in a \vocab{Hilbert space} $\mathcal{H}$, a complex vector space with a complete inner product $\ip{\cdot}{\cdot}$.
\end{axiom}

The exact choice of Hilbert space $\mathcal{H}$ will depend on the system being studied, and there is non-trivial physical systems where both finite and infinite dimensional spaces are used.

\begin{axiom}[Observables]
    The observables of a quantum mechanical system $O$ are given by self-adjoint linear operators $O: \mathcal{H} \rightarrow \mathcal{H}$.
\end{axiom}

\begin{axiom}[Time evolution]
    There is a distinguished quantum observable, the Hamiltonian $H$. Time evolution of states $\ket{\psi(t)} \in \mathcal{H}$ is given by the Schrödinger equation
    $$
    i \hbar \frac{\mathrm{d}}{\mathrm{d} t}|\psi(t)\rangle=H|\psi(t)\rangle.
    $$
    The operator $H$ has eigenvalues that are bounded below.
\end{axiom}

\subsection{Postulates of Quantum Mechanics}

Okay now we are just going to dump a bunch of stuff and say that it's intuitive and makes sense. The truth is that you have to start somewhere and this is where it's going to be.

I'm also going to just use dirac notation because literally everyone actually uses it, and since we are doing things properly there's no reason not to -- this is a mathematics course after all.

\begin{axiom}[States]
    The state of a quantum mechanical system is given by a non-zero vector $\ket{\psi}$ in a complex vector space $\mathcal{H}$ with Hermitian inner product $\langle \cdot, \cdot \rangle$.
\end{axiom}

We won't specify more about what $\mathcal{H}$ is here, as it depends on the system you're working with. There's real world systems where it can make sense to have $\mathcal{H}$ being finite or infinite dimensional.

Also of note: $\mathcal{H}$ is a \emph{vector space}, and the linear combination of states is also a state. It's also a \emph{complex} vector space, where linear combinations allow complex numbers. This is necessary -- I won't elaborate here.

\begin{axiom}[Observables]
    The observables of a quantum mechanical system are given by self-adjoint linear operators on $\mathcal{H}$.
\end{axiom}

\begin{axiom}[Dynamics]
    There is a distinguished quantum observable, the Hamiltonian $H$. Time evolution of states $\ket{\psi(t)} \in \mathcal{H}$ is given by the Schrödinger equation
    $$
    i \hbar \frac{\dd}{\dd t}|\psi(t)\rangle=H|\psi(t)\rangle.
    $$
    The operator $H$ has eigenvalues that are bounded below.
\end{axiom}

The Hamiltonian observable has the physical interpretation in terms of the energy, and boundedness implies there's a stable lowest energy state. 
$\hbar$ is a dimensionless constant called Planck's constant.

It's very much a logical leap but I'm just going to claim that for a particle moving in a potential $U$, we can take the Hamiltonian to be
$$
H = -\frac{\hbar^{2}}{2 m} \nabla^{2}+U.
$$
Good stuff.

We now need a way in our framework to go from what's happening on a quantum scale to what's happening in the real world. You'll notice that in classical mechanics, an observable is a function that takes a state and produces some numerical value -- something that can be thought of as reading out a value from some measurement device. For quantum mechanics, since we work with states in a vector space, if we had two different states with two different values in an observable, since the sum of states is also a state -- what would that observable be? This is a core quantum effect, and the way we deal with it is as follows.

\begin{axiom}[Measurement]
    States for which the value of an observable can be characterized by a well-defined number are the states that are eigenvectors for the corresponding self-adjoint operator. The value of the observable in such a state will be a real number, the eigenvalue of the operator.
\end{axiom}

\begin{axiom}[The Born Rule]
    Given an observable $O$ with two unit-norm states $\ket{\psi_1}$ and $\ket{\psi_2}$ that are eigenvectors of $O$ with distinct eigenvalues $\lambda_1$ and $\lambda_2$,
    $$
    O \ket{\psi_1} = \lambda_1 \ket{\psi_1}, \quad O\ket{\psi_2} = \lambda_2 \ket{\psi_2},
    $$
    the complex linear combination state 
    $$
    \alpha_1 \ket{\psi_1} + \alpha_2 \ket{\psi_2}
    $$
    will not have a well-defined value for the observable $O$. If one attempts to measure this observable, one would get either $\lambda_1$ or $\lambda_2$, with probabilities
    $$
    \frac{\abs{\alpha_1^2}}{\abs{\alpha_1^2} + \abs{\alpha_2^2}}
    $$
    and
    $$
    \frac{\abs{\alpha_2^2}}{\abs{\alpha_1^2} + \abs{\alpha_2^2}}
    $$
    respectively.
\end{axiom}

Note that this axiom implies there is a global phase on states that makes no difference to these probabilities. This is ignored, but one must keep in mind that \emph{relative} differences in phase \emph{can} make a difference.

\subsection{Conservation of Probability}

Because the schrodinger equation

Because the norm of a state is preserved by the Schrodinger equation, we really get probability conservation.

This can be written something like
$$
\frac{\partial}{\partial t} \rho(\vv x, t) + \nabla \cdot J = 0,
$$
where
$$
J(\vv x, t) = 
$$


\end{document}
