\documentclass[a4paper]{scrartcl}

\usepackage[
    fancytheorems, 
    fancyproofs, 
    noindent, 
]{adam}


\title{Quantum Mechanics}
\author{Adam Kelly (\texttt{ak2316@cam.ac.uk})}
\date{\today}

\allowdisplaybreaks

\begin{document}

\maketitle

% This is a short description of the course. It should give a little flavour of what the course is about, and what will be roughly covered in the notes.

This article constitutes my notes for the `Quantum Mechanics' course, held in Michaelmas 2021 at Cambridge. These notes are \emph{not a transcription of the lectures}, and differ significantly in quite a few areas. Still, all lectured material should be covered.



\tableofcontents

\section{Historical Introduction}

\begin{enumerate}
    \item \emph{1801-03}. Interference/diffraction experiments (by Young), found that light was a wave.
    \item \emph{1862-64}. The development of electromagnetism (by Maxwell), found that light was an electromagnetic wave.
    \item \emph{1897}. The discovery of the electron (by Thomson).
    \item \emph{1900}. The Plack law, explaining black body radiation. 
    \item \emph{1905}. The photo-electric effect (by Einstein).
    \item \emph{1909}. Wave-light interference patterns with one photon recorded at a time (by G. I. Taylor).
    \item \emph{1911}. Rutherford's experiment giving an atomic model.
    \item \emph{1913}. Bohr model of the atom.
    \item \emph{1923}. The Compton experiment of x-ray scattering off electrons.
    \item \emph{1923-24}. Wave-particle duality.
    \item \emph{1925-30}. The emergence of Quantum Mechanics (by Heisenberg, Bohr, Jordan, Dirac, Pauli, Schrödinger, \dots).
    \item \emph{1927-28}. Diffraction experiment with electrons (by Davisson, Germer, and Thomson).
\end{enumerate}


\subsection{Particles and Waves in Classical Mechanics}

We will begin by describing the physical background to particles and waves, as studied in classical mechanics.

% \begin{definition}[Particle]
    
% \end{definition}

A \vocab{particle} is an object carrying energy and momentum in an point-like position in space.
Particles are determined by two vectors, $\vv x$ (position) and $\vv v = \vv{\dot{x}}$ (velocity). Newton's second law gave us that
$$
    m \ddot{\vv x} = \vv{F}(\vv{x}(t), \vv{\dot{x}}(t)),
$$
and solving this equation would then describe the position totally, so long as the initial conditions are known.

A \vocab{wave} is any real or complex valued function with periodicity in time and/or space. If the period of the wave is $T$, then it has frequency $1/T$, and we define the \vocab{angular frequency} as $\omega = 2 \pi / T$. If we have a function of $x$ in one dimension, $f(x + \lambda) = f(x)$, we call $\lambda$ the \vocab{wavelength} and $k = 2\pi/\lambda$ the \vocab{wavenumber}.
In three dimensions, we have $f(x) = \exp(i \vv k \cdot \vv x)$, which we call \vocab{plane waves}. We call $\vv k$ the \vocab{wavevector}, and $\lambda = 2\pi/|\vv k|$.

One dimensional waves obey the wave equation
$$
    \frac{\partial^2 f(x, t)}{\partial t^2} - c^2 \frac{\partial^2 f(x, t)}{\partial x^2} = 0,
$$
with $c \in \R$. The solutions are given by
$$
    f(x, t) = A \exp(\pm i kx - i \omega t), 
$$
with $\omega = ck$ and $\lambda = cT$ (the dispersion relations). We call $A$ the amplitude.

The three dimensional wave equation is given by
$$
\frac{\partial^2 f(\vv x, t)}{\partial t^2} - c^2 \nabla^2 f(\vv x, t) = 0,
$$
and the solutions are given by
$$
f(\vv x, t) = A \exp(i \vv k \cdot \vv x - i \omega t),
$$
with $\omega = c |\vv k|$ and $\lambda = cT$.

Other kinds of waves arise as solutions to other governing wave equations, provided other dispersion relations are given.
Also for any governing equation that is linear in $f$, we always have the superposition principle, saying that if $f_1, f_2$ are solutions to the equation, then so is $f = f_1 + f_2$.

% \subsection{Particle-like Behavior of Light-waves}

% % Blackbody radiation

\subsection{Particle-Like Behavior of Light-Waves}

If you heat a body to a temperature $T$, it will emit radiation.
The simplest system which this behavior can be studied in is a \emph{black-body}, a totally absorbing surface. In studying such a system, Kirchoff looked at the intensity of light emitted by the black body as a function of the radiation frequency. He observed a %insertimagehere

\end{document}
