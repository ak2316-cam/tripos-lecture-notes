\documentclass[a4paper]{scrartcl}

\usepackage[
    fancytheorems, 
    fancyproofs, 
    noindent, 
    % nokoma
]{adam}


\usepackage{physics}

\title{Quantum Mechanics}
\author{Adam Kelly (\texttt{ak2316@cam.ac.uk})}
\date{\today}


\allowdisplaybreaks

\begin{document}

\maketitle


Quantum mechanics is the mathematical framework used to describe nature at the scale of subatomic particles. These notes will primarily focus on building up this mathematical framework, and not so much on the actual applications or physical background to it.

This article constitutes my notes for the `Quantum Mechanics' course, held in Michaelmas 2021 at Cambridge. These notes are \emph{not a transcription of the lectures}, and differ significantly\footnote{In particular, these notes are heavily influenced by the `Principles of Quantum Mechanics' course, along with numerous books.} in quite a few areas (more so than some of my other notes). Still, all lectured material should be covered.





\tableofcontents

\section{Foundations of Quantum Mechanics}

\subsection{Classical Mechanics}

Write a bit here about the way we think about classical mechanics, and how this makes the presentation of the QM postulates look reasonable.

{\color{blue}
\begin{itemize}
\item The state of a particle at a time $t$ is specified by two quantities: the position $r(t)$ and momentum $p(t)$.
\item This means that the state of a particle is totally enoded by a point $(x(t), p(t))$ in phase space. 
\item Knowing these variables at a time $t_0$ allows us to then determine the future state of the particle, using Newton's laws or (and this will look similar to what we do) Hamilton's equations
$$
\frac{\dd x}{\dd t} = \frac{\partial H}{\partial p} \quad \text{and} \quad \frac{\dd p}{\dd t} = - \frac{\partial H}{\partial x}.
$$
\item Also for any other physical quantity, we can write them in terms of these two variables. 
\item In particular, a physical quantity $O$ can really by thought of as a function on phase space $O: M \rightarrow \R$. Things like position, kinetic energy, angular momentum and everything else can be written like this. We evaluate $O(x(t), p(t))$ to get the value at some time $t$, but notice the time dependence is baked into $M$ not $O$.
\item Classical mechanics is a game of juggling initial conditions, physical iputs (forces and potentials) and observables. We will see that QM is like this.
\item Generally, QM and CM are different. You can't derive quantum from classical, and any attempt to do so will fail. You should instead use the mathematical framework of CM to build intuition as to the way that the QM framework is built up, and then try and derive your intition from the precise way we will deal with QM.
\end{itemize}
}

\subsection{Postulates of Quantum Mechanics}

We now want to set out the mathematical framework that QM uses. {\color{blue}In particular, it's natural to think about answering the following questions: How is a quantum state represented mathematically? How do we take this representation and calculate physical quantities? Given the state at some time $t_1$, how do we find the state at some time $t_2$? With answers to these we are well on our way.
}

The postulates.

{\color{blue}
\begin{itemize}
    \item \textbf{State}. The state of a quantum mechanical system at some time $t$ is given by a non-zero \vocab{state vector} $\ket{\psi(t)}$ in a Hilbert space $\mathcal{H}$.
    \item \textbf{Observables}. To every physically measurable quantity $O$ (an \vocab{observable}) there is a Hermitian operator $\hat{O}: \mathcal{H} \rightarrow \mathcal{H}$ whose eigenvectors form a complete basis.
    \item \textbf{Measurement}. {\color{red}TODO FILL IN}. The result of a measuring a quantity $O$ results in an eigenvalue of the corresponding operator $\hat{O}$. If we perform a measurement of $O$ on a state $\ket{t}$ and the result is $\lambda$, then the system will be left in a corresponding eigenstate $\ket{\psi_{\lambda}}$ immediately after the measurement.
    \item \textbf{The Born Rule}. Measure discrete and continuous, bake in normalisation and square integrable or something about inner products.
    \item \textbf{Time Evolution}. Time evolution of the state vector $\ket{\psi(t)} \in \mathcal{H}$ is described by the time-dependent Schrodinger equation
    $$
    i \hbar \frac{\partial \ket{\psi(t)}}{\partial t} = H \ket{\psi(t)},
    $$
    where $H$ is the Hamiltonian operator, corresponding to the total energy of the system.
\end{itemize}
}

\end{document}
