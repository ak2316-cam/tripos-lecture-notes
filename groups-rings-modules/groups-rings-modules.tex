%  \documentclass[DIV=12, a4]{scrartcl}
%\documentclass[12pt, a5]{scrartcl}

% \documentclass[a4paper]{report}
% \usepackage[
% % fancytheorems, 
% noindent, 
% %spacingfix, 
% %noheader
% ]{vanilla}


\documentclass[a4paper]{scrreprt}
\usepackage[
fancytheorems, 
noindent, 
% %spacingfix, 
% %noheader,
fancyproofs
]{adam} 

\usepackage{tikz}

% \usepackage{subfig}

\setcounter{chapter}{-1}

\title{Groups, Rings and Modules}
% \subtitle{Adam Kelly}
\author{Adam Kelly}
% \date{Michaelmas 2020}
\date{\today}

\begin{document}

\maketitle

\begin{abstract}
	
	% \vspace{2\baselineskip}
	% {\color{red} None of the notes here have been reviewed at all, and are just exactly what was taken down live in the lectures. I would turn around now and come back in a few days, when I have gone back, cleaned things up, fixed explanations and added some structure.}
	% \vspace{5\baselineskip}

	This set of notes is a work-in-progress account of the course `Groups, Rings and Modules', originally lectured by Dr Tom Fisher in Lent 2020 at Cambridge. These notes are not a transcription of the lectures, but they do roughly follow what was lectured (in content and in structure).

	These notes are my own view of what was taught, and should be somewhat of a superset of what was actually taught. I frequently provide different explanations, proofs, examples, and so on in areas where I feel they are helpful. Because of this, this work is likely to contain errors, which you may assume are my own. If you spot any or have any other feedback, I can be contacted at \href{mailto:ak2316@cam.ac.uk}{ak2316@cam.ac.uk}.


	% {\color{red} Notes written upto lecture 6.}
	% During the creation of this document, I consulted a number of other books and resources. All of these are listed in the bibliography. 

\end{abstract}

\tableofcontents

\clearpage
\chapter{Introduction}



\section{Structure of the Course}

This course is, quite naturally, divided into three sections.

\begin{enumerate}
	\item \emph{Groups}
	
	We will be continuing on from IA Groups, paying particular attention to certain topics such as simple groups, $p$-groups and $p$-subgroups. The main highlight of this part of the course will be the Sylow theorems.

	\item \emph{Rings}
	
	Rings are sets where we can add, subtract and multiply (but not necessarily divide), for example, $\Z$. A ring where division is always possible is a field, for example $\Q$, $\R$ and $\Z/p\Z$ for a prime $p$.

	\item \emph{Modules}
	
	A module is the analog of a vector space where we work over a ring, rather than a field. We will attempt to classify modules over certain `nice' rings. This will allow us to prove the Jordan Normal theorem of matrices and to classify finite abelian groups. 
\end{enumerate}

\section{Books}

As with most mathematics courses in Cambridge, you will not need a textbook to follow this course. What is covered in lectures is enough to do both the example sheets and the examinations for this course. Still, you might find that a textbook can provide a different perspective, additional worked examples, and additional material that you may find informative, helpful or fun.

In particular, the following books are quite relevant/good, but there is no expectation that you will look at these.

\begin{itemize}
	\item P. M. Cohn, \emph{Classical Algebra}.
	
	This covers the whole course (but does have some weird notation).

	\item Hartley \& Hawkes, \emph{Rings, Modules \& Linear Algebra}.
	
	This is a good reference for the `rings and modules' part of the course, but notably doesn't include any content on group theory.
\end{itemize}

You should be able to find all of these books in either your college library or the university library.


\clearpage


\chapter{Groups -- Revision and Basics}

The first algebraic object that we shall consider in this course is one you are likely familiar with -- a group. 

\section{Definitions}

We will begin by defining what a group is.

\begin{definition}[Group]
	A \vocab{group} is a pair ($G$, $*$) consisting of a set $G$ and a binary operation\footnote{Some texts include an additional \emph{closure} axiom, but this is implied by $*$ being a binary operation on $G$.} $* : G \times G \rightarrow G$ satisfying the axioms:
	\begin{itemize}
		\item \emph{Identity}. There is an element $e \in G$ such that $e * g = g * e = g$ for all $g \in G$,
		\item \emph{Inverses}. For every element $g \in G$, there is an element $g^{-1} \in G$ such that $g * g^{-1} = g^{-1} * g = e$.
		\item \emph{Associativity}. The operation $*$ is associative.
	\end{itemize}
\end{definition}

\begin{remark}
	We will usually either use additive or multiplicative notation for groups, and in these cases we will often write $0$ or $1$ for the identity respectively.
\end{remark}

\begin{definition}[Subgroup]
	A subset $H \subseteq G$ is a \vocab{subgroup} of $G$, written $H \leq G$, if it is a group with respect to the operation $*$ defined on $H \times H$.
\end{definition}

There is a way to test the conditions needed for a subset to be a subgroup in just a few lines, but it does have limited utility.

\begin{lemma}[Fast Subgroup Checking Lemma]
A nonempty subset $H \subseteq G$ is a subgroup if $a, b \in H$ implies $a * b^{-1} \in H$.
\end{lemma}
\begin{proof}[Proof Sketch]
	Check that this implies the definition.
\end{proof}

\begin{example}[Examples of Groups]
	The following are all examples of groups.
	\begin{enumerate}[label=(\roman*)]
		\item The additive groups $(\Z, +) \leq (\Q, +) \leq (\R, +)$.
		\item The cyclic group of order $n$, $C_n$.
		\item The dihedral group $D_{2n}$ of the symmetries of a regular $n$-gon.
		\item The symmetric group $S_n$ and alternating group $A_n$, where $S_n$ is the group of permutations of $\{1, 2, \dots, n\}$ and $A_n \leq S_m$ is the group of even permuations. 
		\item The quaternion group $Q_8 = \{\pm 1, \pm i, \pm j, \pm k \}$ with $i^2 = j^2 = k^2 = ijk = -1$.
		\item The matrix groups over some field $F$, $\GL_n(F)$ of all $n \times n$ matrices over $F$ with non-zero determinant, and $\SL_n(F) \leq \GL_n(F)$, the subgroup of matrices with determinant 1.
	\end{enumerate}
\end{example}

\begin{definition}[Direct Product]
	The \vocab{direct product} of groups $G$ and $H$ is $G \times H$ with operation $(g_1, h_1) * (g_2, h_2) = (g_1 g_2, h_1 h_2)$. 
\end{definition}

For a subgroup $H \leq G$, the \vocab{left cosets} of $H$ in $G$ are the sets $gH = \{gh \mid h \in H \}$ where $g \in G$. Recall that these partition $G$, and each has the same cardinality as $H$. From this we deduce Lagrange's theorem.

\begin{theorem}[Lagrange's Theorem]
	Let $G$ be a finite group, and $H$ be a subgroup. Then $|G| = |H| \cdot |G : H|$ where $|G : H|$ is the \vocab{index} of $H$ in $G$, the number of left cosets of $H$ in $G$.
\end{theorem}

It is natural to wonder whether there is a converse to Lagrange's theorem, and it turns out that the converse is \emph{not} true in general. There is a partial converse however.

\begin{theorem}[First Sylow Theorem]
	If $G$ is a group with $|G| = p^a m$ where $p$ is a prime and $p \nmid m$, then there exists $H \leq G$ with $|H| = p^a$.
\end{theorem}

We will prove this theorem later on.

\begin{definition}[Order of an Element]
	Let $G$ be a group and $g \in G$. If there exists $n \geq 1$ such that $g^n = 1$, then the least such $n$ is the \vocab{order} of $g$. If no such $n$ exists, we say $g$ hsa infinite order.
\end{definition}

\begin{remark} 
If $g$ has order $d$, then $g^n = 1 \iff d \mid n$. The proof follows from the division algorithm. Also, $\{1, g, g^2, \dots, g^{d - 1}\} \leq G$ and so if $G$ is finite, then by Lagrange, $d \mid |G|$.
\end{remark}

\begin{definition}[Normal Subgroup]
	A subgroup $H \leq G$ is \vocab{normal} if $g^{-1} H g = H$ for all $g \in G$. We write $H \normal G$ in this case.
\end{definition}

\begin{proposition}[Quotient Group]
	If $H \normal G$, then the set $G/H$ of left cosets of $H$ in $G$ is a group called the \vocab{quotient group} with the operation $g_1 H * g_2 H = (g_1 g_2) H$.
\end{proposition}
\begin{proof}
	We must check that $*$ is well defined. Suppose that $g_1H = g_1' H$ and $g_2H = g_2' H$. Then $g_1' = g_1 h_1$ and $g_2' = g_2 h_2$ for some $h_1, h_2 \in H$. Then we get $g_1' g_2' H = g_1 h_1 g_2 h_2 H = g_1 h_1 g_2 H$. This is equal to $g_1 g_2 H$ if and only if $(g_1 g_2)^{-1} g_1 h_1 g_2 \in H$, that is, if $g_2^{-1} h_1 g_2 \in H$, which follows from the normality of $H$. Now to check the group axioms, note that associativity is inherited, we have the coset $H$ being the identity, and the inverse of $gH$ being $g^{-1} H$. Thus $G/H$ is a group.
\end{proof}

\end{document}
